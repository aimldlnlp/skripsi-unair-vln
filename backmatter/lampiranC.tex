\chapter{LAMPIRAN 3: package.h dan package.cpp}
\section*{package.h}
\scriptsize
\begin{verbatim}
 #ifndef PAC_H
#define PAC_H

//Jumlah yang ada didalam vektor integer $prod
int pac_prod(std::vector<int> S);

//Rem merupakan nilai sisa pembagian $rem
int pac_rem(int a, int b);

//besar adalah paket untuk mencari nilai lebih besar jika > 1, < 0. $besar
std::vector<int> pac_besar(std::vector<int> m, int s);

//Paket untuk menjadikan vektor menjadi matrix Diagonal $Diag
std::vector<std::vector<int> > pac_Diag(std::vector<int> m);

//Matrix to Vector $diag
std::vector<int> pac_diag(std::vector<std::vector<int> > M);
#endif
\end{verbatim}
\normalsize
\section*{package.cpp}
\scriptsize
\begin{verbatim}
#include "edft.h"
#include "package.h"

typedef std::vector<int> vector_int;
typedef std::vector<double> vector_double;
typedef std::vector<vector_int> vector_vector;

//c-%pac_prod Jumlah yang ada didalam vektor integer $prod
int pac_prod(std::vector<int> S)
{
 int hasil=1;
 for(int i=0;i<S.size();i++)
  {
   hasil=hasil*S[i];
  } 
 return hasil;
 S.clear();
}

//c- %pac_rem merupakan nilai sisa pembagian $rem 
int pac_rem(int a, int b)
{
 int hasil;
 div_t divresult;
 divresult=div(a,b);
 hasil = divresult.rem;
 return hasil;
}

//c-%pac_besar merupakan fungsi untuk mengetahui nilai lebih besar, jika
//lebih besar, maka hasilnya satu, jika lebih kecil nilainya 0. $besar
std::vector<int> pac_besar(std::vector<int> m, int s)
{
 std::vector<int> hasil(m.size());
 for(int i=0;i<m.size();i++)
  {
   int sementara;
    if(m[i]>s/2) sementara=1;
      else sementara=0;
   hasil[i]=m[i]-(sementara)*s;
  }
 return hasil;
 hasil.clear(); m.clear();
}

//c-%pac_Diag Package untuk merubah vektor ke matriks Diagonal. $Diag
vector_vector pac_Diag(vector_int m)
{
 vector_vector hasil;
 vector_int sementara(m.size());
 for (int i=0;i<m.size();i++)
  {
   for (int j=0;j<m.size();j++)
    {
      if (i==j) sementara[j]=m[j]; else sementara[j]=0;
    }
   hasil.push_back(sementara);
  }
  return hasil;
  hasil.clear(); m.clear(); sementara.clear();
}

//c- %pac_diag Matrix to Vector $diag
vector_int pac_diag(vector_vector M)
{
 vector_int hasil(M.size());
 for (int i=0;i<3;i++)
  {
   hasil[i]=M[i][i];
  }
 return hasil;
 hasil.clear(); M.clear();	
}
\end{verbatim}


\chapter{LAMPIRAN}
\section*{\textbf{Lampiran 1:} Desain \textit{Prompt} untuk Pembangkitan Tugas LH-VLN \textit{Code-Switching} Indonesia-Inggris}
\label{appendix: prompts}
\vspace{0.5em}

% =========================
% Lampiran 1.1
% =========================
\subsection*{\textbf{Lampiran 1.1:} \textit{Prompt} Sistem Navigasi LH-VLN (Pembangkitan Instruksi)}
\label{appendix:prompt-system-nav}
\begin{tcolorbox}[breakable, colback=white, colframe=black, boxrule=0.5pt]
\small
\textbf{[system]:}\\
You are good at guiding the way. Convert a structured action plan into ONE fluent, practical navigation instruction in code-switching Indonesia--English. Keep it grounded in the provided tags and easy to follow.

\medskip
Style and constraints for the SINGLE output paragraph:
\begin{itemize}
  \item Code-switching ratio: aim for 60--80\% Indonesian and 20--40\% English.
  \item Use imperative voice (e.g., ``Pergi\ldots'', ``Lanjut\ldots'', ``Belok\ldots'', ``move\ldots'', ``turn\ldots''), no slang, no bullet points, no numbering, no explicit word ``step''.
  \item Do NOT use the literal words ``grab'' or ``release''.
  \item Mention ONLY rooms/objects that appear in the input tags or the input target.
  \item Keep to 1--2 regions overall (infer region/room from tags; do not introduce new rooms).
\end{itemize}

\textbf{[user]:}\\
The INPUT is a dictionary with this shape:
\begin{quote}
\{\\
\ \ ``target'': $<$final navigation target object$>$,\\
\ \ ``step\_0'': \{``action'': $<$move\_forward$|$turn\_left$|$turn\_right$>$, ``tags'': [$<$scene tags$>$]\},\\
\ \ \ldots\\
\ \ ``step\_n'': \{``action'': $<$move\_forward$|$turn\_left$|$turn\_right$>$, ``tags'': [$<$scene tags$>$]\}\\
\}
\end{quote}

Guidelines:
\begin{itemize}
  \item For each step, choose ONE most relevant tag to mention (do not list many tags).
  \item Steps with move\_forward should prefer a specific region/room tag; turn\_left/turn\_right may prefer a specific object/landmark tag.
  \item The order of actions in the final instruction MUST follow the steps order.
  \item The last part MUST naturally reach the target object.
  \item Output only the final instruction string (no quotes, no extra text, no JSON, no code fences).
\end{itemize}

Here is a tiny example (for style only, not to be reused verbatim):

INPUT (snippet):
\begin{quote}
\{\\
\ \ ``target'': ``desk'',\\
\ \ ``step\_0'': \{``action'': ``move\_forward'', ``tags'': [``bedroom'']\},\\
\ \ ``step\_1'': \{``action'': ``turn\_left'', ``tags'': [``picture'']\}\\
\}
\end{quote}

DESIRED STYLE (example):\\
Jalan pelan dari bedroom, then turn left di dekat picture, lanjutkan sampai mencapai desk.

\medskip
Now generate the final instruction for the following INPUT. Think briefly, then answer with the instruction only.
\end{tcolorbox}
% \clearpage


% =========================
% Lampiran 1.2
% =========================
\subsection*{\textbf{Lampiran 1.2:} \textit{Prompt} Aturan Subtugas LH-VLN}
\label{appendix:prompt-rules}
\begin{tcolorbox}[breakable, colback=white, colframe=black, boxrule=0.5pt]
\small
The task you create must be decomposed into navigation--interaction subtasks using ONLY the following three function formats:
\begin{itemize}
\item \texttt{Move\_to("<object>\_<regionId>")}: Move to an object within a region. NOTE: \texttt{<object>\_<regionId>} MUST combine the object name and the corresponding region ID of that object in the input scene (e.g., \texttt{"cup\_3"}).
\item \texttt{Grab("<object>")}: Pick up an object after reaching it. NOTE: the \texttt{<object>} MUST be present in the input scene and MUST be portable.
\item \texttt{Release("<object>")}: Place the object currently held at the current location. NOTE: the \texttt{<object>} MUST match the object being held and MUST be present in the input scene.
\end{itemize}

There are several things to note:
\begin{itemize}
\item The output must use code-switching Indonesian--English with a ratio of 60--80\% Indonesian and 20--40\% English.
\item IMPORTANT: Do NOT use the literal words ``grab'' or ``release'' in the \textbf{Task instruction} (natural language). Using \texttt{Grab(...)} and \texttt{Release(...)} as function names in the \textbf{Subtask list} is allowed.
\item Objects mentioned MUST be limited to 1--2 regions total. You must explicitly name these regions in the \textbf{Task instruction}. All regions must exist in the input scene.
\item The overall assignment should be similar to: ``Ambil objek portable dari suatu region, lalu move ke region lain dan letakkan di lokasi tertentu, kemudian ambil objek lain'' (tetap dalam 1--2 region).
\item The task must contain 2--6 subtasks.
\item Subtask ordering must be logically consistent: \texttt{Grab} can only happen after moving to that object; \texttt{Release} can only happen after \texttt{Grab}.
\item The region ID in each \texttt{Move\_to} subtask MUST match the region named in the \textbf{Task instruction}.
\item The task should be practical and reasonable for a household robot, considering the robot characteristics given in the input.
\end{itemize}

Your output MUST be a Python dictionary with exactly two keys:
\begin{itemize}
\item \texttt{dictionary["Task instruction"]}: One conversational, imperative instruction (no bullets/numbering).
\item \texttt{dictionary["Subtask list"]}: A list of subtasks (2--6 items) using the function formats above.
\end{itemize}
\end{tcolorbox}
% \clearpage


% =========================
% Lampiran 1.3
% =========================
\subsection*{\textbf{Lampiran 1.3:} \textit{Prompt} Deskripsi Robot Spot di Simulator}
\label{appendix:robot-spot}
\begin{tcolorbox}[breakable, colback=white, colframe=black, boxrule=0.5pt]
\small
Spot in the simulator is an agile, quadrupedal robot.\\
\textbf{Action:} Spot supports three actions: \texttt{move\_forward}, \texttt{turn\_left}, and \texttt{turn\_right}.\\
\textbf{Sensors:} Spot is equipped with three RGB cameras at a height of 0.5 meters (front, left, and right) to capture embodied images from these directions.\\
\textbf{Mobility:} Spot's four-legged design allows it to navigate challenging terrains, including stairs, rocky surfaces, and cluttered environments.\\
\textbf{Use:} Since Spot is dog-shaped, it can assist humans in domestic scenes. It has a simple robotic arm positioned low (0.5 meters) and can perform basic pick-and-place.
\end{tcolorbox}
% \clearpage


% =========================
% Lampiran 1.4
% =========================
\subsection*{\textbf{Lampiran 1.4:} \textit{Prompt} Deskripsi Robot Fetch di Simulator}
\label{appendix:robot-fetch}
\begin{tcolorbox}[breakable, colback=white, colframe=black, boxrule=0.5pt]
\small
The Fetch robot in the simulator is a versatile mobile robot designed for logistics and material handling.\\
\textbf{Action:} Fetch supports three actions: \texttt{move\_forward}, \texttt{turn\_left}, and \texttt{turn\_right}.\\
\textbf{Sensors:} Fetch is equipped with three RGB cameras at a height of 1 meter (front, left, and right) to capture embodied images from these directions.\\
\textbf{Mobility:} Fetch moves with a wheeled base, so it can only travel on flat ground and cannot traverse rugged terrain.\\
\textbf{Use:} Fetch has a flexible robotic arm mounted higher (1 meter) that can perform more complex pick-and-place operations.
\end{tcolorbox}
% \clearpage


% =========================
% Lampiran 1.5
% =========================
\subsection*{\textbf{Lampiran 1.5:} \textit{Prompt} Sistem Desain Tugas LH-VLN}
\label{appendix:prompt-system-taskdesign}
\begin{tcolorbox}[breakable, colback=white, colframe=black, boxrule=0.5pt]
\small
\textbf{[system]:}\\
You are proficient in planning and dataset design. You will design a practical long-horizon vision-language navigation task with object interaction based on a scene graph and robot characteristics, and then output a consistent subtask decomposition.

\medskip
\textbf{[user]:}\\
There are two parts of the input: \texttt{scene} and \texttt{robot}. The \texttt{scene} includes regions and their objects. The \texttt{robot} describes capabilities and constraints (e.g., mobility, camera height, arm reach), which you must consider.

\medskip
Input format:
\begin{quote}
scene: \{\\
\ \ ``Region 1: <name>'': [<objects>],\\
\ \ ``Region 2: <name>'': [<objects>],\\
\ \ \ldots\\
\}\\
robot: <robot description text>
\end{quote}

\medskip
Your output MUST follow these requirements:
\begin{itemize}
\item Output a Python dictionary with keys \texttt{"Task instruction"} and \texttt{"Subtask list"}.
\item The \texttt{"Task instruction"} must be ONE conversational imperative paragraph in code-switching Indonesian--English (60--80\% Indonesian, 20--40\% English), no bullets/numbering, no slang.
\item Use ONLY objects and regions from the given \texttt{scene}, and limit the task to 1--2 regions total (explicitly named in the instruction).
\item Include 2--6 subtasks using ONLY: \texttt{Move\_to("<object>\_<regionId>")}, \texttt{Grab("<object>")}, \texttt{Release("<object>")}.
\item Do NOT use the literal words ``grab'' or ``release'' in the natural-language instruction (function names in the subtask list are allowed).
\item Ensure logical consistency and portability: you can only pick up portable objects; order must be \texttt{Move\_to} $\rightarrow$ \texttt{Grab} $\rightarrow$ (optional \texttt{Move\_to}) $\rightarrow$ \texttt{Release}.
\item Make the task realistic under robot constraints (e.g., wheeled robots stay on flat indoor areas).
\end{itemize}

Output only the Python dictionary (no extra explanation, no code fences).
\end{tcolorbox}
% \clearpage


% =========================
% Lampiran 1.6
% =========================
\subsection*{\textbf{Lampiran 1.6:} Contoh Lengkap INPUT--OUTPUT Tugas LH-VLN}
\label{appendix:example-io}
\begin{tcolorbox}[breakable, colback=white, colframe=black, boxrule=0.5pt]
\small
Here is an example of the INPUT and OUTPUT:

\medskip
\textbf{INPUT:}
\begin{quote}
scene: \{\\
\ \ ``Region 1: Bedroom'': [``picture'', ``bed'', ``lamp''],\\
\ \ ``Region 2: Bathroom'': [``towel'', ``sink'', ``toilet paper dispenser''],\\
\ \ ``Region 3: Kitchen'': [``pan'', ``cup'', ``plate''],\\
\ \ ``Region 4: Office'': [``board'', ``chair'', ``desk'']\\
\}\\
robot: The Fetch robot in simulator is a versatile mobile robot designed for logistics and material handling. It supports \texttt{move\_forward}, \texttt{turn\_left}, and \texttt{turn\_right}. It has three RGB cameras at 1 meter height and a flexible arm for pick-and-place, but it moves on a wheeled base and stays on flat ground.
\end{quote}

\medskip
\textbf{OUTPUT:}
\begin{quote}
\{\\
\ \ ``Task instruction'': ``Ambil picture dari Bedroom, lalu go to Office dan letakkan di desk; setelah itu, tetap di Office dan ambil chair untuk dipindahkan sesuai kebutuhan.'',\\[2pt]
\ \ ``Subtask list'': [\\
\ \ \ \ ``Move\_to('picture\_1')'',\\
\ \ \ \ ``Grab('picture')'',\\
\ \ \ \ ``Move\_to('desk\_4')'',\\
\ \ \ \ ``Release('picture')'',\\
\ \ \ \ ``Move\_to('chair\_4')'',\\
\ \ \ \ ``Grab('chair')''\\
\ \ ]\\
\}
\end{quote}

\medskip
\textit{Note:} The task uses only two regions (Bedroom and Office), and the subtasks follow a consistent order.
\end{tcolorbox}

\clearpage % atau \newpage

\section*{\textbf{Lampiran 2:} Contoh Data yang Digunakan dan/atau Dihasilkan dari \textit{Pipeline}}
\vspace{0.5em}

\subsection*{\textbf{Lampiran 2.1:} Contoh Metadata \textit{Scene}}
\begin{lstlisting}[basicstyle=\rmfamily\normalsize, breaklines=true]
Region id:_-1, position:[0. 0. 0.]
Region id:_0, position:[0. 0. 0.]
Id:2, name:armchair, position:[4.771444, 3.526179, -4.256199]
Id:3, name:plant, position:[1.6414305, 4.1937037, -4.5211315]
Id:4, name:chest of drawers, position:[1.4356585, 3.5864055, -4.4750156]
...
Region id:_10, position:[0. 0. 0.]
Id:286, name:chest of drawers, position:[-3.629725, 0.44680634, -1.6254207]
Id:293, name:tv, position:[-4.007966, 1.1502895, -1.0507892]
...
Region id:_15, position:[0. 0. 0.]
Id:703, name:toilet paper dispenser, position:[-1.7919614, -2.2921019, 2.544517]
\end{lstlisting}
\clearpage

\subsection*{\textbf{Lampiran 2.2:} Contoh Gambar Navigasi Agen dari \textit{Start} Hingga \textit{Finish}}
\noindent Instruksi: "Pergi ke \textit{laundry room} dan \textit{pick up the towel}, kemudian \textit{move to bathroom and place it neatly on the shelf}"
\begin{figure}[H]
    \centering
    \includegraphics[
        width=1\textwidth,
    ]{images/contact_sheets/contact_sheet_01_step_-1_to_28.png}
    \vspace{-0.5em} % atur sendiri: -0.5em, -1em, -5pt, dll
    \includegraphics[
        width=1\textwidth,
    ]{images/contact_sheets/contact_sheet_02_step_29_to_58.png}
    % \caption{...}
    \label{fig:contact_sheet_1-2}
\end{figure}

\begin{figure}[H]
    \centering
    \includegraphics[
        width=1\textwidth,
    ]{images/contact_sheets/contact_sheet_03_step_59_to_87.png}
    \vspace{-0.5em} % atur sendiri: -0.5em, -1em, -5pt, dll
    \includegraphics[
        width=1\textwidth,
    ]{images/contact_sheets/contact_sheet_04_step_88_to_117.png}
    % \caption{...}
    \label{fig:contact_sheet_3-4}
\end{figure}

\begin{figure}[H]
    \centering
    \includegraphics[
        width=1\textwidth,
    ]{images/contact_sheets/contact_sheet_05_step_118_to_147.png}
    \vspace{-0.5em} % atur sendiri: -0.5em, -1em, -5pt, dll
    \includegraphics[
        width=1\textwidth,
    ]{images/contact_sheets/contact_sheet_06_step_148_to_177.png}
    % \caption{...}
    \label{fig:contact_sheet_5-6}
\end{figure}

\begin{figure}[H]
    \centering
    \includegraphics[
        width=1\textwidth,
    ]{images/contact_sheets/contact_sheet_07_step_178_to_207.png}
    \vspace{-0.5em} % atur sendiri: -0.5em, -1em, -5pt, dll
    \includegraphics[
        width=1\textwidth,
    ]{images/contact_sheets/contact_sheet_08_step_208_to_214.png}
    % \caption{...}
    \label{fig:contact_sheet_7-8}
\end{figure}
\clearpage

\subsection*{\textbf{Lampiran 2.3:} Contoh Format Data JSON untuk Instruksi Tugas dan Subtugas}
\begin{lstlisting}[language=json, basicstyle=\rmfamily\normalsize, breaklines=true]
{
  "Task instruction": "Pergi ke laundry room dan pick up the towel, kemudian move to bathroom and place it neatly on the shelf",
  "Subtask list": [
    "Move_to('towel_0')",
    "Grab('towel')",
    ...
    "Release('towel')"
  ],
  "Robot": "stretch",
  "Scene": "00685-ENiCjXWB6aQ",
  "Object": ["towel", "bathroom shelf"],
  "Region Name": ["Bathroom", "Bathroom"],
  "Region": ["0", "14"],
  "Geo dis": [12.3853874, 25.3590908],
  "trial": {
    "trial_0": {
      "pos": [
        [-2.6512451, 1.9263153, 6.9955611],
        [-2.6512451, 1.6535878, 6.7455611],
        ...,
        [ 3.0860734, 0.1888131, 0.6243353]
      ],
      "yaw": [180, 180, 180, ..., 90],
      "action": ["stop", "move_forward", "move_forward", ..., "stop"]
    },
    "trial_1": { ... }
  }
}
\end{lstlisting}
\clearpage

\subsection*{\textbf{Lampiran 2.4:} Contoh Format Data Pembangkitan Subtugas yang Diperinci}
\begin{lstlisting}[language=json, basicstyle=\rmfamily\normalsize, breaklines=true]
{
    "trajectory path": "task/2/Pergi ke laundry room dan pick up the towel, kemudian move to bathroom and place it neatly on the shelf/success/trial_1",
    "start": 0,
    "end": 69,
    "Robot": "stretch",
    "Scene": "00685-ENiCjXWB6aQ",
    "target": [
        "towel"
    ],
    "Region": [
        "0"
    ],
    "start_pos": [
        -2.6512451171875,
        1.9263153076171875,
        6.995561122894287
    ],
    "start_yaw": 180,
    "Task instruction": "Pergi dari balustrade ke hallway, lalu putar kanan di dekat picture frame, lanjut ke bathroom, kemudian belok kanan sampai mencapai towel."
}
\end{lstlisting}
\clearpage

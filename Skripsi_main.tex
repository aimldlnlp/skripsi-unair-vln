% Template skripsi FTMM Unair V1 2025
% Dibuat oleh Rizki Putra Prastio berdasarkan pedoman skripsi.
% Anda dapat mengubah dokumen ini sesuai dengan panduan yang berlaku. Anda juga diizinkan untuk mengganti, menambah, atau mengurangi package
% yang digunakan jika memang diperlukan.
% Segala bentuk pertanyaan terkait dokumen ini dapat dikirimkan melalui r.p.prastio@ftmm.unair.ac.id atau rpprastio@gmail.com.
% Atau jika anda telah memahami cara menulis dengan LaTeX, anda diizinkan untuk mengedit.
% Kelas Dokumen dengan ukuran huruf 12pt, format buku satu sisi (Tidak bolak balik).
\documentclass[12pt,oneside]{book}
\usepackage[table]{xcolor}
\usepackage{tikz}
\usetikzlibrary{patterns}
\usepackage{pgfplots}
\usetikzlibrary{arrows.meta,positioning,calc,shapes,fit}
\usetikzlibrary{arrows.meta,positioning,matrix,shapes.geometric}
\usepackage{amsmath,amsthm,amssymb,amstext,amsfonts}
\usepackage{tabularx}
\usepackage{lipsum}
\usepackage{svg}        % ini yang penting
\usepackage{enumitem}
\usepackage{amscd}
\usepackage{graphics}
\usepackage{layout}
\usepackage{graphicx}
\usepackage{makeidx}
\usepackage{listings}
\usepackage{longtable}
\usepackage[acronym,nonumberlist]{glossaries}

\lstdefinelanguage{json}{
    morestring=[b]",
    morecomment=[l]{//},
    sensitive=true
}

\lstset{
    language=json,
    basicstyle=\rmfamily\normalsize, % Times New Roman 12pt
    columns=fullflexible,
    breaklines=true
}
\usepackage{subcaption}
\usepackage{caption}
\usepackage[T1]{fontenc}
\usepackage[utf8]{inputenc}
\usepackage{newtxtext,newtxmath}
\usepackage{float}
\usepackage{bm}
\usepackage{calc}
\usepackage{fancyvrb}
\usepackage{newfloat}
\usepackage{adjustbox}
\usepackage{setspace}
\usepackage{mathtools}
\usepackage{rotating} 
\usepackage{array}
\usepackage{booktabs}
\usepackage{threeparttable}
\usepackage{siunitx}
\usepackage{pdflscape}
\usepackage{afterpage}
\usepackage[most]{tcolorbox}
\tcbuselibrary{listings,breakable}
\usepackage{listings}
\usepackage{titlesec}
\usepackage{etoolbox,tocloft}
\usepackage{xr}
\usepackage{algorithm}
\usepackage{algpseudocode} % atau algorithmicx
\usepackage[style=apa, backend=biber]{biblatex}
\DeclareLanguageMapping{english}{english-apa}
\DeclareCaptionFont{tenpt}{\fontsize{10pt}{12pt}\selectfont}
\addbibresource{new_ref.bib}
\DefineBibliographyStrings{english}{
  andothers = {dkk\adddot} % hasil: dkk.
}

\usepackage{indentfirst}
\usepackage{setspace}
\usepackage{textcomp,geometry}
\usepackage{array,tabularx,makecell,pifont,iftex}

% Times New Roman (cross-engine)
\ifPDFTeX
  \usepackage{newtxtext} % Times-like untuk pdfLaTeX
  \newcommand{\tnrfamily}{\rmfamily}
\else
  \usepackage{fontspec}
  \newfontfamily\tnrfamily{Times New Roman}
\fi

\newcommand{\cmark}{\ding{51}}
\newcommand{\xmark}{\ding{55}}
\usepackage{multirow}
\usepackage{pdfpages}
\usepackage{upgreek}
\usepackage{chngcntr}
\usepackage[labelsep=space, skip=2pt]{caption}

\usepackage[bookmarks, pdftitle={Skripsi}, pdfauthor={Rizki Putra Prastio}, bookmarksnumbered=false,hypertexnames=false, colorlinks=true, unicode, linkcolor=black, citecolor=black, urlcolor=black]{hyperref}

\usetikzlibrary{arrows.meta,positioning,fit,calc,shapes.geometric,shadows.blur}
\usetikzlibrary{backgrounds}

\pgfplotsset{compat=1.18}
% Palet sederhana (opsional)
\definecolor{LHblue}{RGB}{36,122,253}
\definecolor{LHgreen}{RGB}{35,171,108}
\definecolor{LHyellow}{RGB}{253,185,73}
\definecolor{LHred}{RGB}{229,80,57}

% Declare a background layer and order layers
\pgfdeclarelayer{bg}
\pgfsetlayers{bg,main}
\pgfdeclarelayer{background}
\pgfsetlayers{background,main}
\DefineVerbatimEnvironment{promptblock}{Verbatim}{%
  frame=single,
  fontsize=\small
}

% Okabe–Ito colors
\definecolor{oiBlue}{RGB}{0,114,178}
\definecolor{oiVermilion}{RGB}{213,94,0}
\definecolor{oiGreen}{RGB}{0,158,115}
\definecolor{oiYellow}{RGB}{240,228,66}

% Column-friendly styles
\tikzset{
  block/.style={
    rounded corners=2pt, draw=black, very thick, fill=black!3,
    align=center, inner sep=3pt, minimum height=8mm, text width=0.42\columnwidth
  },
  small/.style={
    rounded corners=2pt, draw=black, thick, fill=black!2,
    align=center, inner sep=2pt, minimum height=6mm, text width=0.26\columnwidth
  },
  planner/.style={block,   draw=oiBlue,      fill=oiBlue!8},
  memory/.style={small,    draw=oiGreen,     fill=oiGreen!12},
  monitor/.style={small,   draw=oiYellow!70!black, fill=oiYellow!25},
  controller/.style={block, draw=oiVermilion, fill=oiVermilion!10},
  arrow/.style={-{Latex[scale=1.0]}, thick}
}

\DeclareFloatingEnvironment[
    fileext=lod,
    listname={Daftar Kode},
    name=Kode,
    placement=!htbp,
]{kode}

\setlist[enumerate]{labelsep=0.1em}  % adjust spacing here

\makeatletter
\patchcmd{\@makecaption}
  {\vskip\abovecaptionskip}
  {\vspace{6pt}} % or \vspace{0pt}, tune this as needed
  {}{}
\patchcmd{\@makecaption}
  {\vskip\belowcaptionskip}
  {} % remove below caption space
  {}{}
\makeatother

\lstset{
language=tcl,
basicstyle=\normalfont\sffamily,
numbers=left,
numberstyle=\normalfont,
frame=tb,
columns=fullflexible,
showstringspaces=false,
xleftmargin=1.5em
}

\newlength{\glsnamewidth}
\setlength{\glsnamewidth}{0.40\textwidth} % <-- atur ini
\setlength{\glsdescwidth}{0.60\textwidth} % <-- atur ini

\newglossarystyle{longrapi}{%
  \setglossarystyle{long}%
  \renewenvironment{theglossary}{%
    \begin{longtable}{%
      >{\raggedright\arraybackslash}p{\glsnamewidth}%
      >{\raggedright\arraybackslash}p{\glsdescwidth}}%
  }{\end{longtable}}%
  \renewcommand{\glossentry}[2]{%
    \glsentryitem{##1}%
    \textit{\glstarget{##1}{\glossentryname{##1}}}%
    \ifglshassymbol{##1}{\space \glossentrysymbol{##1}}{}%
    & \glossentrydesc{##1}\glspostdescription\space ##2\tabularnewline
  }%
}

\sisetup{
  detect-weight=true,
  detect-family=true,
  group-separator={,},
  group-minimum-digits=4
}

\definecolor{LightCyan}{rgb}{0.88,1,1}
\definecolor{Gray}{gray}{0.9}
\definecolor{Warnaku}{rgb}{0.7,0.45,0.32}
\definecolor{Warnaku2}{RGB}{50,135,100}
\definecolor{phaseLit}{RGB}{220,220,220}   % studi literatur
\definecolor{phaseDes}{RGB}{189,215,238}   % perancangan
\definecolor{phaseImpl}{RGB}{198,224,180}  % implementasi
\definecolor{phaseData}{RGB}{255,242,204}  % pembangkitan dataset
\definecolor{phaseEval}{RGB}{255,230,204}  % evaluasi & analisis
\definecolor{phaseWrite}{RGB}{221,217,238} % penulisan skripsi  
\captionsetup[table]{justification=raggedright, singlelinecheck=false, font=tenpt}

\captionsetup[lstlisting]{%  
  font=small,      % ukuran 10pt
  labelfont=normal,      % bold label "Listing"
  singlelinecheck=false, % allow left alignment
  justification=raggedright % align to the left
}

\captionsetup[figure]{justification=centering, font=tenpt, labelfont=tenpt}

\floatname{algorithm}{Algoritma}

% \captionsetup[algorithm]{font={\fontsize{10pt}{12pt}\selectfont}}
\newcommand{\algfontsize}{\fontsize{10pt}{12pt}\selectfont}

\makeatletter
\renewcommand{\fnum@algorithm}{\algfontsize\bfseries \fname@algorithm~\thealgorithm}
\makeatother

%============================================================================
\makeindex
\makeglossaries
\loadglsentries{frontmatter/glosarium_entries}
\geometry{a4paper,hmargin={4cm,3cm},vmargin=3cm}
% \include{hypen}
% Daftar pemenggalan suku kata dan istilah dalam LaTeX

% \include{hype.indonesia}

% Perintah untuk membuat perintah/variabel baru. 
\newcommand{\var}[2]{\newcommand{#1}{#2}}
% 
% Perintah untuk membuat perintah/variabel baru. Teks yang ditulis dalam 
% perintah ini akan diformat ulang menggunakan huruf kapital. 
\newcommand{\Var}[2]{\newcommand{#1}{\uppercase{#2}}}

% Gambar
\newcommand{\figref}[1]{Gambar~\ref{#1}}
% Persamaaan
\newcommand{\qref}[1]{Persamaan~\ref{#1}}

	\renewcommand{\eqref}[1]{Persamaan~\ref{#1}}
% Tabel
\newcommand{\tabref}[1]{Tabel~\ref{#1}}
%Lampiran


%============================================================================
\renewcommand{\baselinestretch}{1.5}
\renewcommand{\chaptername}{BAB}
\renewcommand{\listfigurename}{\begin{center}\bfseries\fontsize{12}{12}\selectfont DAFTAR GAMBAR\end{center}}
\renewcommand{\listtablename}{\begin{center}\bfseries\fontsize{12}{12}\selectfont DAFTAR TABEL\end{center}}
\renewcommand{\contentsname}{\begin{center}\bfseries\fontsize{12}{12}\selectfont DAFTAR ISI\end{center}}
\renewcommand{\bibname}{DAFTAR PUSTAKA}
\renewcommand{\figurename}{Gambar}
\renewcommand{\tablename}{Tabel}
\renewcommand{\lstlistingname}{Kode}

% ====== Spasi Daftar Isi (TOC) ======
% Jarak sebelum judul TOC dari margin atas
\setlength{\cftbeforetoctitleskip}{0pt}
% Jarak setelah judul TOC ke entri pertama
\setlength{\cftaftertoctitleskip}{10pt}

% Jarak antar entri BAB (chapter) di TOC
\setlength{\cftbeforechapskip}{2pt}
% Jarak antar entri Section di TOC
\setlength{\cftbeforesecskip}{1pt}
% (Opsional) Jarak antar entri Subsection di TOC
\setlength{\cftbeforesubsecskip}{0pt}

% ====== Spasi Daftar Gambar (LOF) ======
% Jarak sebelum judul "DAFTAR GAMBAR" dari margin atas
\setlength{\cftbeforeloftitleskip}{0pt}

% Jarak setelah judul "DAFTAR GAMBAR" ke entri pertama
\setlength{\cftafterloftitleskip}{10pt}

% ====== Spasi Daftar Tabel (LOT) ======
% Jarak sebelum judul "DAFTAR TABEL" dari margin atas
\setlength{\cftbeforelottitleskip}{0pt}

% Jarak setelah judul "DAFTAR TABEL" ke entri pertama
\setlength{\cftafterlottitleskip}{10pt}

\counterwithin{kode}{chapter}

\renewcommand{\thechapter}{\Roman{chapter}}
\renewcommand{\theequation}{\arabic{chapter}.\arabic{equation}} % Format penomoran persamaan. Angka arab bab+angka arab persamaan
\renewcommand{\thefigure}{\arabic{chapter}.\arabic{figure}}% Format penomoran gambar. Angka arab bab+angka arab persamaan
\renewcommand{\thetable}{\arabic{chapter}.\arabic{table}}% Format penomoran tabel. Angka arab bab+angka arab persamaan
\renewcommand\thesection{\arabic{chapter}.\arabic{section}}
\renewcommand{\thekode}{\arabic{chapter}.\arabic{kode}}
\titlespacing*{\chapter}{0pt}{-25pt}{12pt}
%\titleformat{\chapter}{\normalfont\bfseries\centering}
%    {\fontsize{12pt}{12pt}\chaptertitlename\ \thechapter}{12pt}{\Large}

\pagestyle{plain} 

\titleformat{\chapter}[display]
  {\normalfont\bfseries\centering}
  {BAB~\thechapter}{0pt}{\bfseries}
  {}

\setlength{\abovecaptionskip}{6pt}   
\setlength{\belowcaptionskip}{3pt}  

% Mengatur lebar spasi antara text dan objek float
\setlength{\intextsep}{10pt plus 2pt minus 2pt}
\setlength{\textfloatsep}{15pt plus 3pt minus 3pt}
\setlength{\floatsep}{12pt plus 2pt minus 2pt}

\renewcommand{\arraystretch}{1.5} % Setting spasi pada tabel
  

\makeatletter
\renewcommand{\cftchappresnum}{BAB~} % Menambahkan teks "BAB"x	
\renewcommand{\cftchapnumwidth}{4em}     % Pengaturan jarak
\renewcommand{\cftchapleader}{\cftdotfill{\cftdotsep}}  % Menambah titik-titik lurus nomor halaman

% Bagian ini mengatur section pada daftar isi agar rata kiri
\setlength{\cftsecindent}{0pt}
\setlength{\cftsubsecindent}{0pt} % indentasi subsection pada daftar isi
\setlength{\cftsubsubsecindent}{0pt} % indentasi subsection pada daftar isi
\setlength{\cftsecnumwidth}{2.0em}  % Mengatur jarak nomor dengan teks pada daftar isi

% Mengatur indentasi nomor gambar dan tabel
% \renewcommand{\cftfigpresnum}{\hfill}
% \renewcommand{\cfttabpresnum}{\hfill}
% \renewcommand{\cftfigpresnum}{\figurename~} % -> "Gambar 2.1"
% \renewcommand{\cfttabpresnum}{\tablename~}  % -> "Tabel 2.1"
% \renewcommand{\cftfigaftersnum}{\hspace{1em}}
% \renewcommand{\cfttabaftersnum}{\hspace{1em}}
% \setlength{\cftfigindent}{0pt}
% \setlength{\cfttabindent}{0pt}
% \setlength{\cftfignumwidth}{2 em}  % Adjust width as needed
% \setlength{\cfttabnumwidth}{2 em}  % Adjust width as needed
% \makeatother

% === SOLUSI DEFINITIF: Redefinisi \@chapter tanpa addvspace di LOF/LOT ===
\makeatletter
\def\@chapter[#1]#2{\ifnum \c@secnumdepth >\m@ne
	\if@mainmatter
	\refstepcounter{chapter}%
	\typeout{\@chapapp\space\thechapter.}%
	\addcontentsline{toc}{chapter}%
	{\protect\numberline{\thechapter}#1}%
	\else
	\addcontentsline{toc}{chapter}{#1}%
	\fi
	\else
	\addcontentsline{toc}{chapter}{#1}%
	\fi
	\chaptermark{#1}%
	% BARIS INI YANG DIHILANGKAN:
	%\addtocontents{lof}{\protect\addvspace{10\p@}}%
	%\addtocontents{lot}{\protect\addvspace{10\p@}}%
	\if@twocolumn
	\@topnewpage[\@makechapterhead{#2}]%
	\else
	\@makechapterhead{#2}%
	\@afterheading
	\fi}
\makeatother
% === AKHIR REDEFINISI ===

\renewcommand{\cftfigpresnum}{Gambar~~} % Menambahkan kata "Gambar"
\renewcommand{\cfttabpresnum}{Tabel~~}  % Menambahkan kata "Tabel"
\renewcommand{\cftfigaftersnum}{\hspace{1em}}
\renewcommand{\cfttabaftersnum}{\hspace{1em}}
\setlength{\cftfigindent}{0pt}
\setlength{\cfttabindent}{0pt}
\setlength{\cftfignumwidth}{5.8em}  % Diperbesar untuk mengakomodasi "Gambar X.X"
\setlength{\cfttabnumwidth}{4.5em}  % Diperbesar untuk mengakomodasi "Tabel X.X"
\makeatother
%==============================================================
\captionsetup[kode]{  
    labelfont=normalfont,
    textfont=normalfont,
    justification=raggedright,
    singlelinecheck=false,
    font=small,
    position=above
}

\titleformat{\section}
  {\normalfont\bfseries\fontsize{12}{12}\selectfont}
  {\thesection}
  {1em}
  {}

\titleformat{\subsection}
  {\normalfont\fontsize{12}{12}\selectfont}
  {\thesubsection}
  {1em}
  {}

\titleformat{\subsubsection}
  {\normalfont\fontsize{12}{12}\selectfont}
  {\thesubsubsection}
  {1em}
  {}
  
\titlespacing*{\section}
  {0pt}   % No left indent
  {6pt}   % Before
  {0pt}   % After

\titlespacing*{\subsection}
  {0pt}   % No left indent
  {6pt}   % Before
  {0pt}   % After

\titlespacing*{\subsubsection}
  {0pt}   % No left indent
  {6pt}   % Before
  {0pt}   % After

\setcounter{secnumdepth}{3}  % atau 4, tapi 3 sudah cukup untuk subsubsection
\setcounter{tocdepth}{2}     % kalau mau subsubsection muncul di daftar isi

\begin{document}
\frontmatter
\begin{titlepage}
\begin{center}
\fontsize{12}{12} \textbf{SKRIPSI}\\
\bigskip
\bigskip
\begin{singlespace}
	\fontsize{12}{12} \textbf{PEMBANGKITAN \textit{DATASET LONG-HORIZON VISION-LANGUAGE NAVIGATION} MENGGUNAKAN \textit{LARGE LANGUAGE MODELS} UNTUK INSTRUKSI \textit{CODE-SWITCHING} INDONESIA-INGGRIS}\\
\end{singlespace}
\bigskip
\bigskip
\bigskip
\bigskip
\bigskip
\bigskip
\bigskip
\bigskip
\bigskip
\bigskip
\bigskip
\begin{figure}[h]
    \centerline{\includegraphics[width=0.4\textwidth]{images/Lambang-Universitas-Airlangga-bg-transparan.png}}


	
\end{figure}
\bigskip
\bigskip
\bigskip
\bigskip
\bigskip
\bigskip
\bigskip
\bigskip
\begin{singlespace}
	\fontsize{12}{12} \textbf{ZINADINE ZIDAN ALSYAHANA}\\
	\fontsize{12}{12} \textbf{NIM 163221014}\\
\end{singlespace}
\bigskip
\bigskip
\bigskip
\bigskip
\bigskip
\bigskip
\bigskip
\bigskip
\bigskip
\bigskip
\begin{singlespace}
	\fontsize{12}{12} \textbf{PROGRAM SARJANA}\\
	\fontsize{12}{12} \textbf{TEKNIK ROBOTIKA DAN KECERDASAN BUATAN}\\
	\fontsize{12}{12} \textbf{DEPARTEMEN TEKNIK}\\
	\fontsize{12}{12} \textbf{FAKULTAS TEKNOLOGI MAJU DAN MULTIDISIPLIN}\\
	\fontsize{12}{12} \textbf{UNIVERSITAS AIRLANGGA}\\
	\fontsize{12}{12} \textbf{2026}
\end{singlespace}
\end{center}
\end{titlepage} 
\setcounter{page}{2}
\chapter{\fontsize{12}{12}\selectfont LEMBAR PENGESAHAN}
%\thispagestyle{empty}
\begin{center}
\begin{singlespace}
\fontsize{12}{12} \textbf{PEMBANGKITAN \textit{DATASET LONG-HORIZON VISION-LANGUAGE NAVIGATION} MENGGUNAKAN \textit{LARGE LANGUAGE MODELS} UNTUK INSTRUKSI \textit{CODE-SWITCHING} INDONESIA-INGGRIS}\\
\end{singlespace}
\bigskip
\bigskip
\begin{singlespace}
\fontsize{12}{12} \textbf{SKRIPSI}\\
\fontsize{12}{12} \textbf{sebagai Salah Satu Syarat untuk Memperoleh}\\
\fontsize{12}{12} \textbf{Gelar Sarjana Teknik Robotika dan Kecerdasan Buatan}\\
\fontsize{12}{12} \textbf{Fakultas Teknologi Maju Dan Multidisiplin}\\
\fontsize{12}{12} \textbf{Universitas Airlangga}\\
\end{singlespace}
\bigskip
\bigskip
\bigskip

\begin{center}
\begin{minipage}{0.8\linewidth}  % Sesuaikan lebarnya dengan mengganti angka di depan \linewidth
  \fontsize{12}{12}\selectfont
  %Tulis nama anda disini
  Nama \hspace{10.75em} : Zinadine Zidan Alsyahana \\[0.5ex]
  NIM \hspace{11.25em} : 163221014 \\[0.5ex]
  Tanggal Sidang Skripsi \hspace{4.2em}:
\end{minipage}
\end{center}
\bigskip
\bigskip

% \begin{center}
% \begin{minipage}{0.8\linewidth}  % Sesuaikan lebarnya dengan mengganti angka di depan \linewidth
%   \fontsize{12}{12}\selectfont
%   %Tulis nama anda disini
%   Nama \hspace{7.55em} : FAIQ VARIAN GAMAL HASAN \\[0.5ex]
%   NIM \hspace{8.05em} : 163221011 \\[0.5ex]
%   Tanggal Sidang Proposal\hspace{0.5em}: \hspace{1em}Desember 2025
% \end{minipage}
% \end{center}
% \bigskip
% \bigskip


Disetujui oleh:\\
\bigskip
%Tanda tangan pembimbing
\begin{minipage}[t]{0.4\linewidth}
  \centering
  Pembimbing I\\[3cm]
  \makebox[\linewidth]{\underline{Yutika Amelia Effendi, S.Kom., M.Kom., Ph.D}}\\
  NIK. 199404142018083201
\end{minipage}
\hfill
\begin{minipage}[t]{0.4\linewidth}
  \centering
  Pembimbing II\\[3cm]
  \makebox[\linewidth]{\underline{Ir. Asif Ali Zamzami, S.ST., M.Sc., Ph.D., IPM.}}\\
  NIK. 199207222022103101
\end{minipage}

\vspace{0.5cm}

% Tanda tangan KPS
\noindent
\begin{minipage}[t]{\linewidth}
  \centering
  \begin{singlespace}
  Mengetahui\\
  Koordinator Program Studi\\
  S1 Teknik Robotika dan Kecerdasan Buatan\\[3cm]
  \underline{Ir. Asif Ali Zamzami, S.ST., M.Sc., Ph.D., IPM.} \\
  NIK. 199207222022103101
  \end{singlespace}
\end{minipage}
\vspace{1cm}
\bigskip
\bigskip
\end{center}
% \chapter{\fontsize{12}{12}\selectfont PERNYATAAN ORISINALITAS SKRIPSI}

\noindent Saya, John Doe, 123456789, penulis skripsi yang berjudul.............................\\
.....................................................................................\\
......................................................................

\noindent menyatakan bahwa:

\begin{enumerate}[left=0pt, itemsep=0pt,label=\arabic*. ]
	\item Skripsi ini adalah asli dan benar-benar hasil karya sendiri, bukan hasil karya pihak lain dengan mengatasnamakan saya, bukan hasil tiruan atau jiplakan (\textit{plagiarism}) dari karya pihak lain, dan/atau bukan tulisan yang dibuat dengan bantuan kecerdasan buatan.
	\item Skripsi ini belum pernah diajukan untuk mendapatkan gelar akademik, baik di Universitas Airlangga, maupun di perguruan tinggi lainnya.
	\item Dalam Skripsi ini tidak terdapat karya atau pendapat yang telah ditulis atau dipublikasikan orang lain, kecuali secara tertulis dengan jelas dicantumkan sebagai acuan dengan disebutkan nama pengarang dan dicantumkan dalam daftar kepustakaan.
\end{enumerate}

\noindent Pernyataan ini saya buat dengan sebenar-benarnya, dan apabila dikemudian hari terdapat penyimpangan dan ketidakbenaran dalam pernyataan ini, maka saya bersedia menerima sanksi akademik berupa pencabutan gelar yang telah diperoleh karena karya tulis Skripsi ini, serta sanksi-sanksi lainnya sesuai dengan norma dan peraturan yang berlaku di Universitas Airlangga.
\vspace{0.5cm}
\begin{flushright}
    \begin{minipage}{0.4\textwidth}
        \centering
        Surabaya, 32 Desember 2045 \\
        \vspace{1cm}
        {\fontsize{9}{9}\selectfont \color{gray} Meterai Rp10.000}\\
        \vspace{1cm}
        John Doe \\
        NIM 123456789 \\
    \end{minipage}
\end{flushright}
\chapter{\fontsize{12}{12}\selectfont KATA PENGANTAR}

Segala puji bagi Allah SWT, Tuhan yang Maha Pengasih lagi Maha Penyayang, yang telah memberi penulis rahmat, kekuatan, dan inspirasi untuk menyelesaikan proposal skripsi ini.  Dengan rendah hati, penulis ingin mengucapkan terima kasih kepada semua orang yang telah membantu dan mendukung proses penyusunan proposal ini.  Berikut ini adalah pihak-pihak yang telah memberikan dukungan:
\begin{enumerate}[left=0pt, labelsep=0.33cm, itemsep=0pt, topsep=0pt, partopsep=0pt, parsep=0pt]
    \item Orang tua dan keluarga yang senantiasa memberikan doa, kasih sayang, serta dukungan moral sejak penulis lahir, sehingga penulis dapat mencapai tahap ini berkat semangat dan keyakinan mereka.
    \item Ibu Yutika Amelia Effendi, S.Kom., M.Kom., Ph.D. sebagai dosen pembimbing pertama dan Bapak Ir. Asif Ali Zamzami, S.ST., M.Sc., Ph.D., IPM. sebagai dosen pembimbing kedua sekaligus koordinator program studi, yang telah membimbing penulis dalam penyusunan proposal skripsi ini.
    % \item Bapak Muhammad Aldo Setiawan, S.Si., M.Sc (Eng)., selaku dosen wali penulis yang selalu memberikan bimbingan akademik serta motivasi kepada penulis.
    \item Prof. Hsien-I Lin dan Bapak Muhammad Ahsan Fatwaddin Shodiq, S.Pd., M.Sc. selaku pembimbing selama kegiatan \textit{Visiting Researcher} di Hucenrotia Lab, National Yang Ming Chiao Tung University, yang telah memberikan ilmu, pengalaman, dan motivasi kepada penulis.
    \item Teman-teman seperjuangan di Program Studi Teknik Robotika dan Kecerdasan Buatan, Astro24, serta KsKim Udean yang selalu memberikan dukungan, bantuan, semangat, dan pengalaman berharga selama perjalanan akademik penulis.
    \item Alfaizah Wasilah selaku wanita hebat yang hadir ke dalam hidup penulis, yang selalu memberikan semangat, dukungan, serta cinta yang tulus kepada penulis.
\end{enumerate}

Penulis sangat menyadari bahwa proposal skripsi ini masih jauh dari kesempurnaan. Oleh sebab itu, penulis mengharapkan saran dan kritik yang membangun agar lebih baik untuk kedepannya. Semoga proposal skripsi ini dapat memberikan banyak manfaat bagi siapapun yang membacanya.

~\\
\begin{flushright}
    \begin{minipage}{0.4\textwidth}
    \begin{flushright}
        Surabaya, 03 Desember 2025 \\
        Penulis \\
    \end{flushright}
    \end{minipage}
\end{flushright}
% \chapter{\fontsize{12}{12}\selectfont KETENTUAN PENGGUNAAN SKRIPSI}

\noindent Ketentuan hak cipta bagi skripsi yang tidak dipublikasikan, terdaftar, tersedia, serta terbuka untuk umum di Perpustakaan Universitas Airlangga, dimiliki penulis dengan mengikuti aturan HKI yang berlaku di Universitas Airlangga. Referensi kepustakaan diperkenankan dicatat, tetapi pengutipan atau peringkasan hanya dapat dilakukan dengan seizin penulis dan harus disitasi sesuai dengan kaidah ilmiah. Memperbanyak atau menerbitkan sebagian atau seluruh skripsi haruslah seizin Penulis\\

\noindent Sitasi Skripsi ini dapat ditulis sebagai berikut:\\
Nama belakang, Nama depan. (Tahun): Judul Skripsi. Skripsi. Surabaya: Universitas Airlangga.\\

\noindent \textit{Contoh:}\\
Sofiah, A. (2015). Desain dan Implementasi Piranti EMG Mutlikanal Berbasis IIR Filter dalam Penyadapan Sinyal Elektrik Otot Ekstrimitas. Skripsi. Universitas Airlangga.\\

\noindent Sofiah, A. (2015). \textit{Design and Implementation of IIR Filter-Based Multichannel EMG for Myoelectric Signal of Extremity Muscles Recording}. Undergraduate Thesis. Universitas Airlangga.
% \chapter{\fontsize{12}{12}\selectfont ABSTRAK}
\begin{center}
\begin{singlespace}
\textbf{DESAIN DAN REALISASI AGV BERBASI LIDAR DAN ODOMETRI}\\
\end{singlespace}
\begin{singlespace}
Oleh\\
\textbf{John Doe}\\
\textbf{NIM : 1223456789}\\
\textbf{Program Sarjana Teknik Robotika dan Kecerdasan Buatan}\\[0.3cm]
\end{singlespace}
\end{center}

\begin{spacing}{1.2}
\noindent Abstrak berisi poin-poin utama dari skripsi, yaitu latar belakang, tujuan penelitian, metode 
yang  digunakan,  hasil,  dan  kesimpulan.  Abstrak  harus  ditulis  dalam  bahasa  Indonesia  dan 
bahasa Inggris. Jumlah kata 150-250, tidak diperbolehkan memuat kutipan, dan maksimal 5 
kata kunci yang dapat menggambarkan pokok-pokok atau subjek penelitian di akhir abstrak..


\noindent Kata kunci : Kata kunci 1, kata kunci 2, kata kunci 3, kata kunci 4, kata kunci 5.
\end{spacing}

% \chapter{\fontsize{12}{12}\selectfont ABSTRACT}
\begin{center}
\begin{singlespace}
\textbf{WRITE THE TITLE HERE IN ENGLISH}\\
\end{singlespace}
\begin{singlespace}
By\\
\textbf{John Doe}\\
\textbf{Student ID Number : 1223456789}\\
\textbf{Undergraduate Program in Robotics and Artificial Intelligence Engineering}\\[0.3cm]
\end{singlespace}
\end{center}

\begin{spacing}{1.2}
\noindent Write your abstract here.


\noindent Keywords : keyword 1, keyword 2, keyword 3, keyword 4, keyword 5.
\end{spacing}


\begingroup
\singlespacing
%\phantomsection
% \addcontentsline {toc}{chapter}{HALAMAN PENGESAHAN}
% \markboth{HALAMAN PENGESAHAN}{\textbf{HALAMAN PENGESAHAN}}
%\clearpage
\phantomsection
\addcontentsline{toc}{chapter}{DAFTAR ISI} \tableofcontents
\markboth{DAFTAR ISI}{\textbf{DAFTAR ISI}}
\clearpage
\phantomsection
\addcontentsline{toc}{chapter}{DAFTAR GAMBAR} \listoffigures
\markboth{DAFTAR GAMBAR}{\textbf{DAFTAR GAMBAR}}
\clearpage
\phantomsection
\addcontentsline{toc}{chapter}{DAFTAR TABEL} \listoftables
\markboth{DAFTAR TABEL}{\textbf{DAFTAR TABEL}}
% \clearpage
% =========================
% Halaman GLOSARIUM (otomatis)
% =========================
% \clearpage
\phantomsection
\addcontentsline{toc}{chapter}{DAFTAR GLOSARIUM}
\markboth{DAFTAR GLOSARIUM}{\textbf{DAFTAR GLOSARIUM}}

% Jika ingin semua entri tercetak walau belum dipanggil dengan \gls{}, biarkan baris ini:
\glsaddall
\glsaddallunused % opsional: tampilkan semua entri walau belum dipanggil \gls{}

% Cetak glosarium istilah (type=main adalah default)
\printglossary[title={DAFTAR GLOSARIUM}, style=long]

% (Opsional) Cetak daftar singkatan dari \newacronym:
% \clearpage
% \phantomsection
% \addcontentsline{toc}{chapter}{DAFTAR SINGKATAN}
% \markboth{DAFTAR SINGKATAN}{\textbf{DAFTAR SINGKATAN}}
% \printglossary[type=\acronymtype,title={DAFTAR SINGKATAN}]

% \clearpage


%\chapter{DAFTAR SINGKATAN}

\begin{table}[th!]
\centering
\begin{tabular}{ll}
\textbf{DFT} & \textbf{D}ensity \textbf{F}unctional \textbf{T}heory\\[0.2cm]
\textbf{EBS} & \textbf{E}nergy \textbf{B}and \textbf{S}tructure \\[0.2cm]
\textbf{HF} & \textbf{H}artree \textbf{F}ock \\[0.2cm]
\textbf{KS} & \textbf{K}ohn \textbf{S}ham \\[0.2cm]
\textbf{MD} & \textbf{M}olecular \textbf{D}ynamic  \\[0.2cm]
\textbf{SP} & \textbf{S}ampling \textbf{P}oints \\[0.2cm]
\textbf{RLV}& \textbf{R}eciprocal \textbf{L}attice \textbf{V}ector \\[0.2cm]
\textbf{PBE}& \textbf{P}eriodic \textbf{B}oundary \textbf{C}ondition \\[0.2cm]
\textbf{FCC}& \textbf{F}ace \textbf{C}entered \textbf{C}ubic \\[0.2cm]
\textbf{PW} & \textbf{P}lane \textbf{W}ave \\[0.2cm]
\textbf{QE} & \textbf{Q}uantum \textbf{E}spresso \\[0.2cm]
\end{tabular}
\end{table}
% \chapter{DAFTAR SIMBOL}

\begin{table}[htbp]
\centering
\footnotesize % 10pt
\renewcommand{\arraystretch}{1.25}
\begin{tabular}{|c|p{10cm}|}
\hline
\textbf{Simbol} & \textbf{Definisi} \\
\hline
$\langle \cdot \rangle$ 
& Notasi tuple atau struktur terurut. \\
\hline
$\times$ 
& Produk kartesian antara dua himpunan. \\
\hline
$\to$ 
& Pemetaan fungsi dari domain ke kodomain. \\
\hline
$\Delta(\cdot)$
& Himpunan semua distribusi probabilitas atas suatu himpunan. \\
\hline
$\cdot$
& Placeholder untuk argumen fungsi atau distribusi. \\
\hline
$\sim$
& Relasi bahwa suatu variabel diambil dari distribusi tertentu. \\
\hline
$\mathbb{E}[\cdot]$
& Ekspektasi matematis terhadap suatu variabel acak. \\
\hline
$\mathbb{P}(\cdot)$
& Probabilitas suatu peristiwa. \\
\hline
$\sum$
& Operator penjumlahan. \\
\hline
$\prod$
& Operator perkalian produk atas indeks. \\
\hline
$\arg\max$
& Operator pencari argumen yang memaksimalkan suatu fungsi. \\
\hline
$\approx$
& Relasi aproksimasi antara dua nilai. \\
\hline
$|\cdot|$
& Kardinalitas atau panjang suatu urutan. \\
\hline
$o_{1:t}$, $o_{\le t}$
& Notasi sejarah observasi dari langkah awal hingga langkah ke-$t$. \\
\hline
$a_t^{*}$
& Aksi rujukan atau aksi benar pada langkah ke-$t$. \\
\hline
$p_t$
& Probabilitas bahwa aksi pada langkah ke-$t$ benar. \\
\hline
$f_\theta$
& Fungsi keputusan parametrik. \\
\hline
$:$ 
& Penanda bahwa suatu simbol adalah fungsi dari satu himpunan ke himpunan lain. \\
\hline
\end{tabular}
\end{table}
\endgroup

\mainmatter
% \chapter{PENDAHULUAN}
\label{Bab1}

\section{Latar Belakang}
\textit{Vision-Language Navigation} (VLN) mempelajari bagaimana agen \textit{embodied} mengeksekusi instruksi bahasa untuk menavigasi lingkungan 3D. Kebutuhan aplikasi dunia nyata semakin mengarah pada skenario navigasi \textit{long-horizon}, yaitu ketika tujuan global dicapai melalui rangkaian sub-tugas yang saling bergantung secara spasial maupun temporal \parencite{anderson2018r2r,KrantzVLNCE,Song2025}. Pada skenario ini, agen tidak cukup bersikap reaktif langkah demi langkah; agen perlu menjaga konsistensi rencana lintas langkah, menyusun keterampilan secara komposisional, serta tetap tangguh terhadap kegagalan lokal dan ambiguitas linguistik \parencite{anderson2018r2r,KrantzVLNCE,Song2025}. Dalam penelitian ini, \textit{long-horizon} didefinisikan sebagai tugas navigasi multi-target, yakni agen harus mencapai beberapa target berurutan yang membentuk ketergantungan antar-tahap; definisi ini dibedakan dari sekadar trajektori panjang menuju satu tujuan \parencite{gu2022vision,Zhang2024}. Konsekuensinya, episode dapat mencakup puluhan hingga ratusan langkah diskret dan melintasi beberapa ruangan berbeda, sehingga keputusan awal berpotensi memengaruhi keberhasilan tahap-tahap berikutnya \parencite{gu2022vision,Zhang2024}.

\textit{Dataset} fondasional seperti R2R dan RxR memformalkan VLN berbasis instruksi, namun umumnya bersifat \textit{short-horizon}: episode relatif pendek, struktur sub-tugas tidak dieksplisitkan, dan representasi rencana multi-tahap tidak tersedia secara langsung untuk kebutuhan audit serta analisis kegagalan \parencite{anderson2018r2r,ku2020rxr}. Sejalan dengan itu, penelitian mutakhir menegaskan bahwa kemajuan VLN semakin ditentukan oleh kemampuan mengelola ketergantungan keputusan jangka panjang, bukan hanya memetakan instruksi ke tindakan lokal \parencite{Song2025,KrantzVLNCE}. Dengan kata lain, kebutuhan \textit{long-horizon} menuntut pergeseran fokus dari akurasi langkah menuju konsistensi strategi sepanjang episode \parencite{Song2025,KrantzVLNCE}.

Gambar~\ref{fig:timeline-vln} merangkum pergeseran \textit{frontier benchmark} VLN menuju horizon yang makin panjang: dari tugas \textit{indoor} berbasis \textit{navigation graph} (misalnya R2R dan RxR) \parencite{anderson2018r2r,ku2020rxr}, menuju pengaturan yang lebih realistis seperti \textit{continuous control} (VLN-CE) \parencite{KrantzVLNCE}, hingga tugas multi-stage yang secara eksplisit menargetkan navigasi \textit{long-horizon} (misalnya LHPR-VLN) \parencite{Song2025}. Tren ini mengindikasikan bahwa keberhasilan pada \textit{benchmark} \textit{short-horizon} tidak otomatis tertransfer ke \textit{benchmark} \textit{long-horizon}, karena kegagalan kecil di awal episode dapat memicu \textit{error compounding} dan mengunci agen pada lintasan yang menyimpang \parencite{Song2025}. Seiring bertambahnya panjang rangkaian keputusan, kebutuhan akan \textit{memory} untuk mempertahankan konsistensi instruksi bertahap serta kebutuhan \textit{planning} (sering kali hierarkis) menjadi semakin menonjol \parencite{Song2025,KrantzVLNCE}.
\begin{figure}[H]
\centering
\includegraphics[width=1.0\linewidth]{images/fig1_timeline_vln_paperish_v2.2.png}
\caption{Timeline Eskalasi \textit{Horizon} Pada \textit{Benchmark Vision-Language Navigation}}
\label{fig:timeline-vln}
\end{figure}

Implikasi dari pergeseran tersebut adalah perlunya pembaruan rancangan \textit{dataset} untuk VLN \textit{long-horizon}. Jika episode panjang hanya disimpan sebagai pasangan ``instruksi--trajektori'' dan dinilai dengan keberhasilan global, maka sulit menjawab pertanyaan audit seperti: sub-tugas mana yang gagal, dan bagaimana kegagalan awal memengaruhi tahap berikutnya. Karena itu, lintasan sebaiknya dipresentasikan sebagai rangkaian \textit{waypoint} dan sub-tugas yang eksplisit \parencite{KrantzVLNCE,liu2025visualgrounding,liu2024embodiedai,wang-etal-2024-navigating}. 

Sebagai contoh, instruksi panjang dapat dipetakan ke segmen-segmen seperti ``keluar dari ruang tamu'', ``mencapai koridor'', dan ``berhenti di depan meja''; tiap segmen memiliki target dan rujukan instruksi yang jelas. Representasi terstruktur seperti ini memudahkan evaluasi komposisionalitas, diagnosis \textit{error compounding}, dan penilaian agen pada tingkat sub-tugas \parencite{KrantzVLNCE,liu2025visualgrounding,liu2024embodiedai}. Dengan demikian, tantangan utama tidak hanya mencapai tujuan akhir, tetapi menjaga konsistensi perencanaan global dan kontrol lokal serta keselarasan semantik instruksi--persepsi sepanjang episode \parencite{KrantzVLNCE,liu2025visualgrounding,liu2024embodiedai}.

\textit{Framework Long-Horizon Vision-Language Navigation} (LH-VLN) dan NavGen menyediakan fondasi untuk menyintesis episode panjang yang terstruktur dan dapat diaudit. Secara umum, NavGen memfaktorkan instruksi menjadi rencana lintasan berbutir halus dan menyintesis deskripsi tingkat tinggi dari rangkaian tindakan, sehingga keterkaitan antara representasi rencana dan teks menjadi lebih terlacak. Dalam konteks LH-VLN, \textit{dataset} dapat dirancang sebagai pasangan rencana--instruksi dengan jejak transformasi yang eksplisit, sehingga audit semantik dan spasial sepanjang episode dapat dilakukan lebih sistematis. Fondasi ini relevan bagi penelitian ini karena tujuan utama bukan hanya memperpanjang episode, tetapi juga membuat relasi rencana--instruksi dapat ditinjau ulang hingga tingkat sub-tugas \parencite{Song2025}.

Agar episode navigasi \textit{long-horizon} dapat direalisasikan secara konsisten, dibutuhkan lingkungan simulasi yang fotorealistik, efisien, serta memberi kendali operasional atas pemilihan \textit{scene}, konfigurasi sensor, dan pergerakan agen. Habitat menyediakan simulator yang efisien dan terintegrasi \parencite{savva2019habitat}, sementara \textit{dataset} \textit{Habitat-Matterport} 3D (HM3D) menawarkan ribuan rekonstruksi bangunan dengan keragaman \textit{layout} dan fidelitas visual \parencite{HM3D2021}. Kombinasi Habitat dan HM3D menjadi landasan yang memadai untuk membangkitkan episode LH-VLN yang terstruktur, berjangka panjang, dan dapat diinspeksi ulang \parencite{HM3D2021}. Dengan basis simulasi tersebut, proses sintesis episode dapat dikontrol (misal pemilihan \textit{scene} dan target) tanpa bergantung pada akuisisi data dunia nyata.

Kemajuan \textit{Large Language Models} (LLM) membuka peluang \textit{pipeline} generatif \textit{in-the-loop} yang mengaitkan perencanaan, pembangkitan instruksi, serta pemeriksaan kualitas dalam satu alur \parencite{10.1007/978-3-031-73397-0_3}. Literatur \textit{alignment} dan kurasi instruksi seperti \textit{Reinforcement Learning from Human Feedback} (RLHF) dan \textit{self-instruct} menunjukkan bahwa keluaran model dapat diarahkan menuju gaya dan struktur tertentu, yang relevan untuk menghasilkan instruksi bertahap yang konsisten dengan rencana \parencite{Ouyang2022,wang2023selfinstruct,OpenAI2023}. Pada saat yang sama, kemampuan \textit{instruction following} perlu dipantau agar kepatuhan terhadap batasan lingkungan dan rencana tetap terjaga \parencite{Kuchemann2025,Mienye2025LargeLanguageModels, zhang25llm, li24llmtextgen}. Oleh karena itu, penelitian ini memposisikan LLM \textit{in-the-loop} bukan hanya sebagai pembangkit teks, tetapi juga sebagai komponen yang dapat membantu pemeriksaan konsistensi terbatas (misalnya kesesuaian urutan sub-tugas dan keterkaitan rujukan spasial terhadap rencana), tanpa mengklaim validasi kebenaran yang bersifat menyeluruh \parencite{Zhang2024,OpenAI2023}.

Fokus spesifik penelitian ini adalah pembangkitan instruksi \textit{code-switching} Indonesia--Inggris yang terkontrol untuk navigasi \textit{long-horizon}. Dalam praktik komunikasi di Indonesia, \textit{code-switching} jamak muncul pada konteks pendidikan, media, dan ranah profesional; akibatnya, instruksi navigasi yang realistis dapat memuat pergantian bahasa pada berbagai granularitas dan pola \parencite{adilazuarda-2022-indorobusta,Handoyo2024}. Bagi VLN, fenomena ini penting karena label ruang (misal landmark), relasi spasial (misal \textit{next to}, \textit{across}, \textit{towards}), dan tujuan bertahap dapat diekspresikan dalam dua bahasa secara bergantian. Tanpa kontrol yang eksplisit, variasi pola \textit{code-switching} berpotensi menjadi faktor pengganggu analisis: kegagalan agen dapat berasal dari tantangan persepsi/penalaran ruang, atau semata dari perubahan identitas bahasa yang tidak terukur. Karena itu, penelitian ini menargetkan pengaturan rasio dan pola pergantian bahasa secara sistematis agar pengaruhnya terhadap keterpahaman instruksi dan keterlacakan rencana dapat dipelajari secara lebih terukur.

Sebagai ilustrasi instruksi yang relevan untuk VLN, berikut contoh singkat \textit{code-switching} Indonesia--Inggris yang memisahkan urutan tindakan (Indonesia) dan rujukan spasial/landmark (Inggris):
``Mulai dari ruang tamu, jalan lurus sampai kamu melihat \textit{stairs}. Setelah itu belok kanan dan menuju \textit{door} yang \textit{next to} rak buku, lalu berhenti di depan meja kecil di koridor.'' Contoh ini menunjukkan bagaimana landmark dan relasi spasial sering muncul dalam bahasa Inggris, sementara struktur langkah dan pengikat urutan dinyatakan dalam bahasa Indonesia.

Literatur \textit{Natural Language Processing} (NLP), \textit{Automatic Speech Recognition} (ASR), dan \textit{Text-to-Speech} (TTS) untuk Indonesia--Inggris juga menegaskan bahwa pengendalian identitas bahasa per token serta modul \textit{Language Identification} (LID) membantu pemodelan ujaran campuran, yang implikasinya langsung pada desain \textit{dataset} teks navigasi \parencite{Tazakka2024,Handoyo2024,DBLP:journals/access/HidayatullahQLA22}. Dengan demikian, instruksi \textit{code-switching} dalam penelitian ini tidak diperlakukan sebagai variasi kebahasaan semata, melainkan sebagai variabel yang dikontrol dan dianotasi untuk kepentingan eksperimen VLN yang replikabel \parencite{HidayatullahPLM2025}.

\begin{figure}[H]
\centering
\includegraphics[width=1\linewidth]{images/benchmark_cs_ideng/xy_benchmark_map_no_table_readable.pdf}
\caption{Peta \textit{Landscape Benchmark} \textit{Code-Switching} Indonesia--Inggris}
\label{fig:peta-codeswitch-ideng}
\end{figure}

Pada Gambar~\ref{fig:peta-codeswitch-ideng}, ruang benchmark diproyeksikan pada bidang $XY$: sumbu-$X$ (skala log) merepresentasikan \textit{navigation horizon proxy} (rata-rata langkah/\textit{hops}), sedangkan sumbu-$Y$ merepresentasikan intensitas \textit{code-switching} yang dikuantifikasi dengan \textit{Code-Mixing Index (CMI)}~\parencite{gamback-das-2016-comparing}. Titik-titik \textit{benchmarkembodied navigation} (misal R2R~\parencite{anderson2018r2r}, RxR~\parencite{ku2020rxr}, SOON/FAO~\parencite{Zhu_2021_CVPR}, CVDN~\parencite{thomason2020cvdn}, hingga LHPR-VLN~\parencite{Song2025}) terkonsentrasi pada $Y\approx 0$, menandakan bahwa instruksi navigasi pada benchmark utama umumnya monolingual atau tidak memodelkan fenomena \textit{code-switching} secara eksplisit. Pergerakan ke kanan (nilai $X$ makin besar) menunjukkan peningkatan kompleksitas \textit{horizon}, dengan \textit{benchmark long-horizon} (LHPR-VLN~\parencite{Song2025}) berada jauh di sisi kanan namun tetap pada CMI rendah.

Sebaliknya, korpus \textit{text-only} ID--EN yang memang \textit{code-mixed} (ditunjukkan pada inset) memiliki CMI lebih tinggi dan menjadi bukti adanya data \textit{code-switching} yang terukur untuk konteks Indonesia--Inggris~\parencite{barik-etal-2019-normalization,Yulianti2021}, selaras dengan literatur yang menekankan pentingnya pemodelan \textit{code-switching} untuk teknologi bahasa~\parencite{dogruoz-survey} dan isu \textit{robustnes} pada variasi \textit{code-switching} di Indonesia~\parencite{adilazuarda-2022-indorobusta}. Area arsiran ``\textit{Gap focus: long-horizon + code-mixing}'' menandai kekosongan \textit{benchmark}: belum ada \textit{benchmark} VLN yang sekaligus \textit{long-horizon} dan \textit{code-mixed} (ID--EN). Titik target (bintang) mengilustrasikan tujuan penelitian untuk membangkitkan dataset instruksi navigasi horizon panjang dengan fenomena code-switching ID--EN; anotasi ``LLM synthesis'' merepresentasikan strategi menjembatani gap tersebut melalui pembangkitan instruksi/variasi data berbasis LLM, selaras dengan praktik \textit{instruction tuning} dan \textit{self-instruction} pada LLM modern~\parencite{Ouyang2022,wang2023selfinstruct}.

\begin{table}[H]
\centering
\caption{Perbandingan \textit{Dataset} Penelitian Ini dengan Beberapa \textit{Benchmark/Dataset} VLN yang Umum Digunakan.}
\label{tab:gap-dataset}
\setlength{\tabcolsep}{4pt}
\renewcommand{\arraystretch}{1.15}

{\tnrfamily\fontsize{10}{12}\selectfont
\begin{tabularx}{\linewidth}{@{}l*{3}{>{\centering\arraybackslash}X}@{}}
\toprule
\textit{Dataset/Benchmark} & \textit{Long-Horizon} & ID-EN & Audit Sub-Tugas \\
\midrule
R2R \parencite{anderson2018r2r}            & \cellcolor{red!15}\xmark & \cellcolor{red!15}\xmark & \cellcolor{red!15}\xmark \\
RxR \parencite{ku2020rxr}       & \cellcolor{red!15}\xmark & \cellcolor{red!15}\xmark & \cellcolor{red!15}\xmark \\
Touchdown \parencite{chen2019touchdown}      & \cellcolor{yellow!15}$\sim$ & \cellcolor{red!15}\xmark & \cellcolor{red!15}\xmark \\
CVDN \parencite{thomason2020cvdn}           & \cellcolor{yellow!15}$\sim$ & \cellcolor{red!15}\xmark & \cellcolor{red!15}\xmark \\
VLN-CE \parencite{KrantzVLNCE}         & \cellcolor{yellow!15}$\sim$ & \cellcolor{red!15}\xmark & \cellcolor{red!15}\xmark \\
LHPR-VLN \parencite{Song2025}      & \cellcolor{green!15}\cmark & \cellcolor{red!15}\xmark & \cellcolor{green!15}\cmark \\
Penelitian Ini & \cellcolor{green!15}\cmark & \cellcolor{green!15}\cmark & \cellcolor{green!15}\cmark \\
\bottomrule
\end{tabularx}

\vspace{0.25em}
\begin{minipage}{\linewidth}
\footnotesize
Keterangan: \cmark\ = mendukung; \xmark\ = tidak mendukung; $\sim$ = mendekati aspek terkait secara parsial (misalnya episode lebih panjang karena kontrol kontinu atau dialog), namun tidak sepenuhnya memenuhi definisi \textit{long-horizon} multi-target dan/atau belum menyediakan representasi audit sub-tugas eksplisit sesuai definisi penelitian ini.
\end{minipage}
}% end font scope
\end{table}

Tabel~\ref{tab:gap-dataset} menyoroti kesenjangan pada \textit{dataset} instruksi fondasional dan \textit{benchmark} terkait: (i) keterbatasan cakrawala episode (khususnya untuk definisi multi-target), (ii) ketiadaan instruksi bilingual Indonesia--Inggris, dan (iii) belum adanya representasi yang secara eksplisit mendukung \textit{audit sub-tugas}. Walaupun \textit{benchmark} lain turut memperluas spektrum setting (misal Touchdown dan CVDN) \parencite{chen2019touchdown,thomason2020cvdn} serta memperketat realisme kontrol (misal VLN-CE) \parencite{KrantzVLNCE}, kombinasi tiga aspek tersebut tetap jarang tersedia secara bersamaan. Kondisi ini memotivasi rancangan \textit{dataset} yang mengikat rencana, instruksi, dan metadata secara eksplisit agar analisis menjadi lebih terukur pada tingkat sub-tugas \parencite{Song2025,Zhang2024}.

Ruang lingkup penelitian ini berfokus pada pembangkitan \textit{dataset} navigasi \textit{long-horizon} dengan instruksi \textit{code-switching} Indonesia--Inggris tanpa melakukan pelatihan agen berbasis \textit{reinforcement learning}. Walaupun penelitian \textit{embodied} seperti SayCan dan RT-2 menunjukkan integrasi penalaran bahasa dan tindakan pada robot dunia nyata \parencite{Ahn2022,Brohan2023}, kontribusi yang dituju di sini adalah: (i) skema kontrol bahasa untuk \textit{code-switching}; (ii) \textit{pipeline} generatif LLM \textit{in-the-loop} untuk membangkitkan instruksi bertahap yang konsisten dengan rencana; (iii) keterlacakan rencana--instruksi hingga tingkat sub-tugas; serta (iv) kurasi metadata/mekanisme evaluasi yang relevan untuk audit episode \textit{long-horizon} \parencite{Song2025,Zhang2024}.

Penelitian ini menegaskan urgensi pengembangan \textit{dataset} LH-VLN berbasis LLM \textit{in-the-loop} dengan pola \textit{code-switching} Indonesia--Inggris yang dikontrol secara eksplisit. Berbeda dengan korpus \textit{code-switching} yang memotret ujaran natural, penelitian ini merancang peralihan bahasa melalui \textit{prompting} dan kontrol keluaran (misal rasio dan peran bahasa Indonesia--Inggris), sehingga sifat \textit{code-switching} di dalam \textit{dataset} dapat dianalisis dan dimanipulasi secara sistematis \parencite{mondal-etal-2022-cocoa}. \textit{Dataset} tersebut dikembangkan dalam lingkungan simulasi Habitat dan HM3D \parencite{savva2019habitat,HM3D2021}, serta dilengkapi struktur representasi dan metadata untuk mendukung audit sub-tugas. Dengan demikian, penelitian ini diharapkan menjadi pijakan awal bagi ekosistem riset \textit{embodied} AI dan \textit{multilingual instruction following} di Indonesia.

\vspace{0.5em}

\section{Rumusan Masalah}
\label{sec:rumusan-masalah}

Rumusan masalah dalam penelitian ini dirumuskan sebagai berikut:
\begin{enumerate}[label=\alph*., left=0pt, labelsep=0.33cm, itemsep=0pt, topsep=0pt, partopsep=0pt, parsep=0pt]    
    \item Bagaimana merancang dan mengadaptasi \textit{pipeline} generatif bertahap untuk membangkitkan \textit{dataset} \textit{long-horizon vision-language navigation} dengan instruksi \textit{code-switching} Indonesia-Inggris, mulai dari perancangan tugas navigasi multi-target dengan LLM, eksekusi trajektori di simulator Habitat, pemecahan trajektori menjadi segmen yang diberi tag visual secara otomatis, hingga pembangkitan instruksi navigasi natural yang mengikuti urutan langkah tersebut?
    \item Bagaimana menyusun dan menerapkan kerangka evaluasi untuk menilai kualitas \textit{dataset} \textit{long-horizon vision-language navigation} dengan instruksi \textit{code-switching} Indonesia-Inggris yang dihasilkan, baik dari sisi kinerja navigasi agen maupun dari sisi pola \textit{code-switching} pada instruksi?
\end{enumerate}
\vspace{0.5em}

\section{Tujuan Penelitian}
\label{sec:tujuan-penelitian}

Sejalan dengan rumusan masalah di atas, penelitian ini memiliki tujuan sebagai berikut:
\begin{enumerate}[label=\alph*., left=0pt, labelsep=0.33cm, itemsep=0pt, topsep=0pt, partopsep=0pt, parsep=0pt]    
    \item Merancang dan mengadaptasi \textit{pipeline} generatif bertahap untuk membangkitkan \textit{dataset} \textit{long-horizon vision-language navigation} dengan instruksi \textit{code-switching} Indonesia-Inggris di lingkungan simulasi Habitat, yang mencakup tahap perancangan tugas navigasi multi-target berbasis LLM, eksekusi trajektori, pemecahan trajektori menjadi segmen bertag visual, serta pembangkitan instruksi navigasi natural.
    \item Menyusun dan menerapkan kerangka evaluasi untuk menilai kualitas \textit{dataset} yang dihasilkan dengan menggabungkan (1) metrik kinerja navigasi, serta (2) metrik objektif \textit{code-switching} yang merefleksikan distribusi dan intensitas \textit{code-switching} Indonesia-Inggris pada instruksi navigasi.
\end{enumerate}
\vspace{0.5em}

\section{Manfaat Penelitian}
Penelitian ini diharapkan memberikan nilai tambah dari sisi teoretis dan praktis sebagai berikut:
% \subsection*{1. Manfaat Teoretis}
\begin{enumerate}[label=\alph*., left=0pt, labelsep=0.33cm, itemsep=0pt, topsep=0pt, partopsep=0pt, parsep=0pt]
    \item Memberikan kontribusi terhadap pengembangan kajian \textit{Long-Horizon Vision-Language Navigation} (LH-VLN) melalui rancangan \textit{dataset} bilingual Indonesia-Inggris.
    \item Memperkaya literatur mengenai \textit{code-switching} dalam konteks instruksi navigasi.
    \item Menambah khazanah kajian pemanfaatan \textit{Large Language Models} \textit{in-the-loop} untuk pembangkitan \textit{dataset} bilingual Indonesia-Inggris.
    \item Memberikan kontribusi bagi penguatan landasan keilmuan di bidang robotika dan kecerdasan buatan, khususnya pada kajian \textit{embodied} AI yang menghubungkan persepsi visual, pemahaman bahasa alami, dan perencanaan gerak dalam kerangka navigasi \textit{long-horizon}.
% \end{enumerate}

% \subsection*{2. Manfaat Praktis}
% \begin{enumerate}[label=\alph*., left=0pt, labelsep=0.33cm, itemsep=0pt, topsep=0pt, partopsep=0pt, parsep=0pt]
    \item Menyediakan rancangan \textit{pipeline} generatif dan skema evaluasi yang dapat dijadikan acuan bagi peneliti lain dalam membangun \textit{dataset} navigasi bilingual atau \textit{code-switching} pada domain VLN maupun \textit{embodied AI} terkait.
    \item Menjadi langkah awal bagi pengembangan sistem navigasi berbasis bahasa alami dalam konteks bahasa Indonesia dengan memanfaatkan \textit{dataset} yang berkualitas serta membangun ekosistem penelitian \textit{embodied} AI dan \textit{multilingual instruction following} di Indonesia.
\end{enumerate}

\vspace{0.5em}

\section{Batasan Masalah}
Batasan berikut diterapkan agar ruang lingkup penelitian terdefinisi secara tegas dan operasional:
\begin{enumerate}[label=\alph*., left=0pt, labelsep=0.33cm, itemsep=0pt, topsep=0pt, partopsep=0pt, parsep=0pt]
    \item Penelitian ini berfokus pada pembangkitan \textit{dataset} dan \textit{pipeline} generatif LLM \textit{in-the-loop} untuk instruksi navigasi \textit{long-horizon} dan tidak mencakup pelatihan maupun evaluasi kinerja agen navigasi berbasis \textit{reinforcement learning} atau algoritma pengendali lainnya.
    \item Lingkungan yang digunakan dibatasi pada simulator Habitat dengan \textit{scene} yang diambil dari \textit{dataset} HM3D. Lingkungan dunia nyata dan simulator lain berada di luar cakupan penelitian.
    \item Bahasa yang digunakan pada instruksi navigasi dibatasi pada \textit{code-switching} Indonesia-Inggris. Bahasa lain atau variasi \textit{multilingual} di luar pasangan Indonesia-Inggris tidak dibahas.
    \item Modus interaksi dibatasi pada instruksi berbasis teks tertulis; penelitian tidak membahas pemetaan instruksi ke bentuk ujaran (ASR/TTS), multimodalitas tambahan (misalnya gestur), atau antarmuka manusia-robot secara langsung.
    \item \textit{Large Language Models} yang digunakan diperlakukan sebagai \textit{black box} pada tingkat arsitektur. Fokus penelitian terletak pada perancangan \textit{prompt}, templat, pembatasan leksikal, serta prosedur evaluasi, bukan pada pengembangan atau modifikasi internal model bahasa itu sendiri.
    \item Agen navigasi pada penelitian ini bergerak dengan aksi diskret di lingkungan \textit{indoor} berbasis \textit{dataset} HM3D. Aksi manipulasi objek (misalnya \textit{grasping}, \textit{releasing}) maupun model dinamika gerak kontinu tidak dibahas.
    \item Label semantik objek dan \textit{scene} diperoleh dari model pengenal objek pralatih \textit{Recognize Anything (RAM)}. Kualitas dan kelengkapan label mengikuti keterbatasan model tersebut dan tidak dianotasi ulang secara manual atau diverifikasi secara menyeluruh oleh anotator manusia.
    \item Pola \textit{code-switching} pada instruksi navigasi dikontrol melalui perancangan \textit{prompt} LLM (termasuk rasio penggunaan bahasa Indonesia dan Inggris serta peran masing-masing bahasa dalam struktur ujaran) dan tidak dimodelkan langsung dari korpus ujaran alami penutur dwibahasa.
\end{enumerate}

% \chapter{TINJAUAN PUSTAKA}

\section{\textit{Vision–Language Navigation}}

\vspace{0.5em}

\subsection{\textit{Vision--Language Navigation}}

\textit{Vision--Language Navigation} (VLN) adalah tugas memetakan instruksi bahasa alami $x$ menjadi rangkaian aksi agar agen \textit{embodied} mencapai tujuan yang diinginkan \parencite{anderson2018r2r,ku2020rxr}. Dalam VLN, agen berinteraksi secara berulang melalui siklus persepsi--aksi: pada langkah waktu $t$ agen menerima observasi egosentris $o_t$ lalu menghasilkan aksi $a_t$ yang mengubah \textit{pose} dan, akibatnya, distribusi observasi berikutnya. Dua pengaturan utama yang umum digunakan adalah: (i) navigasi diskret berbasis graf sudut pandang seperti R2R dan RxR, di mana agen berpindah antar-\textit{node} (\textit{viewpoint}) dan memilih orientasi diskret \parencite{anderson2018r2r,ku2020rxr}; serta (ii) pengaturan \textit{continuous control} seperti pada VLN-CE, dengan kontrol kecepatan sudut dan linear yang lebih realistis karena memasukkan dinamika, ketidakpastian sensor, dan biaya eksekusi \parencite{KrantzVLNCE}. Perbedaan struktur aksi ini penting bagi desain kebijakan karena menentukan bagaimana instruksi diproyeksikan menjadi rencana navigasi, bagaimana kesalahan terakumulasi, dan bagaimana strategi pemulihan dapat dilakukan pada tiap pengaturan.

\begin{figure}[H]
\centering
\includegraphics[width=0.55\textwidth,trim=0mm 10mm 0mm 10mm,clip]{images/simplex_belief_pomdp.pdf}
\caption{Representasi Simplex Dua-Dimensi untuk \textit{Belief State} pada POMDP dengan Tiga Status Laten.}
\label{fig:belief-simplex}
\end{figure}

Gambar~\ref{fig:belief-simplex} memberi intuisi ruang \textit{belief state} pada POMDP ketika terdapat tiga status laten. Keyakinan agen direpresentasikan sebagai titik pada simplex 2D (segitiga): setiap simpul menyatakan keyakinan penuh pada satu status laten, sedangkan titik di interior segitiga menyatakan kombinasi probabilistik atas ketiga status tersebut. Trajektori titik-titik ${b_t}$ menggambarkan perubahan \textit{belief} dari waktu ke waktu akibat pengaruh dinamika transisi dan bukti observasi yang terus terakumulasi.

Secara umum, VLN dapat dimodelkan sebagai \textit{Partially Observable Markov Decision Process} (POMDP),
yang diringkas oleh tuple pada~\eqref{eq:pomdp_model}:
\begin{equation}
\mathcal{M}
=
\big\langle
\mathcal{S}, \mathcal{A}, \mathcal{O},
T, Z, R, \gamma, b_1
\big\rangle .
\label{eq:pomdp_model}
\end{equation}
\noindent
dengan $\mathcal{S}$ himpunan status laten, $\mathcal{A}$ himpunan aksi, dan $\mathcal{O}$ himpunan observasi egosentris.
Notasi $\Delta(\mathcal{X})$ menyatakan himpunan semua distribusi probabilitas di atas $\mathcal{X}$
(\textit{probability simplex}).
Pada setiap langkah waktu $t$, status laten $s_t \in \mathcal{S}$ menghasilkan observasi $o_t \in \mathcal{O}$,
dan agen memilih aksi $a_t \in \mathcal{A}$.

Pada navigasi diskret R2R/RxR, interpretasi yang lazim adalah $s_t$ mencakup \textit{viewpoint} saat ini beserta orientasi diskret,
$a_t$ memilih perpindahan ke \textit{node} tetangga (serta perubahan orientasi), dan $o_t$ berupa citra/panorama RGB dari sudut pandang agen.
Pada VLN-CE dengan kontrol kontinu, $s_t$ dapat dipahami sebagai \textit{pose} kontinu (misal posisi dan heading) beserta dinamika yang relevan,
$a_t$ berupa pasangan kontrol kecepatan linear--sudut, dan $o_t$ mencakup sensor seperti RGB, kedalaman, dan/atau IMU \parencite{KrantzVLNCE}.
Model ini memiliki fungsi transisi $T$, model observasi $Z$, fungsi ganjaran $R$, faktor diskonto $\gamma \in (0,1]$,
serta \textit{belief} awal $b_1 \in \Delta(\mathcal{S})$.

Fungsi transisi, model observasi, dan ganjaran pada~\eqref{eq:pomdp_model} dinyatakan sebagai:
\begin{align}
T &: \mathcal{S} \times \mathcal{A} \to \Delta(\mathcal{S}),
&
T(s' \mid s, a)
&= \mathbb{P}\!\left(s_{t+1} = s' \mid s_t = s,\, a_t = a\right),
\label{eq:transition}
\\
Z &: \mathcal{S} \times \mathcal{A} \to \Delta(\mathcal{O}),
&
Z(o \mid s', a)
&= \mathbb{P}\!\left(o_{t+1} = o \mid s_{t+1} = s',\, a_t = a\right),
\label{eq:observation}
\\
R &: \mathcal{S} \times \mathcal{A} \to \mathbb{R},
&
r_t
&= R(s_t, a_t).
\label{eq:reward}
\end{align}
\noindent
Karena ketakteramatan parsial, agen tidak mengakses $s_t$ secara langsung.
Sebagai gantinya, agen mempertahankan \textit{belief state} $b_t \in \Delta(\mathcal{S})$,
yaitu distribusi posterior atas status laten pada waktu $t$ (dengan $a_{1:t-1}=(a_1,\dots,a_{t-1})$, misal $b_t(s) = \mathbb{P}(s_t=s \mid o_{1:t}, a_{1:t-1}, x)$).

Satu episode interaksi agen-lingkungan dapat direpresentasikan sebagai trajektori pasangan observasi-aksi $\tau$ pada~\eqref{eq:trajectory}:
\begin{equation}
\tau
=
\big(
(o_1, a_1),
(o_2, a_2),
\ldots,
(o_H, a_H)
\big),
\label{eq:trajectory}
\end{equation}
dengan panjang trajektori atau \textit{horizon} episode $H$ didefinisikan pada~\eqref{eq:horizon}:
\begin{equation}
H = |\tau|.
\label{eq:horizon}
\end{equation}

Agen memilih aksi menggunakan kebijakan yang bersyarat pada instruksi bahasa alami dan sejarah observasi. Pada langkah waktu $t$, pemilihan aksi mengikuti kebijakan stokastik pada~\eqref{eq:policy_def}:
\begin{equation}
a_t
\sim
\pi_{\theta}\left(
\cdot \middle| o_{\le t}, x
\right),
\label{eq:policy_def}
\end{equation}
dengan $\pi_{\theta}$ kebijakan berparameter $\theta$, $o_{\le t}\equiv o_{1:t}=(o_1,\dots,o_t)$ sejarah observasi hingga langkah $t$, dan $x$ instruksi bahasa alami. Dalam praktik, kebijakan umumnya dilatih sebagai distribusi stokastik (untuk memodelkan ketidakpastian dan memfasilitasi optimisasi), namun saat eksekusi dapat digunakan keputusan \textit{greedy} melalui pemilihan $\arg\max$ atas distribusi yang sama.

\begin{figure}[H]
\centering
\begin{subfigure}[t]{0.48\textwidth}
\centering
\includegraphics[width=\textwidth]{images/vln_viewpoint_graph_v4_complex.pdf}
\end{subfigure}
\hfill
\begin{subfigure}[t]{0.48\textwidth}
\centering
\includegraphics[width=\textwidth]{images/vln_viewpoint_graph_ring_v1.pdf}
\end{subfigure}
\caption{Representasi Graf Sudut Pandang pada Tugas Navigasi Diskret R2R/RxR.}
\label{fig:vln-graph}
\end{figure}

Gambar~\ref{fig:vln-graph} memvisualisasikan pengaturan navigasi diskret ala R2R/RxR sebagai graf sudut pandang: simpul (\textit{node}) merepresentasikan lokasi kamera/\textit{viewpoint}, sedangkan sisi (\textit{edge}) menyatakan keterjangkauan gerak antar-\textit{viewpoint}. Episode navigasi dapat dipahami sebagai pemilihan dari simpul awal $S$ menuju simpul tujuan $G$; akibatnya, kesalahan sering termanifestasi sebagai pemilihan \textit{node} atau jalur yang keliru pada struktur topologis tersebut, dan pemulihan dapat terjadi melalui eksplorasi ulang jalur jika konektivitas graf memungkinkan.

Objektif pembelajaran kebijakan adalah memaksimalkan ekspektasi ganjaran kumulatif terdiskonto sepanjang trajektori $\tau$ pada horizon $H$ dari~\eqref{eq:horizon}. Hal ini dirumuskan pada~\eqref{eq:policy_objective}:
\begin{equation}
\max_{\theta}
\mathbb{E}_{\tau \sim \pi_{\theta}}\left[
\sum_{t=1}^{H} \gamma^{t-1} r_t
\right],
\label{eq:policy_objective}
\end{equation}
dengan $r_t$ ganjaran pada langkah $t$ (misal $r_t = R(s_t, a_t)$ dari~\eqref{eq:reward}),
$\mathbb{E}_{\tau \sim \pi_{\theta}}[\cdot]$ ekspektasi atas trajektori $\tau$ yang diinduksi oleh kebijakan $\pi_{\theta}$ pada~\eqref{eq:policy_def},
dan $\gamma$ faktor diskonto seperti pada~\eqref{eq:pomdp_model}.

Di bawah ketakteramatan, penyelarasan antara bahasa dan persepsi menjadi krusial. Agen perlu melakukan \textit{visual grounding} antara token instruksi dengan entitas visual, relasi spasial, dan \textit{affordance} yang relevan bagi navigasi. Model modern memusatkan \textit{visual grounding} melalui penyelarasan lintas-modal berbasis perhatian atau adaptor yang memproyeksikan fitur penglihatan ke ruang semantik bahasa \parencite{liu2025visualgrounding,liu2024embodiedai}. Kualitas \textit{grounding} ini memengaruhi kemampuan agen mempertahankan konsistensi lintas langkah serta melakukan koreksi ketika bukti visual ambigu.

Peran simulator dan data sangat penting dalam VLN. Habitat menyediakan kerangka eksekusi efisien yang mensimulasikan sensor RGB, kedalaman, dan IMU dengan kontrol tingkat rendah yang dapat dikonfigurasi \parencite{savva2019habitat}. Adapun HM3D menghadirkan lingkungan \textit{indoor} yang beragam dengan fidelitas geometrik dan tekstur realistis \parencite{HM3D2021}. Ekosistem ini memungkinkan evaluasi dan pelatihan berskala besar secara praktis, sekaligus memudahkan studi sistematis tentang dampak sensor, dinamika, dan struktur ruang aksi.

\begin{figure}[H]
\centering
\includegraphics[width=0.5\textwidth]{images/vln_trajectory_state_space.pdf}
\caption{Ilustrasi \textit{Reference Trajectory} dan \textit{Agent Trajectory} dengan Akumulasi Kesalahan pada Bidang Posisi 2D.}
\label{fig:vln-trajectory-2d}
\end{figure}

Gambar~\ref{fig:vln-trajectory-2d} mengilustrasikan fenomena \textit{compounding error} secara geometrik pada bidang posisi 2D dengan membandingkan \textit{reference trajectory} (lintasan rujukan) dan \textit{agent trajectory} (lintasan agen) yang mengandung galat kecil per langkah. Deviasi lokal yang tampak kecil pada awal lintasan dapat terakumulasi sehingga posisi akhir menyimpang signifikan dari lintasan rujukan dan berpotensi keluar dari radius toleransi keberhasilan; konsekuensinya, stabilitas eksekusi lokal dan mekanisme koreksi menjadi semakin penting pada episode yang lebih panjang.

Walau demikian, dataset klasik umumnya berfokus pada navigasi \textit{short-horizon} dengan instruksi singkat dan sedikit struktur tindakan eksplisit. Instruksi sering cukup dengan petunjuk lokal pada beberapa langkah awal, tetapi kurang memaksa agen untuk menyusun, memelihara, dan merevisi rencana global \parencite{anderson2018r2r,ku2020rxr}. Akibatnya, analisis dan desain algoritma kerap memprioritaskan pemadanan frasa ke \textit{landmark} sesaat daripada \textit{planning} berlapis dan memori jangka panjang \parencite{Zhang2024}. Kesenjangan ini menuntut formulasi dan data yang mendorong dekomposisi tujuan, pelacakan progres, serta penyelarasan yang dapat diaudit antara rencana, observasi, dan bahasa.

Dalam kerangka pemetaan instruksi ke aksi, pada setiap langkah waktu $t$ didefinisikan fungsi keputusan terparametrisasi $f_{\theta}$ sebagaimana dinyatakan pada~\eqref{code:Equation1}. Fungsi ini memetakan instruksi $x$ dan sejarah observasi hingga langkah tersebut, $o_{1:t} \coloneqq (o_1, \dots, o_t)$, ke sebuah aksi $a_t \in \mathcal{A}$:
\begin{equation}
a_t
=
f_{\theta}\bigl(x, o_{1:t}\bigr)
=
\arg\max_{a \in \mathcal{A}}
\pi_{\theta}\bigl(a \mid o_{1:t}, x\bigr),
\label{code:Equation1}
\end{equation}
\eqref{code:Equation1} mendefinisikan keputusan deterministik (eksekusi \textit{greedy}) dari distribusi kebijakan $\pi_{\theta}$, sementara selama pelatihan dan analisis probabilistik kebijakan tetap dipandang sebagai distribusi atas aksi.

Selanjutnya, untuk tujuan analitis dan membangun intuisi tentang \textit{compounding error}, peluang keberhasilan episodik didefinisikan sebagai peluang bahwa seluruh rangkaian aksi yang dihasilkan dalam satu episode tepat sama dengan aksi rujukan (demonstrasi) $\{a_t^{*}\}_{t=1}^{H}$. Definisi \textit{success} ini tidak dimaksudkan sebagai definisi evaluasi standar VLN yang umumnya berbasis \textit{goal-reaching} (misal kedekatan ke tujuan), melainkan sebagai alat untuk memperlihatkan bagaimana kesalahan per langkah dapat berakumulasi pada horizon panjang. Dengan definisi tersebut, peluang \textit{success} dapat dituliskan pada~\eqref{code:Equation2}:
\begin{equation}
\mathbb{P}(\text{success})
=
\mathbb{P}\Biggl(\bigcap_{t=1}^{H} (a_t = a_t^{*})\Biggr)
\approx
\prod_{t=1}^{H} p_t,
\label{code:Equation2}
\end{equation}
di mana aproksimasi $\prod_{t=1}^{H} p_t$ digunakan untuk menangkap perilaku peluruhan terhadap horizon $H$ di bawah asumsi penyederhanaan, misalnya dengan menganggap kejadian benar per langkah ``hampir independen'' ketika dikondisikan pada sejarah rujukan (\textit{teacher forcing}) dan instruksi, sehingga perubahan distribusi observasi akibat kesalahan sebelumnya diabaikan demi analisis yang lebih transparan.

Probabilitas keberhasilan per langkah $p_t$ pada~\eqref{code:Equation2} didefinisikan pada~\eqref{code:Equation3} sebagai probabilitas bahwa aksi yang dipilih pada langkah ke-$t$ sama dengan aksi rujukan $a_t^{*}$, bersyarat pada instruksi dan sejarah observasi hingga langkah tersebut:
\begin{equation}
p_t
=
\mathbb{P}\left(a_t = a_t^{*}\mid o_{1:t}, x\right)
=
\pi_{\theta}\bigl(a_t^{*}\mid o_{1:t}, x\bigr),
\label{code:Equation3}
\end{equation}
Sehingga, \eqref{code:Equation3} mengukur seberapa besar probabilitas kebijakan $\pi_{\theta}$ menempatkan massa pada aksi rujukan $a_t^{*}$ ketika dikondisikan pada instruksi $x$ dan sejarah observasi $o_{1:t}$. Dengan demikian, hubungan antara fungsi keputusan \eqref{code:Equation1}, probabilitas per langkah \eqref{code:Equation3}, dan aproksimasi peluang episodik \eqref{code:Equation2} menjadi eksplisit sebagai perangkat untuk memahami akumulasi kesalahan pada episode panjang.

\begin{figure}[H]
\centering
\includegraphics[width=0.7\textwidth]{images/vln_compounding.pdf}
\caption{Grafik Pengaruh Panjang Horizon Terhadap Peluang Keberhasilan Episodik Untuk Berbagai Tingkat Akurasi Langkah.}
\label{fig:vln-compounding}
\end{figure}

Gambar~\ref{fig:vln-compounding} menunjukkan bahwa $\mathbb{P}(\text{success})$ menurun tajam ketika horizon $H$ bertambah untuk beberapa tingkat akurasi langkah $p_t$. Intuisi utamanya adalah bahwa keberhasilan episodik (dalam definisi analitis di atas) memerlukan keberhasilan beruntun di setiap langkah, sehingga akumulasi kesalahan menghasilkan peluruhan yang efektifnya bersifat eksponensial terhadap $H$. Implikasi praktisnya adalah kebutuhan strategi seperti \textit{waypointing}, supervisi sub-tujuan, dan \textit{re-planning} agar kesalahan awal tidak terpropagasi sepanjang episode.

Pada skenario \textit{short-horizon}, agen dapat bergantung pada petunjuk lokal di sekitar posisi saat ini. Sebaliknya, pada \textit{long-horizon} agen perlu mendekomposisi tujuan global menjadi sub-tugas berantai, menggabungkan \textit{planning} hierarkis tingkat peta dengan kontrol lokal, mempertahankan memori peta metrik atau topologis, serta merepresentasikan progres terhadap rencana agar konsistensi lintas langkah dan ruangan terjaga \parencite{Song2025}. Pada pengaturan graf diskret, deviasi sering muncul sebagai pemilihan \textit{node} yang keliru dan dapat dipulihkan melalui manuver kembali jika struktur konektivitas mendukung. Pada kontrol kontinu, kesalahan heading atau translasi kecil terakumulasi menjadi \textit{pose drift} yang menurunkan kualitas pengamatan, sehingga estimasi keadaan dan penutupan \textit{loop} berbasis sensor menjadi vital \parencite{KrantzVLNCE}. Ketidakpastian sensor juga meningkatkan kebutuhan kebijakan yang memadukan persepsi yang tangguh dengan kontrol stabil, termasuk pengaturan kepercayaan terhadap instruksi ketika bukti visual tidak konsisten.

\vspace{0.5em}

\subsection{\textit{Long-Horizon Vision–Language Navigation}}
\label{subsubsec:lhvln}

\textit{Long-Horizon Vision–Language Navigation} (LH-VLN) memperluas VLN ke skenario multi-tahap yang menuntut konsistensi keputusan lintas sub-tugas, daya ingat konteks, dan kemampuan \textit{re-planning} lokal. Rangka LH-VLN modern yang relevan dengan penelitian ini memanfaatkan platform NavGen untuk menghasilkan episode multi-tahap yang dapat diaudit, disertai metrik evaluasi per-sub-tugas \parencite{Song2025}. Konsep LH-VLN serta kaitannya dengan tahapan data, tolok ukur, dan rancangan model diilustrasikan pada Gambar~\ref{fig:vln-intro}.

\begin{figure}[H]
    \centering
    \includegraphics[width=0.9\textwidth]{images/1-intro.pdf}    
    \caption{Ilustrasi Konsep \textit{Long-Horizon Vision-Language Navigation} (LH-VLN) \parencite{Song2025}.}
    \label{fig:vln-intro}
\end{figure}

Gambar~\ref{fig:vln-intro} mengilustrasikan alur besar LH-VLN dari sisi perumusan tugas hingga evaluasi: instruksi panjang dipecah menjadi serangkaian sub-tujuan yang saling bergantung, lalu episode multi-tahap dihasilkan secara terstruktur (misalnya melalui NavGen) agar setiap tahap dapat dilacak dan diaudit. Diagram tersebut juga menekankan bahwa keberhasilan LH-VLN tidak hanya ditentukan oleh “sampai tujuan akhir”, tetapi juga oleh ketepatan penyelesaian tiap sub-tugas, kemampuan mempertahankan konteks lintas ruangan/segmen, serta kapasitas mengoreksi rencana ketika terjadi penyimpangan atau ambiguitas observasi.

Untuk menilai keberhasilan penyelesaian tugas kompleks yang terdiri dari beberapa tahap (\textit{multi-step}), digunakan lima metrik: \textit{Success Rate} (SR) pada~\eqref{eq:sr}, \textit{Success weighted by Path Length} (SPL) pada~\eqref{eq:spl}, \textit{Independent Success Rate} (ISR) pada~\eqref{eq:isr}, \textit{Conditional Success Rate} (CSR) pada~\eqref{eq:csr}, dan \textit{CSR weighted by Ground Truth} (CGT) pada~\eqref{eq:cgt} \parencite{Song2025}. Metrik SR dan SPL pada~\eqref{eq:sr}–\eqref{eq:spl} mengevaluasi keberhasilan tugas kompleks secara keseluruhan, sedangkan ISR, CSR, dan CGT pada~\eqref{eq:isr} hingga~\eqref{eq:cgt} memberikan gambaran lebih rinci tentang keberhasilan pada tingkat sub-tugas serta ketergantungan antar-tahap.
\begin{equation}
\label{eq:sr}
\mathrm{SR}
=
\frac{1}{M}\sum_{j=0}^{M} S_j,
\qquad 
S_j
=
\prod_{i=1}^{N} s_{j,i}
\end{equation}
\eqref{eq:sr} mendefinisikan \textit{success rate} (SR) sebagai rata-rata indikator keberhasilan tugas kompleks, di mana $S_j$ bernilai $1$ hanya jika seluruh sub-tugas pada tugas ke-$j$ berhasil ($s_{j,i}=1$ untuk semua $i$).
\begin{equation}
\label{eq:spl}
\mathrm{SPL}
=
\frac{1}{M}\sum_{j=0}^{M}
S_j \,
\frac{L_j^{*}}{\max\!\bigl(L_j^{*},\,\ell_j\bigr)}
\end{equation}
\eqref{eq:spl} mendefinisikan \textit{success weighted by path length} (SPL), yang menimbang keberhasilan $S_j$ dari~\eqref{eq:sr} dengan rasio efisiensi lintasan, yaitu perbandingan antara panjang lintasan rujukan terpendek $L_j^{*}$ dan panjang lintasan aktual yang ditempuh agen $\ell_j$.
\begin{equation}
\label{eq:isr}
\mathrm{ISR}
=
\frac{1}{M\,N}
\sum_{j=0}^{M}
\sum_{i=0}^{N}
s_{j,i}
\end{equation}
\eqref{eq:isr} mendefinisikan \textit{independent success rate} (ISR) sebagai rata-rata keberhasilan sub-tugas secara per-tahap, tanpa memperhitungkan ketergantungan eksplisit antar-tahap seperti yang dilakukan CSR pada~\eqref{eq:csr}.

Untuk mengakomodasi ketergantungan berurutan antar-tahap pada CSR di~\eqref{eq:csr} dan CGT di~\eqref{eq:cgt}, digunakan konvensi $s_{j,0}=0$ untuk semua $j$, sehingga tahap pertama tidak memperoleh bobot tambahan dari tahap sebelumnya.
\begin{equation}
\label{eq:csr}
\mathrm{CSR}
=
\frac{1}{M\,N^{2}}
\sum_{j=0}^{M}
\sum_{i=0}^{N}
s_{j,i}\Bigl(1+(N-1)\,s_{i-1}\Bigr)
\end{equation}
\eqref{eq:csr} mendefinisikan \textit{conditional success rate} (CSR) yang
mengukur keberhasilan sub-tugas dengan menimbang tahap yang sukses lebih tinggi
apabila tahap sebelumnya juga berhasil.  
Istilah $s_{j,i-1}$ pada~\eqref{eq:csr} memodelkan ketergantungan urutan,
sehingga tahap ke-$i$ memperoleh bobot $1$ bila tahap sebelumnya gagal
($s_{j,i-1}=0$) dan bobot $N$ bila tahap sebelumnya berhasil
($s_{j,i-1}=1$), setelah dinormalisasi dengan $N^{2}$.
\begin{equation}
\label{eq:cgt}
\mathrm{CGT}
=
\frac{1}{M\,N}
\sum_{j=0}^{M}
\sum_{i=0}^{N}
\frac{P_i}{P}\;
s_{j,i}\Bigl(1+(N-1)\,s_{j,(i-1)}\Bigr)
\end{equation}
\eqref{eq:cgt} mendefinisikan \textit{CSR weighted by ground truth} (CGT), yang memperluas CSR pada~\eqref{eq:csr} dengan menimbang setiap sub-tugas berdasarkan panjang lintasan rujukan \textit{ground truth} $P_i$. Faktor bobot $\frac{P_i}{P}$ pada~\eqref{eq:cgt} memastikan bahwa tahap dengan lintasan rujukan lebih panjang (dan umumnya lebih sulit) berkontribusi lebih besar terhadap nilai CGT total. Notasi yang digunakan dalam~\eqref{eq:sr}–\eqref{eq:cgt} diringkas pada Tabel~\ref{tab:notation}.
\begin{table}[H]
  \centering
  \caption{Ringkasan Notasi yang Digunakan Pada~\eqref{eq:sr} Hingga \eqref{eq:cgt}.}
  \label{tab:notation}

  % set font size 10pt for the table
  {\fontsize{10}{12}\selectfont
  \begin{tabular}{lp{0.8\linewidth}}
    \hline
    Notasi & Deskripsi \\
    \hline
    $M$
      & Jumlah tugas kompleks, digunakan sebagai faktor normalisasi 
        pada~\eqref{eq:sr}–\eqref{eq:cgt}. \\
    $N$
      & Jumlah sub-tugas per tugas kompleks, digunakan sebagai faktor
        normalisasi pada~\eqref{eq:isr}–\eqref{eq:cgt}. \\
    $S_j \in \{0,1\}$
      & Indikator sukses tugas kompleks ke-$j$, seperti didefinisikan
        pada~\eqref{eq:sr}, bernilai $1$ hanya bila semua $s_{j,i}$ pada
        tugas ke-$j$ bernilai $1$. \\
    $s_{j,i} \in \{0,1\}$
      & Indikator sukses sub-tugas ke-$i$ pada tugas ke-$j$, digunakan dalam
        perhitungan SR (~\eqref{eq:sr}), ISR (~\eqref{eq:isr}), CSR
        (~\eqref{eq:csr}), dan CGT (~\eqref{eq:cgt}). \\
    $L_j^{*}$
      & Panjang lintasan rujukan/terpendek untuk tugas kompleks ke-$j$,
        misalnya agregat lintasan rujukan lintas sub-tugas, digunakan pada
        SPL di~\eqref{eq:spl}. \\
    $\ell_j$
      & Panjang lintasan aktual yang ditempuh agen pada tugas ke-$j$, muncul
        pada SPL di~\eqref{eq:spl}. \\
    $s_{j,i-1}$
      & Memodelkan ketergantungan berurutan pada CSR di~\eqref{eq:csr} dan
        CGT di~\eqref{eq:cgt}, sehingga keberhasilan pada tahap lebih awal
        mempengaruhi bobot tahap berikutnya; dengan konvensi $s_{j,0}=0$
        seperti dijelaskan sebelum~\eqref{eq:csr}. \\
    $P_i$
      & Panjang lintasan \textit{ground-truth} untuk sub-tugas ke-$i$,
        digunakan sebagai bobot kesulitan relatif pada CGT 
        di~\eqref{eq:cgt}. \\
    $P = \sum_{i=1}^{N} P_i$
      & Panjang total lintasan rujukan sepanjang seluruh sub-tugas, sehingga
        faktor $\frac{P_i}{P}$ pada~\eqref{eq:cgt} membentuk distribusi bobot
        yang ter-normalisasi. \\
    \hline
  \end{tabular}
  }
\end{table}

Berbeda dari VLN klasik, LH-VLN menuntut perancangan kebijakan hierarkis yang secara eksplisit memisahkan \textit{high-level planner} dan \textit{low-level controller}. Skema arsitektur lengkapnya ditunjukkan pada Gambar~\ref{fig:lhvln-pipeline}. Perencana tingkat atas menyintesis rangkaian \textit{waypoint} dari instruksi panjang, mengikat token kunci pada entitas peta topologis, serta menetapkan kendala urutan dan dependensi; pengendali tingkat bawah kemudian melakukan pelacakan lokal yang tangguh terhadap \textit{waypoint} sembari memperbarui keyakinan pose dan indikator progres. Untuk mempertahankan konsistensi lintas ruangan dan belokan, memori internal dapat berupa peta diferensiabel, memori graf topologis, atau \textit{plan sketch} simbolik yang dapat diaudit. Ketika deteksi \textit{off-route} atau ambiguitas visual terjadi, modul \textit{self-monitoring} memicu \textit{re-planning} adaptif sehingga jalur yang sudah ditempuh dan sisa instruksi dipertimbangkan ulang tanpa mengorbankan ketepatan terminologi rujukan dalam bahasa \parencite{Song2025}. Contoh \textit{rollout} kualitatif beserta pemicu \textit{re-planning} ditampilkan pada Gambar~\ref{fig:lhvln-qual}. 

\begin{figure}[H]
\centering
\footnotesize
\begin{adjustbox}{max width=\columnwidth,center,trim=0mm 0mm 0mm 0mm,clip}
\begin{tikzpicture}[
  font=\small,
  >=Stealth,
  thick,
  node distance=6mm,
  every node/.style={transform shape},
  % --- Styles ---
  block/.style={
    draw,
    align=center,
    text width=0.23\columnwidth,
    minimum height=12mm,
    inner sep=3mm
  },
  input/.style={
    block,
    line width=0.7pt
  },
  core/.style={
    block,
    line width=1.0pt
  },
  aux/.style={
    block,
    line width=0.7pt
  },
  arrow/.style={->, thick},
  farrow/.style={->, thick, dashed},
  lab/.style={font=\scriptsize, fill=white, inner sep=1pt}
]

\pgfsetlayers{main,background}

% --- Matrix layout (2 rows, left->right) ---
\matrix (m) [matrix of nodes,
             nodes in empty cells,
             row sep=10mm,
             column sep=10mm,
             nodes={anchor=center}]
{
  % High-level row
  \node[input] (instr) {\textbf{Instruction}\\``Go past the sofa,\\then enter the kitchen\\and stop by the sink.''}; &
  \node[core]  (planner) {\textbf{High-Level Planner}\\Waypoints \& ordering}; &
  \node[aux]   (memory) {\textbf{Memory}\\Map/Sketch}; &
  \node[aux]   (monitor) {\textbf{Self-Monitoring}\\Off-route? Ambiguity?}; \\
  % Low-level row
  \node[aux]   (topo) {\textbf{Topo Map}\\Landmarks/Graph}; &
  \node[core]  (controller) {\textbf{Low-Level Controller}\\Local waypoint tracking}; &
  \node[aux]   (perception) {\textbf{Perception}\\RGB-D, poses}; &
  \node[aux]   (actuation) {\textbf{Actuation}\\Velocity cmds}; \\
};

% --- Main left->right flow (solid) ---
\draw[arrow] (instr) -- (planner);

\draw[arrow] (planner) -- node[lab, pos=0.5, yshift=2pt] {waypoints} (controller);

\draw[arrow] (controller) -- (perception);
\draw[arrow] (perception) -- (actuation);

% --- Additional main inputs (solid, still L->R dominant) ---
\draw[arrow] (topo) -- (controller);
\draw[arrow] (topo) to[out=20, in=200] (planner);

\draw[arrow] (memory) -- node[lab, pos=0.5, yshift=2pt] {} (planner);

\draw[arrow] (monitor) to[out=-150, in=-20] node[lab, pos=0.55, xshift=2pt] {alerts} (planner);

% --- Feedback / monitoring loops (dashed, curved) ---
\draw[farrow]
  (perception) to[out=110, in=-70]
  node[lab, pos=0.55, yshift=2pt] {monitoring signals}
  (monitor);

\draw[farrow]
  (monitor) to[out=-200, in=20, looseness=1.1]
  node[lab, pos=0.55, yshift=2pt] {re-plan if needed}
  (planner);

% --- Group boxes (no fill) ---
\begin{pgfonlayer}{background}
  \node[draw, line width=0.5pt, inner sep=5mm,
        fit=(planner)(memory)(monitor),
        label={[font=\scriptsize, xshift=-0mm]above:{High-level (global)}}] (hlbox) {};

  \node[draw, line width=0.5pt, inner sep=5mm,
        fit=(controller)(perception)(actuation),
        label={[font=\scriptsize, xshift=-0mm]below:{Low-level (local)}}] (llbox) {};
\end{pgfonlayer}

\end{tikzpicture}
\end{adjustbox}
\normalsize
\caption{Kebijakan Hierarkis LH-VLN}
\label{fig:lhvln-pipeline}
\end{figure}


Gambar~\ref{fig:lhvln-pipeline} menekankan pemisahan peran antara komponen global dan lokal: pada lapisan \textit{high-level} (global), instruksi bahasa diubah menjadi \textit{waypoint} serta urutannya, dengan bantuan memori (peta/sketsa) agar konteks yang sudah dilalui tetap tersimpan dan dapat dipakai ulang saat membuat keputusan berikutnya. Pada lapisan \textit{low-level} (lokal), pengendali melacak \textit{waypoint} menggunakan masukan persepsi (misal RGB-D dan estimasi pose) untuk menghasilkan aksi kontrol (perintah kecepatan). Panah umpan-balik putus-putus memperlihatkan bagaimana sinyal monitoring dari persepsi masuk ke modul \textit{self-monitoring}; ketika terdeteksi \textit{off-route} atau ambiguitas, modul ini mengirim peringatan yang memicu perencana tingkat atas melakukan \textit{re-planning} sehingga rencana global dapat disesuaikan dengan keadaan aktual.

\begin{figure}[H]
\centering
\begin{tikzpicture}[scale=1.0]
  % Nodes (rooms)
  \node[draw, circle, minimum size=7mm] (A) at (0,0) {\small Hall};
  \node[draw, circle, minimum size=7mm] (B) at (3,0.2) {\small Sofa};
  \node[draw, circle, minimum size=7mm] (C) at (6,0) {\small .};
  \node[draw, circle, minimum size=7mm] (D) at (6,2.2) {\small Kitchen};
  \node[draw, circle, minimum size=7mm] (E) at (8.5,2.2) {\small Sink};

  % Edges (graph)
  \draw[thick, black!40] (A) -- (B) -- (C) -- (D) -- (E);

  % Ground-truth path
  \draw[very thick, oiBlue, dashed] (A) -- (B) -- (C) -- (D) -- (E);

  % Agent path (deviate then re-plan)
  \draw[very thick, oiVermilion, dashed]
      (A) -- (B) -- +(1.4,-0.8) coordinate (off)
      (off) -- (C);
  % Re-plan star and continuation
  \node[star, star points=5, star point ratio=2.5, draw=oiYellow!70!black, fill=oiYellow, minimum size=8pt] at (C) {};
  \draw[very thick, oiVermilion, dashed] (C) -- (D) -- (E);

  % Waypoints
  \foreach \n in {B,C,D,E}{
    \node[draw=oiGreen, fill=oiGreen!15, rounded corners=1pt, inner sep=1pt] at ($( \n)+(0,0.6)$) {\tiny waypoint};
  }

  % Inset instruction box
  \node[align=left, draw=black!30, rounded corners=2pt, fill=black!2, inner sep=3pt]
    at (4.3,-1.2)
    {\scriptsize \textbf{Instruction:} go past the sofa,\\[-1pt]
     \scriptsize turn right into the kitchen, stop by the sink.};

  % Legend
  \begin{scope}[shift={(0,-2.2)}]
    \draw[very thick, oiBlue, dashed] (0,0) -- (0.9,0) node[right, black]{\small Ground-truth};
    \draw[very thick, oiVermilion, dashed] (4.0,0) -- (4.9,0) node[right, black]{\small Agent};
    \node[star, star points=5, star point ratio=2.5, draw=oiYellow!70!black, fill=oiYellow, minimum size=8pt] at (8.0,0) {};
    \node[right] at (8.5,0) {\small Re-plan trigger};
  \end{scope}
\end{tikzpicture}
\caption{\textit{Rollout} Kualitatif LH-VLN}
\label{fig:lhvln-qual}
\end{figure}

Gambar~\ref{fig:lhvln-qual} menggambarkan episode navigasi pada peta topologis sederhana (misal Hall $\rightarrow$ Sofa $\rightarrow$ Kitchen $\rightarrow$ Sink) beserta jalur \textit{ground-truth} (garis solid) dan jalur agen (garis putus-putus). \textit{Waypoint} ditandai pada simpul-simpul kunci untuk menunjukkan target lokal yang harus dicapai agen sejalan dengan instruksi bahasa. Ketika agen menyimpang dari rute yang semestinya (deviasi dari graf utama), simbol bintang menandai titik pemicu \textit{re-planning}; setelah pemicu ini, agen memperbaiki rute dan melanjutkan ke \textit{waypoint} berikutnya hingga mencapai tujuan akhir (misal Sink). Dengan visual seperti ini, sumber kegagalan dapat dianalisis lebih jelas: apakah kesalahan terjadi pada penentuan urutan \textit{waypoint}, pada eksekusi lokal (misal drift pose/looping), atau pada kebutuhan disambiguasi saat memasuki area yang mirip.

Dari sisi data dan pelatihan, episode LH-VLN yang dihasilkan idealnya menyertakan pelabelan sub-tujuan, \textit{progress meter}, serta pelacakan \textit{landmark} untuk mendorong penyelarasan yang dapat diaudit antara rencana, observasi, dan bahasa \parencite{Song2025}. Skema pembelajaran yang efektif menggabungkan kurikulum dari \textit{short} ke \textit{long-horizon}, imitasi di tingkat sub-tujuan, dan penguatan dengan bentuk berbasis progres agar agen belajar menukar kepercayaan antara petunjuk linguistik dan bukti visual ketika domain bergeser atau sensor tidak pasti. Evaluasi multi-metrik (SR/SPL ditambah ISR/CSR/CGT) kemudian memberi sinyal berbeda: apakah kegagalan terutama berasal dari penyusunan rencana global (rendahnya CSR/CGT), pelaksanaan lokal (SPL rendah meski ISR moderat), atau regresi/putaran yang tidak produktif. Praktiknya, laporan kinerja LH-VLN sebaiknya menyajikan \textit{breakdown} per tahap, analisis \textit{failure modes} (misal kesalahan disambiguasi rujukan, \textit{pose drift}, \textit{looping}), dan studi ablatif pada komponen perencanaan, memori, serta \textit{re-planning} agar kemajuan dapat ditelusuri secara sistematis.

\vspace{0.5em}

\section{Lingkungan dan \textit{Dataset} VLN}

\begin{figure}[H]
  \centering

  %==== Subfigure HM3D (atas) ====
  \begin{subfigure}{\textwidth}
    \centering
    % baris-baris subfigure yang sudah kamu punya:
    \begin{subfigure}[t]{0.3\linewidth}
      \centering
      \includegraphics[width=\linewidth,height=4.2cm,keepaspectratio]{images/00004-VqCaAuuoeWk.jpg}
    \end{subfigure}\hfill
    \begin{subfigure}[t]{0.3\linewidth}
      \centering
      \includegraphics[width=\linewidth,height=4.2cm,keepaspectratio]{images/00007-UQuchpekHRJ.png}
    \end{subfigure}\hfill
    \begin{subfigure}[t]{0.3\linewidth}
      \centering
      \includegraphics[width=\linewidth,height=4.2cm,keepaspectratio]{images/00029-4wCTuaUNWEd.png}
    \end{subfigure}

    \vspace{0.4em}

    \begin{subfigure}[t]{0.3\linewidth}
      \centering
      \includegraphics[width=\linewidth,height=4.2cm,keepaspectratio]{images/00033-oPj9qMxrDEa.png}
    \end{subfigure}\hfill
    \begin{subfigure}[t]{0.3\linewidth}
      \centering
      \includegraphics[width=\linewidth,height=4.2cm,keepaspectratio]{images/00036-41FNXLAZZgC.png}
    \end{subfigure}\hfill
    \begin{subfigure}[t]{0.3\linewidth}
      \centering
      \includegraphics[width=\linewidth,height=4.2cm,keepaspectratio]{images/00043-Jfyvj3xn2aJ.png}
    \end{subfigure}

    \caption{}
    \label{fig:hm3dexamples}
  \end{subfigure}

  \vspace{0.5em}

  %==== Subfigure Gibson (bawah) ====
  \begin{subfigure}{\textwidth}
    \centering
    \includegraphics[width=0.9\textwidth]{images/gibson_examples.png}
    \caption{}
    \label{fig:gibsonexamples}
  \end{subfigure}

  % (opsional) caption besar untuk keseluruhan figure
  \caption{Contoh Denah HM3D dan Gibson yang Digunakan dalam Simulasi Habitat. (a) Denah \textit{Top-Down} Adegan HM3D \parencite{HM3D2021}. (b) Contoh Peta \textit{Top-Down} Gibson \parencite{xia2018gibson}.}
  \label{fig:hm3d_gibson}
\end{figure}

Eksekusi kebijakan navigasi berbasis bahasa membutuhkan lingkungan sintetis berperforma tinggi yang bertindak sebagai simulator. Lingkungan ini menyediakan model sensor dan protokol evaluasi yang terukur sehingga \textit{loop} persepsi-aksi dapat direplikasi hingga tingkat siklus yang halus \parencite{savva2019habitat}. Habitat menempatkan diri sebagai ekosistem modular dengan mesin grafika efisien, dukungan sensor RGB, kedalaman, dan \textit{ego-motion}, serta antarmuka eksperimen yang memudahkan pengaturan percobaan. Peningkatan arsitektural, seperti koordinasi tugas, pemuatan adegan asinkron, dan orkestrasi episode, menghasilkan kapasitas tinggi untuk pelatihan dan evaluasi. Komponen tersebut kompatibel lintas korpus adegan, termasuk HM3D dan Gibson \parencite{HM3D2021, xia2018gibson, szot2021habitat2}. Lihat Gambar~\ref{fig:hm3dexamples} untuk contoh denah HM3D dan Gambar~\ref{fig:gibsonexamples} untuk Gibson.
\begin{figure}[htbp]
  \centering
  \begin{subfigure}{0.3\textwidth}
    \includegraphics[width=\linewidth]{images/rxr_teaser_frames/rxr_teaser_000.png}
  \end{subfigure}
  \begin{subfigure}{0.3\textwidth}
    \includegraphics[width=\linewidth]{images/rxr_teaser_frames/rxr_teaser_015.png}
  \end{subfigure}
  \begin{subfigure}{0.3\textwidth}
    \includegraphics[width=\linewidth]{images/rxr_teaser_frames/rxr_teaser_030.png}
  \end{subfigure}

  \begin{subfigure}{0.3\textwidth}
    \includegraphics[width=\linewidth]{images/rxr_teaser_frames/rxr_teaser_045.png}
  \end{subfigure}
  \begin{subfigure}{0.3\textwidth}
    \includegraphics[width=\linewidth]{images/rxr_teaser_frames/rxr_teaser_060.png}
  \end{subfigure}
  \begin{subfigure}{0.3\textwidth}
    \includegraphics[width=\linewidth]{images/rxr_teaser_frames/rxr_teaser_075.png}
  \end{subfigure}

  \begin{subfigure}{0.3\textwidth}
    \includegraphics[width=\linewidth]{images/rxr_teaser_frames/rxr_teaser_090.png}
  \end{subfigure}
  \begin{subfigure}{0.3\textwidth}
    \includegraphics[width=\linewidth]{images/rxr_teaser_frames/rxr_teaser_105.png}
  \end{subfigure}
  \begin{subfigure}{0.3\textwidth}
    \includegraphics[width=\linewidth]{images/rxr_teaser_frames/rxr_teaser_120.png}
  \end{subfigure}

  \caption{Rangkaian \textit{Frame} RxR yang Menampilkan Navigasi Agen dari Titik Awal Hingga Tujuan \parencite{ku2020rxr}.}
  \label{fig:rxrframes} 
\end{figure}

Persepsi dan perencanaan dibantu oleh representasi peta. Agen bekerja pada graf sudut pandang berbasis konfigurasi \textit{viewpoints} yang juga berfungsi sebagai \textit{scene graph} dan \textit{navigation graph}. Setiap simpul menunjukkan lokasi kamera, dan setiap sisi menunjukkan keterjangkauan gerak. Ruang pose dalam konfigurasi kontinu membentuk manifold, yang memerlukan perencanaan lintasan dan estimasi keadaan pada ruang gerak yang lebih kaya. Rancangan kebijakan dipengaruhi oleh perbedaan ini. Pada graf diskret, perencanaan topologis dan tekstur permukaan frasa ke \textit{landmark} cenderung dominan; namun, pada ruang kontinu, kebijakan harus memadukan perencanaan kinodinamik, tekstur permukaan beresolusi tinggi, dan strategi pemulihan deviasi untuk menjaga stabilitas kontrol. Habitat membantu orkestrasi episode dan kompatibel dengan HM3D dan Gibson \parencite{savva2019habitat, HM3D2021, xia2018gibson, szot2021habitat2, wijmans2020vlnce}. Ilustrasi keragaman tata letak yang berdampak pada graf navigasi dapat dilihat pada Gambar~\ref{fig:hm3dexamples} dan Gambar~\ref{fig:gibsonexamples}.
\begin{figure}[H]
  \centering
  % Baris 1
  \begin{subfigure}{0.3\textwidth}
    \includegraphics[width=\linewidth]{images/demo_frames/demo_000.png}
  \end{subfigure}
  \begin{subfigure}{0.3\textwidth}
    \includegraphics[width=\linewidth]{images/demo_frames/demo_006.png}
  \end{subfigure}
  \begin{subfigure}{0.3\textwidth}
    \includegraphics[width=\linewidth]{images/demo_frames/demo_012.png}
  \end{subfigure}

  % Baris 2
  \begin{subfigure}{0.3\textwidth}
    \includegraphics[width=\linewidth]{images/demo_frames/demo_018.png}
  \end{subfigure}
  \begin{subfigure}{0.3\textwidth}
    \includegraphics[width=\linewidth]{images/demo_frames/demo_024.png}
  \end{subfigure}
  \begin{subfigure}{0.3\textwidth}
    \includegraphics[width=\linewidth]{images/demo_frames/demo_030.png}
  \end{subfigure}

  % Baris 3
  \begin{subfigure}{0.3\textwidth}
    \includegraphics[width=\linewidth]{images/demo_frames/demo_036.png}
  \end{subfigure}
  \begin{subfigure}{0.3\textwidth}
    \includegraphics[width=\linewidth]{images/demo_frames/demo_042.png}
  \end{subfigure}
  \begin{subfigure}{0.3\textwidth}
    \includegraphics[width=\linewidth]{images/demo_frames/demo_045.png}
  \end{subfigure}

  \caption{Rangkaian \textit{Frame} R2R yang Menggambarkan Lintasan Rujukan pada Matterport3D \parencite{anderson2018r2r, chang2017matterport3d}.}
  \label{fig:r2rframes}
\end{figure}

R2R dirancang sebagai tolok ukur pada adegan Matterport3D dengan graf sudut pandang diskret dan menyediakan pasangan instruksi berbahasa Inggris serta lintasan target pada simpul kamera yang telah dikuantisasi \parencite{anderson2018r2r, chang2017matterport3d}. Instruksinya menekankan deskripsi rute tingkat lokal. Contoh episode RxR ditunjukkan pada Gambar~\ref{fig:rxrframes}, sedangkan contoh episode R2R ditunjukkan pada Gambar~\ref{fig:r2rframes}. Metrik umum meliputi \textit{Success Rate}, \textit{SPL}, dan \textit{nDTW} yang mengevaluasi ketepatan topologis serta efisiensi lintasan relatif terhadap rujukan. RxR memperluas ke pengaturan multibahasa, menambah variasi panjang dan gaya instruksi, serta memperkaya anotasi jejak rute dan wacana sehingga mendorong pemodelan pemahaman linguistik yang lebih dalam \parencite{ku2020rxr}. Keduanya efektif untuk studi \textit{short-horizon} karena fokus pada beberapa petunjuk visual berurutan di sekitar posisi saat ini. Namun, keterbatasan muncul pada skenario \textit{long-horizon} yang memerlukan perencanaan bertingkat, memori peta, dan pemulihan kesalahan berulang \parencite{Zhang2024}. Instruksi yang relatif ringkas, struktur tindakan yang tidak diekspos secara eksplisit, serta ketiadaan pemetaan satu-ke-satu antara tahapan rencana multi-langkah dan teks menyulitkan diagnosis kegagalan berantai.

Untuk mengukur kesesuaian lintasan prediksi terhadap lintasan rujukan diskret, digunakan kriteria kesamaan topologis antara kedua lintasan tersebut. Secara khusus, kerugian (\textit{loss}) didefinisikan sebagai bentuk komplementer dari \textit{normalized Dynamic Time Warping} (nDTW). Misalkan $\hat{\tau}$ menyatakan lintasan prediksi dan $\tau^{\star}$ lintasan rujukan, maka kerugian topologis didefinisikan pada~\eqref{eq:ndtw1} sebagai
\begin{equation}
\mathcal{L}_{\text{path}}(\hat{\tau}, \tau^{\star}) 
= 1 - \mathrm{nDTW}\!\left(\hat{\tau}, \tau^{\star}\right).
\label{eq:ndtw1}
\end{equation}
Lintasan prediksi $\hat{\tau}$ diasumsikan dihasilkan oleh suatu kebijakan parametrik $\pi_{\theta}$, sedangkan lintasan rujukan $\tau^{\star}$ merupakan lintasan \textit{ground truth}. Hubungan ini dinyatakan eksplisit pada~\eqref{eq:ndtw2}:
\begin{equation}
\hat{\tau} \sim \pi_{\theta}, 
\qquad 
\tau^{\star}\ \text{: lintasan \textit{ground truth}}.
\label{eq:ndtw2}
\end{equation}
dengan $\mathcal{L}_{\text{path}}(\hat{\tau}, \tau^{\star})$ merepresentasikan kerugian yang meminimalkan ketidakselarasan topologis antara lintasan prediksi $\hat{\tau}$ dan lintasan rujukan $\tau^{\star}$ sebagaimana didefinisikan pada~\eqref{eq:ndtw1}. Selain itu, $\mathrm{nDTW}(\hat{\tau}, \tau^{\star})$ merupakan ukuran kesamaan berbasis \textit{normalized dynamic time warping} pada graf sudut pandang antara kedua lintasan tersebut, juga didefinisikan pada~\eqref{eq:ndtw1}. Kebijakan parametrik $\pi_{\theta}$ dengan parameter $\theta$ menginduksi distribusi trajektori sebagaimana dinyatakan pada~\eqref{eq:ndtw2}. Lintasan $\hat{\tau}$ adalah lintasan yang dihasilkan selama eksekusi kebijakan $\pi_{\theta}$, sedangkan $\tau^{\star}$ adalah lintasan rujukan (\textit{ground truth}) yang digunakan sebagai acuan evaluasi, keduanya sebagaimana dijelaskan pada~\eqref{eq:ndtw2}.

Secara praktis, objektif pelatihan tidak hanya bergantung pada satu realisasi trajektori, melainkan dituliskan sebagai ekspektasi atas distribusi trajektori yang diinduksi oleh kebijakan $\pi_{\theta}$. Dengan memanfaatkan definisi kerugian pada~\eqref{eq:ndtw1} dan hubungan trajektori–kebijakan pada~\eqref{eq:ndtw2}, objektif pelatihan dapat dirumuskan seperti pada~\eqref{eq:ndtw_obj} berikut:
\begin{equation}
\min_{\theta}\ \mathbb{E}_{\hat{\tau}\sim \pi_{\theta}}
\Big[\,1 - \mathrm{nDTW}\!\left(\hat{\tau}, \tau^{\star}\right)\,\Big]
\quad \Longleftrightarrow \quad
\max_{\theta}\ \mathbb{E}_{\hat{\tau}\sim \pi_{\theta}}
\Big[\,\mathrm{nDTW}\!\left(\hat{\tau}, \tau^{\star}\right)\,\Big].
\label{eq:ndtw_obj}
\end{equation}

\noindent Pada~\eqref{eq:ndtw_obj}, operator $\mathbb{E}_{\hat{\tau}\sim \pi_{\theta}}[\cdot]$ menyatakan harapan terhadap distribusi trajektori $\hat{\tau}$ yang diinduksi oleh kebijakan $\pi_{\theta}$. Jika nilai kesamaan $\mathrm{nDTW}(\hat{\tau}, \tau^{\star})$ dibatasi pada rentang $[0,1]$, maka dari definisi pada~\eqref{eq:ndtw1} diperoleh bahwa $\mathcal{L}_{\text{path}}(\hat{\tau}, \tau^{\star}) \in [0,1]$ dan nilai minimum $\mathcal{L}_{\text{path}}(\hat{\tau}, \tau^{\star}) = 0$ tercapai ketika lintasan prediksi $\hat{\tau}$ identik secara topologis dengan lintasan rujukan $\tau^{\star}$.

Kerangka LH-VLN yang diilustrasikan pada Gambar~\ref{fig:vln-intro} memfaktorkan tujuan global menjadi \textit{waypoint} dan sub-tugas yang dapat diperiksa kembali, lalu menautkan rencana ke teks melalui pasangan representasi rencana-instruksi yang eksplisit \parencite{Song2025}. Alur generatif dimulai dari perincian rencana granular pada peta topologis maupun metrik, tahap berikutnya menyusun instruksi tingkat tinggi yang mempertahankan ketercakupan \textit{landmark} dan relasi spasial, kemudian dilakukan validasi kepatuhan semantik terhadap adegan aktual \parencite{Song2025,radford2021clip}. LLM \textit{in-the-loop} untuk mengatur gaya, format, dan konsistensi naratif lintas sub-tugas, termasuk pengendalian intensitas deskripsi lokal relatif terhadap arahan global \parencite{OpenAI2023,Lu2023PlanSolve,Yao2023ReAct,Zhang2024}. Dengan demikian, analisis kegagalan berantai dapat mengidentifikasi sub-tugas yang terhenti serta bukti visual yang ambigu \parencite{Song2025,Liu2023GEval}.

% Pengendalian gaya linguistik lintas bahasa diakomodasi melalui pembatasan per token, misalnya \textit{per-token language mask} untuk mengatur rasio frasa bahasa Indonesia terhadap bahasa Inggris, sehingga pola \textit{code-switch} dapat direkayasa sesuai skenario pengguna. Dari sisi komputasi, pemetaan dari tujuan menuju rencana, dari rencana menuju instruksi, dan dari instruksi menuju aksi dirumuskan terlebih dahulu dalam narasi ini, lalu dinyatakan secara formal sebagai~\eqref{eq:lhvln1}--\eqref{eq:lhvln3} berikut.
% \begin{equation}
% \phi:\ \text{Goal} \mapsto W = (w_1,\ldots,w_K),
% \label{eq:lhvln1}
% \end{equation}
% \begin{equation}
% g_{\psi}: (W, \text{scene}) \mapsto x,
% \label{eq:lhvln2}
% \end{equation}
% \begin{equation}
% f_{\theta}: (x, o_{1:t}) \mapsto a_t,
% \label{eq:lhvln3}
% \end{equation}
% \noindent di mana:
% \begin{itemize}
% \item $\phi$ adalah fungsi perencanaan tingkat tinggi yang memetakan tujuan global menjadi urutan \textit{waypoint} yang layak dieksekusi,
% \item $W = (w_1,\ldots,w_K)$ adalah urutan \textit{waypoint} dengan panjang $K$ sebagai target perantara sepanjang lintasan,
% \item $g_{\psi}$ adalah modul generatif yang menyintesis instruksi berbasis konteks adegan,
% \item $x$ adalah representasi instruksi yang menjaga ketercakupan \textit{landmark} dan relasi spasial untuk seluruh $W$,
% \item $f_{\theta}$ adalah kebijakan eksekusi yang memetakan instruksi dan observasi sekuensial $o_{1:t}$ menjadi aksi tingkat rendah $a_t$,
% \item $a_t$ adalah perintah translasi atau rotasi pada waktu $t$ yang tangguh terhadap ketidakpastian sensor.
% \end{itemize}

\vspace{0.5em}

\section{Model Visi Fondasi: \textit{Recognize Anything Model} (RAM)}
\label{subsec:ram_tag2text}

\vspace{0.5em}

\subsection{Pengantar Model Visi Fondasi}

Perkembangan \textit{foundation models} pada ranah bahasa telah diikuti oleh kemunculan \textit{foundation vision models} (\textit{FVM}), yakni model visi berskala besar yang dilatih pada korpora citra atau pasangan citra--teks dalam skala jutaan hingga miliaran contoh dan dirancang untuk dapat diadaptasi secara \textit{zero-shot} atau \textit{few-shot} ke berbagai tugas hilir. Contoh menonjol di ranah visi mencakup Segment Anything Model (SAM) untuk segmentasi \textit{promptable} \parencite{kirillov2023segment} dan ViT-22B sebagai \textit{Vision Transformer} berskala 22 miliar parameter \parencite{pmlr-v202-dehghani23a}, yang menunjukkan bahwa skala parameter dan data juga membawa manfaat signifikan bagi representasi visual generik.

\begin{figure}[H]
    \centering
    \includegraphics[width=0.85\linewidth]{images/visualisasi_2_zero_few_shot_transfer_paperish_v3.pdf}
    \caption{Kurva Kinerja Hilir Terhadap Jumlah Contoh Berlabel Pada Skenario \textit{Zero-Shot} Dan \textit{Few-Shot}}
    \label{fig:zero_few_shot_transfer}
\end{figure}

Gambar~\ref{fig:zero_few_shot_transfer} memperlihatkan tren peningkatan kinerja tugas hilir seiring bertambahnya jumlah demonstrasi berlabel, dengan ketiga metode menunjukkan pola \textit{diminishing returns} pada jumlah \textit{shots} yang lebih besar; model berbasis \textit{foundation vision model} untuk \textit{image tagging} (Tag2Text dan RAM++) memberikan kinerja \textit{zero-shot} yang lebih tinggi dan tetap mempertahankan keunggulan pada \textit{few-shot} dibandingkan baseline, sementara pita ketidakpastian (95\% CI) menggambarkan variasi antar percobaan dan cenderung menyempit ketika jumlah \textit{shots} meningkat.

Dalam konteks \textit{embodied AI} dan Vision--Language Navigation (VLN) long-horizon, \textit{foundation vision model} memainkan peran sebagai \textit{perceptual backbone} yang menyediakan representasi semantik tinggi atas observasi visual agen. Daripada merancang \textit{feature extractor} yang sempit untuk satu tugas navigasi, pendekatan modern cenderung memanfaatkan FVM yang telah dilatih sebelumnya dan kemudian mengekstraksi informasi objek, tata ruang, serta atribut lingkungan dari keluaran model tersebut. Hal ini sangat relevan untuk pipeline berbasis \textit{large language model} (LLM), karena LLM bekerja lebih efektif ketika menerima representasi visual yang sudah diproyeksikan ke ruang simbolik atau linguistik (misalnya, daftar tag semantik atau deskripsi tekstual) alih-alih langsung dari piksel mentah.

Subbab ini memfokuskan pada dua model visi fondasi yang berorientasi pada \textit{image tagging} dan pemetaan citra ke struktur semantik, yakni \textit{Recognize Anything Model} (RAM / RAM++) \parencite{zhang2023recognize,huang2023openset} dan Tag2Text \parencite{huang2023tag2text}. Keduanya menyediakan antarmuka yang sangat cocok untuk menghubungkan observasi visual dalam lingkungan 3D dengan pemodelan bahasa berbasis LLM di VLN \textit{long-horizon}.

\vspace{0.5em}

\subsection{\textit{Recognize Anything Model} (RAM / RAM++)}

\textit{Recognize Anything Model} (RAM) diperkenalkan sebagai \textit{image tagging model} berskala besar yang dirancang untuk \textit{open-set image tagging} \parencite{zhang2023recognize}. Tidak seperti klasifikasi multi-label tradisional yang terbatas pada himpunan label tertutup, RAM memanfaatkan skema pelabelan universal dan mekanisme kueri tekstual sehingga mampu mengenali ribuan kategori umum dan melakukan generalisasi ke kategori baru. Secara konseptual, untuk sebuah citra $x$, RAM menghasilkan himpunan tag semantik sebagaimana ditunjukkan pada~\eqref{eq:ram_tagset}:
\begin{equation}
\hat{p}_{\ell}(x) = \sigma\!\big(z_{\ell}(x)\big),
\qquad
\mathcal{L}(x) = \left\{\ell \in \mathcal{V} \;\middle|\; \hat{p}_{\ell}(x) \ge \tau \right\},
\label{eq:ram_tagset}
\end{equation}
di mana $z_{\ell}(x)$ adalah skor/logit untuk label $\ell$ yang diprediksi dari citra $x$, $\sigma(\cdot)$ adalah fungsi sigmoid, $\hat{p}_{\ell}(x)$ adalah skor kepercayaan (confidence) per-label, $\mathcal{V}$ adalah himpunan label/kueri teks yang digunakan saat inferensi, dan $\tau$ adalah ambang untuk menentukan tag mana yang dipertahankan dalam $\mathcal{L}(x)$.

\begin{figure}[H]
  \centering
  % Sesuaikan path & ekstensi file dengan nama file kamu
  \includegraphics[width=0.8\linewidth]{images/ram_semantic_tag_space_theoretical_en_shortylabel_spacedL.png}
  \caption{Ruang Semantik Tag RAM Sebagai Aproksimasi \textit{Embedding} Label $\mathcal{V}$.}
  \label{fig:ram_semantic_tag_space}
\end{figure}

Gambar~\ref{fig:ram_semantic_tag_space} memvisualisasikan intuisi bahwa keluaran RAM berupa himpunan tag $\mathcal{L}(x)$ dapat dipandang sebagai subset dari ruang label besar $\mathcal{V}$ yang “dipilih” berdasarkan ambang $\tau$, yakni \eqref{eq:ram_tagset}. Sumbu-$x$ menunjukkan tingkat \textit{concreteness} (semakin ke kanan semakin konkret/berwujud), sedangkan sumbu-$y$ merepresentasikan spektrum tipe konsep dari \textit{entity}-level (misal objek) menuju konteks tingkat tinggi (misal \textit{scene} atau atribut kontekstual). Titik-titik dengan bentuk berbeda melambangkan kategori label yang berbeda (objek, \textit{scene}, atribut), dan lingkaran putus-putus menggambarkan “wilayah” tag yang dipertahankan RAM untuk sebuah citra tertentu. Contoh tag seperti \textit{door}, \textit{sofa}, \textit{stairs} (lebih konkret) serta \textit{kitchen}, \textit{corridor}, \textit{bright}, \textit{cluttered} (lebih kontekstual) menunjukkan bahwa RAM tidak hanya mengekstrak objek, tetapi juga merangkum konteks dan atribut yang relevan dalam satu representasi tag semantik yang eksplisit.

RAM dilatih menggunakan kombinasi data \textit{image--text} skala besar dan tag yang diperoleh secara otomatis melalui \textit{text semantic parsing}. Arsitekturnya mengintegrasikan modul \textit{image encoder}, \textit{recognition decoder} untuk multi-label tagging, serta \textit{text generation decoder} untuk captioning \parencite{zhang2023recognize}. Dibandingkan model \textit{vision--language} sebelumnya yang menyelaraskan citra dan teks secara global, RAM melakukan penyelarasan yang lebih halus antara fitur region dan tag, sehingga mampu memprediksi sejumlah besar tag yang relevan dengan ketepatan tinggi. Hasilnya, RAM dapat secara \textit{zero-shot} mengenali lebih dari 6{,}000 kategori umum dan melampaui model multi-label terawasi pada berbagai \textit{benchmark} tagging \parencite{zhang2023recognize}.

RAM++ kemudian diusulkan sebagai pengembangan RAM dengan fokus pada \textit{open-set image tagging} yang lebih kuat \parencite{huang2023openset}. RAM++ mengintegrasikan \textit{multi-grained text supervision} dengan triplet citra--tag--teks, sehingga tidak hanya memanfaatkan supervisi global berupa deskripsi kalimat, tetapi juga supervisi per-tag secara eksplisit. Selain itu, RAM++ memanfaatkan LLM untuk memperkaya deskripsi tag menjadi frasa yang lebih kaya semantik, memperluas cakupan konsep visual yang dapat dicakup dalam pengaturan \textit{open-set}. Secara empiris, RAM++ menunjukkan peningkatan signifikan atas RAM dan model lain pada berbagai skenario, termasuk tag kategori umum, kategori langka, serta frasa interaksi manusia--objek \parencite{huang2023openset}.

Dalam konteks VLN, keunggulan utama RAM/RAM++ terletak pada kemampuannya menghasilkan himpunan tag semantik yang eksplisit untuk setiap observasi visual, yang dapat diformalkan melalui~\eqref{eq:ram_tagset}. Tag-tag ini mencakup informasi objek (misal, \textit{sofa}, \textit{table}, \textit{stairs}), tipe ruangan (misal, \textit{kitchen}, \textit{corridor}), serta atribut konteks (misal, \textit{bright}, \textit{cluttered}). Representasi semacam ini menyediakan \textit{symbolic interface} antara piksel dan bahasa, sehingga memudahkan proses \textit{grounding} antara instruksi linguistik dan entitas visual. Selain itu, representasi ini juga berfungsi untuk menyaring informasi yang relevan bagi navigasi, misalnya hanya menyimpan tag yang berkaitan dengan landmark atau tujuan lokal, serta memasok LLM dengan daftar entitas yang dapat dirujuk secara eksplisit dalam instruksi.

Gambar~\ref{fig:ram_triplet_space} mengilustrasikan mekanisme RAM++ yang mempelajari representasi gabungan citra--tag--teks melalui penyelarasan (\textit{alignment}) berbasis triplet. Fitur citra $x_i$ berada pada \textit{image feature space}, sementara embedding tag $t_j$ berada pada \textit{tag space}; keduanya didorong agar selaras melalui vektor \textit{image--tag alignment} sehingga prediksi tag menjadi lebih \textit{grounded} pada bukti visual. Selain itu, teks (misal \textit{short caption} atau deskripsi yang diperkaya) menyediakan supervisi linguistik yang dipetakan ke representasi $y_k$, sehingga RAM++ tidak hanya menyamakan citra dengan tag secara lokal, tetapi juga mengaitkannya dengan makna bahasa pada level kalimat/frasa. Dengan supervisi multi-butir (per-tag dan global), model belajar sebuah ruang representasi bersama yang konsisten: citra, tag, dan teks saling mengunci secara semantik, yang pada akhirnya memperkuat \textit{open-set image tagging} dan memudahkan integrasi tag eksplisit ke modul penalaran berbasis LLM.

\begin{figure}[H]
    \centering
    \includegraphics[
      width=0.5\linewidth,
      trim=0mm 6mm 0mm 6mm,
      clip
      ]{images/ram_triplet_cube.png}
    \caption{Ilustrasi Ruang Gabungan Citra--Tag--Teks pada RAM++ dan Mekanisme Penyelarasan Image--Tag--Text Triplet.}
    \label{fig:ram_triplet_space}
\end{figure}

Dengan demikian, RAM dan RAM++ dapat dipandang sebagai modul persepsi generik yang menyiapkan fakta-fakta visual eksplisit sebelum LLM menyusun instruksi navigasi natural. Mekanisme penyelarasan antara citra, tag, dan representasi tekstual dalam RAM++ dapat dilihat pada Gambar~\ref{fig:ram_triplet_space}, di mana model mempelajari hubungan eksplisit antara $x_i$, $t_j$, dan $y_k$ melalui skema \textit{image--tag--text triplet}.

\vspace{0.5em}

\section{\textit{Code-Switching} Indonesia-Inggris}

\vspace{0.5em}

\subsection{Tipologi dan Teori Linguistik}

\textit{Code-switching} adalah pemakaian dua bahasa atau lebih dalam satu wacana yang koheren, yang dapat mencakup satu percakapan, satu instruksi navigasi, bahkan satu kalimat tunggal. Literatur klasik biasanya membedakan tiga tipe utama \textit{code-switching}, yaitu \textit{inter-sentential}, \textit{intra-sentential}, dan \textit{tag-switching}, masing-masing dengan fungsi komunikatif yang berbeda dalam interaksi sosial maupun instruksi tugas \parencite{winata2023decades}.

\textit{Inter-sentential} adalah perpindahan bahasa yang terjadi tepat di batas kalimat, misalnya “Belok kanan di ujung koridor. \textit{Then continue straight to the lobby}”. Peralihan seperti ini sering dipakai untuk memberi tanda bahwa penutur sedang menggeser konteks kecil dalam pembicaraan: dari penjelasan yang bersifat umum (dalam bahasa Indonesia) ke bagian yang lebih teknis atau lebih spesifik (dalam bahasa Inggris), atau untuk menyesuaikan pilihan bahasa dengan audiens yang lebih nyaman menggunakan bahasa tertentu \parencite{winata2023decades}.

\textit{Intra-sentential} terjadi ketika perpindahan bahasa muncul di dalam satu kalimat atau klausa, contohnya “Jalan lurus sampai \textit{glass door}, lalu belok kiri ke \textit{main desk}”. Pada pola ini, bahasa Indonesia biasanya menjadi kerangka tata bahasa utama, sedangkan kata atau frasa tertentu, terutama nama objek fisik dan \textit{landmark}, tetap dalam bahasa Inggris agar rujukannya lebih tepat dan tidak membingungkan. Secara praktis, hal ini juga memudahkan penutur karena tidak perlu mencari padanan lokal untuk istilah teknis yang sudah umum dipakai dalam bahasa Inggris.

Sementara itu, \textit{tag-switching} adalah penyisipan \textit{tag}, partikel diskursus, atau penanda sikap yang relatif lepas dari struktur kalimat, misalnya “Naik tangga ke lantai dua, \textit{okay}?”. Unsur seperti ini biasanya dipakai untuk mengecek pemahaman, melunakkan nada perintah, atau membangun kedekatan interpersonal, tanpa mengubah inti informasi yang disampaikan \parencite{winata2023decades}. Tabel~\ref{tab:cs-ringkas} menyajikan contoh tambahan untuk \textit{inter-sentential}, \textit{intra-sentential}, dan \textit{tag-switching}, sedangkan Gambar~\ref{fig:cs-mindmap} merangkum ketiganya dalam bentuk peta konsep.

\begin{table}[htbp]
\centering
\caption{Contoh \textit{Code-Switch}: \textit{Inter-Sentential}, \textit{Intra-Sentential}, dan \textit{Tag-Switching}.}
\label{tab:cs-ringkas}
\begingroup
\fontsize{10pt}{12pt}\selectfont
\setlength{\tabcolsep}{8pt}
\renewcommand{\arraystretch}{1.15}
\begin{tabular}{@{}l p{0.8\textwidth}@{}}
\toprule
\textbf{Target} & \textbf{Contoh} \\
\midrule

\multirow{5}{*}{\textit{Inter-sentential}} 
& ``Buka aplikasinya dulu. \textit{Then sign in with your company account}.'' \\
& ``Kita mulai jam dua. \textit{After that, we'll move to the Q\&A}.'' \\
& ``Silakan isi formulir. \textit{If anything is unclear, contact us}.'' \\
& ``Gunakan pintu samping. \textit{Security will guide you from there}.'' \\
& ``Dokumen ringkas ada di email. \textit{Full details are in the appendix}.'' \\
\midrule

\multirow{4}{*}{\textit{Intra-sentential}} 
& ``Setelah \textit{login}, buka \textit{settings} lalu aktifkan \textit{two-factor authentication}.'' \\
& ``Ambil \textit{elevator} ke lantai tiga, lalu belok kanan ke \textit{records office}.'' \\
& ``Pastikan \textit{dataset}-nya sudah di-\textit{clean} sebelum \textit{deploy}.'' \\
& ``Kalau \textit{deadline}-nya maju, kita \textit{rescope deliverable} utama.'' \\
\midrule

\multirow{4}{*}{\textit{Tag-switching}} 
& ``Sudah cek lampiran, \textit{okay}?'' \\
& ``Itu masih fleksibel, \textit{right}?'' \\
& ``Tolong percepat \textit{handover}-nya, \textit{please}.'' \\
& ``Kalau ragu, tanya di kanal \textit{support}, \textit{deal}?'' \\

\bottomrule
\end{tabular}
\endgroup
\end{table}

Teori batas perpindahan menyoroti bahwa tidak semua titik di dalam struktur kalimat sama-sama aman untuk dicampur. Secara umum terdapat larangan kuat pada batas morfem terikat, misalnya afiks bahasa Indonesia yang langsung menempel pada bentuk dasar bahasa Inggris, serta kehati-hatian pada batas frasa yang membawa fungsi gramatikal penting. Dalam kerangka \textit{matrix language} dan \textit{embedded language}, satu bahasa berperan sebagai \textit{matrix} yang memasok kerangka morfosintaks global yang mencakup urutan kata dasar, infleksi, penandaan aspek, dan elemen fungsi gramatikal lain. Bahasa lain hadir sebagai unsur tersisip (\textit{embedded}) yang biasanya berupa konten leksikal seperti nomina konkret, nama tempat, dan label objek. Dengan demikian, afiks dan penandaan fungsi gramatikal diharapkan mengikuti bahasa \textit{matrix}, sedangkan bahasa \textit{embedded} lebih aman jika dimasukkan sebagai unit leksikal bebas tanpa diinfleksikan menurut morfologi bahasa Indonesia \parencite{winata2023decades}. Akibatnya, \textit{code-switch} di posisi inti internal kata, misalnya menyisipkan leksem Inggris lalu menambahkan afiks derivatif Indonesia yang tidak sesuai, dipandang berisiko tinggi karena berpotensi melanggar batas morfologi. Sementara itu, sisipan pada posisi leksikal bebas seperti nomina \textit{landmark} atau nama ruangan cenderung dinilai aman, karena tidak mengganggu integritas morfosintaks bahasa \textit{matrix} \parencite{winata2023decades}.

\begin{figure}[htbp]
  \centering
  \includegraphics[width=1.0\linewidth]{images/cs-summary.png}
  \caption{Peta Konsep Tipologi \textit{Code-Switching} Indonesia–Inggris}
  \label{fig:cs-mindmap}
\end{figure}

Dalam konteks bahasa Indonesia, motivasi sosiolinguistik untuk \textit{code-switching} Indonesia-Inggris tidak hanya bersifat gaya, tetapi juga berkaitan dengan identitas sosial, domain penggunaan, dan strategi efisiensi kognitif. Pertama, penutur sering menggunakan unsur bahasa Inggris untuk mengindeks identitas profesional atau afiliasi komunitas tertentu, misalnya lingkungan kampus, kantor multinasional, atau komunitas permainan daring, sehingga pemilihan bahasa menjadi sinyal keanggotaan kelompok. Kedua, terdapat penyesuaian domain. Istilah teknis dan fasilitas bangunan sering kali tersedia, terdokumentasi, atau ditempel sebagai penanda ruangan dalam bahasa Inggris, sehingga mempertahankannya dalam bentuk asli dianggap lebih akurat dan ekonomis bagi pendengar \parencite{Tazakka2024}. Ketiga, ada keuntungan efisiensi kognitif karena penutur tidak perlu melakukan pencarian padanan leksikal lokal yang mungkin kurang dikenal atau malah memperkenalkan ambiguitas baru. Dalam instruksi navigasi untuk VLN, pola \textit{code-switch} ini sangat berguna. Bahasa Indonesia mempertahankan kejelasan tindakan prosedural langkah demi langkah, sementara nama objek fisik dan \textit{landmark} dibiarkan dalam bahasa Inggris agar agen tidak keliru mengenali referen fisik di lingkungan \parencite{gu2022vision}. Masalahnya, masih jarang ada \textit{dataset} navigasi \textit{code-switch} yang menyediakan kontrol pola dan rasio secara eksplisit; banyak \textit{benchmark} VLN dan lingkungan \textit{embodied} populer fokus pada monolingual/multilingual tanpa pengaturan rasio \textit{switch} yang terkontrol \parencite{gu2022vision,HM3D2021}.

\vspace{0.5em}

\subsection{Pemodelan Komputasional}
\label{subsec:pemodelan-komputasional}

\textit{Code-switching} Indonesia--Inggris pada instruksi navigasi menuntut pemisahan bahasa yang stabil:
istilah \textit{landmark} perlu dipertahankan spesifik (sering berbahasa Inggris), sementara struktur tindakan/prosedur
tetap mudah diikuti (umumnya berbahasa Indonesia). Karena itu, \textit{language identification} per-token (LID)
dimodelkan sebagai \textit{sequence labeling}: sebuah \textit{encoder} kontekstual (misal BiLSTM/Transformer) menghasilkan
representasi token yang kemudian diprediksi sebagai label bahasa, dengan opsi pemodelan terstruktur (misal CRF)
untuk menghindari pergantian label yang terlalu rapat \parencite{winata2023decades,Handoyo2024,dhawan2023unified,kargaran2024masklid}.

Diberikan himpunan data latih
$\mathcal{D}=\{(x^{(i)}_{1:T_i},y^{(i)}_{1:T_i})\}_{i=1}^{N}$ berisi $N$ sekuens. Sekuens ke-$i$ memiliki panjang
$T_i\in\mathbb{N}$, token masukan $x^{(i)}_{1:T_i}=(x^{(i)}_1,\dots,x^{(i)}_{T_i})$, dan label kebenaran
$y^{(i)}_{1:T_i}=(y^{(i)}_1,\dots,y^{(i)}_{T_i})$. Himpunan label bahasa dinyatakan sebagai
$\mathcal{Y}=\{\textsf{id},\textsf{en}\}$. Representasi kontekstual token pada posisi $t$ didefinisikan oleh \textit{encoder} berparameter $\theta$ sebagaimana pada
\eqref{eq:encoder242}:
\begin{equation}
\label{eq:encoder242}
h^{(i)}_t=f_\theta\!\big(x^{(i)}_{1:T_i},t\big)\in\mathbb{R}^d,
\end{equation}
dengan $d$ dimensi \textit{embedding} dan $\mathbb{R}$ himpunan bilangan real. Untuk memetakan representasi $h$ ke skor label,
\textit{scoring head} $g_\omega:\mathbb{R}^d\to\mathbb{R}^{|\mathcal{Y}|}$ (misal proyeksi linear) menghasilkan vektor skor,
dan skor emisi untuk label $y$ adalah komponen vektor tersebut:
\begin{equation}
\label{eq:skoremisi242}
g_\omega(h)\doteq Wh+b,\qquad
s_{\omega}(y,h)\doteq \bigl(g_\omega(h)\bigr)_y,
\end{equation}
dengan $W\in\mathbb{R}^{|\mathcal{Y}|\times d}$ dan $b\in\mathbb{R}^{|\mathcal{Y}|}$.

Agar notasi transisi valid pada $t=1$,
didefinisikan label awal $y_0=\langle\textsc{bos}\rangle$ dan himpunan label diperluas menjadi
$\widetilde{\mathcal{Y}}=\{\langle\textsc{bos}\rangle\}\cup\mathcal{Y}$.
Matriks transisi $A\in\mathbb{R}^{|\widetilde{\mathcal{Y}}|\times|\mathcal{Y}|}$ memiliki entri $A_{y',y}$ yang
menyatakan preferensi berpindah dari label sebelumnya $y'\in\widetilde{\mathcal{Y}}$ ke label saat ini $y\in\mathcal{Y}$.
Seluruh parameter CRF-LID diringkas sebagai $\vartheta \doteq (\theta,\omega,A)$, dengan simbol ``$\doteq$'' berarti
``didefinisikan sebagai''.
(Opsi tambahan yang lazim adalah menambahkan state akhir $\langle\textsc{eos}\rangle$ untuk memodelkan transisi terakhir,
namun tidak esensial untuk pelabelan per-token pada formulasi ini.)

Probabilitas sekuens label pada CRF \textit{linear-chain} untuk satu sekuens $(x_{1:T},y_{1:T})$ dinyatakan pada
\eqref{eq:lid_nll1}, dengan representasi $h_t$ diambil dari \textit{encoder} dan transisi menggunakan $A_{y_{t-1},y_t}$:
\begin{equation}
\label{eq:lid_nll1}
\begin{aligned}
p_{\vartheta}\!\big(y_{1:T}\mid x_{1:T}\big)
&=\frac{\exp\!\Big(\sum_{t=1}^{T}\big[s_{\omega}(y_t,h_t)+A_{y_{t-1},y_t}\big]\Big)}
{Z_{\vartheta}(x_{1:T})},\\
h_t&=f_{\theta}(x_{1:T},t),
\qquad
y_0=\langle\textsc{bos}\rangle.
\end{aligned}
\end{equation}
Faktor normalisasi (\textit{partition function}) yang menormalkan probabilitas pada \eqref{eq:lid_nll1} didefinisikan pada
\eqref{eq:lid_nll12}:
\begin{equation}
\label{eq:lid_nll12}
\begin{aligned}
Z_{\vartheta}(x_{1:T})
&=\sum_{y'_{1:T}\in\mathcal{Y}^{T}}
\exp\!\Big(\sum_{t=1}^{T}\big[s_{\omega}(y'_t,h_t)+A_{y'_{t-1},y'_t}\big]\Big),
\qquad y'_0=\langle\textsc{bos}\rangle.
\end{aligned}
\end{equation}
Pada \eqref{eq:lid_nll12}, $\mathcal{Y}^{T}$ menyatakan himpunan semua sekuens label panjang $T$ (yakni \textit{cartesian power}),
dan $y'_{1:T}$ adalah variabel \textit{dummy} untuk menjumlahkan seluruh kemungkinan sekuens label.
Objektif pelatihan berupa \textit{negative log-likelihood} (NLL) dinormalisasi per sekuens didefinisikan pada
\eqref{eq:lid_nll3}:
\begin{equation}
\label{eq:lid_nll3}
L_{\text{LID}}(\vartheta)
= -\frac{1}{N}\sum_{i=1}^{N}\log p_{\vartheta}\!\big(y^{(i)}_{1:T_i}\mid x^{(i)}_{1:T_i}\big).
\end{equation}

\textit{Encoder} menyediakan fitur kontekstual $h_t$ untuk tiap token, emisi $s_\omega(y_t,h_t)$ mendukung keputusan label lokal,
dan transisi $A_{y_{t-1},y_t}$ memberi bias agar label tidak bolak-balik terlalu rapat, sehingga segmen bahasa cenderung utuh.

Normalisasi per token (misal membagi dengan $\sum_i T_i$) juga lazim dipakai bila ingin menyeimbangkan kontribusi sekuens panjang dan pendek,
tanpa mengubah bentuk dasar pada \eqref{eq:lid_nll3}. Komponen transisi $A_{y_{t-1},y_t}$ yang muncul pada \eqref{eq:lid_nll1}
mengatur preferensi agar segmen bahasa cenderung utuh, sehingga \textit{switching} tidak terjadi di setiap token
\parencite{kargaran2024masklid,winata2023decades}. Jika dependensi antarlabel tidak diinginkan, model menjadi kasus khusus dengan meniadakan
peran transisi (misal menetapkan $A=0$), sehingga \eqref{eq:lid_nll1} memfaktorkan prediksi per token melalui emisi saja dan ekuivalen dengan
klasifikasi token independen berbasis \textit{softmax} atas skor $g_\omega(h_t)$.

Untuk pembangkitan instruksi VLN, selain kelancaran, diperlukan dua kendali operasional: (i) rasio bahasa agar porsi Inggris cukup untuk menyebut
\textit{landmark} tanpa mengaburkan prosedur berbahasa Indonesia, dan (ii) kepatuhan batas morfologi Indonesia agar sisipan Inggris tidak melekat
pada afiks terikat. Kendali ini dapat diimplementasikan pada tahap dekoding sebagai penalti/kendala, misalnya melalui automata untuk kamus
\textit{landmark} dan aturan morfologi \parencite{Koo2024AutomataConstraints,BeurerKellner2024Domino}.
Dalam praktik, penalti dapat dihitung \emph{post-hoc} untuk \textit{re-ranking} kandidat hasil \textit{beam search}, atau secara \emph{incremental}
selama pencarian (misal mengukur pelanggaran pada prefiks sekuens).

Misalkan sebuah model generatif berparameter $\phi$ memberi skor log-probabilitas $\log p_{\phi}(w\mid c)$ untuk konteks VLN $c$, dengan
$w\in\mathcal{V}^*$ sekuens token dari kosakata $\mathcal{V}$. Kosakata $\mathcal{V}$ mencakup token khusus \textsc{eos} (\textit{end-of-sequence})
yang menandai berhenti, dan $\mathcal{V}^*$ menyatakan himpunan semua sekuens hingga panjang berhingga dari token $\mathcal{V}$.
Untuk menghindari bias keluaran pendek, definisikan sekuens tanpa token \textsc{eos} sebagai
$w_{\neg \text{\textsc{eos}}}$ (yakni $w$ setelah menghapus token \textsc{eos}). Normalisasi panjang pada token non-\textsc{eos}:
\begin{equation}
  \label{eq:normalizenoneos}
  T(w)\doteq |w_{\neg \text{\textsc{eos}}}|,\qquad
  \ell_{\phi}(w;c)\doteq \frac{1}{T(w)}\log p_{\phi}(w\mid c).
\end{equation}

Rasio Inggris dihitung sebagai proporsi token pada $w_{\neg \textsc{eos}}$ yang diprediksi berlabel $\textsf{en}$ oleh modul LID (misal CRF pada
\eqref{eq:lid_nll1}--\eqref{eq:lid_nll12} atau aturan leksikon konsisten), dan didefinisikan eksplisit pada \eqref{eq:rho_en_def}:
\begin{equation}
\label{eq:rho_en_def}
\rho_{\text{en}}(w)\doteq \frac{1}{T(w)}\sum_{t=1}^{T(w)} \mathbf{1}[\hat{y}_t=\textsf{en}],
\end{equation}
dengan $\hat{y}_{1:T(w)}$ sekuens label hasil LID untuk token pada $w_{\neg \textsc{eos}}$, dan $\mathbf{1}[\cdot]$ fungsi indikator bernilai 1 jika kondisi
benar dan 0 jika salah. Target rasio dinyatakan sebagai $\rho^\star\in[0,1]$, dan deviasi rasio didefinisikan pada \eqref{eq:delta_rho}:
\begin{equation}
\label{eq:delta_rho}
\Delta_{\rho}(w)\doteq \bigl|\rho_{\text{en}}(w)-\rho^\star\bigr|.
\end{equation}

Kepatuhan terhadap kamus \textit{landmark} dan batas morfologi dirangkum oleh dua penguji boolean
$\mathrm{FA}:\mathcal{V}^*\to\{\text{true},\text{false}\}$ dan $\mathrm{MORPH}:\mathcal{V}^*\to\{\text{true},\text{false}\}$.
Secara operasional, $\mathrm{FA}(w)$ dapat diimplementasikan sebagai automata/\textit{gazetteer} yang memvalidasi kemunculan entitas \textit{landmark}
(terutama \textsf{en}) sesuai daftar dan batas kata; sedangkan $\mathrm{MORPH}(w)$ memeriksa larangan afiks Indonesia melekat pada sisipan Inggris.
Contoh pelanggaran morfologi mencakup bentuk seperti \textit{di-turn}, \textit{turn-kan}, \textit{left-nya}, atau \textit{parkingnya},
yakni ketika prefiks/sufiks terikat Indonesia melekat pada token yang semestinya dipertahankan sebagai \textit{landmark}/kata Inggris.
Pada praktiknya, pemeriksaan ini biasanya dilakukan pada bentuk kata (setelah detokenisasi ringan atau pada batas \textit{wordpiece} yang stabil)
agar tidak salah menghukum pemenggalan subkata oleh BPE.

Pelanggaran keduanya dirumuskan sebagai indikator pada \eqref{eq:delta_fa_morph}:
\begin{equation}
\label{eq:delta_fa_morph}
\Delta_{\mathrm{FA}}(w)\doteq 1-\mathbf{1}[\mathrm{FA}(w)],\qquad
\Delta_{\mathrm{MORPH}}(w)\doteq 1-\mathbf{1}[\mathrm{MORPH}(w)],
\end{equation}
sehingga $\Delta_{\mathrm{FA}}(w),\Delta_{\mathrm{MORPH}}(w)\in\{0,1\}$.

Dengan bobot penalti $\lambda=\{\lambda_{\rho},\lambda_{\mathrm{FA}},\lambda_{\mathrm{MORPH}}\}$ (umumnya
$\lambda_{\rho},\lambda_{\mathrm{FA}},\lambda_{\mathrm{MORPH}}\ge 0$), penalti total didefinisikan pada \eqref{eq:total_penalty}:
\begin{equation}
\label{eq:total_penalty}
\Omega_{\lambda}(w)\doteq
\lambda_{\rho}\,\Delta_{\rho}(w)
+\lambda_{\mathrm{FA}}\,\Delta_{\mathrm{FA}}(w)
+\lambda_{\mathrm{MORPH}}\,\Delta_{\mathrm{MORPH}}(w).
\end{equation}
Keluaran terpilih $\hat{w}$ kemudian diperoleh melalui dekoding berpenalti pada \eqref{eq:vln_decode}:
\begin{equation}
\label{eq:vln_decode}
\hat{w}
=\arg\max_{w\in\mathcal{V}^*}
\Bigl\{
\ell_{\phi}(w;c)-\Omega_{\lambda}(w)
\Bigr\}.
\end{equation}
Pada \eqref{eq:vln_decode}, penalti rasio $\lambda_{\rho}$ mendorong komposisi bahasa mendekati target $\rho^\star$,
sedangkan $\lambda_{\mathrm{FA}}$ dan $\lambda_{\mathrm{MORPH}}$ menurunkan skor kandidat yang melanggar kamus \textit{landmark} atau aturan
morfologi. Kendala keras dapat dipandang sebagai kasus batas dengan menaikkan penalti pelanggaran hingga sangat besar, atau dengan membatasi ruang
pencarian hanya pada kandidat yang memenuhi kendala; target $\rho^\star$ juga dapat ditala sesuai lingkungan \parencite{BeurerKellner2024Domino,Willard2023Outlines}.
Secara komputasional, optimisasi \eqref{eq:vln_decode} umumnya didekati dengan \textit{beam search} atau sampling terkontrol, lalu kandidat di-\textit{re-rank}
menggunakan penalti di atas.

\begin{figure}[H]
  \centering
  \includegraphics[width=0.7\linewidth]{images/embedding_plane_3d_scatter_complex.pdf}
  \caption{Ruang \textit{Embedding} 3D Token yang Memperlihatkan Subklaster Bahasa Indonesia dan Inggris, Bidang Keputusan Perkiraan, Serta Lintasan \textit{Code-Switching} Antarwilayah (Skema Konseptual)}
  \label{fig:lid-embedding-space}
\end{figure}

Gambar~\ref{fig:lid-embedding-space} bersifat ilustratif untuk menunjukkan intuisi di balik LID:
token cenderung membentuk subklaster berbeda untuk Indonesia dan Inggris pada ruang \textit{embedding}, sehingga sebuah bidang keputusan perkiraan
dapat memisahkan wilayah label secara relatif stabil. Lintasan \textit{code-switching} tampak sebagai pergerakan representasi yang menyeberangi batas
antarwilayah (bukan loncatan acak di setiap token), selaras dengan tujuan pada \eqref{eq:lid_nll1}--\eqref{eq:lid_nll3} untuk menghasilkan segmen bahasa
yang koheren. Secara ringkas, pemisahan berbasis representasi mempermudah keputusan lokal melalui emisi pada \eqref{eq:skoremisi242}, sementara (ketika digunakan)
komponen transisi $A$ pada \eqref{eq:lid_nll1} menstabilkan batas segmen agar tidak terlalu rapat \parencite{winata2023decades,dhawan2023unified,kargaran2024masklid}.

Agar relevan untuk VLN, keluaran LID tidak berhenti pada label bahasa.
Token berlabel \textsf{id} dipetakan ke predikat tindakan dan relasi spasial (instruksi prosedural) yang stabil bagi \textit{planner},
sedangkan token berlabel \textsf{en} menandai entitas fisik/\textit{landmark} yang dipertahankan apa adanya untuk mengurangi ambiguitas \textit{grounding}
perseptual \parencite{gu2022vision}.

Secara praktis, ketahanan LID diperkuat oleh representasi subkata/karakter untuk menangani variasi ejaan dan OOV pada nama ruang
\parencite{winata2023decades,dhawan2023unified,kargaran2024masklid}. Pemeriksaan morfologi ringan dapat mencegah pelabelan \textsf{en} pada token yang jelas
berafiks Indonesia, konsisten dengan kendala morfologis pada \eqref{eq:vln_decode} \parencite{winata2023decades,Koo2024AutomataConstraints,BeurerKellner2024Domino}.
Leksikon domain/\textit{gazetteer} dapat menambah bias positif untuk istilah \textit{landmark}, namun tetap harus tunduk pada kepatuhan batas kata agar tidak
memicu sisipan yang tidak gramatikal \parencite{Koo2024AutomataConstraints,BeurerKellner2024Domino}.

\vspace{0.5em}

\section{\textit{LLM in-the-loop} untuk Pembangkitan Data}

\vspace{0.5em}

\subsection{Konsep Umum dan Peran LLM sebagai Pembangkit Instruksi}

\textit{Large Language Models} (LLM) seperti GPT-4, LLaMA, dan Qwen menunjukkan kemampuan generatif yang kuat dalam berbagai tugas pemrosesan bahasa alami melalui antarmuka berbasis \textit{prompt} \parencite{Touvron2023,Dubey2024,qwen25,Mienye2025LargeLanguageModels}. Dalam penelitian ini, istilah \textit{LLM in-the-loop} untuk pembangkitan data merujuk pada penggunaan LLM sebagai komponen aktif di dalam sebuah \textit{pipeline} pembangkitan data, bukan hanya sebagai model inferensi yang menjawab kueri siap pakai. LLM dipanggil berulang kali untuk mensintesis data baru, seperti pasangan instruksi dan trajektori, instruksi dan konteks visual, atau deskripsi langkah-langkah, berdasarkan konteks struktural yang diberikan oleh sistem.

\begin{figure}[H]
    \centering
    \includegraphics[width=0.6\linewidth]{images/ptheta_space_theoretical_v2.pdf}
    \caption{Ilustrasi Ruang Instruksi $p_{\theta}(x \mid c)$ dan Subhimpunan Instruksi Valid $\mathcal{X}_{\mathrm{valid}}(c)$ yang Dihasilkan Melalui Struktur Tambahan Dalam Konteks, Seperti Rencana Eksplisit, Informasi Semantik Visual, Serta Filter Heuristik.}
    \label{fig:ptheta-space}
\end{figure}

Gambar~\ref{fig:ptheta-space} mengilustrasikan ruang keluaran instruksi yang dapat dihasilkan oleh LLM, yang dimodelkan sebagai distribusi generatif $p_{\theta}(x\mid c)$ pada konteks $c$. Sumbu horizontal merepresentasikan \textit{plan fidelity} (seberapa selaras instruksi dengan urutan rencana/aksi), sedangkan sumbu vertikal merepresentasikan \textit{style/lexical diversity} (keragaman gaya dan pilihan leksikal). Elips besar menggambarkan keseluruhan dukungan (ragam) instruksi yang mungkin muncul dari $p_{\theta}(x\mid c)$, sementara elips putus-putus di dalamnya merepresentasikan subhimpunan instruksi valid $\mathcal{X}_{\mathrm{valid}}(c)$ yang memenuhi kendala keselarasan rencana, korespondensi entitas, dan aturan format. Panah menunjukkan bahwa penambahan struktur dalam konteks (misalnya rencana eksplisit, informasi semantik visual, serta \textit{filter} heuristik) berfungsi sebagai mekanisme pengarah yang memperkecil ruang efektif keluaran menuju area yang lebih valid, sehingga instruksi yang dihasilkan cenderung lebih dapat dieksekusi tanpa mengorbankan keragaman secara berlebihan.

Pembedaan antara LLM sebagai model inferensi dan LLM sebagai pembangkit data penting untuk memperjelas kontribusi arsitektur semacam ini. Pada skenario inferensi klasik, LLM menerima masukan berupa teks (misalnya pertanyaan atau perintah) dan menghasilkan keluaran tunggal yang langsung digunakan pengguna. Sebaliknya, pada skenario \textit{in-the-loop} untuk pembangkitan data, keluaran LLM diperlakukan sebagai bahan baku: ia dikumpulkan, diformat, dan disaring untuk kemudian menjadi bagian dari korpus pelatihan, \textit{benchmark}, atau antarmuka instruksi bagi agen lain (misalnya agen navigasi). Dengan demikian, LLM berperan sebagai mesin generatif berskala besar yang beroperasi di atas representasi simbolik dari lingkungan dan tugas, bukan sebagai modul jawaban sekali pakai.

Dalam domain VLN, kebutuhan akan instruksi berkualitas tinggi dan beragam telah lama diidentifikasi sebagai faktor kunci dalam kinerja agen\parencite{anderson2018r2r,ku2020rxr,qi2020reverie,thomason2020cvdn,chen2019touchdown}. Dataset seperti R2R, RxR, REVERIE, dan CVDN mengilustrasikan bagaimana pasangan trajektori--instruksi dikumpulkan dan dikurasi secara manual sebelum berkembang menjadi \textit{benchmark} standar\parencite{anderson2018r2r,ku2020rxr,qi2020reverie,thomason2020cvdn}. Survei terbaru menunjukkan bahwa ekosistem VLN semakin bergeser ke arah pemanfaatan model fondasi dan LLM untuk mengurangi biaya anotasi manual dan memperkaya variasi instruksi, termasuk pada pengaturan lingkungan kontinu dan tugas jangka panjang\parencite{gu2022vision,Zhang2024,Song2025}. Dalam kerangka ini, \textit{LLM in-the-loop} diposisikan sebagai pengganti atau pelengkap anotator manusia dalam menghasilkan deskripsi bahasa natural yang konsisten dengan trajektori dan konteks visual.

Salah satu pola yang banyak diadopsi adalah penggunaan LLM \textit{pre-trained} sebagai pembangkit instruksi berbasis \textit{prompt}, tanpa melakukan perubahan parameter model. Pendekatan ini sejalan dengan paradigma di mana model yang sudah dilatih berskala besar dimanfaatkan kembali untuk menghasilkan instruksi atau caption baru hanya melalui desain \textit{prompt} dan pemberian contoh (\textit{in-context learning})\parencite{Mienye2025LargeLanguageModels}. Dalam pengaturan ini, LLM dikondisikan pada deskripsi tugas, skema lingkungan, atau meta-data simbolik lain, kemudian diminta untuk menghasilkan instruksi, penjelasan, atau deskripsi skenario. Keluaran LLM dapat digunakan sebagai data pelatihan tambahan, sebagai set instruksi sintetis untuk evaluasi, atau sebagai komponen tekstual yang diberikan kepada agen navigasi.

Literatur di luar VLN telah mengeksplorasi skema di mana LLM diminta menghasilkan instruksi dari beberapa contoh awal dan deskripsi tugas, misalnya dalam kerangka pembangkitan instruksi otomatis seperti Self-Instruct\parencite{wang2023selfinstruct}. Meskipun karya tersebut pada akhirnya menggunakan instruksi sintetis untuk membangun model lain, bagian yang relevan bagi penelitian ini adalah bukti bahwa LLM yang dibekukan mampu menghasilkan instruksi beranotasi diri dalam jumlah besar dengan hanya mengandalkan desain \textit{prompt} dan beberapa contoh acuan. Survei LLM dan model fondasi juga menekankan bahwa kontrol berbasis \textit{prompt} terhadap gaya, struktur, dan isi keluaran merupakan karakteristik inti dari model-model ini\parencite{Mienye2025LargeLanguageModels,Kuchemann2025}.

Dalam VLN dan tugas \textit{embodied} lain, terdapat tradisi sebelumnya dalam membangkitkan instruksi dari trajektori melalui model \textit{speaker} yang memetakan urutan aksi ke bahasa natural\parencite{anderson2018r2r}. Model-model tersebut biasanya dilatih secara \textit{supervised} di atas korpus trajektori--instruksi yang sudah ada. \textit{LLM in-the-loop} dapat dipandang sebagai generalisasi dari ide \textit{speaker}: representasi trajektori atau rencana (misalnya urutan tindakan navigasi, waypoint, dan objek yang dikunjungi) disuplai sebagai konteks teks kepada LLM, yang kemudian menghasilkan instruksi yang menjelaskan cara mengeksekusi trajektori tersebut. Bedanya, LLM tidak dibatasi oleh ruang pelatihan dataset awal dan mampu menggabungkan pengetahuan dunia yang luas, gaya bahasa variatif, serta kemampuan multibahasa.

Dimensi penting lain dari penggunaan \textit{LLM in-the-loop} adalah \textit{planning-aware \textit{prompt}ing}, yaitu strategi pemberian konteks di mana LLM tidak hanya menerima deskripsi global tugas, tetapi juga struktur rencana atau urutan aksi eksplisit. Karya-karya di domain penalaran menggunakan LLM, seperti Plan-and-Solve \textit{Prompt}ing dan ReAct, menunjukkan bahwa memisahkan langkah perencanaan dan aksi, atau menggabungkan penalaran dengan tindakan eksplisit, dapat meningkatkan konsistensi dan interpretabilitas keluaran model\parencite{Lu2023PlanSolve,Yao2023ReAct}. Diadaptasikan ke konteks VLN, ide ini berarti bahwa konteks $c$ yang diberikan ke LLM tidak hanya berupa deskripsi lingkungan, tetapi juga urutan langkah simbolik (misalnya TURN-LEFT, GO-FORWARD, ENTER-KITCHEN) dan daftar objek atau ruangan yang relevan.

Secara konseptual, LLM kemudian dimodelkan sebagai distribusi generatif yang dituliskan sebagai Persamaan~\ref{eq:distribusi_generatif}:
\begin{equation}
  \label{eq:distribusi_generatif}
x \sim p_{\theta}(x \mid c),
\end{equation}
di mana $x$ adalah instruksi bahasa natural dan $c$ mencakup rencana/urutan aksi, informasi objek, serta deskripsi semantik lingkungan. Hal ini dapat diilustrasikan sebagai Gambar~\ref{fig:ptheta-space} dan Gambar~\ref{fig:instruction-manifold}. Tujuan perancangan \textit{prompt} adalah memastikan bahwa $x$ memelihara urutan langkah yang tepat, menjaga korespondensi antara aksi simbolik dan deskripsi natural, serta menghasilkan instruksi yang dapat dieksekusi oleh agen di dalam simulator atau lingkungan nyata. Survei VLN menekankan bahwa menjaga keselarasan antara representasi trajektori, instruksi, dan dinamika lingkungan menjadi kunci keberhasilan agen dalam pengaturan kontinu dan jangka panjang\parencite{wijmans2020vlnce,KrantzVLNCE,Zhang2024,Song2025}.

\begin{figure}[H]
    \centering
    \includegraphics[width=0.55\linewidth]{images/instruction_manifold.png}%
    % ganti dengan nama file gambar kamu, tanpa ekstensi jika pakai pdf/png
    \caption{Ilustrasi Ruang Instruksi dan Subhimpunan Instruksi Valid Yang Dikondisikan Pada Konteks.}
    \label{fig:instruction-manifold}
\end{figure}

Gambar~\ref{fig:instruction-manifold} memformalkan gagasan bahwa, untuk setiap konteks $c$, hanya sebagian kecil dari ruang instruksi yang benar-benar valid dan relevan. Titik $c$ melambangkan sebuah konteks (misalnya kombinasi lingkungan, tujuan, dan rencana), sementara ruang abu-abu merepresentasikan kandidat instruksi yang mungkin dihasilkan pada ruang representasi (misalnya koordinat laten atau fitur linguistik). Wilayah bertanda $\mathcal{X}_{\mathrm{valid}}$ menunjukkan \textit{manifold} atau subruang instruksi yang konsisten dengan konteks tersebut, yaitu instruksi yang menyebut entitas valid, selaras dengan urutan aksi, dan tidak melanggar batasan tugas. Panah dari $c$ menuju $\mathcal{X}_{\mathrm{valid}}$ menekankan peran pengondisian (melalui \textit{prompt} dan struktur konteks) sebagai pemetaan yang “mengunci” generasi agar berfokus pada subhimpunan yang dapat dieksekusi, sehingga proses \textit{LLM in-the-loop} dapat dipahami sebagai upaya sistematis untuk mendekatkan sampel $x\sim p_{\theta}(x\mid c)$ ke wilayah valid alih-alih menjelajah ruang instruksi secara bebas.

\vspace{0.5em}

\subsection{Kontrol Output, Integrasi Semantik, dan Filter Heuristik}

Di luar struktur rencana, keberhasilan skema \textit{LLM in-the-loop} sangat bergantung pada kemampuan untuk mengontrol gaya, format, dan leksikon keluaran melalui desain \textit{prompt}. Laporan teknis GPT-4 dan dokumentasi model fondasi lain menunjukkan bahwa instruksi mengenai persona, tingkat formalitas, jenis register, dan struktur output (misalnya daftar bernomor, format JSON, atau paragraf tunggal) dapat diikuti cukup konsisten oleh model berskala besar\parencite{Touvron2023,Dubey2024,qwen25}. Survei tentang LLM juga mencatat bahwa kontrol semacam ini biasanya dicapai sepenuhnya di tingkat \textit{prompt} dan contoh dalam konteks, tanpa perlu mengubah parameter model\parencite{Mienye2025LargeLanguageModels}. Dalam pengaturan pembangkitan data, hal ini memungkinkan peneliti menspesifikasikan dengan eksplisit bahwa instruksi harus, misalnya, terdiri dari satu paragraf pendek, menghindari istilah teknis, atau mengikuti format penomoran langkah-langkah aksi.

Kontrol terhadap leksikon dan gaya dapat diperluas ke ranah multibahasa maupun \textit{code-switching}. Literatur mengenai \textit{code-switching} dalam NLP dan pengolahan wicara menunjukkan bahwa pencampuran bahasa mengikuti pola sintaktis dan pragmatis tertentu, serta dapat dievaluasi menggunakan berbagai metrik kompleksitas dan indeks campuran bahasa\parencite{winata2023decades,SrivastavaSingh2021CALCS,Chi2024TIndex,Suresh2025CSSum}. Penelitian tentang pengenalan dan sintesis wicara \textit{code-switching} juga menekankan pentingnya membedakan fungsi masing-masing bahasa untuk kategori kata tertentu, seperti penggunaan bahasa lokal untuk nama tempat atau objek, dan bahasa global untuk verba aksi atau kata kerja pendek \parencite{dhawan2023unified,Handoyo2024}. Temuan-temuan ini menyediakan dasar teoretis untuk memberi instruksi eksplisit kepada LLM agar menghasilkan teks dwibahasa atau \textit{code-switching} sesuai aturan yang ditetapkan, misalnya dengan menggunakan bahasa A (misalnya bahasa Indonesia) sebagai bahasa dominan untuk struktur kalimat dan narasi, menggunakan bahasa B (misalnya bahasa Inggris) untuk verba aksi atau istilah navigasi pendek (seperti \textit{turn left}, \textit{go straight}), serta mempertahankan rasio tertentu antara token dalam bahasa A dan bahasa B atau membatasi kemunculan istilah-istilah tertentu.

Dalam skema \textit{LLM in-the-loop}, aturan-aturan ini diwujudkan langsung ke dalam \textit{prompt}, misalnya melalui instruksi tekstual yang menjelaskan proporsi bahasa, daftar kata yang boleh atau tidak boleh digunakan, serta contoh instruksi \textit{code-switching} yang diinginkan. Karena LLM modern telah dilatih pada korpus multibahasa berskala besar\parencite{Touvron2023,qwen25}, mereka umumnya mampu mengikuti pola \textit{code-switching} yang diminta tanpa modifikasi parameter. Hal ini memungkinkan pembangkitan dataset instruksi yang mencerminkan praktik bahasa pengguna akhir (misalnya penutur bilingual) sekaligus tetap terkontrol secara sistematis berdasarkan metrik dan prinsip linguistik dari literatur code-switching \parencite{winata2023decades,Chi2024TIndex,Suresh2025CSSum}.

Aspek penting lain dari \textit{LLM in-the-loop} untuk tugas visi-bahasa adalah integrasi dengan informasi simbolik dan semantik yang diekstraksi dari lingkungan visual. Di ranah VLN, berbagai platform simulasi seperti Habitat dan Gibson menyediakan representasi terstruktur berupa peta 3D, graf navigasi, dan anotasi semantik ruangan/objek yang dapat diolah menjadi bentuk simbolik\parencite{savva2019habitat,szot2021habitat2,HM3D2021,xia2018gibson}. Dataset seperti HM3D dan turunan-turunannya lebih lanjut memperkaya lingkungan dengan label semantik yang padat\parencite{HM3D2021}. Survei VLN menekankan bahwa model modern cenderung memanfaatkan kombinasi representasi geometrik, semantik, dan bahasa untuk mencapai penalaran yang lebih kuat \parencite{gu2022vision,Zhang2024,Song2025}.

Secara bersamaan, ekosistem model visi modern menyediakan komponen yang mampu mengekstraksi tag semantik serta struktur objek dari citra. CLIP, Recognize Anything, dan Tag2Text, misalnya, memanfaatkan hubungan antara teks dan visual untuk menghasilkan deskripsi objek atau kumpulan tag yang dapat digunakan sebagai kata kunci\parencite{radford2021clip,zhang2023recognize,huang2023tag2text,huang2023openset}. Segment Anything menghasilkan segmentasi objek yang dapat diterjemahkan menjadi daftar instansi lengkap dengan posisi dan kategorinya\parencite{kirillov2023segment}. Survei mengenai \textit{text-guided 3D visual grounding} juga menekankan pentingnya representasi simbolik dan semantik yang menghubungkan objek, ruang, dan deskripsi teks \parencite{liu2025visualgrounding}. Dalam kerangka \textit{LLM in-the-loop}, keluaran dari model visi tersebut dapat diubah menjadi representasi tekstual terstruktur, seperti daftar objek beserta atributnya atau rangkuman tag untuk setiap ruangan, yang kemudian diberikan kepada LLM sebagai bagian dari konteks $c$.

Pola umum yang muncul adalah sebagai berikut: (i) sistem visi menghasilkan representasi simbolik (daftar objek, label ruangan, graf konektivitas, atau urutan aksi navigasi), (ii) representasi tersebut diformat ke dalam teks yang eksplisit (misalnya ``\textit{From the hallway, go past a red sofa into the kitchen with a large table}'') dan dilampirkan dalam \textit{prompt}, lalu (iii) LLM diminta untuk menghasilkan instruksi, penjelasan, atau narasi yang konsisten dengan representasi tersebut. Dengan cara ini, pemodelan visual tetap ditangani oleh model visi khusus, sementara LLM digunakan untuk melakukan penalaran dan perumusan instruksi di ruang teks, memanfaatkan kemampuannya dalam memahami struktur, koherensi diskursif, dan pengetahuan dunia\parencite{gu2022vision,Zhang2024,Song2025,radford2021clip,huang2023tag2text,liu2025visualgrounding}.

Karena LLM beroperasi sebagai pembangkit yang relatif tidak terikat, dibutuhkan mekanisme untuk memastikan bahwa keluaran yang dihasilkan konsisten dengan pengetahuan struktural tentang lingkungan dan tugas. Alih-alih menggunakan penilai otomatis yang kompleks, banyak skema pembangkitan data berbasis LLM mengandalkan \textit{filter heuristik} sederhana yang sesuai dengan domain. Tradisi kurasi dataset VLN memberikan contoh awal: instruksi yang tidak menyebutkan objek penting, yang mengandung rujukan ke entitas tidak valid, atau yang tidak konsisten dengan trajektori sering kali dibuang atau diperbaiki melalui prosedur berbasis aturan dan verifikasi manual\parencite{anderson2018r2r,ku2020rxr,qi2020reverie}. Prinsip serupa dapat diadaptasi ke konteks \textit{LLM in-the-loop} dengan langkah-langkah seperti memeriksa apakah semua objek yang disebut dalam instruksi terdapat dalam daftar entitas valid yang diekstraksi dari lingkungan, memastikan bahwa struktur keluaran (misalnya format daftar langkah, JSON, atau pola frasa tertentu) memenuhi spesifikasi yang diberikan dalam \textit{prompt}, serta menyaring keluaran yang mengandung kata atau frasa yang dilarang (misalnya rujukan eksplisit ke informasi di luar lingkungan simulasi).

\begin{figure}[H]
    \centering
    \includegraphics[width=0.7\linewidth]{images/fig_heuristic_filter.png}
    \caption{Contoh Ruang Fitur Untuk Filter Heuristik Pada Keluaran \textit{LLM In-The-Loop}}
    \label{fig:heuristic-filter}
\end{figure}

Gambar~\ref{fig:heuristic-filter} menunjukkan contoh ruang fitur sederhana untuk menerapkan \textit{filter heuristik} pada keluaran \textit{LLM in-the-loop}. Setiap titik merepresentasikan satu instruksi yang dihasilkan, dipetakan ke dua metrik: \textit{entity coverage} pada sumbu-$x$ (cakupan penyebutan entitas/objek relevan dari konteks) dan \textit{format compliance} pada sumbu-$y$ (kepatuhan terhadap spesifikasi format keluaran, misalnya pola langkah bernomor atau skema JSON). Garis putus-putus membentuk ambang batas minimal untuk kedua metrik, sehingga area kanan-atas menjadi \textit{accepted region} (instruksi diterima) karena memenuhi cakupan entitas dan kepatuhan format secara simultan, sedangkan titik di luar wilayah tersebut ditolak. Visualisasi ini menekankan bahwa filter heuristik tidak perlu menilai kualitas semantik secara penuh; cukup dengan indikator domain-spesifik yang mudah dihitung untuk menyingkirkan keluaran yang jelas tidak konsisten atau tidak dapat digunakan.

Filter semacam ini tidak menilai kualitas instruksi secara semantik penuh, tetapi menjamin konsistensi dasar antara keluaran LLM dan struktur tugas. Dalam praktiknya, filter heuristik dapat digabungkan dengan desain \textit{prompt} yang hati-hati untuk mengurangi frekuensi keluaran yang melanggar aturan, sehingga sebagian besar pembangkit LLM langsung \textit{usable}, sementara sebagian kecil diperbaiki atau dibuang. Secara konseptual, prosedur ini dapat dipandang sebagai penetapan daerah keputusan pada ruang fitur yang menggabungkan cakupan entitas dan kepatuhan format, seperti ditunjukkan pada Gambar~\ref{fig:heuristic-filter}.

\vspace{0.5em}

\section{Metrik Evaluasi}
\label{sec:metrik-evaluasi}
Bagian ini mendeskripsikan metrik yang digunakan untuk mengevaluasi kualitas data
LH-VLN \parencite{Song2025} serta
karakteristik \textit{code-switching} (ID-EN) pada instruksi yang dihasilkan.
Seluruh metrik dikelompokkan menjadi dua kategori utama:
(i) metrik efektivitas--efisiensi navigasi dan tugas, serta
(ii) metrik linguistik \textit{code-switching}.

\vspace{0.5em}

\subsection{Metrik Efektivitas dan Efisiensi Navigasi-Tugas}
\label{subsec:nav_metrics}

Untuk mengevaluasi efektivitas dan efisiensi penyelesaian tugas navigasi multi-tahap, penelitian ini menggunakan enam metrik utama, yaitu \textit{time cost}, \textit{fail tasks}, \textit{mean task step}, \textit{mean navigation step}, \textit{mean task success rate}, dan \textit{mean navigation success rate}. Misalkan $\mathcal{T}$ adalah himpunan seluruh tugas yang dieksekusi.

\vspace{0.5em}

\subsubsection{\textit{Time cost}}
Time cost digunakan untuk mengukur efisiensi komputasi atau waktu eksekusi kebijakan navigasi. Untuk setiap tugas $j \in \mathcal{T}_{\text{eval}}$ (tugas yang berhasil dieksekusi di simulator), dicatat waktu mulai $t^{\text{start}}_j$, waktu selesai $t^{\text{end}}_j$, serta jumlah langkah navigasi total $N^{\text{step}}_j$ yang dibutuhkan untuk menyelesaikan seluruh sub-navigasi dalam tugas tersebut. Time cost rata-rata didefinisikan sebagai
\begin{equation}
    \text{TimeCost}
    = \frac{1}{\left|\mathcal{T}_{\text{eval}}\right|}
      \sum_{j \in \mathcal{T}_{\text{eval}}}
      \frac{t^{\text{end}}_j - t^{\text{start}}_j}{N^{\text{step}}_j}.
    \label{eq:time_cost}
\end{equation}
Dengan demikian, Persamaan~\ref{eq:time_cost} menyatakan rata-rata waktu per langkah navigasi, yang memungkinkan perbandingan efisiensi antar-tugas dengan panjang lintasan yang berbeda-beda.

\vspace{0.5em}

\subsubsection{\textit{Fail tasks}}
Tidak semua instruksi tugas dapat dieksekusi dengan sukses, misalnya ketika target tidak dapat dijangkau oleh perencana lintasan. Himpunan tugas yang gagal didefinisikan sebagai kumpulan tugas yang tidak dapat diselesaikan dalam batasan yang dimiliki oleh sistem pemrosesan. Jumlah \textit{fail tasks} merujuk pada banyaknya elemen di dalam himpunan tersebut. Selain itu, deskripsi instruksi untuk setiap tugas yang gagal disimpan guna memungkinkan analisis kualitatif terhadap pola-pola kegagalan.

\vspace{0.5em}

\subsubsection{\textit{Mean task step}}
Untuk setiap tugas yang selesai dengan sukses, total langkah yang dibutuhkan untuk menyelesaikan seluruh rangkaian sub-tugas dinotasikan sebagai $T_j = N^{\text{step}}_j$. Jika $\mathcal{S} \subseteq \mathcal{T}$ adalah himpunan tugas yang berhasil diselesaikan, dengan $N_{\text{succ}} = |\mathcal{S}|$, maka \textit{mean task step} didefinisikan sebagai
\begin{equation}
    \text{MeanTaskStep}
    = \frac{1}{N_{\text{succ}}}
      \sum_{j \in \mathcal{S}} T_j.
    \label{eq:mean_task_step}
\end{equation}
Dengan kata lain, Persamaan~\ref{eq:mean_task_step} mengukur rata-rata panjang lintasan (dalam satuan langkah) untuk menyelesaikan satu tugas secara utuh.

\vspace{0.5em}

\subsubsection{\textit{Mean navigation step}}
Satu tugas biasanya terdiri atas beberapa episode navigasi menuju objek-objek target perantara. Misalkan $L_{j,k}$ adalah jumlah langkah yang dibutuhkan pada navigasi ke-$k$ dari tugas $j$, dan himpunan seluruh navigasi berhasil yang teramati dinotasikan sebagai $\mathcal{N}$ dengan $N_{\text{nav}} = |\mathcal{N}|$. Maka \textit{mean nav step} didefinisikan sebagai
\begin{equation}
    \text{MeanNavStep}
    = \frac{1}{N_{\text{nav}}}
      \sum_{(j,k) \in \mathcal{N}} L_{j,k}.
    \label{eq:mean_nav_step}
\end{equation}
Persamaan~\ref{eq:mean_nav_step} menangkap rata-rata panjang lintasan pada level satu episode navigasi (satu tujuan objek), sehingga lebih sensitif terhadap kesulitan tiap subtugas dibandingkan metrik pada tingkat tugas penuh.

\vspace{0.5em}

\subsubsection{\textit{Mean task success rate}}
Efektivitas sistem pada tingkat tugas utuh diukur melalui \textit{task success rate}. Misalkan $N_{\text{task}}^{\text{valid}}$ adalah jumlah tugas yang dieksekusi secara valid (misalnya, seluruh tugas dikurangi tugas-tugas pada himpunan $\mathcal{F}$), maka \textit{mean task SR} didefinisikan sebagai
\begin{equation}
    \text{MeanTaskSR}
    = \frac{N_{\text{succ}}}{N_{\text{task}}^{\text{valid}}}.
    \label{eq:mean_task_sr}
\end{equation}
Dengan demikian, Persamaan~\ref{eq:mean_task_sr} menyatakan proporsi tugas yang berhasil diselesaikan sepenuhnya terhadap seluruh tugas yang benar-benar terealisasi di simulator.

\vspace{0.5em}

\subsubsection{\textit{Mean navigation success rate}}
Selain keberhasilan pada tingkat tugas, penting pula untuk menilai seberapa sering agen berhasil mencapai setiap target perantara. Misalkan $N_{\text{nav}}^{\text{succ}}$ adalah jumlah episode navigasi yang selesai dengan sukses dan $N_{\text{nav}}^{\text{total}}$ adalah jumlah seluruh episode navigasi yang seharusnya dijalankan (misalnya berdasarkan panjang daftar objek target pada setiap tugas), maka \textit{mean nav SR} didefinisikan sebagai
\begin{equation}
    \text{MeanNavSR}
    = \frac{N_{\text{nav}}^{\text{succ}}}{N_{\text{nav}}^{\text{total}}}.
    \label{eq:mean_nav_sr}
\end{equation}
Persamaan~\ref{eq:mean_nav_sr} memberikan ukuran granular terhadap stabilitas dan keandalan agen dalam menyelesaikan setiap navigasi parsial, terlepas dari apakah seluruh rangkaian tugas berhasil diselesaikan.

Kombinasi metrik pada Persamaan~\ref{eq:time_cost}--\ref{eq:mean_nav_sr} memungkinkan analisis yang seimbang antara efektivitas (melalui \textit{success rate} dan \textit{fail tasks}) dan efisiensi (melalui \textit{time cost}, \textit{mean task step}, dan \textit{mean nav step}) pada skenario navigasi-tugas multi-objek.

\vspace{0.5em}

\subsection{Metrik \textit{Code-Switching} Indonesia--Inggris}
\label{subsec:cs_metrics}

Pada implementasi evaluasi, seluruh metrik \textit{code-switching} dihitung
berdasarkan penandaan bahasa (\textit{language identification}, LID) pada level token
untuk setiap ujaran. Setiap token diberi salah satu label:
bahasa Indonesia (ID), bahasa Inggris (EN), atau tidak diketahui (UNK),
dengan mengombinasikan kamus kata dan model LID berbasis \textit{langid}
yang dibatasi pada pasangan bahasa Indonesia--Inggris
\parencite{kargaran2024masklid,dhawan2023unified,SrivastavaSingh2021CALCS}.
Tokenisasi menggunakan token berbasis kata (huruf alfabet), dan seluruh token
dinormalisasi (misal \textit{lowercasing}) sebelum proses LID.
Dalam seluruh metrik berikut, hanya token berlabel ID atau EN yang digunakan;
token UNK dikeluarkan dari urutan, sehingga perhitungan dilakukan pada subsekuens ID/EN.
Akibatnya, dua token ID/EN yang semula dipisahkan oleh UNK dapat menjadi bersebelahan
dalam urutan terfilter dan berkontribusi pada perhitungan \textit{switch} maupun \textit{span}.

Misalkan korpus terdiri atas $U$ ujaran.
Untuk ujaran ke-$u$ dengan $u \in \{1,\dots,U\}$, setelah membuang token UNK,
diperoleh urutan label bahasa yang dinotasikan sebagai
$\mathbf{z}^{(u)}$ $= \big(z^{(u)}_1, z^{(u)}_2, \dots, z^{(u)}_{n_u}\big)$
dengan $z^{(u)}_i \in \{\mathrm{ID}, \mathrm{EN}\}$ dan $n_u$ menyatakan panjang urutan terfilter.
Jumlah token per bahasa pada ujaran $u$ masing-masing dinotasikan
$c^{(u)}_{\mathrm{ID}}$ dan $c^{(u)}_{\mathrm{EN}}$.

Pada tingkat korpus, untuk metrik berbasis distribusi bahasa, digunakan total hitungan token 
yang dinyatakan sebagai~\eqref{eq:corpus_counts}:
\begin{equation}
c_{\mathrm{ID}}=\sum_{u=1}^{U} c^{(u)}_{\mathrm{ID}},
\qquad
c_{\mathrm{EN}}=\sum_{u=1}^{U} c^{(u)}_{\mathrm{EN}}.
\label{eq:corpus_counts}
\end{equation}
Untuk metrik berbasis transisi dan struktur lokal (misal \textit{switch} dan \textit{span}),
nilai dihitung pada setiap ujaran $\mathbf{z}^{(u)}$ lalu diringkas (misal rata-rata),
dan transisi tidak dihitung melintasi batas ujaran.

\vspace{0.5em}

\subsubsection{\textit{M-index} dan \textit{I-index}}
Misalkan pada suatu deret token terdapat $c_{\mathrm{EN}}$ token bahasa Inggris dan
$c_{\mathrm{ID}}$ token bahasa Indonesia. Probabilitas empirik tiap bahasa didefinisikan sebagai~\eqref{eq:lang-prob}:
\begin{equation}
p_{\mathrm{EN}}
= \frac{c_{\mathrm{EN}}}{c_{\mathrm{EN}} + c_{\mathrm{ID}}},\qquad
p_{\mathrm{ID}}
= \frac{c_{\mathrm{ID}}}{c_{\mathrm{EN}} + c_{\mathrm{ID}}}.
\label{eq:lang-prob}
\end{equation}
Berdasarkan Persamaan~\ref{eq:lang-prob}, M-index untuk dua bahasa didefinisikan sebagai~\eqref{eq:Mnorm}:
\begin{equation}
M
= \frac{1 - (p_{\mathrm{EN}}^2 + p_{\mathrm{ID}}^2)}
       {p_{\mathrm{EN}}^2 + p_{\mathrm{ID}}^2}.
\label{eq:Mnorm}
\end{equation}
Nilai $M$ meningkat ketika distribusi bahasa semakin seimbang dan menurun ketika satu bahasa
mendominasi \parencite{SrivastavaSingh2021CALCS}.
Dalam evaluasi, $M$ dihitung pada tingkat korpus menggunakan $c_{\mathrm{ID}}$ dan $c_{\mathrm{EN}}$
pada Persamaan~\ref{eq:corpus_counts}.

Untuk I-index, pada ujaran $u$ dengan urutan label $\mathbf{z}^{(u)}$,
I-index mengukur proporsi perpindahan bahasa antar token berurutan yang
dinyatakan pada Persamaan~\ref{eq:Inorm}:
\begin{equation}
I^{(u)}
= \frac{\#\{\,i \mid 2 \le i \le n_u,\ z^{(u)}_i \neq z^{(u)}_{i-1}\,\}}
       {n_u - 1},
\qquad \text{untuk } n_u \ge 2.
\label{eq:Inorm}
\end{equation}
Definisikan himpunan ujaran valid untuk I-index sebagai
$\mathcal{U}_I = \{\,u \in \{1,\dots,U\} \mid n_u \ge 2\,\}$.
Evaluasi melaporkan (i) rata-rata I-index per-ujaran yang dinyatakan sebagai~\eqref{eq:MeanI}:
\begin{equation}
\mathrm{MeanI}
=
\frac{1}{|\mathcal{U}_I|}
\sum_{u \in \mathcal{U}_I} I^{(u)},
\label{eq:MeanI}
\end{equation}
serta (ii) versi berbobot tanpa melintasi batas ujaran yang dinyatakan sebagai~\eqref{eq:Iweighted}:
\begin{equation}
\mathrm{I\text{-}weighted}
=
\frac{\sum_{u \in \mathcal{U}_I}\#\{\,i \mid 2 \le i \le n_u,\ z^{(u)}_i \neq z^{(u)}_{i-1}\,\}}
     {\sum_{u \in \mathcal{U}_I}(n_u - 1)}.
\label{eq:Iweighted}
\end{equation}

\vspace{0.5em}

\subsubsection{\textit{Burstiness} dan \textit{Memory}}
Dari urutan label $\mathbf{z}^{(u)}$ pada setiap ujaran, dibentuk deret \textit{span}
monolingual, yaitu segmen token berturut-turut dengan label bahasa yang sama.
Misalkan panjang span ke-$k$ pada ujaran $u$ adalah $r^{(u)}_k$, sehingga deret panjang span
ditulis $\mathbf{r}^{(u)}=(r^{(u)}_1,r^{(u)}_2,\dots,r^{(u)}_{m_u})$,
dengan $m_u$ adalah jumlah span pada ujaran $u$.

\textit{Burstiness} dan \textit{Memory} pada ujaran $u$ didefinisikan sebagai~\eqref{eq:burst-memory}:
\begin{equation}
\mathrm{Burstiness}^{(u)}
= \frac{\sigma_{\mathbf{r}^{(u)}}-\mu_{\mathbf{r}^{(u)}}}{\sigma_{\mathbf{r}^{(u)}}+\mu_{\mathbf{r}^{(u)}}},
\qquad
\mathrm{Memory}^{(u)}
= \mathrm{corr}\!\left(\{(r^{(u)}_k, r^{(u)}_{k+1})\}_{k=1}^{m_u-1}\right),
\label{eq:burst-memory}
\end{equation}
dengan $\mu_{\mathbf{r}^{(u)}}$ dan $\sigma_{\mathbf{r}^{(u)}}$ masing-masing adalah rata-rata
dan simpangan baku panjang span, serta $\mathrm{corr}(\cdot)$ adalah korelasi Pearson
yang dihitung atas pasangan berurutan $(r^{(u)}_k, r^{(u)}_{k+1})$ untuk $k=1,\dots,m_u-1$.
Jika $m_u < 2$, maka $\mathrm{Memory}^{(u)}$ tidak terdefinisi; jika $\mu_{\mathbf{r}^{(u)}}+\sigma_{\mathbf{r}^{(u)}}=0$
atau $m_u=0$, maka $\mathrm{Burstiness}^{(u)}$ tidak terdefinisi. Dalam kasus tersebut,
nilai metrik pada ujaran tersebut dikecualikan dari peringkasan.

Definisikan $\mathcal{U}_B$ sebagai himpunan ujaran yang memiliki Burstiness terdefinisi dan
$\mathcal{U}_M$ sebagai himpunan ujaran yang memiliki Memory terdefinisi.
Evaluasi melaporkan rata-rata per-ujaran yang dinyatakan sebagai~\eqref{eq:mean_burst_memory}:
\begin{equation}
\begin{aligned}
\mathrm{MeanBurstiness}
&=
\frac{1}{|\mathcal{U}_B|}
\sum_{u \in \mathcal{U}_B} \mathrm{Burstiness}^{(u)},\\
\mathrm{MeanMemory}
&=
\frac{1}{|\mathcal{U}_M|}
\sum_{u \in \mathcal{U}_M} \mathrm{Memory}^{(u)}.
\end{aligned}
\label{eq:mean_burst_memory}
\end{equation}

\vspace{0.5em}

\subsubsection{\textit{Code-Mixing Index} (CMI)}
Code-Mixing Index (CMI) mengukur tingkat pencampuran bahasa dengan mempertimbangkan
proporsi bahasa dominan \parencite{SrivastavaSingh2021CALCS}.
Untuk suatu deret token dengan $c_{\mathrm{EN}}$ dan $c_{\mathrm{ID}}$,
CMI didefinisikan sebagai~\eqref{eq:CMI}:
\begin{equation}
\mathrm{CMI}
= 100 \times
\frac{(c_{\mathrm{EN}} + c_{\mathrm{ID}}) - \max\!\left(c_{\mathrm{EN}},c_{\mathrm{ID}}\right)}
{c_{\mathrm{EN}} + c_{\mathrm{ID}}}.
\label{eq:CMI}
\end{equation}
Evaluasi melaporkan tiga ringkasan:
(i) CMI tingkat korpus (menggunakan $c_{\mathrm{EN}},c_{\mathrm{ID}}$ global pada
Persamaan~\ref{eq:corpus_counts}, seolah seluruh token adalah satu ujaran),
(ii) rata-rata CMI pada semua ujaran yang memiliki setidaknya satu token ID atau EN,
dan (iii) rata-rata CMI pada ujaran yang benar-benar bercampur bahasa (CMI $> 0$).

\vspace{0.5em}

\subsubsection{\textit{Language Entropy} (LE) dan \textit{Span Entropy} (SE)}
Language Entropy (LE) mengukur keragaman distribusi bahasa berdasarkan proporsi token per bahasa.
Dengan probabilitas $p_j$ pada Persamaan~\ref{eq:lang-prob}, LE didefinisikan sebagai~\eqref{eq:LE}:
\begin{equation}
\mathrm{LE}
= -\sum_{j\in\{\mathrm{ID},\mathrm{EN}\}} p_j \log_2 p_j.
\label{eq:LE}
\end{equation}
Dalam evaluasi, LE dihitung pada tingkat korpus menggunakan distribusi token global
pada Persamaan~\ref{eq:corpus_counts}.

Span Entropy (SE) mengukur keragaman panjang span monolingual pada level ujaran.
Untuk ujaran $u$, definisikan fungsi massa peluang panjang span $p^{(u)}(\ell)$ sebagai
proporsi span dalam $\mathbf{r}^{(u)}$ yang memiliki panjang $\ell$ didefinisikan sebagai~\eqref{eq:span_pmf}:
\begin{equation}
p^{(u)}(\ell)
=
\frac{\#\{\,k \mid 1 \le k \le m_u,\ r^{(u)}_k = \ell\,\}}
     {m_u},
\qquad \text{untuk } m_u \ge 1.
\label{eq:span_pmf}
\end{equation}
Dengan demikian, SE untuk ujaran $u$ didefinisikan sebagai~\eqref{eq:SE}:
\begin{equation}
\mathrm{SE}^{(u)}
= -\sum_{\ell:\,p^{(u)}(\ell)>0} p^{(u)}(\ell)\log_2 p^{(u)}(\ell).
\label{eq:SE}
\end{equation}
Nilai SE tinggi menunjukkan distribusi panjang span yang lebih beragam, sedangkan SE rendah
menunjukkan panjang span cenderung seragam \parencite{SrivastavaSingh2021CALCS}.
Jika $m_u=0$ (tidak ada token ID/EN), maka $\mathrm{SE}^{(u)}$ tidak terdefinisi dan dikecualikan dari peringkasan.
Definisikan $\mathcal{U}_{SE}=\{\,u \in \{1,\dots,U\} \mid m_u \ge 1\,\}$, dan evaluasi melaporkan~\eqref{eq:mean_se}:
\begin{equation}
\mathrm{MeanSE}
=
\frac{1}{|\mathcal{U}_{SE}|}
\sum_{u \in \mathcal{U}_{SE}} \mathrm{SE}^{(u)}.
\label{eq:mean_se}
\end{equation}

\vspace{0.5em}

\subsubsection{\textit{T-Index}}
T-Index bertujuan mengukur sejauh mana pilihan \textit{code-switch} suatu kata konsisten
dengan preferensi sistem penerjemah mesin (MT) \parencite{Chi2024TIndex}.
Untuk setiap ujaran $u$, definisikan himpunan indeks titik \textit{switch} yang dinyatakan sebagai~\eqref{eq:switch-set}:
\begin{equation}
S^{(u)} =
\{\, i \mid 2 \le i \le n_u,\ z^{(u)}_i \neq z^{(u)}_{i-1} \,\}.
\label{eq:switch-set}
\end{equation}
Untuk setiap $i \in S^{(u)}$, ambil token (kata) $w^{(u)}_i$ pada posisi $i$,
label bahasa saat ini $z^{(u)}_i$, dan label bahasa sebelumnya $z^{(u)}_{i-1}$.
Kata $w^{(u)}_i$ dianggap sebagai masukan bahasa sumber $z^{(u)}_i$ dan diterjemahkan
ke bahasa target $z^{(u)}_{i-1}$ menggunakan model MarianMT bilingual sesuai arah
(ID$\rightarrow$EN atau EN$\rightarrow$ID).
Misalkan $y^{(u)}_i = (y_{i,1},\dots,y_{i,|y^{(u)}_i|})$ adalah deret token keluaran teratas.
Model MT menyediakan probabilitas bersyarat per token
$P_{\mathrm{MT}}(y_{i,t} \mid y_{i,<t}, w^{(u)}_i)$.
Dalam metrik ini, $\log(\cdot)$ menyatakan logaritma natural.
Skor MT untuk titik \textit{switch} didefinisikan sebagai rata-rata log-probabilitas token yang dihasilkan sebagaimana pada Persamaan~\ref{eq:tindex-score}:
\begin{equation}
\mathrm{score}_{\mathrm{MT}}\!\left(w^{(u)}_i, z^{(u)}_i \rightarrow z^{(u)}_{i-1}\right)
= \frac{1}{|y^{(u)}_i|}
  \sum_{t=1}^{|y^{(u)}_i|}
  \log P_{\mathrm{MT}}\!\left(y_{i,t} \mid y_{i,<t}, w^{(u)}_i\right).
\label{eq:tindex-score}
\end{equation}
T-Index tingkat korpus didefinisikan sebagai rata-rata skor MT pada seluruh titik switch yang valid:
\begin{equation}
\mathrm{TIndex}
= \frac{1}{|S'|}
  \sum_{(u,i) \in S'}
  \mathrm{score}_{\mathrm{MT}}\!\left(w^{(u)}_i, z^{(u)}_i \rightarrow z^{(u)}_{i-1}\right),
\label{eq:TIndex}
\end{equation}
dengan $S' \subseteq \bigcup_{u=1}^{U} \{(u,i): i \in S^{(u)}\}$ adalah himpunan titik switch yang
berhasil dihitung skornya (misalnya, kasus ketika MT gagal atau skor tidak terdefinisi diabaikan).
Normalisasi terhadap panjang keluaran $|y^{(u)}_i|$ mengakomodasi efek panjang terjemahan secara implisit.
Nilai T-Index yang lebih tinggi mengindikasikan bahwa pilihan \textit{code-switch} pada titik-titik tersebut
lebih selaras dengan preferensi sistem MT dalam memetakan kata ke bahasa yang berseberangan.
\vspace{0.5em}

\section{Penelitian Terdahulu}
Pada beberapa tahun terakhir, penelitian VLN bergerak ke dua sumbu utama. Sumbu pertama berfokus pada \textit{long-horizon} \textit{embodied} AI yang memperpanjang urutan tindakan dan penalaran lintas sub-tugas, didorong oleh platform serta \textit{benchmark} baru dan integrasi LLM. \textit{Benchmark} dan platform \textit{embodied} seperti Habitat, VLN-CE, dan HM3D mendorong skenario yang lebih realistis dan berkelanjutan \parencite{savva2019habitat,wijmans2020vlnce,HM3D2021}, sedangkan kerangka \textit{long-horizon} terbaru menegaskan kebutuhan perencanaan berjenjang dan penguraian sub-tugas \parencite{Song2025}. Sumbu kedua menekankan perluasan kemampuan bahasa dari monolingual ke multilingual, dengan dataset seperti RxR yang menegaskan peran keragaman bahasa dalam VLN \parencite{ku2020rxr}.

Secara paralel, LLM dimanfaatkan untuk menghasilkan instruksi sintetis dan meningkatkan kemampuan penalaran agen, selaras dengan praktik \textit{self-instruction} dan kemajuan model fondasi \parencite{wang2023selfinstruct}. Dalam ranah \textit{code-switching}, kemajuan pada TTS dan ASR, termasuk untuk pasangan Indonesia–Inggris, menunjukkan kesiapan ekosistem data dan teknik untuk membangun instruksi \textit{code-switching} yang natural \parencite{Handoyo2024,dhawan2023unified}. Meski banyak kemajuan, masih terdapat celah berupa ketiadaan dataset VLN \textit{long-horizon} dengan instruksi \textit{code-switching} Indonesia–Inggris yang dibangkitkan secara sistematis oleh LLM, yang menjadi kontribusi utama penelitian ini \parencite{gu2022vision,Zhang2024}. Untuk rangkuman penelitian terkait dan posisi kontribusi, lihat Tabel~\ref{tab:rlwork}.

\begin{sidewaystable}[t]
\centering
\caption{Ringkasan Penelitian Terkait}
\label{tab:rlwork}
{\fontsize{10pt}{12pt}\selectfont
\setlength{\tabcolsep}{4pt}
\newcolumntype{Y}{>{\raggedright\arraybackslash}X}
\begin{tabularx}{\textwidth}{|p{3.2cm}|p{3.2cm}|p{3.0cm}|p{3.0cm}|p{2.0cm}|Y|}
\hline
\textbf{Karya (Tahun)} & \textbf{Fokus/Tugas} & \textbf{Dataset/Platform} & \textbf{Peran LLM} & \textbf{Bahasa} & \textbf{Catatan Relevansi} \\
\hline
R2R (2018) \parencite{anderson2018r2r} & VLN instruksi natural (dasar) & Matterport3D / R2R & — (dataset) & EN & Titik awal VLN; jalur pendek berbasis instruksi. \\
\hline
RxR (2020) \parencite{ku2020rxr} & VLN multibahasa \& alignment kata–waktu & Matterport3D / RxR & — (dataset) & Multi & Menegaskan pentingnya dukungan multilingual. \\
\hline
Habitat (2019) \parencite{savva2019habitat} & Platform simulasi \textit{embodied} AI & Habitat & — (platform) & — & Infrastruktur simulasi berkecepatan tinggi. \\
\hline
VLN-CE (2020) \parencite{wijmans2020vlnce} & VLN di lingkungan kontinu & VLN-CE & — (benchmark) & EN & Lebih realistis; pergerakan kontinu. \\
\hline
HM3D (2021) \parencite{HM3D2021} & Dunia 3D skala besar untuk \textit{embodied} & HM3D & — (dataset) & — & Variasi struktur/tekstur, untuk generalisasi. \\
\hline
Touchdown (2019) \parencite{chen2019touchdown} & Navigasi bahasa di peta jalan & Street View & — (dataset) & EN & Instruksi panjang di lingkungan kota. \\
\hline
CVDN (2020) \parencite{thomason2020cvdn} & VLN dengan dialog interaktif & CVDN & — (dataset) & EN & Menambah dimensi dialog pada VLN. \\
\hline
REVERIE (2020) \parencite{qi2020reverie} & \textit{Remote object grounding} + nav & REVERIE & — (dataset) & EN & Memadukan grounding objek dengan navigasi. \\
\hline
Long-Horizon VLN (2025) \parencite{Song2025} & \textit{Long-horizon} multi-tahap & (platform/benchmark LH-VLN) & Pemanfaatan LLM untuk perencanaan & EN & Menekankan decomposisi sub-tugas dan evaluasi jangka panjang. \\
\hline
Self-Instruct (2023) \parencite{wang2023selfinstruct} & Pembangkitan instruksi sintetis & — & LLM sebagai \textit{instructor} & — & Prinsip pembuatan instruksi sintetis untuk skala data. \\
\hline
GPT-4 Report (2023) \parencite{OpenAI2023} & Model fondasi multimodal & — & Kapabilitas penalaran/komposisi & — & Landasan integrasi LLM pada agen VLN. \\
\hline
TTS ID–EN (2024) \parencite{Handoyo2024} & Sintesis ujaran \textit{code-switching} & STEN-TTS & LID-BERT + TTS multibahasa & ID/EN & Mendukung naturalitas instruksi \textit{code-switching}. \\
\hline
ASR CS (2023) \parencite{dhawan2023unified} & Pengenalan ujaran \textit{code-switching} & Beragam korpus CS & Arsitektur terpadu (ASR+LID) & Multi & Indikasi kematangan teknik CS untuk pipeline bahasa. \\
\hline
\end{tabularx}
}
\end{sidewaystable}
% \chapter{METODOLOGI PENELITIAN}
\label{Bab3}

\section{Waktu dan Tempat Penelitian}
Penelitian ini akan dilaksanakan dari bulan Agustus 2025 hingga April 2026. 
Kegiatan penelitian dilakukan secara utama di 
Human-Centered Robotics and Automation Laboratory (Hucenrotia Lab), EE806, 
National Yang Ming Chiao Tung University yang berlokasi di 
No.~1001 Daxue Rd., East Dist., Hsinchu City 300, Taiwan, 
serta didukung oleh kegiatan penulisan dan diskusi akademik di 
Gedung Kuliah Bersama Kampus C Universitas Airlangga, 
Jalan Dr.~Ir.~H.~Soekarno, Kecamatan Mulyorejo, Surabaya, Indonesia.

\vspace{0.5em}

\section{Alat dan Bahan Penelitian}
\label{sec:alat-bahan}

\vspace{0.5em}

\subsection{Alat Penelitian}
Alat yang digunakan dalam penelitian ini meliputi perangkat keras dan perangkat lunak 
sebagai berikut.
\vspace{0.5em}
% \begin{enumerate}[itemsep=0pt,parsep=0pt,topsep=0pt,label=\alph*.]

\subsubsection{Komputer/Laptop Penelitian}
Seluruh eksperimen dijalankan pada sebuah mesin \textit{remote desktop} yang 
terhubung dengan \textit{server} komputasi berperforma tinggi. Spesifikasi lengkap 
berdasarkan hasil pengukuran sistem adalah sebagai berikut.
\begin{enumerate}[label=\alph*., left=0pt, labelsep=0.33cm, itemsep=0pt, topsep=0pt, partopsep=0pt, parsep=0pt]
    \item Sistem operasi: Ubuntu 24.04.3 LTS dengan kernel 6.8.0-87-generic dan arsitektur x86\_64.
    \item Prosesor: Intel(R) Core(TM) i9-14900K dengan 32 CPU logis (24 \textit{core}, 2 \textit{thread} per \textit{core}).
    \item Memori utama (RAM): 125~GB.
    \item GPU: NVIDIA RTX A5000 dengan kapasitas memori 24{,}6~GiB.
    \item Media penyimpanan:
        \begin{enumerate}[label=\arabic*., leftmargin=*, labelsep=0.33cm, itemsep=0pt, parsep=0pt, topsep=0pt]
            \item Satu buah \textit{HDD} SATA berkapasitas 14{,}6~TB yang dikonfigurasi sebagai volume utama sistem.
            \item Satu buah \textit{SSD NVMe} berkapasitas 1{,}8~TB untuk \textit{cache} dan berkas kerja eksperimen.
        \end{enumerate}
    \item Lingkungan eksekusi: sesi \textit{remote desktop} dan 
          terminal SSH dengan \textit{virtual environment} Python.
\end{enumerate}

\vspace{0.5em}

\subsubsection{Simulator Habitat}
Penelitian ini menggunakan \textit{simulator} Habitat \parencite{szot2021habitat2} yang menyediakan 
lingkungan virtual fotorealistis untuk mengeksekusi \textit{episode} navigasi 
\textit{long-horizon}. Pustaka habitat-sim digunakan untuk:
\begin{enumerate}[label=\alph*., left=0pt, labelsep=0.33cm, itemsep=0pt, topsep=0pt, partopsep=0pt, parsep=0pt]
    \item Menginisialisasi agen robotik dengan sensor RGB-D dan parameter gerak.
    \item Mengakses navmesh dan modul \textit{Greedy Geodesic Follower} 
          sebagai \textit{planner} \textit{ground-truth} lintasan geodesik terpendek.
    \item Merekam trajektori agen, termasuk posisi, orientasi, dan aksi pada 
          setiap \textit{timestep}.
\end{enumerate}

\vspace{0.5em}

\subsubsection{Pustaka Pemrograman dan Lingkungan Pengembangan}
Implementasi \textit{pipeline} menggunakan bahasa pemrograman Python 
(versi~3.11). Pustaka utama yang digunakan antara lain:
\begin{enumerate}[label=\alph*., left=0pt, labelsep=0.33cm, itemsep=0pt, topsep=0pt, partopsep=0pt, parsep=0pt]
    \item PyTorch untuk pemanggilan LLM 
          dan model pendukung lainnya.
    \item transformers untuk antarmuka model Qwen dan 
          model MarianMT.
    \item numpy, pandas, dan scipy 
          untuk pemrosesan numerik dan statistik.
    \item langid untuk deteksi bahasa otomatis tingkat token.
\end{enumerate}

\vspace{0.5em}

\subsubsection{\textit{Large Language Model} (LLM)}
Penelitian ini menggunakan keluarga \textit{Large Language Model} 
Qwen-Instruct sebagai komponen utama \textit{in-the-loop}. 
Secara khusus, model Qwen2.5-7B-Instruct 
digunakan dalam dua peran:
\begin{enumerate}[label=\alph*., left=0pt, labelsep=0.33cm, itemsep=0pt, topsep=0pt, partopsep=0pt, parsep=0pt]
    \item LLM pembangkit tugas 
          $\mathcal{G}_{\text{task}}$, yang menghasilkan konfigurasi 
          tugas LH-VLN (instruksi global dan daftar subtugas) 
          berdasarkan deskripsi \textit{scene} dan konfigurasi robot.
    \item LLM pembangkit instruksi 
          $\mathcal{G}_{\text{instr}}$, yang menghasilkan instruksi 
          navigasi \textit{long-horizon} bilingual 
          (Indonesia--Inggris) berdasarkan trajektori dan 
          konteks visual yang telah disegmentasi.
\end{enumerate}

Pemilihan Qwen didasarkan pada beberapa pertimbangan:
\begin{enumerate}[label=\alph*., left=0pt, labelsep=0.33cm, itemsep=0pt, topsep=0pt, partopsep=0pt, parsep=0pt]
    \item Mendukung banyak bahasa dan memiliki performa yang baik 
          pada bahasa Indonesia dan Inggris, sehingga sesuai 
          untuk skenario \textit{code-switching} \parencite{koto23llm,nusacrowd, nusax}.
    \item Tersedia sebagai model \textit{open-weight}, sehingga 
          eksperimen dapat dijalankan secara lokal tanpa ketergantungan 
          pada layanan \textit{cloud} \parencite{qwen25}.
    \item Ukuran model yang sedang (\textit{medium-sized}) memungkinkan 
          waktu inferensi yang realistis pada GPU tunggal \parencite{qlora}.
\end{enumerate}

\vspace{0.5em}

\subsubsection{Model Pendukung Lain}
Selain LLM utama, penelitian ini memanfaatkan beberapa model tambahan:
\begin{enumerate}[label=\alph*., left=0pt, labelsep=0.33cm, itemsep=0pt, topsep=0pt, partopsep=0pt, parsep=0pt]
    \item Model MarianMT (id--en dan en--id) untuk menghitung 
          \textit{T-index}, yaitu skor kelayakan kata di titik 
          \textit{code-switching} berdasarkan log-probabilitas terjemahan.
    \item langid dan kamus kata bahasa Indonesia serta 
          bahasa Inggris untuk penandaan bahasa (\textit{Language 
          Identification}, LID) di tingkat token.
\end{enumerate}
\vspace{0.5em}

% \end{enumerate}

\subsection{Bahan Penelitian}
\label{lab:bahan-penelitian}

\begin{figure}[ht]
\centering

\begin{subfigure}{0.32\textwidth}
    \centering
    \includegraphics[width=\linewidth]{images/hm3d/1.png}
\end{subfigure}
\begin{subfigure}{0.32\textwidth}
    \centering
    \includegraphics[width=\linewidth]{images/hm3d/2.png}
\end{subfigure}
\begin{subfigure}{0.32\textwidth}
    \centering
    \includegraphics[width=\linewidth]{images/hm3d/3.png}
\end{subfigure}

\begin{subfigure}{0.32\textwidth}
    \centering
    \includegraphics[width=\linewidth]{images/hm3d/4.png}
\end{subfigure}
\begin{subfigure}{0.32\textwidth}
    \centering
    \includegraphics[width=\linewidth]{images/hm3d/5.png}
\end{subfigure}
\begin{subfigure}{0.32\textwidth}
    \centering
    \includegraphics[width=\linewidth]{images/hm3d/6.png}
\end{subfigure}

\begin{subfigure}{0.32\textwidth}
    \centering
    \includegraphics[width=\linewidth]{images/hm3d/7.png}
\end{subfigure}
\begin{subfigure}{0.32\textwidth}
    \centering
    \includegraphics[width=\linewidth]{images/hm3d/8.png}
\end{subfigure}
\begin{subfigure}{0.32\textwidth}
    \centering
    \includegraphics[width=\linewidth]{images/hm3d/9.png}
\end{subfigure}

\caption{Contoh \textit{Scene} dari HM3D \textit{Dataset}.}
\label{fig:hm3d_dollhouse}
\end{figure}

Bahan penelitian yang digunakan berupa kumpulan \textit{scene} tiga dimensi 
dari HM3D Dataset \parencite{HM3D2021}. Dalam penelitian ini, digunakan sebanyak 181 \textit{scene} 
yang masing-masing merepresentasikan tata ruang interior yang realistis, 
lengkap dengan informasi geometris, furnitur, dan objek-objek rumah tangga. 
Pemilihan 181 \textit{scene} tersebut dilakukan secara terarah agar mencakup 
variasi tata letak (seperti apartemen, rumah, dan kantor), tingkat 
kompleksitas konektivitas antar-ruangan, serta keberagaman kategori objek 
yang berpotensi menjadi target navigasi. Gambar~\ref{fig:hm3d_dollhouse} memperlihatkan beberapa contoh tampilan 
\textit{dollhouse} \textit{scene} HM3D yang digunakan dalam penelitian.
\vspace{0.5em}

\section{Prosedur Penelitian}
\label{sec:prosedur-penelitian}

\vspace{0.5em}

\subsection{Model Pendekatan Penelitian}
Penelitian ini termasuk dalam kategori penelitian rekayasa 
(\textit{engineering research}) dengan fokus pada perancangan dan evaluasi 
sebuah \textit{pipeline} generatif untuk menghasilkan \textit{episode} 
\textit{Long-Horizon Vision--Language Navigation} (LH-VLN) bilingual, 
yaitu instruksi navigasi \textit{code-switching} Indonesia--Inggris 
yang terhubung dengan trajektori navigasi di simulator. 
Perancangan \textit{pipeline} ini diinspirasi oleh kerangka (LH-VLN) dan mekanisme
pembangkitan \textit{episode} terstruktur (NavGen) \parencite{Song2025}.

Secara konseptual, cara kerja penelitian mengikuti alur:
(1) perumusan kebutuhan dan perancangan arsitektur \textit{pipeline} 
serta metrik evaluasi,
(2) implementasi dan integrasi komponen perangkat lunak 
(\textit{simulator} Habitat, LLM, dan modul evaluasi),
(3) pembangkitan dan kurasi dataset \textit{episode} LH-VLN bilingual 
melalui \textit{pipeline} generatif,
(4) analisis kuantitatif dan kualitatif terhadap dataset,
serta (5) penarikan kesimpulan berdasarkan rumusan masalah 
pada Bab~\ref{Bab1}.

% Pada eksekusi nyata, setiap \textit{episode} navigasi dibatasi oleh 
% parameter max\_step pada simulator. Dalam penelitian ini 
% batas langkah maksimum per navigasi ditetapkan pada 500 langkah, 
% dengan batas total langkah per tugas sekitar 2.000 langkah 
% atau batas waktu komputasi sekitar 60 detik, mana yang tercapai lebih dahulu. 
% Nilai ini dikalibrasi kembali pada tahap uji coba awal 
% agar total waktu eksekusi seluruh eksperimen tetap realistis 
% dengan sumber daya komputasi yang tersedia.

\vspace{0.5em}

\subsection{Alur Penelitian}
Alur kerja penelitian secara keseluruhan digambarkan pada 
Gambar~\ref{fig:pipeline-bilingual} yang memuat tahapan utama 
mulai dari persiapan hingga penyusunan kesimpulan. 
Secara naratif, alur pada Gambar~\ref{fig:pipeline-bilingual} 
dapat dijelaskan sebagai berikut.
Tahap pertama adalah persiapan dan studi literatur, yakni 
pengumpulan dan pengkajian referensi terkait VLN, LH-VLN, 
\textit{code-switching}, serta penggunaan LLM dalam \textit{robotics}.
Tahap kedua adalah perumusan kebutuhan dan perancangan 
\textit{pipeline} serta metrik evaluasi yang akan digunakan 
(lihat Subbab~\ref{sec:metrik-evaluasi}).
\begin{figure}[H]
    \centering
    \includegraphics[
        width=1\textwidth,
        trim=60 105 15 110, %kiri bawah kanan atas
        clip
    ]{images/alur-kerja-penelitian2.png}
    \caption{\textit{Flowchart} Alur Kerja Penelitian yang Diusulkan.}
    \label{fig:pipeline-bilingual}
\end{figure}

Tahap ketiga adalah implementasi dan uji coba awal, 
yakni merealisasikan komponen \textit{pipeline} di atas simulator Habitat, 
menyusun \textit{prompt} Qwen untuk tugas dan instruksi, 
serta menguji eksekusi beberapa \textit{episode} percobaan untuk mengkalibrasi 
parameter (batas langkah, jumlah sampel, ambang rasio bahasa, dan lain-lain).

Tahap keempat adalah pengumpulan data melalui eksekusi 
\textit{pipeline} generatif untuk menghasilkan kumpulan tugas dan instruksi 
navigasi bilingual. Tahap kelima adalah analisis dan evaluasi 
terhadap dataset tersebut baik secara kuantitatif maupun kualitatif. 
Tahap terakhir adalah penyusunan hasil dan penarikan kesimpulan
yang dikaitkan kembali dengan rumusan masalah dan tujuan penelitian.

\vspace{0.5em}

\subsection{Desain Ukuran dan Distribusi \textit{Dataset}}
\label{subsec:desain-dataset}

Agar \textit{dataset} yang dihasilkan memiliki cakupan yang cukup kaya namun tetap realistis dikerjakan dengan sumber daya komputasi yang tersedia, ukuran dan distribusi \textit{dataset} dirancang secara eksplisit pada beberapa level: level \textit{scene}, level tugas, level \textit{episode} sukses, serta distribusi jumlah subtugas dan panjang episode. Subbab ini menjabarkan target kuantitatif dan pertimbangannya. Desain ini dapat diringkas pada Gambar~\ref{fig:skema-dataset}. Adapun contoh data yang dihasilkan maupun yang digunakan pada penelitian ini dapat dilihat pada Lampiran~2.1 hingga Lampiran~2.4.
\begin{figure}[H]
    \centering
    \includegraphics[
        width=1\textwidth,
        trim=20 175 20 175, %kiri bawah kanan atas
        clip
    ]{images/skema-dataset.pdf}
    \caption{Skema Desain Ukuran dan Distribusi \textit{Dataset} pada Berbagai Level: 
    \textit{Scene}, Tugas, \textit{Episode}, dan Segmen.}
    \label{fig:skema-dataset}
\end{figure}
\vspace{0.5em}

\subsubsection{Jumlah \textit{Scene}}

Penelitian ini memanfaatkan 181 \textit{scene} rumah tangga dari HM3D \parencite{HM3D2021} yang telah disinggung pada Subbab~\ref{lab:bahan-penelitian}. Setiap \textit{scene} direpresentasikan sebagai graf navigasi tiga dimensi yang dapat dieksplorasi oleh agen. Kumpulan \textit{scene} ini dituliskan ke dalam bentuk metadata yang dapat dilihat pada Lampiran~2.1. Penggunaan 181 \textit{scene} dipilih dengan pertimbangan bahwa semakin banyak \textit{scene} yang digunakan, semakin beragam pula tata letak ruangan, tipe furnitur, dan konfigurasi objek yang muncul dalam trajektori. Selain itu, instruksi navigasi yang dihasilkan diharapkan mampu melakukan generalisasi dengan baik, sehingga tidak bergantung pada satu atau dua rumah tertentu, melainkan terdistribusi pada banyak lingkungan yang berbeda.

\vspace{0.5em}

\subsubsection{Jumlah Tugas}

Untuk setiap \textit{scene}, generator tugas $G_{\text{task}}$ dirancang untuk membangkitkan rata-rata lima tugas LH-VLN yang berbeda. Dengan demikian, target jumlah tugas total adalah:
\begin{equation*}
    181 \text{ \textit{scene}} \times 5 \text{ tugas/\textit{scene}} = 905 \text{ tugas}.
\end{equation*}
Lima tugas per \textit{scene} dipilih sebagai kompromi antara cakupan yang cukup luas, di mana setiap \textit{scene} muncul dalam beberapa konfigurasi tugas yang berbeda, serta kemampuan dan pemanggilan LLM yang tetap terjangkau untuk satu mesin GPU selama periode penelitian.

% Tugas yang terbukti tidak dapat dieksekusi (misalnya karena target tidak dapat dijangkau oleh agen) akan dibuang dan, jika perlu, digantikan oleh tugas baru pada \textit{scene} yang sama hingga mendekati target sekitar 900 tugas.

\vspace{0.5em}

\subsubsection{Jumlah \textit{Episode} Sukses}

Setiap tugas dieksekusi oleh agen dalam simulator Habitat untuk menghasilkan trajektori. Satu eksekusi lengkap terhadap sebuah tugas disebut satu \textit{episode}. Karena tidak semua eksekusi akan berhasil (gagal menemukan jalur, melebihi batas langkah, atau melampaui batas waktu), penelitian ini tidak mensyaratkan bahwa setiap tugas harus menghasilkan tepat satu \textit{episode} sukses.

Desain yang digunakan adalah sebagai berikut:
\begin{enumerate}[label=\alph*., left=0pt, labelsep=0.33cm, itemsep=0pt, topsep=0pt, partopsep=0pt, parsep=0pt]
    \item Untuk setiap tugas, agen diperbolehkan beberapa percobaan eksekusi hingga batas maksimum tertentu.
    \item \textit{Episode} yang berhasil menyelesaikan seluruh subtugas dalam batas langkah dan waktu disebut \textit{episode} sukses dan dimasukkan ke himpunan $D_{\text{traj}}$.
    \item \textit{Episode} gagal hanya digunakan untuk menghitung statistik kegagalan, tetapi tidak diteruskan ke tahap segmentasi dan pembangkitan instruksi.
\end{enumerate}

Dengan konfigurasi tersebut, penelitian ini menargetkan sekitar 500 \textit{episode} sukses yang akhirnya membentuk \textit{dataset} utama $D^*$. Angka ini dianggap realistis dengan mempertimbangkan adanya sebagian tugas yang tidak dapat dieksekusi atau sulit diselesaikan oleh agen, serta keterbatasan waktu komputasi yang tersedia.

\vspace{0.5em}

\subsubsection{Sebaran Jumlah Subtugas per Tugas}

Generator tugas $G_{\text{task}}$ dibatasi untuk menghasilkan tugas dengan 2 sampai 6 subtugas (aksi simbolik seperti \textit{Move}, \textit{Grab}, dan \textit{Release}). Rentang ini dipilih untuk menyeimbangkan dua kebutuhan: (i) tugas cukup panjang untuk merepresentasikan skenario \textit{long-horizon}, tetapi (ii) tidak terlalu panjang sehingga tingkat kegagalan eksekusi meningkat dan biaya komputasi per episode menjadi tidak efisien.

Untuk menjaga variasi kompleksitas sekaligus memastikan setiap kategori memiliki jumlah sampel yang memadai untuk analisis, target sebaran jumlah subtugas per tugas ditetapkan sebagai berikut:
\begin{enumerate}[label=\alph*., left=0pt, labelsep=0.33cm, itemsep=0pt, topsep=0pt, partopsep=0pt, parsep=0pt]
    \item Sekitar 70\% tugas memiliki 3--4 subtugas (kasus \textit{typical} yang diharapkan menjadi mayoritas sehingga metrik evaluasi stabil),
    \item Sekitar 25\% tugas memiliki 2 subtugas (tugas relatif sederhana sebagai pembanding dan untuk menjaga keberagaman tingkat kesulitan),
    \item Sekitar 5\% tugas memiliki 5--6 subtugas (tugas lebih kompleks/\textit{long-horizon}, namun dibatasi agar tidak mendominasi karena lebih rentan gagal dan lebih mahal secara komputasi).
\end{enumerate}

\noindent Sebaran ini direalisasikan melalui pengendalian generasi dan kurasi: (1) sebelum menghasilkan tugas, sistem menentukan target jumlah subtugas $k \in \{2,3,4,5,6\}$ berdasarkan proporsi di atas, lalu (2) \textit{prompt} pada $G_{\text{task}}$ menginstruksikan LLM untuk menghasilkan tepat $k$ aksi simbolik. Keluaran kemudian divalidasi; apabila jumlah subtugas tidak sesuai atau melanggar aturan format, proses generasi diulang (\textit{resampling}) hingga komposisi dataset mendekati target sebaran dalam toleransi yang ditentukan.

\vspace{0.5em}

\subsubsection{Panjang \textit{Episode} dan Jumlah Segmen}

Panjang \textit{episode} diukur dalam jumlah langkah simulasi (\textit{time step}) yang diambil agen dari awal hingga seluruh subtugas terselesaikan atau batas langkah tercapai. Untuk menjamin karakter \textit{long-horizon}, perancangan \textit{episode} mengikuti prinsip berikut:
\begin{enumerate}[label=\alph*., left=0pt, labelsep=0.33cm, itemsep=0pt, topsep=0pt, partopsep=0pt, parsep=0pt]
    \item \textit{Episode} dibatasi oleh maksimum jumlah langkah 500 per \textit{episode}.
    \item Melalui desain tugas multi-subtugas dan pemilihan target yang memiliki jarak geodesik cukup besar, diharapkan panjang \textit{episode} sukses rata-rata berada di kisaran 150-300 langkah, dengan sebagian \textit{episode} yang lebih kompleks melebihi 300 langkah namun tetap berada di bawah batas 500 langkah.
    \item Secara praktis, targetnya adalah setidaknya sepertiga \textit{episode} sukses memiliki panjang trajektori di atas 200 langkah sehingga sifat \textit{long-horizon} benar-benar tercermin di dalam \textit{dataset}.
\end{enumerate}

\textit{Episode} sukses kemudian dipecah oleh modul segmentasi $S_{\text{seg}}$ menjadi beberapa segmen berdasarkan perubahan target lokal dan pola aksi. Dengan rata-rata 3-4 subtugas per tugas, setiap \textit{episode} diharapkan menghasilkan sekitar 4-6 segmen yang berbeda (misalnya 1--2 segmen per subtugas, bergantung pada dinamika navigasi). 
Jika target \textit{episode} sukses sekitar 500, maka jumlah segmen yang dihasilkan berada pada kisaran
\begin{equation*}
    500 \times 4 = 2{.}000 \quad \text{sampai} \quad 500 \times 6 = 3{.}000 \text{ segmen}.
\end{equation*}
Dengan demikian, \textit{dataset} akhir $D^*$ ditargetkan memuat sekitar 2{.}000-3{.}000 segmen beserta instruksi navigasi bilingual yang menyertainya. Jumlah segmen pada skala ini dinilai cukup untuk:
\begin{enumerate}[label=\alph*., left=0pt, labelsep=0.33cm, itemsep=0pt, topsep=0pt, partopsep=0pt, parsep=0pt]
    \item Melakukan analisis statistik \textit{code-switching} yang stabil di tingkat korpus, dan
    \item Mengevaluasi variasi pola bahasa lintas \textit{scene}, lintas konfigurasi robot, dan lintas tingkat kompleksitas tugas.
\end{enumerate}

\vspace{0.5em}

\subsection{Teknik Pengumpulan Data}

Teknik pengumpulan data pada penelitian ini berupa eksperimen komputasional yang dijalankan di atas simulator Habitat. Seluruh data diperoleh sebagai keluaran dari tiga komponen utama: \textit{pipeline} generatif \textit{LLM in-the-loop}, modul simulasi navigasi, dan modul pemrosesan trajektori. Secara ringkas, alur pengumpulan data ditunjukkan pada Gambar~\ref{fig:pipeline-dc}, sedangkan ringkasan jenis data yang dikumpulkan dirangkum pada Tabel~\ref{tab:jenis-data}. 

\begin{table}[ht]
    \centering
    {\fontsize{10}{12}\selectfont
    \renewcommand{\arraystretch}{1.15}

    \caption{Ringkasan Jenis Data yang Ada Pada \textit{Pipeline}.}
    \label{tab:jenis-data}

    \begin{tabularx}{\textwidth}{X X X}
        \toprule
        \textbf{Jenis data} & \textbf{Digunakan pada} & \textbf{Contoh/format} \\
        \midrule
        Metadata \textit{scene} dan objek HM3D
        & Perancangan tugas dan pemilihan target
        & ID \textit{scene}, kategori objek, region ruangan \\

        Deskripsi tugas LH-VLN
        & Eksekusi simulasi dan analisis tugas
        & Berkas konfigurasi berisi \textit{task instruction}, daftar subtugas, dan target \\

        Log trajektori navigasi
        & Perhitungan metrik navigasi
        & Urutan posisi $(x,y,z)$, orientasi, aksi, waktu/\textit{timestep}, status keberhasilan, dan referensi citra kamera \\

        Segmen trajektori dan konteks visual
        & Pembangkit instruksi
        & Potongan trajektori dengan indeks langkah $[u,v]$, target lokal, urutan aksi, dan \textit{scene tags} \\

        Instruksi navigasi bilingual + LID
        & Analisis \textit{code-switching} dan evaluasi bahasa
        & Teks instruksi + label bahasa per token (definisi token konsisten; lihat uraian LID) \\
        \bottomrule
    \end{tabularx}
    }
\end{table}

Pada tahap awal, peneliti menetapkan konfigurasi lingkungan dan parameter eksperimen. Secara formal didefinisikan himpunan \textit{scene} HM3D yang digunakan $\mathcal{S}$, himpunan jenis robot $\mathcal{R}$, serta parameter skala data yang mengaitkan proses pembangkitan dan hasil akhir sebagai berikut:
\begin{itemize}[left=0pt, labelsep=0.33cm, itemsep=0pt, topsep=0pt]
    \item $N \in \mathbb{N}$: target jumlah entri dataset akhir $\mathcal{D}^*$ \textit{setelah} seluruh tahap penyaringan (validasi konfigurasi, keterjangkauan geodesik, serta seleksi \textit{success episode}). Dengan kata lain, $N$ adalah target ukuran hasil akhir yang diinginkan.
    \item $L \in \mathbb{N}$: jumlah percobaan pembangkitan konfigurasi tugas (\textit{generation attempts}) menggunakan LLM tugas $\mathcal{G}_{\mathrm{task}}$. Tahap ini menghasilkan himpunan konfigurasi valid $\mathcal{D}_{\mathrm{config}}$ dengan ukuran $|\mathcal{D}_{\mathrm{config}}|\leq L$.
    \item $M \in \mathbb{N}$: jumlah percobaan eksekusi simulasi per tugas (\textit{episode attempts}) pada simulator, misalnya untuk variasi \textit{seed} atau kondisi awal yang ditentukan pada konfigurasi eksperimen. Tahap ini menghasilkan himpunan trajektori sukses $\mathcal{D}_{\mathrm{traj}}$ dengan ukuran yang bergantung pada tingkat keberhasilan eksekusi.
\end{itemize}
Selain itu ditentukan pula batas langkah maksimum $K_{\max}$, batas waktu eksekusi (\textit{timeout}), jarak keberhasilan $d_{\mathrm{succ}}$, serta parameter lain terkait jumlah percobaan per tugas. Seluruh parameter disimpan dalam berkas konfigurasi dan diberikan sebagai argumen saat pemanggilan program.

Secara konseptual, alur ukuran data mengikuti rantai berikut:
\[
L \;\rightarrow\; |\mathcal{D}_{\mathrm{config}}| \;\rightarrow\; |\mathcal{D}_{\mathrm{traj}}| \;\rightarrow\; |\mathcal{E}| \;\rightarrow\; |\mathcal{D}^*|\approx N,
\]
dengan $\mathcal{E}$ menyatakan himpunan \textit{episode} beserta instruksi bilingual yang dihasilkan, dan $\mathcal{D}^*$ menyatakan struktur dataset akhir yang menggabungkan seluruh komponen (lihat bagian penutup subbab ini).

\begin{figure}[H]
\centering
\includegraphics[width=0.9\textwidth,
    trim={0mm 0mm 0mm 0mm},
    clip]{images/flowchart-pengumpulan-data.pdf}
\caption{\textit{Flowchart} Alur Pengumpulan Data Melalui \textit{Pipeline}.}
\label{fig:pipeline-dc}
\end{figure}

% \paragraph{(1) Pembangkitan konfigurasi tugas LH-VLN dengan $\mathcal{G}_{\mathrm{task}}$.}
Langkah pertama pengumpulan data adalah membangkitkan konfigurasi tugas LH-VLN menggunakan LLM tugas $\mathcal{G}_{\mathrm{task}}$. Pada setiap percobaan pembangkitan $i=1,\dots,L$, sistem mengambil sampel \textit{scene} $s \in \mathcal{S}$ dan robot $r \in \mathcal{R}$, kemudian menyusun \textit{prompt} tugas yang terstruktur. \textit{Prompt} ini mencakup:
\begin{enumerate}[label=(\arabic*), left=0pt, labelsep=0.33cm, itemsep=0pt, topsep=0pt]
    \item \textit{System message} yang mendefinisikan peran model sebagai perancang tugas navigasi,
    \item Bagian aturan (\textit{rules}) yang menentukan format keluaran JSON, jumlah dan jenis subtugas, serta batasan objek/langkah,
    \item Satu contoh lengkap (\textit{one-shot example}) yang memetakan deskripsi \textit{scene} ke keluaran yang diinginkan, dan
    \item Deskripsi \textit{scene} dan konfigurasi robot yang aktual.
\end{enumerate}
Rincian naskah sistem, himpunan aturan, serta contoh \textit{one-shot} untuk $\mathcal{G}_{\mathrm{task}}$ disajikan pada Lampiran~1.

Keluaran $\mathcal{G}_{\mathrm{task}}$ berupa teks mentah $y_{\mathrm{raw}}$ kemudian di-\textit{parse} menjadi struktur JSON $t$. Struktur $t$ memuat instruksi tugas global (\textit{task instruction}), daftar subtugas simbolik \textit{berbasis navigasi} (misalnya \textit{Move\_to}, \textit{Turn}, \textit{Stop}), daftar objek, serta daftar region. Setiap konfigurasi $t$ selanjutnya melalui validasi sintaksis dan semantik berikut:
\begin{enumerate}[label=\alph*., left=0pt, labelsep=0.33cm, itemsep=0pt, topsep=0pt, partopsep=0pt, parsep=0pt]
    \item Struktur JSON harus dapat diurai tanpa kesalahan.
    \item Seluruh objek dan region yang dirujuk pada subtugas harus terdapat dalam metadata \textit{scene} $s$ yang bersesuaian.
    \item Tugas tidak boleh bergantung pada penamaan teknis yang tidak natural untuk pengguna manusia (misalnya label numerik abstrak untuk region).
\end{enumerate}
Konfigurasi yang dinyatakan valid disimpan sebagai berkas konfigurasi tugas dan membentuk himpunan $\mathcal{D}_{\mathrm{config}}$.

% \paragraph{(2) Eksekusi tugas di Habitat dengan $\mathcal{S}_{\mathrm{sim}}$ dan definisi operasional keterlaksanaan.}
Setiap konfigurasi tugas $t \in \mathcal{D}_{\mathrm{config}}$ kemudian dieksekusi pada simulator Habitat menggunakan modul $\mathcal{S}_{\mathrm{sim}}$. Untuk setiap subtugas dalam $t$, perencana jalur geodesik (\textit{Greedy Geodesic Follower}) menghitung lintasan terpendek pada \textit{navmesh} dan menghasilkan urutan aksi navigasi.

\begin{algorithm}[!ht]
\caption{\algfontsize{Pembangkitan LH-VLN Bilingual (ID--EN)}}
\label{alg:lhvln-bilingual-pipeline}
\begingroup\fontsize{10}{12}\selectfont
\begin{algorithmic}[1]
\Require scene $\mathcal{S}$ (dengan metadata), robot $\mathcal{R}$, simulator $\mathcal{S}_{\mathrm{sim}}$,
        LLM konfigurasi tugas $\mathcal{G}_{\mathrm{task}}$,
        LLM instruksi $\mathcal{G}_{\mathrm{instr}}$ + prompt bilingual,
        $L, M, K_{\max}, K_{\min}, d_{\mathrm{succ}}$,
        model tag scene $\mathcal{V}_{\mathrm{scene}}$,
        segmentor aksi $\mathcal{S}_{\mathrm{seg}}^{\mathrm{aksi}}$.
\Ensure $\mathcal{D}_{\mathrm{cfg}}$ (konfigurasi valid),
        $\mathcal{D}_{\mathrm{traj}}$ (trajektori sukses),
        $\mathcal{E}$ (episode + instruksi bilingual).

\State $\mathcal{D}_{\mathrm{cfg}}\gets\emptyset$;\;
       $\mathcal{D}_{\mathrm{traj}}\gets\emptyset$;\;
       $\mathcal{E}\gets\emptyset$

\Statex \textbf{Stage 1: Generate task configurations} 
\For{$i\gets 1$ \textbf{to} $L$}
  \State $(s,r)\sim(\mathcal{S},\mathcal{R})$
  \State $y\gets \mathcal{G}_{\mathrm{task}}\!\left(\Phi(s,r)\right)$
  \State $t\gets \textsc{ParseJSON}(y)$
  \If{$t=\bot$ \textbf{or} $\neg\textsc{ValidTask}(t,s)$}
    \State \textbf{continue}
  \EndIf
  \State $t\gets \textsc{AugmentMeta}(t,s,r)$
  \State $\mathcal{D}_{\mathrm{cfg}}\gets \mathcal{D}_{\mathrm{cfg}}\cup\{t\}$
\EndFor

\Statex \textbf{Stage 2: Shortest-path rollout in simulator} 
\ForAll{$t\in\mathcal{D}_{\mathrm{cfg}}$}
  \For{$m\gets 1$ \textbf{to} $M$}
    \If{$\neg\textsc{GeodesicReachable}(t,\mathcal{S}_{\mathrm{sim}})$}
      \State \textbf{break}
    \EndIf
    \State $\tau\gets \textsc{ShortestPathRollout}(t,\mathcal{S}_{\mathrm{sim}},K_{\max})$
    \If{$\textsc{Success}(\tau,t,d_{\mathrm{succ}})$}
      \State $\mathcal{D}_{\mathrm{traj}}\gets \mathcal{D}_{\mathrm{traj}}\cup\{\tau\}$
    \EndIf
  \EndFor
\EndFor

\Statex \textbf{Stage 3: Action segmentation + visual context} 
\ForAll{$\tau\in\mathcal{D}_{\mathrm{traj}}$}
  \State $\mathcal{Z}\gets \textsc{Segment}(\tau,\mathcal{S}_{\mathrm{seg}}^{\mathrm{aksi}},K_{\min})$
  \ForAll{$\zeta\in\mathcal{Z}$}
    \State $T_{\mathrm{scene}}\gets \mathcal{V}_{\mathrm{scene}}(\zeta.\mathcal{I})$
    \State $z\gets(\text{id},u,v,\tau^\star,\text{aksi},T_{\mathrm{scene}})$
    \State $\textsc{StoreSegment}(z)$
  \EndFor
\EndFor

\Statex \textbf{Stage 4: Bilingual instruction synthesis} 
\ForAll{episode $j$ dibentuk oleh $\Gamma(\cdot)$}
  \State $(\mathbf{c}^{(j)},\mathbf{m}^{(j)})\gets \textsc{BuildContext}(j)$
  \State $p^{(j)}\gets \Psi(\mathbf{c}^{(j)},\mathbf{m}^{(j)},\text{rules}_{\mathrm{cs}})$
  \State $y^{(j)}\gets \mathcal{G}_{\mathrm{instr}}(p^{(j)})$
  \State $\mathcal{E}\gets \mathcal{E}\cup\{(\mathbf{m}^{(j)},\mathbf{c}^{(j)},y^{(j)})\}$
\EndFor

\State \Return $\mathcal{D}_{\mathrm{cfg}},\mathcal{D}_{\mathrm{traj}},\mathcal{E}$
\end{algorithmic}
\endgroup
\end{algorithm}

Algoritma~\ref{alg:lhvln-bilingual-pipeline} merangkum \textit{pipeline} pembangkitan dataset LH-VLN bilingual (ID--EN) yang terdiri dari empat tahap utama. Pada bagian pertama, sistem melakukan $L$ kali percobaan pembangkitan konfigurasi tugas dengan mengambil sampel pasangan $(s,r)$ dari himpunan \textit{scene} $\mathcal{S}$ dan robot $\mathcal{R}$, lalu memanggil LLM konfigurasi tugas $\mathcal{G}_{\mathrm{task}}$ untuk menghasilkan keluaran mentah yang diurai menjadi JSON $t$; konfigurasi yang gagal di-\textit{parse} atau tidak lolos validasi (\textsc{ValidTask}) dibuang, sedangkan konfigurasi valid diperkaya metadata (\textsc{AugmentMeta}) dan dikumpulkan ke $\mathcal{D}_{\mathrm{cfg}}$. Pada bagian kedua, setiap konfigurasi $t\in\mathcal{D}_{\mathrm{cfg}}$ dieksekusi pada simulator $\mathcal{S}_{\mathrm{sim}}$ hingga $M$ kali percobaan; tugas terlebih dahulu dicek keterjangkauannya pada \textit{navmesh} (\textsc{GeodesicReachable}), kemudian dilakukan \textit{rollout} lintasan terpendek (\textsc{ShortestPathRollout}) dengan batas langkah $K_{\max}$, dan hanya \textit{episode} yang memenuhi kriteria keberhasilan (\textsc{Success}) dengan ambang jarak $d_{\mathrm{succ}}$ yang disimpan sebagai trajektori sukses ke $\mathcal{D}_{\mathrm{traj}}$. Selanjutnya pada bagian ketiga, setiap trajektori sukses $\tau\in\mathcal{D}_{\mathrm{traj}}$ disegmentasi menjadi unit rencana lokal menggunakan segmentor aksi $\mathcal{S}_{\mathrm{seg}}^{\mathrm{aksi}}$ dengan panjang minimum $K_{\min}$; untuk setiap segmen, konteks visual dihimpun dari citra $\zeta.\mathcal{I}$ dan diproses oleh model tag adegan $\mathcal{V}_{\mathrm{scene}}$ untuk menghasilkan \textit{scene tags} yang kemudian disimpan bersama metadata segmen. Terakhir, pada bagian keempat, segmen-segmen dikelompokkan menjadi \textit{episode} (melalui $\Gamma(\cdot)$), dibangun konteks dan metadata (\textsc{BuildContext}), disusun \textit{prompt} instruksi dengan aturan \textit{code-switch} ($\mathrm{rules}_{\mathrm{cs}}$) melalui $\Psi(\cdot)$, lalu LLM instruksi $\mathcal{G}_{\mathrm{instr}}$ menghasilkan instruksi navigasi bilingual $y^{(j)}$; pasangan (metadata, konteks, instruksi) dikumpulkan dalam himpunan $\mathcal{E}$ sebagai keluaran akhir \textit{pipeline}.

Definisi operasional tugas dapat dieksekusi pada penelitian ini adalah: sebuah tugas $t$ dinyatakan \textit{dapat dieksekusi} jika untuk setiap subgoal/target yang diminta dalam $t$, jarak geodesik yang dihitung pada \textit{navmesh} bersifat hingga (finite). Jika jarak geodesik terhadap salah satu target tak terhingga, maka tugas diklasifikasikan sebagai tugas yang tidak dapat dieksekusi dan tidak digunakan pada tahap berikutnya. Kriteria ini memeriksa keterjangkauan berbasis navigasi pada \textit{navmesh} dan tidak dimaksudkan sebagai pemeriksaan kelayakan lain di luar cakupan tersebut.

Untuk tugas yang dapat dieksekusi, agen dijalankan langkah demi langkah mengikuti lintasan rencana hingga salah satu kondisi tercapai:
\begin{enumerate}[label=(\arabic*), left=0pt, labelsep=0.33cm, itemsep=0pt, topsep=0pt]
    \item Semua target berhasil dicapai dengan jarak akhir di bawah ambang $d_{\mathrm{succ}}$ sebelum jumlah langkah mencapai $K_{\max}$, atau
    \item jumlah langkah mencapai $K_{\max}$ tanpa keberhasilan penuh (atau eksekusi dihentikan oleh \textit{timeout} sesuai konfigurasi).
\end{enumerate}
Pada setiap langkah $k$, sistem merekam log terstruktur yang mencakup: posisi tiga dimensi $(x,y,z)$, orientasi, aksi yang diambil, penanda waktu/\textit{timestep}, status eksekusi, serta citra kamera. Penyimpanan dilakukan dalam berkas terstruktur (misalnya JSON/CSV) dan/atau jalur berkas citra (misalnya \textit{path}) yang dapat direferensikan kembali pada tahap pemrosesan trajektori; detail format file mengikuti konfigurasi eksperimen.

Hanya trajektori yang berhasil (\textit{success episode}) yang dimasukkan ke himpunan trajektori rujukan $\mathcal{D}_{\mathrm{traj}}$. \textit{Episode} yang gagal tidak dimasukkan ke $\mathcal{D}_{\mathrm{traj}}$, namun tetap dapat digunakan untuk ringkasan metrik navigasi pada tahap analisis.

% \paragraph{(3) Segmentasi trajektori dan pembentukan konteks visual dengan $\mathcal{S}_{\mathrm{seg}}$ dan $\mathcal{V}_{\mathrm{scene}}$.}
Dari setiap trajektori sukses $d \in \mathcal{D}_{\mathrm{traj}}$, modul segmentasi $\mathcal{S}_{\mathrm{seg}}$ menurunkannya menjadi serangkaian segmen yang lebih pendek. Segmentasi dilakukan agar setiap segmen merepresentasikan satu unit rencana lokal yang koheren. Secara operasional, sistem membentuk segmen berdasarkan dua pemicu utama:
\begin{enumerate}[label=(\arabic*), left=0pt, labelsep=0.33cm, itemsep=0pt, topsep=0pt]
    \item Perubahan target lokal: segmen diakhiri ketika target/subgoal lokal berganti (sesuai urutan subtugas dalam konfigurasi $t$).
    \item Pola aksi : $\mathcal{S}_{\mathrm{seg}}$ dapat memecah lebih lanjut berdasarkan pola aksi agar rangkaian aksi di dalam segmen tetap koheren sebagai satu langkah/sub-rencana.
\end{enumerate}
Setiap segmen direpresentasikan oleh rentang indeks langkah $[u,v]$ pada trajektori asal. Segmen dengan panjang kurang dari $K_{\min}$ (dalam jumlah langkah) dibuang, sedangkan segmen yang lolos disimpan beserta identitas trajektori asal, target lokal, dan urutan aksi pada rentang tersebut.

Untuk setiap segmen $[u,v]$, sistem menghimpun citra kamera yang terkait dengan langkah-langkah pada rentang tersebut, lalu memprosesnya menggunakan model visi adegan $\mathcal{V}_{\mathrm{scene}}$ (misalnya \textit{Recognize Anything Model} \parencite{huang2023openset}) untuk menghasilkan himpunan \textit{scene tags} $T_{\mathrm{scene}}(u,v)$. Secara operasional, himpunan citra yang dipakai (misalnya subset frame tertentu atau seluruh frame pada segmen) serta cara agregasi \textit{tags} lintas multi-frame (misalnya pemilihan \textit{top-}$(k)$ atau ambang skor) ditentukan pada konfigurasi eksperimen. Keluaran yang disimpan minimal berupa daftar \textit{tags}; apabila model menyediakan skor/kepercayaan, informasi tersebut dapat ikut disimpan sebagai bagian dari log terstruktur.

Hasil tahap ini adalah daftar segmen
\[
L_{\mathrm{seg}}=\{z^{(\ell)}\}_{\ell=1}^{L_{\mathrm{seg}}}, 
\]
dengan $L_{\mathrm{seg}}$ menyatakan jumlah segmen total yang dihasilkan dari seluruh trajektori sukses. Setiap entri $z^{(\ell)}$ memuat: identitas trajektori asal, indeks awal--akhir segmen $[u,v]$, target lokal, urutan aksi, dan himpunan \textit{scene tags}.

% \paragraph{(4) Pembentukan \textit{episode} dan pembangkitan instruksi bilingual dengan $\mathcal{G}_{\mathrm{instr}}$.}
Berdasarkan daftar segmen $L_{\mathrm{seg}}$, sistem membangun \textit{episode} LH-VLN dengan mengelompokkan segmen yang memiliki trajektori asal dan target global yang sama. Untuk setiap \textit{episode}, dibentuk struktur konteks navigasi yang memuat target global $\tau$, serta urutan pasangan (\textit{actions}, \textit{scene tags}) untuk seluruh segmen dalam \textit{episode}. Struktur konteks ini dikombinasikan dengan metadata \textit{episode} (ID \textit{scene}, jenis robot, objek target, region, serta posisi dan orientasi awal--akhir) untuk menyusun \textit{prompt} instruksi bagi LLM $\mathcal{G}_{\mathrm{instr}}$.

\textit{Prompt} instruksi meliputi:
\begin{enumerate}[label=(\arabic*), left=0pt, labelsep=0.33cm, itemsep=0pt, topsep=0pt]
    \item \textit{System message} yang menjelaskan bahwa model harus menghasilkan instruksi navigasi untuk robot,
    \item himpunan aturan bahasa yang mengatur pola \textit{code-switch} (misalnya bahasa Indonesia digunakan untuk struktur langkah dan relasi gerak, sedangkan nama objek atau landmark dipertahankan dalam bahasa Inggris),
    \item Representasi rencana navigasi sebagai urutan langkah dengan ringkasan aksi dan konteks visual, serta
    \item Satu contoh \textit{one-shot} yang menunjukkan pasangan rencana--instruksi bilingual yang diharapkan.
\end{enumerate}
Template lengkap \textit{prompt} untuk $\mathcal{G}_{\mathrm{instr}}$, termasuk aturan \textit{code-switch} dan contoh \textit{one-shot}, disajikan pada Lampiran~1.

Keluaran $\mathcal{G}_{\mathrm{instr}}$ berupa teks instruksi navigasi bilingual untuk satu \textit{episode} LH-VLN, dengan panjang sekitar 1--3 kalimat dan berkisar 18--35 token kata. Pada subbab ini, istilah \textit{token kata} merujuk pada tokenisasi berbasis kata (misalnya pemisahan berdasarkan spasi/tanda baca) yang konsisten dengan statistik panjang instruksi yang dilaporkan. Parameter inferensi (misalnya suhu, \textit{top-p}, dan batas token baru) disesuaikan agar instruksi tetap ringkas namun mencakup seluruh langkah penting, sesuai konfigurasi eksperimen.

% \paragraph{(5) Perakitan dataset akhir dan anotasi LID.}
Sebagai langkah akhir pengumpulan data, seluruh komponen yang dihasilkan digabungkan ke dalam satu struktur dataset. Setiap entri dalam dataset akhir $\mathcal{D}^*$ memuat komponen-komponen yang selaras dengan Tabel~\ref{tab:jenis-data}, yaitu:
\begin{itemize}[left=0pt, labelsep=0.33cm, itemsep=0pt, topsep=0pt]
    \item Metadata \textit{scene} dan robot: ID \textit{scene}, konfigurasi robot, serta metadata target/region yang relevan.
    \item Deskripsi tugas LH-VLN: instruksi tugas global (\textit{task instruction}) dan daftar subtugas/target dalam konfigurasi $t$.
    \item Log trajektori: log langkah demi langkah yang mencakup posisi, orientasi, aksi, waktu/\textit{timestep}, status, dan referensi citra kamera.
    \item Segmen trajektori + konteks visual: daftar segmen $z^{(\ell)}$ beserta indeks $[u,v]$, target lokal, urutan aksi, dan $T_{\mathrm{scene}}(u,v)$.
    \item Instruksi bilingual + anotasi LID: instruksi yang dihasilkan $\mathcal{G}_{\mathrm{instr}}$ dan label bahasa per token.
\end{itemize}
Anotasi bahasa per token dilakukan menggunakan prosedur identifikasi bahasa (LID) yang dijelaskan pada subbab analisis data, sehingga instruksi dilengkapi label bahasa (Indonesia/Inggris) untuk kepentingan analisis \textit{code-switching}. Untuk konsistensi dengan LID, token yang digunakan pada anotasi ini mengikuti definisi tokenisasi yang sama (token berbasis kata) sebagaimana dipakai pada statistik panjang instruksi.
\vspace{0.5em}

\subsection{Teknik Analisis Data}
\label{sec:teknik-analisis-data}
Teknik analisis data dalam penelitian ini menggabungkan pendekatan 
kuantitatif dan kualitatif. 
Analisis kuantitatif berfokus pada metrik navigasi dan metrik 
\textit{code-switching} yang dihitung secara objektif, sedangkan 
analisis kualitatif menekankan interpretasi linguistik dan 
keterbacaan instruksi.

\vspace{0.5em}

\subsubsection{Analisis Kuantitatif}
Analisis kuantitatif dilakukan dengan menghitung berbagai metrik 
navigasi dan metrik \textit{code-switching} terhadap kumpulan 
instruksi pada $\mathcal{D}^*$.
Untuk setiap tugas yang berhasil dieksekusi, akan dihitung beberapa metrik
navigasi berdasarkan \textit{time cost, fail tasks, mean navigation step, mean
task success rate}, dan \textit{mean navigation success rate}. 
Metrik-metrik tersebut menggambarkan kualitas perencanaan 
dan efisiensi eksekusi navigasi pada \textit{pipeline} yang diusulkan.

Sebelum menghitung metrik \textit{code-switching}, dilakukan 
penandaan bahasa untuk setiap token pada instruksi 
$d_{\text{ins}}$. 
Prosedurnya sebagai berikut:
\begin{enumerate}[label=\alph*., left=0pt, labelsep=0.33cm, itemsep=0pt, topsep=0pt, partopsep=0pt, parsep=0pt]
    \item Teks instruksi dinormalisasi dan ditokenisasi pada 
          tingkat kata alfabetis.
    \item Jika token terdapat pada kamus bahasa Indonesia, 
          maka diberi label id; 
          jika terdapat pada kamus bahasa Inggris, diberi label en.
    \item Jika tidak terdapat pada kedua kamus, 
          digunakan langid yang dibatasi hanya pada label 
          id dan en. 
          Apabila skor kepercayaan di bawah ambang (misalnya 0{,}8), 
          token diberi label unk.
\end{enumerate}
Metrik \textit{code-switching} dihitung hanya pada token 
yang berlabel id atau en, sementara 
token unk diabaikan.
Berdasarkan urutan label bahasa yang diperoleh, 
akan dihitung metrik-metrik meliputi M-index, 
Language Entropy, I-index, Burstiness, Memory,
Span Entropy, Code-Mixing Index, dan T-index.
Kumpulan metrik ini memberikan gambaran objektif mengenai 
rasio bahasa, pola peralihan, dan konsistensi 
\textit{code-switching} pada dataset instruksi. 
Detail perhitungan metrik-metrik tersebut dapat ditemukan pada 
Subbab~\ref{sec:metrik-evaluasi}.

\vspace{0.5em}

\subsubsection{Analisis Kualitatif dan Penafsiran Hasil}
Selain analisis numerik, penelitian ini juga mencakup peninjauan 
kualitatif terhadap subset episode. 
Peneliti memilih beberapa contoh instruksi dari berbagai 
\textit{scene} dan tingkat kompleksitas untuk dianalisis 
secara manual, mencakup:
\begin{enumerate}[label=\alph*., left=0pt, labelsep=0.33cm, itemsep=0pt, topsep=0pt, partopsep=0pt, parsep=0pt]
    \item Keterbacaan instruksi bagi pembaca manusia.
    \item Kejelasan referensi objek dan landmark di lingkungan.
    \item Kesesuaian urutan kalimat dengan urutan trajektori.
    \item Kewajaran pola \textit{code-switching} (misalnya 
          penggunaan bahasa Inggris terutama pada nama objek 
          dan terminologi teknis).
\end{enumerate}
Hasil analisis kualitatif digunakan untuk menafsirkan angka-angka 
pada metrik kuantitatif dan mengidentifikasi contoh kasus 
yang berhasil maupun yang masih bermasalah.

\vspace{0.5em}

\subsection{Uji Coba, Evaluasi, dan Penyimpulan}

\vspace{0.5em}

\subsubsection{Uji Coba dan Evaluasi}
Sebelum eksperimen utama dilakukan, penelitian diawali dengan 
uji coba awal (\textit{pilot study}) menggunakan sejumlah kecil tugas. 
Tahap ini bertujuan untuk:
\begin{enumerate}[label=\alph*., left=0pt, labelsep=0.33cm, itemsep=0pt, topsep=0pt, partopsep=0pt, parsep=0pt]
    \item Memastikan setiap modul \textit{pipeline} (pembangkitan tugas, 
          simulasi, segmentasi, pembangkitan instruksi, dan evaluasi) 
          berfungsi dengan benar.
    \item Mengkalibrasi parameter penting seperti batas langkah, 
          \textit{timeout}, jumlah sampel instruksi per trajektori, 
          dan ambang skor $r_{\text{cs}}$ serta $s_{\text{nav}}$.
\end{enumerate}

Setelah \textit{pipeline} stabil, dilakukan eksperimen utama 
dengan menjalankan pembangkitan dan eksekusi tugas dalam jumlah besar 
pada subset scene HM3D yang telah ditentukan. 
Hasil eksekusi dicatat sebagai \textit{success tasks} dan \textit{fail tasks}.
Secara operasional, sebuah tugas $t$ dikategorikan sebagai \textit{success task}
apabila terdapat setidaknya satu percobaan eksekusi (\textit{episode attempt})
yang menyelesaikan seluruh subtugas dalam batas langkah $K_{\max}$ dan
\textit{timeout}, serta jarak geodesik akhir terhadap target berada di bawah
ambang $d_{\mathrm{succ}}$.
Sebaliknya, $t$ dikategorikan sebagai \textit{fail task} apabila
(i) salah satu target memiliki jarak geodesik tak hingga pada \textit{navmesh}
(tugas tidak dapat dieksekusi), atau
(ii) seluruh percobaan eksekusi hingga batas $M$ berakhir tanpa memenuhi
kriteria keberhasilan (misalnya mencapai $K_{\max}$ atau terkena \textit{timeout}).
\textit{Fail tasks} tetap dicatat untuk kebutuhan statistik kegagalan,
namun tidak diteruskan ke tahap segmentasi dan pembangkitan instruksi.

Ambang batas keberhasilan dataset ditetapkan sebagai berikut:
\begin{enumerate}[label=\alph*., left=0pt, labelsep=0.33cm, itemsep=0pt, topsep=0pt, partopsep=0pt, parsep=0pt]
    \item Setiap navigasi mesti berakhir pada jarak geodesik 
          di bawah 1 meter dari objek target \parencite{Song2025}.
    \item \textit{Task Success Rate} dan \textit{Navigation Success Rate} 
          harus berada pada rentang tinggi (mendekati satu) 
          pada himpunan \textit{success tasks}.
    \item Rasio \textit{success tasks} terhadap total kandidat tugas 
          diharapkan melebihi sekitar 50\%, sehingga dataset akhir 
          tetap cukup besar dan berkualitas.
    \item Rasio token bahasa Inggris $r_{\text{cs}}$ 
          diinstruksikan berada pada interval 
          $\alpha_{\min}$--$\alpha_{\max}$ (misalnya 0{,}20--0{,}40) 
          untuk memastikan pola \textit{code-switching} yang wajar.
\end{enumerate}

Seluruh metrik navigasi dan metrik \textit{code-switching} 
kemudian dihitung hanya pada himpunan \textit{success tasks}. 
Hal ini memastikan bahwa analisis linguistik dilakukan pada instruksi 
yang terbukti dapat dijalankan oleh agen di simulator, sehingga 
tingkat halusinasi instruksi relatif rendah \parencite{dogruoz-survey, hidayatullah2023peerj,tarunesh-etal-2021-machine}.

\vspace{0.5em}

\subsubsection{Cara Penyimpulan}
Penyimpulan hasil penelitian dilakukan melalui beberapa langkah sistematis:
\begin{enumerate}[label=\alph*., left=0pt, labelsep=0.33cm, itemsep=0pt, topsep=0pt, partopsep=0pt, parsep=0pt]
    \item Mengumpulkan nilai seluruh metrik utama (navigasi dan 
          \textit{code-switching}) dan menyajikannya dalam bentuk 
          tabel maupun grafik ringkas.
    \item Mengaitkan hasil numerik dengan temuan kualitatif, 
          misalnya dengan menampilkan contoh instruksi yang 
          memiliki skor baik maupun buruk.
    \item Membandingkan karakteristik dataset yang dihasilkan 
          dengan karakteristik yang diharapkan dari perspektif 
          penelitian VLN dan \textit{code-switching}.
    \item Menyusun kesimpulan yang menjawab rumusan masalah, 
          serta memberikan saran pengembangan untuk penelitian selanjutnya.
\end{enumerate}
Dengan demikian, proses uji coba, evaluasi, dan penyimpulan berjalan 
secara terstruktur dan transparan.

\vspace{0.5em}

\section{Jadwal Pelaksanaan}
\label{sec:jadwal_pelaksanaan}

Penelitian ini direncanakan berlangsung selama sembilan bulan efektif,
mulai Agustus 2025 sampai dengan April 2026. Kegiatan penelitian
dikelompokkan ke dalam beberapa tahapan teknis, yaitu:
(1) studi literatur dan analisis \textit{state-of-the-art},
(2) perancangan arsitektur \textit{pipeline} LH-VLN berbasis LLM 
    beserta skema evaluasinya,
(3) implementasi lingkungan eksperimen dan integrasi komponen perangkat lunak,
(4) pembangkitan dan kurasi dataset instruksi navigasi 
    \textit{code-switching} Indonesia--Inggris,
(5) pelaksanaan eksperimen evaluasi dan analisis hasil, serta
(6) penyusunan dan finalisasi naskah skripsi.
Rincian jadwal pelaksanaan setiap tahapan ditunjukkan pada 
Tabel~\ref{tab:jadwal_pelaksanaan}.

\begin{table}[H]
  \centering
  \caption{Jadwal Pelaksanaan Penelitian}
  \label{tab:jadwal_pelaksanaan}
  \begingroup
  \fontsize{10}{12}\selectfont
  \setlength{\tabcolsep}{3pt}
  \renewcommand{\arraystretch}{1.2}
  \begin{tabularx}{\textwidth}{|c|
    >{\raggedright\arraybackslash}X|
    c|c|c|c|c|c|c|c|c|}
    \hline
    \multirow{2}{*}{\textbf{No.}} &
    \multirow{2}{*}{\textbf{Kegiatan}} &
    \multicolumn{9}{c|}{\textbf{Bulan 2025--2026}} \\
    \cline{3-11}
    & & \textbf{Ags} & \textbf{Sep} & \textbf{Okt} &
      \textbf{Nov} & \textbf{Des} & \textbf{Jan} &
      \textbf{Feb} & \textbf{Mar} & \textbf{Apr} \\
    \hline
    1. & Studi literatur dan analisis
         state-of-the-art LH-VLN, NavGen,
         dan code-switching &
      \cellcolor{phaseLit} &
      \cellcolor{phaseLit} &
      \cellcolor{phaseLit} &
      & & & & & \\
    \hline
    2. & Perancangan arsitektur pipeline
         LH-VLN berbasis LLM in-the-loop
         dan rancangan metrik evaluasi dataset &
      \cellcolor{phaseDes} &
      \cellcolor{phaseDes} &
      \cellcolor{phaseDes} &
      & & & & & \\
    \hline
    3. & Implementasi lingkungan eksperimen
         (Habitat--HM3D, integrasi NavGen, LLM,
         dan model pengenalan objek/scene) &
      & & \cellcolor{phaseImpl} &
      \cellcolor{phaseImpl} &
      \cellcolor{phaseImpl} &
      & & & \\
    \hline
    4. & Pembangkitan dan kurasi dataset
         instruksi navigasi long-horizon
         code-switching Indonesia--Inggris &
      & & & \cellcolor{phaseData} &
      \cellcolor{phaseData} &
      \cellcolor{phaseData} &
      \cellcolor{phaseData} &
      & \\
    \hline
    5. & Eksperimen evaluasi kualitas dataset
         (kepatuhan rencana--instruksi, rasio dan pola
         code-switch, serta keragaman episode) dan
         analisis hasil &
      & & & & \cellcolor{phaseEval} &
      \cellcolor{phaseEval} &
      \cellcolor{phaseEval} &
      \cellcolor{phaseEval} &
      \\
    \hline
    6. & Penyusunan, revisi, dan finalisasi
         naskah skripsi berdasarkan hasil eksperimen &
      & & & & & \cellcolor{phaseWrite} &
      \cellcolor{phaseWrite} &
      \cellcolor{phaseWrite} &
      \cellcolor{phaseWrite} \\
    \hline
  \end{tabularx}
  \endgroup
\end{table}

\chapter{HASIL DAN PEMBAHASAN}
\label{Bab4}

\section{Analisis Teks Instruksi Berbasis Frasa Objek}
\label{sec:analisis-teks-frasa-objek}
\vspace{0.5em}

\subsection{Ringkasan Data dan Unit Analisis}
\label{subsec:ringkasan-data-unit-analisis}
Unit analisis pada bagian ini adalah instruksi \textit{natural language} (perintah tekstual) yang digunakan pada episode navigasi. Data dipisahkan menjadi dua kelompok berdasarkan hasil eksekusi tugas, yaitu \textit{success tasks} (506 instruksi) dan \textit{fail tasks} (433 instruksi). Dari setiap instruksi, frasa objek (objek yang diminta untuk diambil/dimanipulasi) diekstrak, lalu dihitung frekuensinya per kelompok. Frekuensi ini menjadi dasar untuk dua analisis utama pada subseksi berikutnya, yaitu: (1) distribusi frekuensi frasa objek melalui plot \textit{Zipf}, serta (2) pemetaan kemunculan frasa objek terhadap \textit{target room} yang disebut dalam instruksi. Sebagai pelengkap deskriptif, bagian ini juga menyajikan ringkasan objek dominan melalui \textit{Top-20} frekuensi dan \textit{wordcloud} pada masing-masing kelompok.

\begin{table}[H]
\centering
\caption{Ringkasan Data dan Statistik Frasa Objek}
\label{tab:summary-frasa-objek}
\fontsize{10pt}{12pt}\selectfont
\setlength{\tabcolsep}{3pt}
\resizebox{\linewidth}{!}{%
\begin{tabular}{lrrrrrr}
\toprule
\textbf{Group} & \textbf{Instr.} & \textbf{Total Phrases} & \textbf{Unique Phrases} & \textbf{Phrases/Instr.} & \textbf{Top-10 Share} & \textbf{Top-20 Share} \\
\midrule
Success & 506 & 961 & 242 & 1,90 & 37,9\% & 49,7\% \\
Fail    & 433 & 841 & 238 & 1,94 & 35,1\% & 46,0\% \\
\bottomrule
\end{tabular}}
\end{table}

Tabel~\ref{tab:summary-frasa-objek} menunjukkan bahwa kepadatan target objek pada instruksi \textit{success} dan \textit{fail} relatif sebanding, yaitu sekitar 1,9 frasa objek per instruksi. Dengan kata lain, sebagian besar instruksi hanya menargetkan sedikit objek secara eksplisit. Di sisi lain, distribusi frasa objek pada kedua kelompok tampak sangat terkonsentrasi: 10 frasa teratas mencakup lebih dari sepertiga seluruh kemunculan frasa, dan 20 frasa teratas mencakup mendekati setengah korpus. Pola ini mengindikasikan karakter \textit{long-tail}, yaitu sebagian kecil frasa muncul sangat sering sementara banyak frasa lainnya muncul jarang. Konsekuensinya, interpretasi pada frasa berfrekuensi sangat rendah perlu dilakukan dengan hati-hati ketika dibandingkan antar-kelompok. Temuan ringkas ini menjadi dasar pembacaan plot \textit{Zipf} pada subseksi berikutnya.
\vspace{0.5em}

\subsection{Distribusi Frekuensi Frasa Objek (\textit{Zipf})}
\label{subsec:zipf}
Hukum \textit{Zipf} menyatakan bahwa dalam korpus bahasa, frekuensi kata/frasa cenderung berbanding terbalik dengan peringkat kemunculannya: frasa yang menempati peringkat tinggi (sering) akan jauh lebih banyak dibanding frasa pada peringkat rendah (jarang). Konsekuensinya, distribusi kosakata umumnya bersifat \textit{heavy-tailed}/\textit{long-tail}. Untuk memeriksa pola ini pada frasa objek, digunakan plot \textit{rank--frequency} pada skala log--log \parencite{lavi-rotbain2022learnability,mikhaylovskiy-2025-zipfs,yokoi2024zipfian}.

Gambar~\ref{fig:zipf-frasa-objek} menampilkan plot \textit{rank--frequency} frasa objek pada skala log--log. Sumbu-$x$ merepresentasikan peringkat frasa (frasa diurutkan dari yang paling sering hingga paling jarang), sedangkan sumbu-$y$ merepresentasikan frekuensi kemunculan. Kurva yang menurun tajam di awal dan membentuk ekor panjang menunjukkan bahwa distribusi frasa objek pada kedua kelompok konsisten dengan pola \textit{Zipf}/\textit{long-tail}: sebagian kecil frasa objek mendominasi kemunculan, sedangkan mayoritas frasa berada pada frekuensi rendah. Sebagai ilustrasi, frasa yang paling sering muncul adalah \textit{towel} (91 kali pada \textit{success tasks}, 70 kali pada \textit{fail tasks}), selaras dengan konsentrasi pada frasa populer yang juga tercermin pada Tabel~\ref{tab:summary-frasa-objek}.

Plot \textit{Zipf} berguna terutama untuk memahami struktur kosakata secara global, bukan untuk menyoroti frasa tertentu. Pertama, visualisasi ini menegaskan bahwa kosakata frasa objek bersifat \textit{heavy-tailed}, sehingga pembacaan perbedaan antar-kelompok pada frasa yang sangat jarang perlu kehati-hatian. Kedua, secara visual bentuk kurva \textit{success} dan \textit{fail} tampak serupa, yang mengindikasikan bahwa perbedaan hasil eksekusi tidak terutama tercermin sebagai pergeseran besar pada struktur distribusi kosakata frasa objek secara keseluruhan. Dengan demikian, pembeda yang lebih informatif lebih masuk akal untuk ditelusuri pada ringkasan frasa dominan (\textit{Top-20}) dan konteks kemunculan frasa terhadap \textit{target room} pada subseksi berikutnya.
\vspace{0.5em}

\begin{figure}[H]
  \centering
  \includegraphics[page=2,width=\linewidth]{images/visualisasi_paperish_fullwidth_notitles_colorful.pdf}
  \caption{Distribusi Zipf Frekuensi Frasa Objek pada Instruksi}
  \label{fig:zipf-frasa-objek}
\end{figure}

\subsection{Frasa Objek Paling Sering Muncul (Top-20)}
\label{subsec:top20}
Untuk memberikan ringkasan yang mudah dibaca, Gambar~\ref{fig:top20-success} dan Gambar~\ref{fig:top20-fail} menyajikan 20 frasa objek dengan frekuensi tertinggi pada masing-masing kelompok. Batang yang lebih panjang menunjukkan frasa tersebut lebih sering muncul pada kelompok terkait, sehingga visualisasi ini dapat dibaca sebagai ringkasan ``objek apa'' yang paling sering menjadi target manipulasi pada \textit{success tasks} maupun \textit{fail tasks}.

Tumpang tindih pada frasa dominan (misalnya \textit{towel}, \textit{picture}, \textit{book}) menunjukkan bahwa sejumlah objek rumah tangga menjadi target manipulasi secara konsisten pada kedua kelompok. Temuan ini mengindikasikan bahwa perbedaan hasil eksekusi tidak hanya ditentukan oleh identitas objek yang disebut, melainkan juga dipengaruhi oleh komponen instruksi lain (misalnya struktur langkah atau penunjuk arah) serta kondisi lingkungan/visual. Studi diagnostik pada tugas navigasi berbasis instruksi menunjukkan agen dapat bergantung pada token objek sekaligus token arah, dan kinerja dapat berubah ketika input visual/lingkungan diperturbasi; hal ini menegaskan bahwa faktor visual dan aspek navigasi turut berperan, bukan hanya ``objek apa'' yang disebut \parencite{Zhu2022DiagnosingVLN}. Karena itu, \textit{Top-20} dipakai terutama sebagai ringkasan deskriptif, sedangkan pembacaan konteks yang lebih kaya dilakukan melalui pemetaan asosiasi objek--ruang pada subseksi berikutnya dan dilengkapi oleh analisis kualitatif \textit{fail tasks} pada Bagian~4.2.4.

\begin{figure}[H]
  \centering
  \includegraphics[page=5,width=\linewidth]{images/visualisasi_paperish_fullwidth_notitles_colorful.pdf}
  \caption{Top-20 Frasa Objek Terbanyak pada \textit{Success Tasks}}
  \label{fig:top20-success}
\end{figure}

\begin{figure}[H]
  \centering
  \includegraphics[page=6,width=\linewidth]{images/visualisasi_paperish_fullwidth_notitles_colorful.pdf}
  \caption{Top-20 Frasa Objek Terbanyak pada \textit{Fail Tasks}}
  \label{fig:top20-fail}
\end{figure}
\vspace{0.5em}

\subsection{Asosiasi Frasa Objek dan Target \textit{Room}}
\label{subsec:heatmap}
Selain frekuensi global, analisis ini memetakan asosiasi frasa objek terhadap \textit{target room} yang dirujuk dalam instruksi. \textit{Target room} diambil dari frasa preposisional yang mengikuti penanda tujuan seperti \textit{to/ke/into} dan dicocokkan ke daftar \textit{room} (misalnya \textit{bedroom}, \textit{bathroom}, \textit{kitchen}, \textit{living room}, \textit{office}, \textit{dining}, dan \textit{garage}). Apabila penanda tujuan tidak eksplisit, rujukan \textit{room} juga dapat diambil dari kemunculan nama \textit{room} di dalam instruksi. Untuk setiap frasa objek, \textit{heatmap} dinormalisasi per baris sehingga setiap baris dapat dibaca sebagai proporsi distribusi kemunculan frasa tersebut pada berbagai \textit{room}.

Gambar~\ref{fig:heatmap-success} dan Gambar~\ref{fig:heatmap-fail} menampilkan \textit{heatmap} asosiasi frasa objek--ruang pada masing-masing kelompok. Cara membacanya adalah: pilih satu frasa pada satu baris, lalu periksa kolom \textit{room} dengan intensitas warna tertinggi untuk melihat konteks ruang yang paling dominan bagi frasa tersebut. Normalisasi per baris membuat perbandingan antar-\textit{room} untuk frasa yang sama menjadi jelas. Secara umum, visualisasi ini memperlihatkan bahwa sebagian frasa objek memiliki konteks ruang yang relatif konsisten (proporsi terkonsentrasi pada satu \textit{room}), sementara frasa lain lebih menyebar lintas \textit{room}, sehingga konteks ruangnya lebih bervariasi.

Untuk menonjolkan pergeseran konteks ruang antar-kelompok, Gambar~\ref{fig:heatmap-delta} menampilkan \textit{delta heatmap} (proporsi \textit{fail} dikurangi \textit{success}). Nilai positif menunjukkan pasangan objek--ruang yang relatif lebih sering muncul pada instruksi gagal, sedangkan nilai negatif menunjukkan pasangan yang relatif lebih sering muncul pada instruksi berhasil. Matriks ko-okurensi/ketergantungan konteks semacam ini umum digunakan dalam visualisasi teks untuk menangkap asosiasi berbasis kemunculan bersama, meskipun interpretasinya tetap harus mempertimbangkan skala dan normalisasi \parencite{Skeppstedt2024WordRain}. Secara praktis, \textit{delta heatmap} membantu mengidentifikasi bahwa perbedaan antara \textit{fail} dan \textit{success} cenderung terlokalisasi pada subset pasangan objek--ruang tertentu, alih-alih menjadi perubahan yang merata di seluruh pasangan. Dengan demikian, pembeda yang lebih informatif bukan semata ``objek apa'' yang disebut, melainkan ``objek tersebut muncul dalam konteks ruang tujuan yang bagaimana''.

\begin{figure}[H]
  \centering
  \includegraphics[page=7,width=\linewidth]{images/visualisasi_paperish_fullwidth_notitles_colorful.pdf}
  \caption{Asosiasi Frasa Objek dan \textit{Target Room} pada \textit{Success Tasks}}
  \label{fig:heatmap-success}
\end{figure}

\begin{figure}[H]
  \centering
  \includegraphics[page=8,width=\linewidth]{images/visualisasi_paperish_fullwidth_notitles_colorful.pdf}
  \caption{Asosiasi Frasa Objek dan \textit{Target Room} pada \textit{Fail Tasks}}
  \label{fig:heatmap-fail}
\end{figure}

\begin{figure}[H]
  \centering
  \includegraphics[page=9,width=\linewidth]{images/visualisasi_paperish_fullwidth_notitles_colorful.pdf}
  \caption{Perbedaan Asosiasi Frasa Objek dan \textit{Target Room} Fail Minus Success}
  \label{fig:heatmap-delta}
\end{figure}

Interpretasi dilakukan dengan prinsip: (1) sel bernilai tinggi menunjukkan pasangan frasa objek--ruang yang relatif dominan, (2) pola baris menunjukkan konsistensi konteks ruang suatu objek, dan (3) \textit{delta heatmap} menyoroti pasangan yang lebih sering muncul pada \textit{fail} dibanding \textit{success}. Karena \textit{delta} dihitung setelah normalisasi baris, perubahan kecil pada frasa yang jarang dapat terlihat besar; oleh sebab itu, \textit{delta heatmap} dipakai terutama sebagai indikator eksploratif untuk menyaring pasangan objek--ruang yang layak diperiksa lebih lanjut.
\vspace{0.5em}

\subsection{Ringkasan Leksikal Frasa Objek Melalui \textit{Wordcloud}}
\label{subsec:wordcloud}

Sebagai ringkasan leksikal, \textit{wordcloud} digunakan untuk memperlihatkan frasa objek yang dominan pada masing-masing kelompok. Pada visualisasi ini, ukuran kata/frasa merepresentasikan frekuensi relatif kemunculan: frasa yang lebih sering muncul akan ditampilkan dengan ukuran lebih besar, sehingga pola dominasi frasa dapat ditangkap dengan cepat. Secara umum, ringkasan visual ini konsisten dengan temuan kuantitatif pada \textit{Top-20}, yaitu dominasi sejumlah kecil frasa objek terhadap korpus.

Meskipun efektif sebagai ringkasan cepat, \textit{wordcloud} tidak dirancang untuk perbandingan kuantitatif yang presisi. Secara khusus, tata letak spasial pada \textit{wordcloud} umumnya tidak membawa makna semantik, dan pembaca cenderung mengandalkan ukuran font sebagai satu-satunya isyarat besaran; hal ini dapat menurunkan akurasi ketika tugas pembaca menuntut penilaian magnitudo yang lebih tepat \parencite{Skeppstedt2024WordRain}. Karena itu, \textit{wordcloud} pada bagian ini diperlakukan sebagai pelengkap interpretasi, sedangkan kesimpulan utama tetap ditopang oleh bukti kuantitatif pada subseksi sebelumnya (misalnya \textit{Top-20} dan \textit{heatmap}).

\begin{figure}[H]
  \centering
  \includegraphics[page=10,width=\linewidth, trim={0mm 18mm 0mm 18mm}, clip]{images/visualisasi_paperish_fullwidth_notitles_colorful.pdf}
  \caption{\textit{Wordcloud} Frasa Objek pada \textit{Success Tasks}}
  \label{fig:wc-success}
\end{figure}

\begin{figure}[H]
  \centering
  \includegraphics[page=11,width=\linewidth, trim={0mm 18mm 0mm 18mm}, clip]{images/visualisasi_paperish_fullwidth_notitles_colorful.pdf}
  \caption{\textit{Wordcloud} Frasa Objek pada \textit{Fail Tasks}}
  \label{fig:wc-fail}
\end{figure}

\section{Evaluasi Efektivitas dan Efisiensi Navigasi--Tugas}
\label{sec:eval_nav_task}

\vspace{0.5em}

\subsection{Protokol Evaluasi dan Kriteria Validitas Eksekusi}
\label{subsec:nav_task_protocol}

Subbagian ini menjelaskan protokol evaluasi untuk mengukur efektivitas (keberhasilan) dan efisiensi (biaya waktu per langkah) pada skenario navigasi--tugas.
Seluruh eksperimen dijalankan pada simulator Habitat sebagai platform penelitian \textit{embodied AI} \parencite{savva2019habitat}.

Pada proses eksekusi, kandidat tugas yang dicoba dapat menghasilkan tiga keluaran praktis berikut.
Pertama, kandidat yang tidak dapat dievaluasi secara valid, misalnya target tidak terjangkau (\textit{unreachable}), target/koordinat tidak terdefinisi,
atau verifikasi keberhasilan tidak dapat dilakukan secara konsisten.
Kasus ini dicatat sebagai \textit{fail tasks} untuk diagnosis keterbatasan lingkungan dan kualitas kandidat instruksi, namun tidak dimasukkan ke perhitungan
\textit{success rate} karena tidak merepresentasikan percobaan evaluasi yang setara.
Kedua, kandidat yang dapat dievaluasi dan berhasil menyelesaikan rangkaian target hingga selesai (sukses end-to-end).
Ketiga, kandidat yang dapat dievaluasi namun gagal menyelesaikan rangkaian target karena eksekusi berhenti pada batas langkah maksimum (\textit{timeout}).
Dengan konvensi ini, metrik berbasis durasi dan langkah dilaporkan pada tugas yang sukses, sedangkan metrik berbasis \textit{success rate} dihitung pada tugas/episode
yang benar-benar dapat dievaluasi.

Untuk setiap tugas yang sukses, dicatat waktu mulai ($t_{start}$), waktu selesai ($t_{end}$), dan total langkah navigasi ($N_{step}$).
\textit{Time cost} didefinisikan sebagai waktu rata-rata per langkah untuk menyelesaikan tugas, yaitu durasi eksekusi dibagi jumlah langkah yang ditempuh.
Definisi ini memudahkan perbandingan efisiensi antar tugas dengan panjang lintasan yang berbeda, karena waktu total dinormalisasi terhadap jumlah langkah.

Pada bidang langkah--waktu, setiap tugas dapat dipetakan sebagai titik $(N_{step}, \Delta t)$, dengan $\Delta t=t_{end}-t_{start}$.
Secara geometris, \textit{time cost} per tugas dapat dipahami sebagai kemiringan garis dari titik asal menuju titik tugas tersebut.
Oleh karena itu, garis iso-\textit{time cost} dapat dipahami sebagai himpunan titik dengan kemiringan konstan, sehingga memudahkan interpretasi dan perbandingan
efisiensi lintasan antar tugas. Semakin curam kemiringan titik terhadap titik asal, semakin besar waktu yang dibutuhkan per langkah.
Ilustrasi interpretasi tersebut ditunjukkan pada Gambar~\ref{fig:proto_timecost_plane_en}.

\begin{figure}[H]
  \centering
  \includegraphics[width=\linewidth]{images/nav-tugas/fig_protocol_timecost_plane_en_v3.pdf}
  \caption{Ilustrasi Bidang Langkah--Waktu dan Garis Iso-\textit{Time Cost} (Kemiringan Merepresentasikan Waktu per Langkah)}
  \label{fig:proto_timecost_plane_en}
\end{figure}

Efektivitas dilaporkan pada dua level untuk membedakan kegagalan lokal dan kegagalan akumulatif pada rangkaian episode.
Pertama, \textit{task success rate} mengukur proporsi keberhasilan penyelesaian tugas end-to-end pada task yang \textit{dapat dievaluasi}.
Kedua, \textit{navigation success rate} mengukur proporsi keberhasilan mencapai target pada level navigasi parsial (episode/sub-goal) pada episode yang \textit{dapat dievaluasi}.
Pada bab ini, satu ``episode'' didefinisikan sebagai segmen navigasi menuju satu target antara (sub-goal) yang ditetapkan pipeline; satu tugas end-to-end dapat terdiri
dari beberapa episode berurutan.

Pembentukan \textit{success tasks} dan \textit{fail tasks} dapat dipandang sebagai pemetaan hasil eksekusi simulator ke dua lapisan berbeda (sukses/gagal).
Untuk memperjelas konsep ini, Gambar~\ref{fig:proto_partition_3d_en} menampilkan ilustrasi ruang 3D di mana tugas sukses dipetakan pada lapisan $z=1$ dan tugas gagal
dipetakan pada lapisan $z=0$. Visualisasi ini menegaskan bahwa metrik-metrik kuantitatif durasi/langkah dihitung pada tugas sukses, sedangkan tugas gagal tetap dicatat
sebagai statistik kegagalan dan bahan analisis kualitas instruksi serta keterbatasan lingkungan.

\begin{figure}[H]
  \centering
  \includegraphics[width=\linewidth]{images/nav-tugas/fig_protocol_3d_partition_en_v3.pdf}
  \caption{Ilustrasi Ruang Eksekusi 3D yang Menunjukkan Partisi Menjadi \textit{Success Tasks} dan \textit{Fail Tasks}}
  \label{fig:proto_partition_3d_en}
\end{figure}

\vspace{0.5em}

\subsection{Hasil Evaluasi Kuantitatif}
\label{subsec:nav_task_results}

Subbagian ini melaporkan hasil evaluasi kuantitatif untuk metrik efektivitas--efisiensi navigasi--tugas.
Metrik berbasis durasi dan jumlah langkah (misal \textit{step} dan \textit{time cost}) dilaporkan pada \textit{success tasks} agar merepresentasikan eksekusi yang valid.
Sementara itu, metrik berbasis rasio keberhasilan dihitung pada tugas/episode yang dapat dievaluasi sesuai protokol pada subbagian sebelumnya.

Tabel~\ref{tab:navtask_summary} merangkum metrik utama yang diperoleh.

\begin{table}[H]
\centering
\caption{Ringkasan Metrik Efektivitas--Efisiensi Navigasi--Tugas}
\label{tab:navtask_summary}
\fontsize{10}{12}\selectfont
\begin{tabular*}{\textwidth}{@{\extracolsep{\fill}} l p{0.52\textwidth} r @{}}
\toprule
\textbf{Metrik} & \textbf{Deskripsi} & \textbf{Nilai} \\
\midrule
\textit{Mean task step} &
Rata-rata jumlah langkah untuk menyelesaikan satu tugas end-to-end pada \textit{success tasks}. &
233{,}70 \\
\textit{Mean navigation step} &
Rata-rata jumlah langkah navigasi pada episode yang berhasil mencapai sub-goal di dalam rangkaian tugas. &
72{,}85 \\
\textit{Mean task success rate} &
Proporsi keberhasilan penyelesaian tugas end-to-end terhadap seluruh tugas yang \textit{dapat dievaluasi}. &
0{,}9945 \\
\textit{Mean navigation success rate} &
Proporsi keberhasilan mencapai target pada level episode terhadap seluruh episode yang \textit{dapat dievaluasi}. &
0{,}9982 \\
\bottomrule
\end{tabular*}
\end{table}

Tabel~\ref{tab:navtask_summary} menunjukkan bahwa keberhasilan pada level tugas dan level episode sama-sama sangat tinggi.
Perbedaan kecil antara \textit{Mean task success rate} dan \textit{Mean navigation success rate} mengindikasikan bahwa sebagian kecil kegagalan end-to-end
lebih mungkin muncul akibat akumulasi risiko pada rangkaian episode, bukan karena episode navigasi parsial sering gagal secara individual.

Perbedaan rata-rata jumlah langkah pada level tugas dan level episode memberikan indikasi bahwa satu tugas end-to-end terdiri dari beberapa episode navigasi.
Sebagai indikator kasar kompleksitas rangkaian episode, rasio berbasis rata-rata langkah menghasilkan aproksimasi jumlah episode efektif per tugas sekitar 3{,}21.
Nilai ini tidak dimaksudkan sebagai estimator eksak jumlah episode diskret pada setiap tugas, melainkan indikator makro untuk mengkuantifikasi tingkat \textit{long-horizon}.
Perbandingan rata-rata langkah pada kedua level ditunjukkan pada Gambar~\ref{fig:steps_lollipop_v2}, di mana \textit{Mean task step} lebih besar karena merupakan
akumulasi beberapa episode dalam satu rangkaian tugas.

\begin{figure}[H]
  \centering
  \includegraphics[width=\linewidth]{images/nav-tugas/fig_results_steps_lollipop_v2.pdf}
  \caption{Perbandingan \textit{Mean Task Step} dan \textit{Mean Navigation Step} Menggunakan Lollipop Plot}
  \label{fig:steps_lollipop_v2}
\end{figure}

Nilai \textit{success rate} pada rezim mendekati 1 sulit dibedakan secara visual apabila diplot langsung sebagai SR.
Untuk meningkatkan keterbacaan, hasil divisualisasikan dalam bentuk \textit{failure rate} (1--SR), sehingga perbedaan kecil tetap terlihat tanpa memerlukan pemotongan skala.
Gambar~\ref{fig:failure_rate_lollipop_v2} memperlihatkan bahwa \textit{failure rate} pada level tugas sedikit lebih tinggi daripada level episode,
yang konsisten dengan sifat tugas berantai: kegagalan pada satu episode dapat menggagalkan keseluruhan tugas end-to-end.

\begin{figure}[H]
  \centering
  \includegraphics[width=\linewidth]{images/nav-tugas/fig_results_failure_rate_lollipop_v2.pdf}
  \caption{Perbandingan \textit{Failure Rate} pada Level Tugas dan Level Navigasi Menggunakan Lollipop Plot}
  \label{fig:failure_rate_lollipop_v2}
\end{figure}

Log \textit{time cost} memuat 512 nilai, dan terdapat 6 nilai sangat kecil ($<0{,}01$) pada rentang 0{,}0002196 hingga 0{,}0003034 waktu/step.
Nilai ini terpaut sangat jauh dari skala distribusi utama: pada data bersih, nilai minimum adalah 0{,}093813 dan persentil 5\% adalah 0{,}131844,
sehingga keenam nilai tersebut ratusan kali lebih kecil daripada batas bawah sebaran utama.
Nilai sangat kecil ini muncul pada percobaan yang tidak selesai (\textit{timeout}) atau dicatat dengan normalisasi durasi yang berbeda dari definisi \textit{time cost}
pada tugas sukses, sehingga tidak sebanding untuk membandingkan efisiensi per langkah antar tugas.
Oleh karena itu, analisis efisiensi \textit{time cost} dilaporkan pada tugas sukses dengan definisi yang konsisten, dan keenam nilai tersebut dikeluarkan agar ringkasan statistik
tidak terdistorsi, dengan tetap melaporkan kriteria penyaringan secara eksplisit \parencite{benShachar2024outliers}.
Setelah penyaringan dengan ambang $<0{,}01$, jumlah sampel menjadi 506 dan konsisten dengan jumlah \textit{success tasks}.

Distribusi \textit{time cost} sebelum dan sesudah penyaringan ditunjukkan pada Gambar~\ref{fig:timecost_hist_raw_clean_v6}.
Perbedaan utama antara kedua distribusi adalah adanya massa probabilitas yang menempel di dekat nol pada data mentah, yang hilang setelah penyaringan,
sementara sebaran utama tetap serupa. Pita IQR serta penanda kuantil menunjukkan bahwa pembersihan tidak mengubah karakteristik distribusi utama.

\begin{figure}[H]
  \centering
  \includegraphics[width=\linewidth]{images/nav-tugas/fig_results_timecost_hist_raw_vs_clean_v6.pdf}
  \caption{Distribusi \textit{Time Cost} Sebelum dan Sesudah Penyaringan \textit{outlier} ($<0{,}01$) dengan Penanda Kuantil dan Pita IQR}
  \label{fig:timecost_hist_raw_clean_v6}
\end{figure}

Tabel~\ref{tab:nav_timecost_stats} melaporkan statistik robust untuk \textit{time cost} pada data bersih.
Median dan IQR digunakan sebagai ringkasan yang stabil terhadap ekor distribusi, sedangkan rentang P5--P95 menggambarkan sebaran utama dengan mengabaikan ekor ekstrem,
konsisten dengan praktik statistik nonparametrik berbasis distribusi empiris \parencite{henze2024edf}.
Secara praktis, nilai tipikal \textit{time cost} berada di sekitar median $\approx 0{,}20$ waktu/step.

\begin{table}[H]
\centering
  \caption{Statistik Robust \textit{Time Cost} pada \textit{Success Tasks} (Setelah Penyaringan \textit{outlier} $<0{,}01$).}
\label{tab:nav_timecost_stats}
\fontsize{10}{12}\selectfont
\begin{tabular*}{\textwidth}{@{\extracolsep{\fill}} l p{0.52\textwidth} r @{}}
\toprule
\textbf{Statistik} & \textbf{Deskripsi} & \textbf{Nilai} \\
\midrule
Mean &
Rata-rata \textit{time cost} (waktu per langkah) pada distribusi yang telah dibersihkan. &
0{,}20775 \\
Std &
Derajat variasi \textit{time cost} di sekitar nilai rata-rata. &
0{,}05924 \\
Median &
Nilai tengah (robust terhadap outlier), merepresentasikan \textit{time cost} tipikal. &
0{,}19712 \\
P5--P95 &
Rentang kuantil 5\% sampai 95\% untuk menggambarkan sebaran utama (mengabaikan ekor ekstrem). &
0{,}13184 -- 0{,}31202 \\
IQR (P25--P75) &
Rentang kuantil 25\% sampai 75\% sebagai ukuran sebaran robust (\textit{interquartile range}). &
0{,}16401 -- 0{,}24140 \\
\bottomrule
\end{tabular*}
\end{table}

Untuk memberikan konteks variasi efisiensi per langkah, data bersih menunjukkan bahwa \textit{time cost} minimum adalah 0{,}09381 (waktu/step paling cepat)
dan maksimum adalah 0{,}51850 (waktu/step paling lambat), sehingga ekstrem atas sekitar 5{,}53$\times$ lebih besar daripada ekstrem bawah.
Perbedaan ini menunjukkan bahwa meskipun perilaku tipikal berkisar di sekitar 0{,}20 waktu/step, terdapat kasus-kasus tertentu yang secara signifikan lebih mahal per langkah.

Sebagai pelengkap, Gambar~\ref{fig:timecost_ecdf_v2} menampilkan kurva ECDF dari \textit{time cost} pada data bersih.
Sebagai contoh pembacaan, median 0{,}19712 berarti sekitar 50\% tugas sukses memiliki \textit{time cost} tidak lebih dari 0{,}19712,
dan persentil 95\% sebesar 0{,}31202 berarti sekitar 95\% tugas sukses berada pada \textit{time cost} di bawah 0{,}31202 \parencite{henze2024edf}.

\begin{figure}[H]
  \centering
  \includegraphics[width=\linewidth]{images/nav-tugas/fig_results_timecost_ecdf_v2.pdf}
  \caption{Kurva ECDF \textit{Time Cost} pada Data Bersih dengan Penanda P5, Median, dan P95.}
  \label{fig:timecost_ecdf_v2}
\end{figure}

\vspace{0.5em}

\subsection{Efektivitas vs Efisiensi pada Skenario \textit{Long-Horizon}}
\label{subsec:nav_task_discussion}

Nilai \textit{Mean task step} sebesar 233{,}70 menunjukkan bahwa satu tugas yang berhasil umumnya memerlukan ratusan langkah.
Pada konteks navigasi berantai, besaran ini menandakan skenario \textit{long-horizon} yang terdiri atas beberapa episode/sub-goal berurutan.
Artinya, efektivitas tidak cukup dinilai dari kemampuan menyelesaikan satu episode secara terpisah, melainkan dari konsistensi agen mempertahankan keberhasilan
dari awal hingga akhir rangkaian. Pada rezim seperti ini, kesalahan kecil yang tampak lokal dapat berdampak sistemik karena muncul berulang di sepanjang urutan aksi.
Temuan ini sejalan dengan benchmark modern agen \textit{embodied} yang menekankan diagnosis lebih rinci dibanding sekadar \textit{final success rate}
\parencite{li2024eai}.

Indikator agregat memperlihatkan bahwa satu tugas memuat sekitar 3{,}21 episode.
Konsekuensinya, kegagalan pada satu episode saja sudah cukup untuk menghentikan keberhasilan end-to-end, walaupun episode lain berjalan baik.
Dengan kata lain, semakin panjang rangkaian, semakin rapuh keberhasilan total terhadap gangguan kecil pada level lokal.
Literatur \textit{long-horizon planning} juga melaporkan pola serupa: ketika horizon bertambah, efek error perencanaan cenderung menumpuk dan menurunkan performa end-to-end
meskipun kemampuan komponen lokal terlihat kuat \parencite{deps2023neurips,egoplan2023neurips}.

Selisih kecil antara \textit{Mean navigation success rate} (0{,}9982) dan \textit{Mean task success rate} (0{,}9945) konsisten dengan pembacaan reliabilitas berantai.
Keberhasilan episode yang sangat tinggi tetap dapat menghasilkan keberhasilan tugas yang sedikit lebih rendah karena tugas menuntut keberhasilan beruntun pada beberapa fase.
Model \textit{phased-mission systems} dan analisis sistem seri--paralel juga menyoroti pola akumulasi risiko pada rangkaian komponen/fase
\parencite{zhang2024msp_pms,torrado2024series_parallel_dependent}.
Pada penelitian ini, analogi reliabilitas dipakai sebagai kerangka interpretasi konseptual untuk menjelaskan kecenderungan tersebut, bukan untuk mengklaim independensi penuh antar episode.

Gambar~\ref{fig:disc_risk_contour_dense} merangkum relasi akumulasi risiko dalam format bidang XY yang lebih analitis.
Sumbu horizontal merepresentasikan keberhasilan pada level episode, sedangkan sumbu vertikal merepresentasikan panjang horizon efektif (jumlah episode).
Area kontur memperlihatkan kecenderungan keberhasilan end-to-end pada kombinasi kedua besaran tersebut, dan titik observasi menandai hasil eksperimen yang diperoleh.
Pembacaan visual ini menegaskan bahwa ketika horizon bertambah, penurunan kecil pada keberhasilan level episode dapat lebih mudah terlihat dampaknya pada level tugas.

\begin{figure}[H]
  \centering
  \includegraphics[width=0.96\linewidth]{images/nav-tugas/fig_discussion_risk_contour_2d_v4_dense_left.pdf}
  \caption{Visualisasi Bidang XY untuk Akumulasi Risiko End-to-End berdasarkan Keberhasilan per Episode dan Panjang Horizon Efektif, Disertai Titik Observasi Eksperimen.}
  \label{fig:disc_risk_contour_dense}
\end{figure}

Gambar~\ref{fig:disc_risk_curve_panels} memberikan sudut pandang komplementer dalam format kurva pada bidang XY.
Panel utama memperlihatkan bagaimana keberhasilan end-to-end berubah terhadap keberhasilan episode pada beberapa horizon efektif, sedangkan panel \textit{zoom}
menunjukkan bahwa nilai terukur berada dekat dengan kecenderungan kurva di sekitar titik observasi.
Dengan pembacaan ini, pesan utamanya tetap naratif: horizon yang lebih panjang membuat sistem lebih sensitif terhadap gangguan kecil, sehingga reliabilitas lintas episode dan efisiensi runtime perlu dijaga bersamaan.

\begin{figure}[H]
  \centering
  \includegraphics[width=0.96\linewidth]{images/nav-tugas/fig_discussion_risk_curve_2d_v4_panels.pdf}
  \caption{Kurva Bidang XY Akumulasi Risiko pada Beberapa Horizon Efektif dan Panel Diagnostik di Sekitar Titik Observasi untuk Membaca Sensitivitas Keberhasilan End-to-End.}
  \label{fig:disc_risk_curve_panels}
\end{figure}

Karena skenario bersifat \textit{long-horizon}, efisiensi per langkah berkontribusi langsung terhadap total runtime end-to-end.
Selain itu, \textit{time cost} sensitif terhadap artefak pencatatan pada ekor distribusi; oleh sebab itu, ringkasan robust (median, IQR, dan rentang P5--P95)
lebih informatif daripada hanya mean untuk merepresentasikan perilaku tipikal sistem secara stabil \parencite{benShachar2024outliers,henze2024edf}.

\vspace{0.5em}

\subsection{Analisis Kualitatif \textit{Fail Tasks} dan Akar Masalah}
\label{subsec:nav_task_fail_analysis}

\textit{Fail tasks} merepresentasikan kandidat instruksi yang tidak dapat dieksekusi hingga memenuhi kriteria keberhasilan end-to-end, baik karena kendala \textit{reachability},
kegagalan perencana lintasan, mismatch batasan aksi lingkungan, maupun ambiguitas instruksi yang mengganggu \textit{grounding}.
Secara kualitatif, kegagalan yang diamati dapat dirangkum ke dalam empat \textit{failure mode} berikut:
\begin{enumerate}[label=(\arabic*), left=0pt, labelsep=0.33cm, itemsep=0pt, topsep=0pt]
  \item F1: Objek non-manipulable namun diminta diambil, misalnya objek statis/terpasang.
  \item F2: Target manipulasi tidak valid, misalnya objek besar atau tidak memenuhi prasyarat aksi pada lingkungan.
  \item F3: Target penempatan tidak valid/ambigu, misalnya permukaan/\textit{receptacle} tidak valid sehingga verifikasi keberhasilan tidak konsisten.
  \item F4: Referensi entitas/area ambigu, sehingga pemetaan instruksi ke target aksi tidak deterministik.
\end{enumerate}

Untuk memperjelas relasi antara \textit{failure mode} dan kategori akar penyebab, Gambar~\ref{fig:failmode_rootcause_matrix} menyajikan matriks
\textit{Failure Mode} $\times$ \textit{Root Cause}.
Setiap sel mengodekan tingkat kontribusi penyebab terhadap suatu \textit{failure mode} secara kualitatif (\textit{None/Low/Medium/High}).
Pemetaan ini disusun sebagai ringkasan interpretatif dari inspeksi log eksekusi dan aturan validitas target pada simulator, sehingga bertujuan menonjolkan pola dominan.

\begin{figure}[H]
  \centering
  \includegraphics[width=\textwidth]{images/nav-tugas/fig_failmode_rootcause_matrix_2d_color_clean_v4.pdf}
  \caption{Matriks Kualitatif \textit{Failure Mode} $\times$ \textit{Root Cause} untuk Mengidentifikasi Akar Penyebab Dominan pada \textit{Fail Tasks}.}
  \label{fig:failmode_rootcause_matrix}
\end{figure}

Secara umum, F1--F2 paling sering berasosiasi dengan ketidakvalidan target aksi dan batasan interaksi lingkungan, sedangkan F3 lebih dipengaruhi oleh ketidakvalidan
target penempatan dan isu verifikasi keberhasilan.
Adapun F4 didominasi oleh ambiguitas instruksi yang menghambat \textit{grounding} dan pemilihan target aksi secara deterministik.
Benchmark modern untuk agen berbasis LLM juga menekankan bahwa kegagalan dapat berasal dari berbagai kategori error (mis.\ \textit{planning error} dan \textit{affordance error}),
sehingga diagnosis komponen menjadi penting untuk memahami bottleneck end-to-end \parencite{li2024eai,shi2024opex}.
Selain itu, studi terkini pada \textit{embodied instruction following} menunjukkan bahwa \textit{grounding} keterampilan lintas domain dan dekomposisi instruksi berpengaruh terhadap
keterlaksanaan tindakan di lingkungan target \parencite{shin2024semgro,cohen2024rlg}.

Temuan \textit{fail tasks} dapat dimanfaatkan sebagai sinyal untuk meningkatkan kualitas kandidat instruksi dan reliabilitas evaluasi melalui \textit{quality gate}:
(i) filter berbasis validitas aksi untuk instruksi \textit{pick up/retrieve} agar hanya target yang memenuhi prasyarat aksi yang dipilih;
(ii) validasi \textit{reachability} sebelum finalisasi instruksi (dan \textit{resampling} bila gagal);
(iii) \textit{whitelist} permukaan/\textit{receptacle} yang valid untuk aksi penempatan agar verifikasi konsisten; serta
(iv) normalisasi instruksi ambigu menjadi target eksplisit dan terverifikasi untuk mengurangi kegagalan akibat \textit{grounding ambiguity}
\parencite{shi2024opex,shin2024semgro,cohen2024rlg}.

\vspace{0.5em}

\section{Evaluasi \textit{Code-Switching} Indonesia--Inggris}
\label{sec:cs_evaluation}

\vspace{0.5em}

\subsection{Protokol Pelabelan Bahasa dan Aturan Agregasi}
\label{subsec:protocol_lid_agg}

Setiap instruksi ditokenisasi menjadi urutan token secara konsisten pada seluruh dataset. Tokenisasi mencakup pemisahan tanda baca sebagai token tersendiri karena keputusan ini memengaruhi metrik berbasis urutan, khususnya metrik yang menghitung perubahan bahasa antar token bertetangga. Setelah tokenisasi, setiap token diberi label bahasa $\in \{\text{ID},\text{EN},\text{UNK}\}$ menggunakan skema \textit{back-off} bertahap yang memprioritaskan keputusan deterministik agar proses pelabelan stabil dan mudah direproduksi.

Pada tahap pertama, token diberi label ID apabila ditemukan pada kamus Bahasa Indonesia dan diberi label EN apabila ditemukan pada kamus Bahasa Inggris. Apabila sebuah token muncul pada kedua kamus (kasus ambigu), label ditentukan dengan aturan \textit{tie-break} yang ditetapkan pada implementasi agar keputusan tetap deterministik. Pada tahap kedua, apabila token tidak tercakup pada kedua kamus, token diprediksi menggunakan model \textit{Language Identification} (LangID). Prediksi hanya diterima sebagai ID atau EN apabila skor keyakinan memenuhi ambang $\ge 0{,}8$. Terakhir, token yang tidak teridentifikasi oleh kamus dan tidak mencapai ambang keyakinan pada tahap LangID diberi label UNK.

Token berlabel UNK merepresentasikan token yang tidak dapat dipetakan secara andal ke kelas ID atau EN, misalnya token \textit{domain-specific}, gabungan kata tanpa spasi, nama objek/ruangan yang tidak umum, simbol, atau token \textit{out-of-vocabulary}. Dalam dataset ini, contoh token yang termasuk kategori UNK antara lain \textit{"headsculpturehunting"}, \textit{"otherroom"}, dan \textit{"hosehead"}. Pada evaluasi metrik \textit{code-switching}, token UNK dikeluarkan dari urutan label bahasa sehingga seluruh metrik dihitung hanya berdasarkan urutan label yang terdiri dari ID dan EN. Keputusan ini mengurangi noise dari token ambigu, namun dapat menurunkan jumlah transisi yang terhitung apabila UNK muncul di sekitar titik perpindahan bahasa.

Gambar~\ref{fig:token_label_timeline_switchpoints} mengilustrasikan pelabelan bahasa pada tingkat token beserta definisi titik perpindahan bahasa. Pada ilustrasi tersebut, titik perpindahan bahasa (ID$\leftrightarrow$EN) didefinisikan sebagai perubahan label pada pasangan token bertetangga setelah token UNK dikeluarkan dari urutan label. Secara formal, untuk urutan label hasil penyaringan $\mathbf{z}_i=(z_{i,1},\dots,z_{i,n_i})$ dengan $z_{i,j}\in\{\text{ID},\text{EN}\}$, titik perpindahan muncul pada posisi $j$ ketika $z_{i,j}\neq z_{i,j+1}$ untuk $j=1,\dots,n_i-1$. Sebagai contoh pada gambar, aliran token mentah berisi 8 token menjadi 7 token setelah UNK dihapus, lalu menghasilkan 3 titik perpindahan pada urutan ID/EN yang tersaring.
\vspace{0.4em}

\begin{figure}[H]
    \centering
    \resizebox{\linewidth}{!}{%
    \begin{tikzpicture}[
        x=0.9cm, y=0.9cm,
        tok/.style={draw=black!35, rounded corners=1pt, minimum width=1.35cm, minimum height=0.62cm, font=\scriptsize, align=center},
        idtok/.style={tok, fill=oiGreen!18},
        entok/.style={tok, fill=oiBlue!16},
        unktok/.style={tok, fill=gray!18},
        sw/.style={draw=oiVermilion, line width=0.9pt, dashed}
    ]
        % Raw token sequence (with UNK kept)
        \node[font=\scriptsize\bfseries, anchor=east] at (-0.3,1.7) {Raw token stream};
        \node[idtok]  (r1) at (1,1.7) {ambil\\ID};
        \node[entok]  (r2) at (2.6,1.7) {the\\EN};
        \node[entok]  (r3) at (4.2,1.7) {blue\\EN};
        \node[entok]  (r4) at (5.8,1.7) {towel\\EN};
        \node[idtok]  (r5) at (7.4,1.7) {di\\ID};
        \node[entok]  (r6) at (9.0,1.7) {bathroom\\EN};
        \node[unktok] (r7) at (10.6,1.7) {otherroom\\UNK};
        \node[entok]  (r8) at (12.2,1.7) {now\\EN};

        % Filtered token stream (UNK removed)
        \node[font=\scriptsize\bfseries, anchor=east] at (-0.3,0.35) {Filtered stream (ID/EN)};
        \node[idtok]  (f1) at (1,0.35) {ambil\\ID};
        \node[entok]  (f2) at (2.6,0.35) {the\\EN};
        \node[entok]  (f3) at (4.2,0.35) {blue\\EN};
        \node[entok]  (f4) at (5.8,0.35) {towel\\EN};
        \node[idtok]  (f5) at (7.4,0.35) {di\\ID};
        \node[entok]  (f6) at (9.0,0.35) {bathroom\\EN};
        \node[entok]  (f7) at (10.6,0.35) {now\\EN};

        % Switch points after UNK removal
        \draw[sw] (1.8,-0.15) -- (1.8,0.85);
        \node[font=\scriptsize, text=oiVermilion, anchor=south] at (1.8,0.88) {ID$\rightarrow$EN};

        \draw[sw] (6.6,-0.15) -- (6.6,0.85);
        \node[font=\scriptsize, text=oiVermilion, anchor=south] at (6.6,0.88) {EN$\rightarrow$ID};

        \draw[sw] (8.2,-0.15) -- (8.2,0.85);
        \node[font=\scriptsize, text=oiVermilion, anchor=south] at (8.2,0.88) {ID$\rightarrow$EN};

        % Legend
        \node[idtok, minimum width=0.95cm, minimum height=0.45cm] at (2.0,-0.65) {};
        \node[font=\scriptsize, anchor=west] at (2.55,-0.65) {ID token};
        \node[entok, minimum width=0.95cm, minimum height=0.45cm] at (4.8,-0.65) {};
        \node[font=\scriptsize, anchor=west] at (5.35,-0.65) {EN token};
        \node[unktok, minimum width=0.95cm, minimum height=0.45cm] at (7.6,-0.65) {};
        \node[font=\scriptsize, anchor=west] at (8.15,-0.65) {UNK token};
        \draw[sw] (10.25,-0.65) -- (10.95,-0.65);
        \node[font=\scriptsize, anchor=west] at (11.05,-0.65) {switch point};

        % Figure note
        \node[font=\scriptsize, text=gray!70, align=center] at (6.6,-1.30)
        {Transition counting is performed on the filtered ID/EN sequence (UNK excluded).};
    \end{tikzpicture}
    }
    \caption{Linimasa Pelabelan Bahasa Tingkat Token dan Titik Perpindahan Bahasa.}
    \label{fig:token_label_timeline_switchpoints}
\end{figure}

Evaluasi membedakan metrik yang bersifat global dan metrik yang sensitif terhadap urutan token. Metrik berbasis komposisi bahasa dihitung pada tingkat korpus untuk merepresentasikan distribusi global penggunaan bahasa Indonesia dan Inggris. Sebaliknya, metrik berbasis transisi dan struktur lokal dihitung per-\textit{utterance} karena bergantung pada urutan token di dalam instruksi, kemudian dirangkum dengan statistik agregat seperti rata-rata, median, dan kuartil. Pemisahan prosedur ini penting agar karakteristik lokal tidak terdistorsi apabila urutan token dari banyak instruksi digabung begitu saja.

Untuk menghindari bias akibat pemotongan instruksi, transisi bahasa tidak dihitung melintasi batas \textit{utterance}. Artinya, jumlah transisi hanya dihitung di dalam satu instruksi, dan batas antar instruksi diperlakukan sebagai pemutus urutan. Jika seluruh instruksi digabung menjadi satu rangkaian panjang, perubahan label bahasa pada pertemuan akhir instruksi $i$ dan awal instruksi $i{+}1$ dapat keliru terhitung sebagai transisi, padahal perubahan tersebut merupakan artefak agregasi, bukan fenomena \textit{code-switching} di dalam instruksi yang sama.

Sebagai ilustrasi teknis, Gambar~\ref{fig:dummy_xy_iindex_aggregation} menampilkan visualisasi untuk memperlihatkan relasi antara panjang \textit{utterance} efektif dan intensitas perpindahan bahasa per-\textit{utterance}, sekaligus perbedaan level agregasi yang benar dan yang bias.

\begin{figure}[H]
    \centering
    \begin{tikzpicture}
        \begin{axis}[
            width=\linewidth,
            height=0.56\linewidth,
            xmin=0, xmax=40,
            ymin=0, ymax=0.95,
            xlabel={Effective \textit{Utterance} Length $(n_i - 1)$},
            ylabel={Per-\textit{Utterance} \textit{I-index} $(I_i)$},
            axis line style={black!70},
            tick style={black!70},
            grid=major,
            grid style={gray!30},
            minor grid style={gray!15},
            minor x tick num=1,
            minor y tick num=1,
            ytick distance=0.1,
            tick label style={font=\small},
            label style={font=\small},
            legend style={
                font=\small,
                at={(0.02,0.98)},
                anchor=north west,
                draw=black!20,
                fill=white,
                fill opacity=0.85,
                text opacity=1,
                rounded corners=1.5pt
            },
            clip=false
        ]
            \addplot+[
                only marks,
                mark=*,
                mark size=1.8pt,
                color=oiBlue,
                fill=oiBlue!70,
                fill opacity=0.55,
                draw opacity=0.8
            ] coordinates {
                (3,0.18) (4,0.34) (5,0.12) (6,0.40) (7,0.26) (8,0.47)
                (9,0.31) (10,0.52) (11,0.29) (12,0.57) (13,0.35) (14,0.60)
                (16,0.41) (18,0.63) (20,0.45) (22,0.66) (25,0.49) (28,0.70)
                (32,0.55) (36,0.74)
            };
            \addlegendentry{Illustrative \textit{Utterance} Samples}

            \addplot+[domain=0:40, samples=2, very thick, color=oiGreen, dashed] {0.43};
            \addlegendentry{$\overline{I}_{\text{utt}}$ (Per-\textit{Utterance} Mean)}

            \addplot+[domain=0:40, samples=2, very thick, color=oiBlue, dash dot] {0.40};
            \addlegendentry{$I_w$ (Weighted Aggregation, No Cross-Boundary Transitions)}

            \addplot+[domain=0:40, samples=2, very thick, color=oiVermilion, densely dotted] {0.48};
            \addlegendentry{$I_{\text{concat}}$ (Biased Under Corpus Concatenation)}

            \node[
                font=\scriptsize,
                anchor=west,
                fill=white,
                fill opacity=0.9,
                text opacity=1,
                inner sep=2pt,
                rounded corners=1pt,
                draw=black!20
            ]
            at (axis cs:23,0.505) {$\Delta_{\text{boundary}} = I_{\text{concat}} - I_w$};
        \end{axis}
    \end{tikzpicture}
    \caption{Ilustrasi Agregasi \textit{I-index} Per-\textit{Utterance} dan Efek Bias Lintas Batas.}
    \label{fig:dummy_xy_iindex_aggregation}
\end{figure}

\subsection{Statistik Dasar Dataset}
\label{subsec:basic_stats}

Dataset yang digunakan pada evaluasi ini terdiri dari 506 \textit{utterance} dengan total 12.881 token, dengan definisi token mengikuti aturan tokenisasi pada Subbab~\ref{subsec:protocol_lid_agg} termasuk pemisahan tanda baca sebagai token tersendiri. Rata-rata panjang instruksi adalah sekitar 25,46 token per \textit{utterance}. Berdasarkan prosedur pelabelan bahasa tingkat token, sebanyak 12.510 token berhasil terklasifikasi sebagai bahasa ID atau EN, sedangkan 371 token dilabeli UNK. Dengan demikian, cakupan pelabelan pada tingkat token adalah 97,12\% dan proporsi token UNK adalah 2,88\%. Ringkasan statistik dasar disajikan pada Tabel~\ref{tab:basic_stats} dan komposisi token divisualisasikan pada Gambar~\ref{fig:basic_stats_tokens}.

Pada token yang terklasifikasi (ID/EN), distribusi bahasa menunjukkan dominasi token berbahasa Inggris. Secara agregat, terdapat 9.365 token EN dan 3.145 token ID pada token berlabel, yang setara dengan proporsi EN 74,86\% dan ID 25,14\%. Komposisi ini menjadi konteks untuk membaca metrik global (keseimbangan dan derajat pencampuran) sekaligus metrik lokal yang menangkap seberapa aktif perpindahan bahasa terjadi di dalam instruksi.

\begin{table*}[t]
\centering
\caption{Ringkasan Statistik Dasar Dataset}
\label{tab:basic_stats}
\fontsize{10}{12}\selectfont
\begin{tabularx}{\textwidth}{@{}l r X@{}}
\toprule
\textbf{Statistik} & \textbf{Nilai} & \textbf{Catatan} \\
\midrule
Jumlah \textit{utterance} & 506 & Total instruksi pada dataset. \\
Total token & 12.881 & Total token hasil tokenisasi (termasuk tanda baca). \\
Rata-rata token per \textit{utterance} & 25{,}46 & $12.881/506$. \\
Token berlabel (ID/EN) & 12.510 & Token terklasifikasi ID/EN. \\
Token UNK & 371 & Token tidak terklasifikasi sebagai ID/EN. \\
Cakupan pelabelan & 97{,}12\% & $12.510/12.881$. \\
Proporsi UNK & 2{,}88\% & $371/12.881$. \\
\midrule
Token EN / ID (pada token berlabel) & 9.365 / 3.145 & Komposisi pada token ID/EN. \\
Proporsi EN / ID (pada token berlabel) & 74{,}86\% / 25{,}14\% & Komposisi relatif pada token ID/EN. \\
\bottomrule
\end{tabularx}
\end{table*}

\begin{figure}[H]
    \centering
    \includegraphics[width=\textwidth]{images/csideng/vis_basic_stats_stackedbar_2col_span.pdf}
    \caption{Komposisi Token Dataset Berdasarkan Label Bahasa (EN, ID, UNK).}
    \label{fig:basic_stats_tokens}
\end{figure}

\subsection{Keseimbangan Bahasa menggunakan \textit{M-index} dan \textit{Language Entropy}}
\label{subsec:mindex_entropy}

Keseimbangan distribusi bahasa pada tingkat korpus dievaluasi menggunakan \textit{M-index} dan \textit{language entropy}. Secara konseptual, \textit{M-index} merepresentasikan derajat keseimbangan proporsi dua bahasa, sedangkan \textit{language entropy} mengukur keragaman distribusi bahasa dan mencapai nilai maksimum ketika kedua bahasa memiliki proporsi yang sama. Kedua metrik ini dihitung pada tingkat korpus dengan menggunakan token berlabel (ID/EN).

Pada dataset ini, \textit{M-index} bernilai 0,603579 dan \textit{language entropy} bernilai 0,813488 bit. Nilai tersebut sejalan dengan dominasi bahasa Inggris, namun tetap menunjukkan keragaman yang relatif tinggi karena bahasa Indonesia hadir dalam porsi yang tidak kecil pada token berlabel. Gambar~\ref{fig:mindex_entropy_vs_pen} memvisualisasikan perilaku kedua metrik sebagai fungsi proporsi EN dan menempatkan titik observasi dataset dalam konteks tersebut.

\begin{figure}[H]
    \centering
    \includegraphics[width=\textwidth]{images/csideng/vis_mindex_entropy_vs_pen_hatchet_v4_en_2col_span.pdf}
    \caption{Perilaku \textit{M-index} dan \textit{Language Entropy} sebagai Fungsi Proporsi Token EN, dengan Zona Heuristik Mendekati Seimbang.}
    \label{fig:mindex_entropy_vs_pen}
\end{figure}

\subsection{Derajat Pencampuran menggunakan \textit{Code-Mixing Index}}
\label{subsec:cmi}

Derajat pencampuran bahasa diringkas menggunakan \textit{Code-Mixing Index} (CMI), yang secara intuitif menggambarkan porsi bahasa non-dominan dalam korpus atau dalam sebuah instruksi. Pada kasus dua bahasa, CMI berada pada rentang 0 hingga 50; nilai yang lebih tinggi menunjukkan porsi bahasa non-dominan yang lebih besar. Dalam evaluasi ini, CMI dihitung pada token berlabel (ID/EN) sehingga konsisten dengan definisi proporsi bahasa setelah UNK dikeluarkan.

Hasil evaluasi menunjukkan CMI pada tingkat korpus sebesar 25,1399, sedangkan rata-rata CMI pada tingkat \textit{utterance} adalah 25,1511. Nilai sekitar 25 mengindikasikan bahwa bahasa non-dominan (ID) muncul sekitar seperempat dari token berlabel, sehingga dataset bersifat EN-dominan tetapi tetap mempertahankan keberadaan token Indonesia yang bermakna. Distribusi CMI per-\textit{utterance} diringkas melalui fungsi kuantil pada Gambar~\ref{fig:cmi_quantilefunction}, yang membantu melihat variasi pencampuran antar instruksi tanpa mengandalkan detail rumus.

\begin{figure}[H]
    \centering
    \includegraphics[width=\textwidth]{images/csideng/vis_cmi_quantilefunction_final_v5_legendcolormatch_2col_span.pdf}
    \caption{Fungsi Kuantil CMI per-\textit{Utterance} dengan Referensi Rata-Rata dan Nilai Korpus.}
    \label{fig:cmi_quantilefunction}
\end{figure}

\subsection{Intensitas Perpindahan Bahasa menggunakan \textit{I-index}}
\label{subsec:iindex}

Intensitas perpindahan bahasa diukur menggunakan \textit{I-index}, yang secara konseptual merepresentasikan proporsi pasangan token bertetangga yang berbeda bahasa (ID$\leftrightarrow$EN) di dalam sebuah instruksi. Metrik ini dihitung per-\textit{utterance} pada urutan label ID/EN setelah token UNK dikeluarkan, kemudian dirangkum pada tingkat dataset menggunakan rata-rata per-\textit{utterance} serta agregasi berbobot berdasarkan panjang instruksi.

Pada dataset ini, rata-rata \textit{I-index} per-\textit{utterance} adalah 0,383364 dan nilai tertimbang pada tingkat korpus adalah 0,376624. Kedua nilai yang berdekatan menunjukkan bahwa pembobotan berdasarkan panjang instruksi tidak mengubah estimasi intensitas perpindahan secara substansial. Secara interpretatif, sekitar 38\% pasangan token bertetangga pada urutan ID/EN merupakan transisi antar bahasa, sehingga dataset tidak hanya menampilkan pencampuran token, tetapi juga kaya akan perpindahan bahasa di dalam instruksi. Ringkasan distribusi \textit{I-index} divisualisasikan pada Gambar~\ref{fig:iindex_cdf} untuk memperlihatkan variasi antar instruksi.

\begin{figure}[H]
    \centering
    \includegraphics[width=\textwidth]{images/csideng/vis_iindex_cdf_colorful_v4_median_rb_2col_span.pdf}
    \caption{Ringkasan Distribusi \textit{I-index} (CDF Berbasis Kuantil) dengan Referensi Rata-Rata dan Nilai Tertimbang.}
    \label{fig:iindex_cdf}
\end{figure}

\subsection{Struktur \textit{Span} menggunakan \textit{Span Entropy}, \textit{Burstiness}, dan \textit{Memory}}
\label{subsec:span_metrics}

Untuk memahami bentuk switching secara lokal, urutan label bahasa pada tingkat token dipetakan menjadi \textit{span} monolingual, yaitu segmen token berturut-turut dengan label bahasa yang sama. Dari representasi ini, evaluasi merangkum keragaman panjang span, tingkat ketidak-ekstreman variasi panjang span, serta keterkaitan panjang span bertetangga. Seluruh metrik berbasis span dihitung pada urutan label ID/EN setelah token UNK dikeluarkan, sehingga konsisten dengan protokol evaluasi. Penghapusan token UNK dapat mengubah struktur lokal secara ringan, misalnya membuat dua span berbahasa sama menjadi bersebelahan dan kemudian tergabung, sehingga interpretasi metrik berbasis span mempertimbangkan kondisi ini.

Keragaman panjang span diringkas menggunakan \textit{span entropy}. Pada dataset ini, rata-rata \textit{span entropy} adalah 1,536138 bit dengan median 1,551098 bit, yang mengindikasikan bahwa panjang span bervariasi dan switching tidak selalu berbentuk selang-seling satu token. Variasi \textit{span entropy} antar instruksi ditunjukkan pada Gambar~\ref{fig:spanentropy_cdf}. Tingkat \textit{burstiness} rata-rata adalah $-0,040680$ yang berada dekat nol, sehingga variasi panjang span secara umum tidak bersifat ekstrem. Sementara itu, \textit{memory} rata-rata adalah $-0,304733$ yang bernilai negatif, menunjukkan kecenderungan \textit{anti-correlation} di mana span yang lebih panjang cenderung diikuti span berikutnya yang lebih pendek (dan sebaliknya). Ringkasan sebaran \textit{burstiness} dan \textit{memory} ditampilkan pada Gambar~\ref{fig:burstiness_memory_summary}.

\begin{figure}[H]
    \centering
    \includegraphics[width=\textwidth]{images/csideng/vis_spanentropy_cdf_v6_best_2col_span.pdf}
    \caption{Ringkasan Distribusi \textit{Span Entropy} pada Tingkat \textit{Utterance}.}
    \label{fig:spanentropy_cdf}
\end{figure}

\begin{figure}[H]
    \centering
    \includegraphics[width=\textwidth]{images/csideng/vis_burstiness_memory_summary_v6_best_2col_span.pdf}
    \caption{Ringkasan Sebaran \textit{Burstiness} dan \textit{Memory} pada Tingkat \textit{Utterance}.}
    \label{fig:burstiness_memory_summary}
\end{figure}

\subsection{Kualitas Titik \textit{Switch} menggunakan \textit{T-index}}
\label{subsec:tindex}

Selain mengukur intensitas dan struktur perpindahan bahasa, evaluasi ini menilai kualitas titik perpindahan menggunakan \textit{T-index} berbasis skor log-probabilitas dari model terjemahan mesin pada konteks titik \textit{switch}. Skor ini merepresentasikan preferensi model terhadap keluaran terjemahan pada konteks titik perpindahan dan dirangkum sebagai nilai rata-rata pada seluruh titik \textit{switch} di korpus. Karena skala skor log-probabilitas bergantung pada model dan domain, \textit{T-index} diperlakukan sebagai indikator komparatif internal untuk membaca sebaran kualitas switching.

Berdasarkan evaluasi pada dataset, diperoleh \textit{T-index} sebesar $-0,5985$. Distribusi skor pada titik \textit{switch} ditampilkan pada Gambar~\ref{fig:tindex_scorecdf}, termasuk pemisahan berdasarkan arah perpindahan (ID$\rightarrow$EN dan EN$\rightarrow$ID) untuk melihat potensi asimetri kualitas. Gambar~\ref{fig:tindex_density_vs_quality} selanjutnya menganalisis hubungan kualitas titik \textit{switch} terhadap kepadatan \textit{switch} pada tingkat \textit{utterance}, sehingga dapat diperiksa apakah instruksi dengan switching sangat sering cenderung menghasilkan titik perpindahan yang relatif kurang natural menurut indikator model terjemahan atau sebaliknya.

\begin{figure}[H]
    \centering
    \includegraphics[width=\textwidth]{images/csideng/vis_tindex_scorecdf_v3_clean_2col_span.pdf}
    \caption{Distribusi Skor Log-Probabilitas pada Titik \textit{Switch} untuk \textit{T-index}.}
    \label{fig:tindex_scorecdf}
\end{figure}

\begin{figure}[H]
    \centering
    \includegraphics[width=\textwidth]{images/csideng/vis_tindex_density_vs_quality_v6_smalllabels_2col_span.pdf}
    \caption{Kualitas Titik \textit{Switch} terhadap Kepadatan \textit{Switch} pada Tingkat \textit{Utterance}.}
    \label{fig:tindex_density_vs_quality}
\end{figure}

\subsection{Ringkasan Hasil dan Profil \textit{Code-Switching} Dataset}
\label{subsec:summary_and_profile}

Tabel~\ref{tab:cs_metrics_summary} merangkum metrik utama evaluasi \textit{code-switching} ID--EN pada dataset. Secara global, distribusi bahasa pada token berlabel menunjukkan dominasi bahasa Inggris (74,86\%) dengan kehadiran bahasa Indonesia yang tetap substantif (25,14\%), yang konsisten dengan nilai \textit{M-index} 0,603579 dan \textit{language entropy} 0,813488 bit. Derajat pencampuran bahasa yang diringkas oleh CMI berada di kisaran 25 baik pada tingkat korpus maupun rata-rata per-\textit{utterance}, mengindikasikan bahwa pencampuran tidak hanya disebabkan oleh sedikit instruksi ekstrem. Pada tingkat lokal, \textit{I-index} menunjukkan intensitas switching yang cukup tinggi, dengan sekitar 38\% pasangan token bertetangga merupakan transisi antar bahasa, sehingga dataset kaya akan switching \textit{intra-utterance}. Analisis berbasis span menunjukkan \textit{chunking} monolingual yang moderat, variasi panjang span yang beragam, \textit{burstiness} yang tidak ekstrem, serta kecenderungan \textit{anti-correlation} pada span bertetangga. Terakhir, kualitas titik \textit{switch} menurut indikator model terjemahan diringkas oleh \textit{T-index} sebesar $-0,5985$, yang digunakan terutama sebagai pembacaan komparatif internal terhadap sebaran kualitas switching.

Sebagai batasan, seluruh metrik switching dihitung setelah token UNK dikeluarkan untuk mengurangi noise, sehingga jumlah transisi terhitung dapat sedikit berkurang apabila UNK muncul di sekitar titik switching. Selain itu, interpretasi \textit{T-index} bersifat \textit{model-dependent}; untuk menilai kewajaran switching secara lebih kuat, indikator ini dapat dilengkapi dengan pembanding model lain atau evaluasi manusia pada pekerjaan lanjutan.

\begin{table*}[!t]
\centering
\caption{Ringkasan Metrik Evaluasi \textit{Code-Switching} ID--EN pada Dataset}
\label{tab:cs_metrics_summary}
\fontsize{10}{12}\selectfont
\setlength{\tabcolsep}{6pt}
\renewcommand{\arraystretch}{1.05}
\begin{tabularx}{\textwidth}{@{}>{\raggedright\arraybackslash}X l r@{}}
\toprule
\textbf{Metrik} & \textbf{Level} & \textbf{Nilai} \\
\midrule
\multicolumn{3}{@{}l}{\textbf{A. Statistik Dasar}} \\
Jumlah \textit{utterance} & Korpus & 506 \\
Total token & Korpus & 12.881 \\
Token berlabel (ID/EN) & Korpus & 12.510 \\
Token \textbf{UNK} & Korpus & 371 \\
Token EN / ID (pada token berlabel) & Korpus & 9.365 / 3.145 \\
Proporsi EN / ID (pada token berlabel) & Korpus & 74,86\% / 25,14\% \\
\addlinespace

\multicolumn{3}{@{}l}{\textbf{B. Keseimbangan Bahasa (Global)}} \\
$M$-index & Korpus & 0,603579 \\
\textit{Language entropy} & Korpus & 0,813488 bit \\
\addlinespace

\multicolumn{3}{@{}l}{\textbf{C. Derajat Pencampuran}} \\
CMI & Korpus & 25,1399 \\
CMI & Per-utt (mean) & 25,1511 \\
\addlinespace

\multicolumn{3}{@{}l}{\textbf{D. Intensitas Perpindahan Bahasa}} \\
$I$-index & Per-utt (mean) & 0,383364 \\
$I$-index & Korpus (weighted) & 0,376624 \\
\addlinespace

\multicolumn{3}{@{}l}{\textbf{E. Struktur Lokal Berbasis \textit{Span}}} \\
\textit{Span entropy} & Per-utt (mean) & 1,536138 bit \\
\textit{Burstiness} & Per-utt (mean) & $-0,040680$ \\
\textit{Memory} & Per-utt (mean) & $-0,304733$ \\
\addlinespace

\multicolumn{3}{@{}l}{\textbf{F. Kualitas Titik \textit{Switch} (indikator MT)}} \\
\textit{T-index} & Korpus & $-0,5985$ \\
\bottomrule
\end{tabularx}
\end{table*}

\begin{figure}[H]
    \centering
    \includegraphics[width=\textwidth]{images/csideng/vis_cs_metrics_summary_profile_v1_2col_span.pdf}
    \caption{Profil Ringkas Metrik \textit{Code-Switching} dalam Skala Ternormalisasi.}
    \label{fig:cs_metrics_profile}
\end{figure}

\begin{figure}[H]
    \centering
    \includegraphics[width=\textwidth]{images/csideng/vis_cmi_vs_iindex_hexbin_v4_ultraclean_2col_span.pdf}
    \caption{Kepadatan CMI dan \textit{I-index} pada Tingkat \textit{Utterance}.}
    \label{fig:cmi_vs_iindex_hexbin}
\end{figure}

\section{Representasi Segmentasi Aksi dan Konteks Visual pada Episode \textit{Long-Horizon}}
\vspace{0.5em}

\subsection{Unit Representasi: Episode, Subtugas, dan Segmen Aksi}
Satu episode \textit{long-horizon} terdiri dari instruksi global yang secara operasional dapat diuraikan menjadi beberapa subtugas berorientasi target (misalnya mencapai beberapa objek secara berurutan). Untuk setiap subtugas target, trajektori navigasi dipadatkan menjadi urutan segmen $\{s_i\}_{i=1}^{m}$. Setiap segmen $s_i$ didefinisikan sebagai tuple pada Persamaan~\ref{eq:segment_tuple} yang merangkum rentang langkah, label aksi, serta ringkasan konteks visual pada interval tersebut.
\begin{equation}
s_i = \left(\textit{start}_i, \textit{end}_i, a_i, \mathcal{T}_i\right),
\label{eq:segment_tuple}
\end{equation}
di mana $\textit{start}_i$ dan $\textit{end}_i$ menyatakan rentang indeks langkah pada trajektori, $a_i$ menyatakan label aksi pada segmen (misalnya \textit{move\_forward}, \textit{turn\_left}, \textit{turn\_right}), dan $\mathcal{T}_i$ menyatakan himpunan tag konteks visual (\textit{scene tags}) yang merangkum \textit{landmark}/ruang/objek dominan pada rentang segmen tersebut. Representasi ini berperan sebagai \textit{intermediate representation} yang menjembatani keluaran simulator (numerik dan bertaraf rendah) menuju instruksi bahasa natural (bilingual) yang tetap \textit{grounded}.
\vspace{0.5em}

\subsection{Skala Data dan Indikator Kompleksitas \textit{Long-Horizon}}
Pada eksperimen ini diperoleh 401 episode unik (berbasis instruksi global) yang terurai menjadi 1096 subtugas berorientasi target. Total segmen yang terbentuk adalah 5253 segmen, dengan 265 target unik. Rata-rata jumlah segmen per subtugas adalah 4{,}79 (median 5; maksimum 21). Panjang segmen (dalam langkah) memiliki median 6, persentil ke-90 sebesar 25, dan maksimum 116 langkah. Ringkasan statistik skala data dan kompleksitas episode ditampilkan pada Tabel~\ref{tab:lhvln_stats_overall}.

\begin{table}[H]
\centering
\caption{Ringkasan Statistik Representasi Segmen dan Kompleksitas Episode}
\label{tab:lhvln_stats_overall}
{\fontsize{10}{12}\selectfont
\renewcommand{\arraystretch}{1.15}
\setlength{\tabcolsep}{10pt}
\begin{tabularx}{\textwidth}{@{} >{\raggedright\arraybackslash}X r @{}}
\toprule
\textbf{Metrik} & \textbf{Nilai} \\
\midrule
Jumlah episode unik (instruksi global) & 401 \\
Jumlah subtugas berorientasi target & 1096 \\
Jumlah segmen total & 5253 \\
Jumlah target unik & 265 \\
Rata-rata segmen per subtugas (median; maksimum) & 4{,}79 (5; 21) \\
Panjang segmen (langkah): median; P90; maksimum & 6; 25; 116 \\
Panjang horizon per episode (langkah): rata-rata; median; P90; maksimum & 155{,}77; 140; 268; 474 \\
Jumlah tag konteks visual unik & 291 \\
Tag unik per episode: rata-rata (median; maksimum) & 23{,}92 (23; 58) \\
\bottomrule
\end{tabularx}}
\end{table}

Panjang horizon episode dihitung sebagai $\textit{max\_end} - \textit{min\_start} + 1$ pada indeks langkah. Distribusi horizon pada Gambar~\ref{fig:horizon_dist_tail_bottom} menunjukkan nilai rata-rata 155{,}77 langkah dan median 140 langkah, dengan \textit{tail} yang menonjol pada P90 = 268 langkah hingga maksimum 474 langkah. Secara kuantitatif, terdapat 41 episode (10{,}22\%) pada wilayah \textit{tail} (horizon $\ge$ P90) dan 21 episode (5{,}24\%) pada horizon $\ge$ P95 (335 langkah). Pola ini memperkuat bahwa episode yang dibentuk memenuhi karakteristik \textit{long-horizon} dalam arti memiliki rentang langkah yang panjang serta variasi tingkat kesulitan antar-episode.

\begin{figure}[H]
\centering
\includegraphics[width=\textwidth, trim={0mm 2mm 0mm 6.2mm}, clip]{images/fig1_horizon_distribution_tail_bottom_color.pdf}
\caption{Distribusi Panjang Horizon Episode dan \textit{Tail} (Horizon $\ge$ P90).}
\label{fig:horizon_dist_tail_bottom}
\end{figure}

\subsection{Struktur Multitarget pada Episode \textit{Long-Horizon}}
Struktur multi-tahap tercermin dari banyaknya target per episode. Distribusi jumlah target per episode adalah: 1 target (6{,}23\%), 2 target (31{,}67\%), 3 target (47{,}38\%), 4 target (12{,}72\%), 5 target (1{,}25\%), dan 6 target (0{,}75\%). Dominasi episode dengan 2--4 target menunjukkan bahwa mayoritas instruksi global memerlukan rangkaian pencapaian target berurutan, yang sejalan dengan tujuan pembangkitan dataset \textit{long-horizon}. Ringkasan distribusi tersebut disajikan pada Tabel~\ref{tab:targets_per_episode}.

\begin{table}[H]
\centering
\caption{Distribusi Jumlah Target per Episode}
\label{tab:targets_per_episode}
{\fontsize{10}{12}\selectfont
\renewcommand{\arraystretch}{1.15}
\setlength{\tabcolsep}{10pt}
\begin{tabular*}{\textwidth}{@{\extracolsep{\fill}} c r r @{}}
\toprule
\textbf{Jumlah Target} & \textbf{Jumlah Episode} & \textbf{Persentase} \\
\midrule
1 & 25  & 6{,}23\% \\
2 & 127 & 31{,}67\% \\
3 & 190 & 47{,}38\% \\
4 & 51  & 12{,}72\% \\
5 & 5   & 1{,}25\% \\
6 & 3   & 0{,}75\% \\
\bottomrule
\end{tabular*}}
\end{table}

\subsection{Karakteristik Segmentasi Aksi}
Segmentasi bertujuan memadatkan rangkaian aksi primitif menjadi unit instruksional yang lebih stabil secara semantik. Distribusi label aksi pada level segmen (Gambar~\ref{fig:action_label_dist_fig}) menunjukkan dominasi \textit{move\_forward} sebesar 59{,}91\%, diikuti \textit{turn\_right} (16{,}30\%) dan \textit{turn\_left} (16{,}24\%). Selain itu, terdapat variasi label \textit{make a left turn} (3{,}90\%) dan \textit{make a right turn} (3{,}66\%). Keberadaan variasi sinonim ini mengindikasikan perlunya normalisasi label pada pra-pemrosesan (misalnya pemetaan \textit{make a left/right turn} ke \textit{turn\_left/right}) agar konsisten dengan himpunan aksi yang digunakan pada pemodelan maupun evaluasi.

\begin{table}[H]
\centering
\caption{Distribusi Label Aksi pada Segmen (Lima Label Teratas)}
\label{tab:action_label_dist_table}
{\fontsize{10}{12}\selectfont
\renewcommand{\arraystretch}{1.15}
\setlength{\tabcolsep}{10pt}
\begin{tabular*}{\textwidth}{@{\extracolsep{\fill}} l r r @{}}
\toprule
\textbf{Label Aksi} & \textbf{Jumlah Segmen} & \textbf{Persentase} \\
\midrule
\textit{move\_forward}      & 3147 & 59{,}91\% \\
\textit{turn\_right}        & 856  & 16{,}30\% \\
\textit{turn\_left}         & 853  & 16{,}24\% \\
\textit{make a left turn}   & 205  & 3{,}90\% \\
\textit{make a right turn}  & 192  & 3{,}66\% \\
\bottomrule
\end{tabular*}}
\end{table}

\begin{figure}[H]
\centering
\includegraphics[width=\textwidth, trim={0mm 5.5mm 0mm 6.2mm}, clip]{images/fig2_action_label_distribution_color.pdf}
\caption{Distribusi Label Aksi pada Level Segmen.}
\label{fig:action_label_dist_fig}
\end{figure}

\subsection{Kekayaan Konteks Visual sebagai Pembatas \textit{Grounding}}
Setiap segmen disertai tag konteks visual yang merangkum informasi lingkungan pada rentang langkah segmen tersebut. Pada dataset ini, hampir seluruh segmen memiliki tepat lima tag konteks visual (hanya satu segmen memiliki empat tag), sehingga representasi konteks bersifat konsisten antarsegmen. Secara agregat terdapat 291 tag unik. Pada level episode, jumlah tag unik memiliki rata-rata 23{,}92 (median 23) dan maksimum 58. Distribusi beserta kurva kumulatif (CDF) pada Gambar~\ref{fig:unique_tags_cdf} memperlihatkan bahwa 90\% episode memiliki tag unik $\le$ 36 (P90 = 36) dan 95\% episode memiliki tag unik $\le$ 41 (P95 = 41). Dengan demikian, episode \textit{long-horizon} tidak hanya panjang dari sisi langkah, tetapi juga kaya variasi konteks lintas ruang dan \textit{landmark}.

Dalam sintesis instruksi bilingual berbasis LLM, tag konteks visual berfungsi sebagai \textit{constraint} eksplisit untuk menjaga agar instruksi yang dihasilkan tetap \textit{grounded}, yakni hanya merujuk pada ruang/\textit{landmark}/objek yang benar-benar teramati pada episode. Pembatasan eksplisit ini relevan untuk menekan risiko \textit{hallucination} berupa penambahan detail lingkungan yang tidak didukung oleh observasi visual episode.

\begin{figure}[H]
\centering
\includegraphics[width=\textwidth, trim={0mm 2mm 0mm 6.3mm}, clip]{images/fig3_unique_tags_distribution_color_cdf.pdf}
\caption{Distribusi Jumlah \textit{Unique Scene Tags} per Episode dan Kurva CDF.}
\label{fig:unique_tags_cdf}
\end{figure}

\subsection{Analisis Kompleksitas Episode pada Ruang Fitur 3D dan Proyeksi 2D}
Untuk memahami variasi kompleksitas episode secara kompak, setiap episode diproyeksikan ke ruang fitur 3D: panjang horizon (sumbu-$x$), jumlah segmen (sumbu-$y$), dan jumlah target (sumbu-$z$). Gambar~\ref{fig:complexity_3d} menampilkan sebaran episode dengan pewarnaan berdasarkan jumlah target. Secara statistik, korelasi Pearson menunjukkan hubungan positif yang kuat antara horizon dan jumlah segmen ($r=0{,}780$), serta korelasi positif antara horizon dan jumlah target ($r=0{,}644$) dan antara segmen dan target ($r=0{,}569$). Artinya, episode dengan lebih banyak target cenderung memiliki horizon lebih panjang dan tersusun dari lebih banyak segmen, meskipun variasi struktur jalur tetap muncul pada target yang sama.

Agar interpretasi pada dokumen statis lebih mudah, proyeksi 2D ditampilkan secara terpisah: horizon vs segmen (Gambar~\ref{fig:proj_xy}), horizon vs target (Gambar~\ref{fig:proj_xz}), dan segmen vs target (Gambar~\ref{fig:proj_yz}). Proyeksi tersebut membantu mengidentifikasi (i) episode dengan horizon panjang yang juga memiliki banyak segmen (wilayah kanan-atas pada \textit{projection} XY), serta (ii) kecenderungan peningkatan horizon/segmen seiring bertambahnya target, tanpa meniadakan adanya tumpang tindih (\textit{overlap}) antar-kelas target.

\begin{figure}[H]
\centering
\includegraphics[width=\textwidth, trim={0mm 5mm 0mm 20mm}, clip]{images/fig4_episode_complexity_3d_color.pdf}
\caption{Kompleksitas Episode pada Ruang Fitur 3D (Diwarnai Berdasarkan Jumlah Target).}
\label{fig:complexity_3d}
\end{figure}

\begin{figure}[H]
\centering
\includegraphics[width=\textwidth]{images/fig4_xy_projection_color.pdf}
\caption{\textit{Projection} XY: Horizon terhadap Jumlah Segmen (Diwarnai Berdasarkan Jumlah Target).}
\label{fig:proj_xy}
\end{figure}

\begin{figure}[H]
\centering
\includegraphics[width=\textwidth]{images/fig4_xz_projection_color.pdf}
\caption{\textit{Projection} XZ: Horizon terhadap Jumlah Target.}
\label{fig:proj_xz}
\end{figure}

\begin{figure}[H]
\centering
\includegraphics[width=\textwidth]{images/fig4_yz_projection_color.pdf}
\caption{\textit{Projection} YZ: Jumlah Segmen terhadap Jumlah Target.}
\label{fig:proj_yz}
\end{figure}

\subsection{Ilustrasi Kualitatif Segmentasi Aksi per Subtugas}
Selain ringkasan kuantitatif, visualisasi timeline segmentasi memperlihatkan bagaimana segmen aksi terbentuk dan tersusun per subtugas target. Pada setiap timeline, sumbu-$x$ menyatakan indeks langkah, sumbu-$y$ menyatakan urutan target/subtugas, warna merepresentasikan label aksi, dan anotasi singkatan (mis. MF, TL, TR) memudahkan pembacaan pola aksi. Tujuh contoh pada Gambar~\ref{fig:timeline_ex1}--\ref{fig:timeline_ex7} dipilih untuk merepresentasikan variasi horizon dari kuantil rendah hingga mendekati P90.

Secara ringkas, ketujuh contoh tersebut memperlihatkan beberapa pola penting: (i) episode dengan horizon moderat dapat memiliki jumlah segmen yang cukup tinggi jika terjadi banyak pergantian aksi belok, (ii) episode dengan horizon panjang dapat terbentuk dari segmen yang relatif sedikit namun berdurasi panjang (kompresi kuat), dan (iii) pada episode mendekati \textit{tail}, satu target tertentu dapat mendominasi jumlah segmen sehingga memperbesar kompleksitas eksekusi subtugas.

\begin{figure}[H]
\centering
\includegraphics[width=\textwidth]{images/fig5_timeline_example_01.pdf}
\caption{Contoh 1 Timeline Segmentasi Aksi per Target (Horizon 72, Segmen 9, Target 3).}
\label{fig:timeline_ex1}
\end{figure}

\begin{figure}[H]
\centering
\includegraphics[width=\textwidth]{images/fig5_timeline_example_02.pdf}
\caption{Contoh 2 Timeline Segmentasi Aksi per Target (Horizon 103, Segmen 13, Target 3).}
\label{fig:timeline_ex2}
\end{figure}

\begin{figure}[H]
\centering
\includegraphics[width=\textwidth]{images/fig5_timeline_example_03.pdf}
\caption{Contoh 3 Timeline Segmentasi Aksi per Target (Horizon 129, Segmen 11, Target 3).}
\label{fig:timeline_ex3}
\end{figure}

\begin{figure}[H]
\centering
\includegraphics[width=\textwidth]{images/fig5_timeline_example_04.pdf}
\caption{Contoh 4 Timeline Segmentasi Aksi per Target (Horizon 146, Segmen 9, Target 3).}
\label{fig:timeline_ex4}
\end{figure}

\begin{figure}[H]
\centering
\includegraphics[width=\textwidth]{images/fig5_timeline_example_05.pdf}
\caption{Contoh 5 Timeline Segmentasi Aksi per Target (Horizon 163, Segmen 11, Target 3).}
\label{fig:timeline_ex5}
\end{figure}

\begin{figure}[H]
\centering
\includegraphics[width=\textwidth]{images/fig5_timeline_example_06.pdf}
\caption{Contoh 6 Timeline Segmentasi Aksi per Target (Horizon 193, Segmen 7, Target 3).}
\label{fig:timeline_ex6}
\end{figure}

\begin{figure}[H]
\centering
\includegraphics[width=\textwidth]{images/fig5_timeline_example_07.pdf}
\caption{Contoh 7 Timeline Segmentasi Aksi per Target (Horizon 267, Segmen 17, Target 3).}
\label{fig:timeline_ex7}
\end{figure}

\subsection{Pemeriksaan Konsistensi Internal dan Keterlacakan}
Agar hasil dapat dipertanggungjawabkan secara ilmiah, dilakukan pemeriksaan konsistensi internal dan keterlacakan sebagai berikut:
\begin{enumerate}[label=(\arabic*), left=0pt, labelsep=0.33cm, itemsep=0pt, topsep=0pt]
  \item Konsistensi kardinalitas: jumlah segmen total pada level segmen konsisten dengan penjumlahan jumlah segmen pada level episode (5253 segmen), dan jumlah subtugas konsisten dengan pasangan unik episode--target (1096 subtugas).
  \item Integritas referensial: setiap segmen dapat ditelusuri kembali ke subtugas target dan episode induknya melalui \textit{identifier} episode/subtugas, sehingga tidak terdapat segmen ``yatim'' (\textit{orphan segments}).
  \item Konsistensi agregat: metrik ringkasan episode (misalnya horizon dan jumlah segmen) dapat dihitung ulang dari data \textit{segment-level} tanpa perbedaan.
  \item Audit proses: jejak pembentukan subtugas menyimpan informasi rentang langkah dan instruksi subtugas, sehingga setiap instruksi dapat ditelusuri ke urutan segmen serta konteks visualnya.
\end{enumerate}
Pemeriksaan ini memastikan bahwa representasi hasil segmentasi dan konteks visual bersifat konsisten, dapat direproduksi, dan dapat diaudit.
\vspace{0.5em}

\subsection{Implikasi terhadap Evaluasi Navigasi dan Analisis \textit{Code-Switching}}
Representasi subtugas target memungkinkan evaluasi yang lebih granular dibanding hanya menilai keberhasilan pada akhir episode. Metrik navigasi dapat dihitung per subtugas lalu diagregasi pada level episode, sehingga analisis kegagalan dapat dikaitkan dengan bagian trajektori yang spesifik (misalnya segmen belokan berulang atau perpindahan antar ruang). Pada sisi linguistik, instruksi bilingual yang disintesis dari representasi terstruktur yang sama memungkinkan evaluasi kualitas \textit{code-switching} (misalnya rasio peralihan bahasa, konsistensi referensi target, serta kewajaran titik \textit{switch}) secara lebih terkontrol, karena keluaran LLM dibatasi oleh rencana aksi dan konteks visual episode.
\vspace{0.5em}

% \chapter{KESIMPULAN DAN SARAN}
\label{Bab5}
\section{Kesimpulan}
Berdasarkan rancangan dan pengujian perangkat yang telah dibuat dapat ditarik kesimpulan seperti berikut:
\begin{enumerate}[left=0pt, labelsep=0.5em,itemsep=0pt,label=\arabic*. ]
	\item Kesimpulan no 1.
	\item Kesimpulan no 2.
	\item dan seterusnya...
\end{enumerate}

\section{Saran}
Hasil rancangan masih mengandung kekurangan untuk diperbaiki. Beberapa saran diberikan untuk perbaikan pada penelitian selanjutnya seperti berikut:
\begin{enumerate}[left=0pt, labelsep=0.5em,itemsep=0pt,label=\arabic*. ]
	\item Saran no 1.
	\item Saran no 2.
	\item dan seterusnya...
\end{enumerate}


\backmatter
\phantomsection
\addcontentsline{toc}{chapter}{DAFTAR PUSTAKA}

\printbibliography[title={DAFTAR PUSTAKA}]
\chapter{LAMPIRAN}
\section*{\textbf{Lampiran 1:} Desain \textit{Prompt} untuk Pembangkitan Tugas LH-VLN \textit{Code-Switching} Indonesia-Inggris}
\label{appendix: prompts}
\vspace{0.5em}

% =========================
% Lampiran 1.1
% =========================
\subsection*{\textbf{Lampiran 1.1:} \textit{Prompt} Sistem Navigasi LH-VLN (Pembangkitan Instruksi)}
\label{appendix:prompt-system-nav}
\begin{tcolorbox}[breakable, colback=white, colframe=black, boxrule=0.5pt]
\small
\textbf{[system]:}\\
You are good at guiding the way. Convert a structured action plan into ONE fluent, practical navigation instruction in code-switching Indonesia--English. Keep it grounded in the provided tags and easy to follow.

\medskip
Style and constraints for the SINGLE output paragraph:
\begin{itemize}
  \item Code-switching ratio: aim for 60--80\% Indonesian and 20--40\% English.
  \item Use imperative voice (e.g., ``Pergi\ldots'', ``Lanjut\ldots'', ``Belok\ldots'', ``move\ldots'', ``turn\ldots''), no slang, no bullet points, no numbering, no explicit word ``step''.
  \item Do NOT use the literal words ``grab'' or ``release''.
  \item Mention ONLY rooms/objects that appear in the input tags or the input target.
  \item Keep to 1--2 regions overall (infer region/room from tags; do not introduce new rooms).
\end{itemize}

\textbf{[user]:}\\
The INPUT is a dictionary with this shape:
\begin{quote}
\{\\
\ \ ``target'': $<$final navigation target object$>$,\\
\ \ ``step\_0'': \{``action'': $<$move\_forward$|$turn\_left$|$turn\_right$>$, ``tags'': [$<$scene tags$>$]\},\\
\ \ \ldots\\
\ \ ``step\_n'': \{``action'': $<$move\_forward$|$turn\_left$|$turn\_right$>$, ``tags'': [$<$scene tags$>$]\}\\
\}
\end{quote}

Guidelines:
\begin{itemize}
  \item For each step, choose ONE most relevant tag to mention (do not list many tags).
  \item Steps with move\_forward should prefer a specific region/room tag; turn\_left/turn\_right may prefer a specific object/landmark tag.
  \item The order of actions in the final instruction MUST follow the steps order.
  \item The last part MUST naturally reach the target object.
  \item Output only the final instruction string (no quotes, no extra text, no JSON, no code fences).
\end{itemize}

Here is a tiny example (for style only, not to be reused verbatim):

INPUT (snippet):
\begin{quote}
\{\\
\ \ ``target'': ``desk'',\\
\ \ ``step\_0'': \{``action'': ``move\_forward'', ``tags'': [``bedroom'']\},\\
\ \ ``step\_1'': \{``action'': ``turn\_left'', ``tags'': [``picture'']\}\\
\}
\end{quote}

DESIRED STYLE (example):\\
Jalan pelan dari bedroom, then turn left di dekat picture, lanjutkan sampai mencapai desk.

\medskip
Now generate the final instruction for the following INPUT. Think briefly, then answer with the instruction only.
\end{tcolorbox}
% \clearpage


% =========================
% Lampiran 1.2
% =========================
\subsection*{\textbf{Lampiran 1.2:} \textit{Prompt} Aturan Subtugas LH-VLN}
\label{appendix:prompt-rules}
\begin{tcolorbox}[breakable, colback=white, colframe=black, boxrule=0.5pt]
\small
The task you create must be decomposed into navigation--interaction subtasks using ONLY the following three function formats:
\begin{itemize}
\item \texttt{Move\_to("<object>\_<regionId>")}: Move to an object within a region. NOTE: \texttt{<object>\_<regionId>} MUST combine the object name and the corresponding region ID of that object in the input scene (e.g., \texttt{"cup\_3"}).
\item \texttt{Grab("<object>")}: Pick up an object after reaching it. NOTE: the \texttt{<object>} MUST be present in the input scene and MUST be portable.
\item \texttt{Release("<object>")}: Place the object currently held at the current location. NOTE: the \texttt{<object>} MUST match the object being held and MUST be present in the input scene.
\end{itemize}

There are several things to note:
\begin{itemize}
\item The output must use code-switching Indonesian--English with a ratio of 60--80\% Indonesian and 20--40\% English.
\item IMPORTANT: Do NOT use the literal words ``grab'' or ``release'' in the \textbf{Task instruction} (natural language). Using \texttt{Grab(...)} and \texttt{Release(...)} as function names in the \textbf{Subtask list} is allowed.
\item Objects mentioned MUST be limited to 1--2 regions total. You must explicitly name these regions in the \textbf{Task instruction}. All regions must exist in the input scene.
\item The overall assignment should be similar to: ``Ambil objek portable dari suatu region, lalu move ke region lain dan letakkan di lokasi tertentu, kemudian ambil objek lain'' (tetap dalam 1--2 region).
\item The task must contain 2--6 subtasks.
\item Subtask ordering must be logically consistent: \texttt{Grab} can only happen after moving to that object; \texttt{Release} can only happen after \texttt{Grab}.
\item The region ID in each \texttt{Move\_to} subtask MUST match the region named in the \textbf{Task instruction}.
\item The task should be practical and reasonable for a household robot, considering the robot characteristics given in the input.
\end{itemize}

Your output MUST be a Python dictionary with exactly two keys:
\begin{itemize}
\item \texttt{dictionary["Task instruction"]}: One conversational, imperative instruction (no bullets/numbering).
\item \texttt{dictionary["Subtask list"]}: A list of subtasks (2--6 items) using the function formats above.
\end{itemize}
\end{tcolorbox}
% \clearpage


% =========================
% Lampiran 1.3
% =========================
\subsection*{\textbf{Lampiran 1.3:} \textit{Prompt} Deskripsi Robot Spot di Simulator}
\label{appendix:robot-spot}
\begin{tcolorbox}[breakable, colback=white, colframe=black, boxrule=0.5pt]
\small
Spot in the simulator is an agile, quadrupedal robot.\\
\textbf{Action:} Spot supports three actions: \texttt{move\_forward}, \texttt{turn\_left}, and \texttt{turn\_right}.\\
\textbf{Sensors:} Spot is equipped with three RGB cameras at a height of 0.5 meters (front, left, and right) to capture embodied images from these directions.\\
\textbf{Mobility:} Spot's four-legged design allows it to navigate challenging terrains, including stairs, rocky surfaces, and cluttered environments.\\
\textbf{Use:} Since Spot is dog-shaped, it can assist humans in domestic scenes. It has a simple robotic arm positioned low (0.5 meters) and can perform basic pick-and-place.
\end{tcolorbox}
% \clearpage


% =========================
% Lampiran 1.4
% =========================
\subsection*{\textbf{Lampiran 1.4:} \textit{Prompt} Deskripsi Robot Fetch di Simulator}
\label{appendix:robot-fetch}
\begin{tcolorbox}[breakable, colback=white, colframe=black, boxrule=0.5pt]
\small
The Fetch robot in the simulator is a versatile mobile robot designed for logistics and material handling.\\
\textbf{Action:} Fetch supports three actions: \texttt{move\_forward}, \texttt{turn\_left}, and \texttt{turn\_right}.\\
\textbf{Sensors:} Fetch is equipped with three RGB cameras at a height of 1 meter (front, left, and right) to capture embodied images from these directions.\\
\textbf{Mobility:} Fetch moves with a wheeled base, so it can only travel on flat ground and cannot traverse rugged terrain.\\
\textbf{Use:} Fetch has a flexible robotic arm mounted higher (1 meter) that can perform more complex pick-and-place operations.
\end{tcolorbox}
% \clearpage


% =========================
% Lampiran 1.5
% =========================
\subsection*{\textbf{Lampiran 1.5:} \textit{Prompt} Sistem Desain Tugas LH-VLN}
\label{appendix:prompt-system-taskdesign}
\begin{tcolorbox}[breakable, colback=white, colframe=black, boxrule=0.5pt]
\small
\textbf{[system]:}\\
You are proficient in planning and dataset design. You will design a practical long-horizon vision-language navigation task with object interaction based on a scene graph and robot characteristics, and then output a consistent subtask decomposition.

\medskip
\textbf{[user]:}\\
There are two parts of the input: \texttt{scene} and \texttt{robot}. The \texttt{scene} includes regions and their objects. The \texttt{robot} describes capabilities and constraints (e.g., mobility, camera height, arm reach), which you must consider.

\medskip
Input format:
\begin{quote}
scene: \{\\
\ \ ``Region 1: <name>'': [<objects>],\\
\ \ ``Region 2: <name>'': [<objects>],\\
\ \ \ldots\\
\}\\
robot: <robot description text>
\end{quote}

\medskip
Your output MUST follow these requirements:
\begin{itemize}
\item Output a Python dictionary with keys \texttt{"Task instruction"} and \texttt{"Subtask list"}.
\item The \texttt{"Task instruction"} must be ONE conversational imperative paragraph in code-switching Indonesian--English (60--80\% Indonesian, 20--40\% English), no bullets/numbering, no slang.
\item Use ONLY objects and regions from the given \texttt{scene}, and limit the task to 1--2 regions total (explicitly named in the instruction).
\item Include 2--6 subtasks using ONLY: \texttt{Move\_to("<object>\_<regionId>")}, \texttt{Grab("<object>")}, \texttt{Release("<object>")}.
\item Do NOT use the literal words ``grab'' or ``release'' in the natural-language instruction (function names in the subtask list are allowed).
\item Ensure logical consistency and portability: you can only pick up portable objects; order must be \texttt{Move\_to} $\rightarrow$ \texttt{Grab} $\rightarrow$ (optional \texttt{Move\_to}) $\rightarrow$ \texttt{Release}.
\item Make the task realistic under robot constraints (e.g., wheeled robots stay on flat indoor areas).
\end{itemize}

Output only the Python dictionary (no extra explanation, no code fences).
\end{tcolorbox}
% \clearpage


% =========================
% Lampiran 1.6
% =========================
\subsection*{\textbf{Lampiran 1.6:} Contoh Lengkap INPUT--OUTPUT Tugas LH-VLN}
\label{appendix:example-io}
\begin{tcolorbox}[breakable, colback=white, colframe=black, boxrule=0.5pt]
\small
Here is an example of the INPUT and OUTPUT:

\medskip
\textbf{INPUT:}
\begin{quote}
scene: \{\\
\ \ ``Region 1: Bedroom'': [``picture'', ``bed'', ``lamp''],\\
\ \ ``Region 2: Bathroom'': [``towel'', ``sink'', ``toilet paper dispenser''],\\
\ \ ``Region 3: Kitchen'': [``pan'', ``cup'', ``plate''],\\
\ \ ``Region 4: Office'': [``board'', ``chair'', ``desk'']\\
\}\\
robot: The Fetch robot in simulator is a versatile mobile robot designed for logistics and material handling. It supports \texttt{move\_forward}, \texttt{turn\_left}, and \texttt{turn\_right}. It has three RGB cameras at 1 meter height and a flexible arm for pick-and-place, but it moves on a wheeled base and stays on flat ground.
\end{quote}

\medskip
\textbf{OUTPUT:}
\begin{quote}
\{\\
\ \ ``Task instruction'': ``Ambil picture dari Bedroom, lalu go to Office dan letakkan di desk; setelah itu, tetap di Office dan ambil chair untuk dipindahkan sesuai kebutuhan.'',\\[2pt]
\ \ ``Subtask list'': [\\
\ \ \ \ ``Move\_to('picture\_1')'',\\
\ \ \ \ ``Grab('picture')'',\\
\ \ \ \ ``Move\_to('desk\_4')'',\\
\ \ \ \ ``Release('picture')'',\\
\ \ \ \ ``Move\_to('chair\_4')'',\\
\ \ \ \ ``Grab('chair')''\\
\ \ ]\\
\}
\end{quote}

\medskip
\textit{Note:} The task uses only two regions (Bedroom and Office), and the subtasks follow a consistent order.
\end{tcolorbox}

\clearpage % atau \newpage

\section*{\textbf{Lampiran 2:} Contoh Data yang Digunakan dan/atau Dihasilkan dari \textit{Pipeline}}
\vspace{0.5em}

\subsection*{\textbf{Lampiran 2.1:} Contoh Metadata \textit{Scene}}
\begin{lstlisting}[basicstyle=\rmfamily\normalsize, breaklines=true]
Region id:_-1, position:[0. 0. 0.]
Region id:_0, position:[0. 0. 0.]
Id:2, name:armchair, position:[4.771444, 3.526179, -4.256199]
Id:3, name:plant, position:[1.6414305, 4.1937037, -4.5211315]
Id:4, name:chest of drawers, position:[1.4356585, 3.5864055, -4.4750156]
...
Region id:_10, position:[0. 0. 0.]
Id:286, name:chest of drawers, position:[-3.629725, 0.44680634, -1.6254207]
Id:293, name:tv, position:[-4.007966, 1.1502895, -1.0507892]
...
Region id:_15, position:[0. 0. 0.]
Id:703, name:toilet paper dispenser, position:[-1.7919614, -2.2921019, 2.544517]
\end{lstlisting}
\clearpage

\subsection*{\textbf{Lampiran 2.2:} Contoh Gambar Navigasi Agen dari \textit{Start} Hingga \textit{Finish}}
\noindent Instruksi: "Pergi ke \textit{laundry room} dan \textit{pick up the towel}, kemudian \textit{move to bathroom and place it neatly on the shelf}"
\begin{figure}[H]
    \centering
    \includegraphics[
        width=1\textwidth,
    ]{images/contact_sheets/contact_sheet_01_step_-1_to_28.png}
    \vspace{-0.5em} % atur sendiri: -0.5em, -1em, -5pt, dll
    \includegraphics[
        width=1\textwidth,
    ]{images/contact_sheets/contact_sheet_02_step_29_to_58.png}
    % \caption{...}
    \label{fig:contact_sheet_1-2}
\end{figure}

\begin{figure}[H]
    \centering
    \includegraphics[
        width=1\textwidth,
    ]{images/contact_sheets/contact_sheet_03_step_59_to_87.png}
    \vspace{-0.5em} % atur sendiri: -0.5em, -1em, -5pt, dll
    \includegraphics[
        width=1\textwidth,
    ]{images/contact_sheets/contact_sheet_04_step_88_to_117.png}
    % \caption{...}
    \label{fig:contact_sheet_3-4}
\end{figure}

\begin{figure}[H]
    \centering
    \includegraphics[
        width=1\textwidth,
    ]{images/contact_sheets/contact_sheet_05_step_118_to_147.png}
    \vspace{-0.5em} % atur sendiri: -0.5em, -1em, -5pt, dll
    \includegraphics[
        width=1\textwidth,
    ]{images/contact_sheets/contact_sheet_06_step_148_to_177.png}
    % \caption{...}
    \label{fig:contact_sheet_5-6}
\end{figure}

\begin{figure}[H]
    \centering
    \includegraphics[
        width=1\textwidth,
    ]{images/contact_sheets/contact_sheet_07_step_178_to_207.png}
    \vspace{-0.5em} % atur sendiri: -0.5em, -1em, -5pt, dll
    \includegraphics[
        width=1\textwidth,
    ]{images/contact_sheets/contact_sheet_08_step_208_to_214.png}
    % \caption{...}
    \label{fig:contact_sheet_7-8}
\end{figure}
\clearpage

\subsection*{\textbf{Lampiran 2.3:} Contoh Format Data JSON untuk Instruksi Tugas dan Subtugas}
\begin{lstlisting}[language=json, basicstyle=\rmfamily\normalsize, breaklines=true]
{
  "Task instruction": "Pergi ke laundry room dan pick up the towel, kemudian move to bathroom and place it neatly on the shelf",
  "Subtask list": [
    "Move_to('towel_0')",
    "Grab('towel')",
    ...
    "Release('towel')"
  ],
  "Robot": "stretch",
  "Scene": "00685-ENiCjXWB6aQ",
  "Object": ["towel", "bathroom shelf"],
  "Region Name": ["Bathroom", "Bathroom"],
  "Region": ["0", "14"],
  "Geo dis": [12.3853874, 25.3590908],
  "trial": {
    "trial_0": {
      "pos": [
        [-2.6512451, 1.9263153, 6.9955611],
        [-2.6512451, 1.6535878, 6.7455611],
        ...,
        [ 3.0860734, 0.1888131, 0.6243353]
      ],
      "yaw": [180, 180, 180, ..., 90],
      "action": ["stop", "move_forward", "move_forward", ..., "stop"]
    },
    "trial_1": { ... }
  }
}
\end{lstlisting}
\clearpage

\subsection*{\textbf{Lampiran 2.4:} Contoh Format Data Pembangkitan Subtugas yang Diperinci}
\begin{lstlisting}[language=json, basicstyle=\rmfamily\normalsize, breaklines=true]
{
    "trajectory path": "task/2/Pergi ke laundry room dan pick up the towel, kemudian move to bathroom and place it neatly on the shelf/success/trial_1",
    "start": 0,
    "end": 69,
    "Robot": "stretch",
    "Scene": "00685-ENiCjXWB6aQ",
    "target": [
        "towel"
    ],
    "Region": [
        "0"
    ],
    "start_pos": [
        -2.6512451171875,
        1.9263153076171875,
        6.995561122894287
    ],
    "start_yaw": 180,
    "Task instruction": "Pergi dari balustrade ke hallway, lalu putar kanan di dekat picture frame, lanjut ke bathroom, kemudian belok kanan sampai mencapai towel."
}
\end{lstlisting}
\clearpage

% \include{backmatter/lampiranB}
% \chapter{LAMPIRAN 3: package.h dan package.cpp}
\section*{package.h}
\scriptsize
\begin{verbatim}
 #ifndef PAC_H
#define PAC_H

//Jumlah yang ada didalam vektor integer $prod
int pac_prod(std::vector<int> S);

//Rem merupakan nilai sisa pembagian $rem
int pac_rem(int a, int b);

//besar adalah paket untuk mencari nilai lebih besar jika > 1, < 0. $besar
std::vector<int> pac_besar(std::vector<int> m, int s);

//Paket untuk menjadikan vektor menjadi matrix Diagonal $Diag
std::vector<std::vector<int> > pac_Diag(std::vector<int> m);

//Matrix to Vector $diag
std::vector<int> pac_diag(std::vector<std::vector<int> > M);
#endif
\end{verbatim}
\normalsize
\section*{package.cpp}
\scriptsize
\begin{verbatim}
#include "edft.h"
#include "package.h"

typedef std::vector<int> vector_int;
typedef std::vector<double> vector_double;
typedef std::vector<vector_int> vector_vector;

//c-%pac_prod Jumlah yang ada didalam vektor integer $prod
int pac_prod(std::vector<int> S)
{
 int hasil=1;
 for(int i=0;i<S.size();i++)
  {
   hasil=hasil*S[i];
  } 
 return hasil;
 S.clear();
}

//c- %pac_rem merupakan nilai sisa pembagian $rem 
int pac_rem(int a, int b)
{
 int hasil;
 div_t divresult;
 divresult=div(a,b);
 hasil = divresult.rem;
 return hasil;
}

//c-%pac_besar merupakan fungsi untuk mengetahui nilai lebih besar, jika
//lebih besar, maka hasilnya satu, jika lebih kecil nilainya 0. $besar
std::vector<int> pac_besar(std::vector<int> m, int s)
{
 std::vector<int> hasil(m.size());
 for(int i=0;i<m.size();i++)
  {
   int sementara;
    if(m[i]>s/2) sementara=1;
      else sementara=0;
   hasil[i]=m[i]-(sementara)*s;
  }
 return hasil;
 hasil.clear(); m.clear();
}

//c-%pac_Diag Package untuk merubah vektor ke matriks Diagonal. $Diag
vector_vector pac_Diag(vector_int m)
{
 vector_vector hasil;
 vector_int sementara(m.size());
 for (int i=0;i<m.size();i++)
  {
   for (int j=0;j<m.size();j++)
    {
      if (i==j) sementara[j]=m[j]; else sementara[j]=0;
    }
   hasil.push_back(sementara);
  }
  return hasil;
  hasil.clear(); m.clear(); sementara.clear();
}

//c- %pac_diag Matrix to Vector $diag
vector_int pac_diag(vector_vector M)
{
 vector_int hasil(M.size());
 for (int i=0;i<3;i++)
  {
   hasil[i]=M[i][i];
  }
 return hasil;
 hasil.clear(); M.clear();	
}
\end{verbatim}


% \chapter{LAMPIRAN 4: setup.cpp dan setup.h}
\section*{setup.h}
\scriptsize
\begin{verbatim}
 /*
#include <vector>
#include <iostream>
#include <stdlib.h>
#include <math.h>
#include "package.h"
#include "out.h"
#include "opmat.h"
#include "OpenBlas/include/cblas.h"
#include <string>
//#include <fstream.h>
*/

#define PI 3.1415926535897932384626433832795028841971693993751058209749445923078164062862089986280348253421170679

void set_setup(std::vector<int> S, std::vector<int> Latice, double *rout,double *g2out,double *gout);

//c-%set_slice() paket untuk menampilkan hasil slice
std::vector<std::vector<double> > set_slice(std::vector<double> dat, std::vector<int> S,int l,int k,int *ba,int *ko);

//c-%set_dplot: untuk membuat data yang bisa di plot.
void set_dplot(std::vector<double> dat,std::vector<int> S);
\end{verbatim}
\normalsize

\section*{setup.cpp}
\scriptsize
\begin{verbatim}
 //Memasukkan include paket

#include "edft.h"
#include "package.h"
#include "out.h"
#include "opmat.h"

#include "setup.h"

//c-Memasukkan definisi
#define PI 3.1415926535897932384626433832795028841971693993751058209749445923078164062862089986280348253421170679

//c-%set_setup
void set_setup(std::vector<int> S, std::vector<int> Latice, double *rout, double *g2out,double *gout)
{
 //c-Cara cepat untuk mendefinisikan vektor dan matrik 
 typedef std::vector<int> vector_int; 
 typedef std::vector<double> vector_double;
 typedef std::vector<vector_int> vector_vector;
 typedef std::vector<vector_double> vector_vector_double;

 int batas=pac_prod(S); //c-definisi %batas integer
 vector_int ms(batas), m1(batas), m2(batas), m3(batas); //c- %ms, %m1, %m2, %m3 vektor integer

 for (int i=0;i<batas;i++) //c-%ms(batas) o-ms=[0:prod(s)-1]
  { 
   ms[i]=i;
  }

 for (int i=0;i<batas;i++)
  {
   m1[i]=pac_rem(ms[i],S[0]); //o-rem(ms,S(1))
   m2[i]=pac_rem(floor(ms[i]/S[0]),S[1]); //o-m2=rem()
   m3[i]=pac_rem(floor(ms[i]/(S[0]*S[1])),S[2]);
  }  
// KeluarVector(m2);

 //c-Membuat vector M menjadi sebuah matriks. //o-M=[m1,m2,m3]
 vector_vector Mi; //c- %Mi matrix integer
 Mi.push_back(m1); //c- memasukkan vektor m1 kedalam M
 Mi.push_back(m2); //c-memasukkan vektor m2 kedalam M
 Mi.push_back(m3); //c-memasukkan vektor m3 kedalam M
// KeluarMatrixDouble(opmat_intodoub(Mi,Mi.size(),m1.size()),Mi.size(),m1.size());

 vector_vector_double M; //c-deklarasi matriks %M double
 M=opmat_transpose_double(opmat_intodoub(Mi,Mi.size(),m1.size()),Mi.size(),m1.size());
// KeluarMatrixDouble(M,M.size(),Mi.size()); std::cout<<"\n \n";
 
 //c-deklarasi vector baru
 vector_int n1(batas),n2(batas),n3(batas); //c-%n1, %n2, %3 vektor integer.

 n1=pac_besar(m1,S[0]);
 n2=pac_besar(m2,S[1]);
 n3=pac_besar(m3,S[2]);

 //o-N=[n1,n2,n3]
 vector_vector Ni; //c-%Ni matrix integer
 Ni.push_back(n1); 
 Ni.push_back(n2);
 Ni.push_back(n3);
// KeluarMatrix(Ni,Ni.size(),n1.size());

 vector_vector_double N; //c-%N matriks double
 N=opmat_transpose_double(opmat_intodoub(Ni,Ni.size(),n1.size()),Ni.size(),n1.size());  
// KeluarMatrixDouble(N,N.size(),Ni.size());

 vector_vector R; //c-%R matriks integer
 R=pac_Diag(Latice);  //c-vektor %Lattice dimasukkan dalam matriks R 
// KeluarMatrix(R,R.size(),R.size()); 


//c-Memasukkan sampling poin R dan G 
 vector_vector_double diS;
 diS=opmat_inverse(pac_Diag(S)); //c-%diS matriks integer dari Diag(S)yang di inverse
// KeluarMatrixDouble(diS,diS.size(),diS.size()); std::cout<<"\n \n"; 

//c-perkalian matriks M dengan diS %MdiS
 vector_vector_double MdiS;
// MdiS=opmat_kali(M,M.size(),Mi.size(),diS,diS.size(),diS.size());
// KeluarMatrixDouble(MdiS,M.size(),diS.size()); std::cout<<"\n \n";

//c-Perkalian matriks M dengan dis versi Blas
//c-Mengubah M dan Mdis kedalam vektor.
 vector_double Mv(Mi.size()*M.size());
 vector_double diSv(diS.size()*diS.size());
 vector_double MdiSv(M.size()*diS.size());
 Mv=opmat_mtv(M,M.size(),Mi.size());
 diSv=opmat_mtv(diS,diS.size(),diS.size()); 
// KeluarVectorDouble(Mv); std::cout<<"\n";KeluarVectorDouble(diSv);
//c-perkalian matriks M dg diS dengan cblas
 cblas_dgemm(CblasRowMajor,CblasNoTrans,CblasNoTrans,M.size(),diS.size(),diS.size(),1.0,&Mv[0],Mi.size(),&diSv[0],diS.size(),1.0,&MdiSv[0],diS.size());
// KeluarVectorDouble(MdiSv); std::cout<<"\n";
// KeluarMatrixDouble(opmat_vtm(MdiSv,M.size(),Mi.size()),M.size(),Mi.size());
 
 vector_vector_double r; //c-%r matrix double perkalian antara M dan S 
 vector_vector_double Rd; //c-%Rd deklarasi matriks R double
 Rd=opmat_transpose_double(opmat_intodoub(R,R.size(),R.size()),R.size(),R.size());
// KeluarMatrixDouble(Rd,R.size(),R.size());

//c- mencari real space, menyiapkan Rd bentuk vektor
 vector_double Rdv(Rd.size(),Rd.size());
 Rdv=opmat_mtv(Rd,Rd.size(),Rd.size());
 vector_double rv(M.size()*Mi.size()); //c-%rv deklarasi vektor rv, r dalam bentuk vektor
 cblas_dgemm(CblasRowMajor,CblasNoTrans,CblasNoTrans,M.size(),Rd.size(),Rd.size(),1.0,&MdiSv[0],Mi.size(),&Rdv[0],Rd.size(),1.0,&rv[0],Rd.size());

// KeluarMatrixDouble(opmat_vtm(rv,M.size(),Mi.size()),M.size(),Mi.size());
 
 //c-operasi mencari matriks r (real space sample point) //o-r=M(Diag(S))^{-1}R'
 r=opmat_vtm(rv,M.size(),Mi.size());
// KeluarMatrixDouble(r,M.size(),Rd.size());

 vector_vector_double Rinv; //c-Membuat matriks interger invers dari R
 Rinv=opmat_inverse(R);

 vector_vector_double Gi; //c-%Gi matriks latice sementara N*Rinv
 vector_vector_double G; //c-%G matriks lattice vector
 vector_double Nv(N.size()*Ni.size()); //c-%Nv matriks N dalam bentuk vector
 Nv=opmat_mtv(N,N.size(),Ni.size()); //c-konversi dari matriks->vector N
 vector_double Giv(N.size()*Ni.size()); //c-%Giv Gi dalam bentuk vector
 vector_double Gv(N.size()*Ni.size()); //c-%Gv G dalam bentuk Vector
 vector_double Rinvv(R.size()*R.size()); //c-%Rinvv Rinv dalam bentuk vektor.
 Rinvv=opmat_mtv(Rinv,R.size(),R.size()); 
 //c- menghitung Giv=N*Rinv
 cblas_dgemm(CblasRowMajor,CblasNoTrans,CblasNoTrans,N.size(),R.size(),R.size(),1.0,&Nv[0],Ni.size(),&Rinvv[0],Rinv.size(),1.0,&Giv[0],Rinv.size());
// KeluarMatrixDouble(opmat_vtm(Giv,N.size(),Ni.size()),N.size(),Ni.size());
//c-menghitung matriks G
 cblas_daxpy(N.size()*Ni.size(),2*PI,&Giv[0],1,&Gv[0],1);
// out_dm(opmat_vtm(Gv,N.size(),Ni.size()),N.size(),Ni.size()); std::cout<<"\n";
 G=opmat_vtm(Gv,N.size(),Ni.size());
// KeluarMatrixDouble(G,N.size(),Rinv.size());
 
 //c-cek kuadrat
 vector_vector_double G2;
// G2=opmat_kuadrat(G,N.size(),Rinv.size()); 
// KeluarMatrixDouble(G2,N.size(),Rinv.size());
 vector_double G2v(Gv.size());

 //c-versi vector.
// cblas_dgemv(CblasRowMajor,CblasNoTrans,1,1,1.0,&Gv[0],1,&Gv[0],1,1.0,&G2v[0],N.size());
 G2v=opmat_vkuadrat(Gv);
 G2=opmat_vtm(G2v,M.size(),Mi.size()); 
//KeluarVectorDouble(G2v);
// KeluarMatrixDouble(opmat_vtm(G2v,N.size(),Ni.size()),N.size(),Ni.size());

 //c-sum matriks to vector
 vector_double g2;
 g2=opmat_sumd(G2,N.size(),Rinv.size());
// out_dv(g2);
// std::cout<<"\n"<<rv.size();

 for (int i=0;i<rv.size();i++)
   {
     rout[i]=rv[i];
   }
 
 for (int i=0;i<g2.size();i++)
   {
     g2out[i]=g2[i]; 
   }

 vector_double gv;
 gv=opmat_mtv(G,N.size(),3);
 
 for (int i=0;i<rv.size();i++)
   {
    gout[i]=gv[i];
   }

 //c-Membersihkan memori yang telah selesai di gunakan.
 ms.clear(); Mi.clear(); M.clear(); n1.clear(); Ni.clear();
 N.clear(); R.clear(); diS.clear(); MdiS.clear();
 Mv.clear(); diSv.clear(); MdiSv.clear(); r.clear();
 Rd.clear(); Rdv.clear(); Rinv.clear(); Gi.clear();
 G.clear(); Nv.clear(); Giv.clear(); Gv.clear(); Rinvv.clear();  
 rv.clear(); g2.clear();
}

//c-%set_slice() paket untuk menampilkan hasil slice
std::vector<std::vector<double> > set_slice(std::vector<double> dat, std::vector<int> S, int l, int k, int *ba, int *ko)
{
 std::vector<std::vector<std::vector<double> > > T;
 T=opmat_vtt(dat,S[0],S[1],S[2]);
 std::vector<double> V(S[0]*S[1]*S[2]);
 std::vector<std::vector<double> > hasil;

 int x,y,z; 
 x=S[0]; y=S[1]; z=S[2];
 
 if(k==1)
   {
    for(int i=0;i<y;i++)
      {
       for(int j=0;j<z;j++)
         {
          V[j]=T[j][i][l];
         }
      hasil.push_back(V);
      }
   ba[0]=y; ko[0]=z;
   } else
 if(k==2)
   { 
    for(int i=0;i<x;i++)
     {
       for(int j=0;j<z;j++)
         {
          V[j]=T[j][l][i]; 
         }
       hasil.push_back(V);
     }
   ba[0]=x; ko[0]=z;
   } else
 if(k==3)
   {
    for(int i=0;i<x;i++)
      {
       for(int j=0;j<y;j++)
         {
          V[j]=T[l][j][i];
         }
        hasil.push_back(V);
      }
    ba[0]=x; ko[0]=y;
   }

 return hasil;
 hasil.clear(); V.clear(); T.clear(); dat.clear(); S.clear();
}

//c-%set_dplot: untuk membuat data yang bisa di plot.
void set_dplot(std::vector<double> dat,std::vector<int> S)
{ 
  int batas;
  std::vector<std::vector<double> > M,sementara,hasil;
  int b,k;
  for(int mat=1;mat<=3;mat++)
    {
     std::cout<<"iter ke"<<mat<<"\n"; 
    
     M=set_slice(dat,S,0,mat,&b,&k);
     batas=b*k;   
     std::vector<double> V(batas),V2(batas),V3(batas);
      
     for(int i=0;i<b;i++)
       {
        for(int j=0;j<k;j++)
          {
           V[i*k+j]=i*1.0;
           V2[i*k+j]=j*1.0;
           V3[i*k+j]=M[i][j];
          }
       } 
 
     sementara.push_back(V);
     sementara.push_back(V2);
     sementara.push_back(V3);
  
     hasil=opmat_transpose_double(sementara,3,batas);
    // out_dm(hasil,batas,3);
     std::string q="view_";
     fout_dm(hasil,q,mat,batas,3);  
     
     hasil.clear(); V.clear(); V2.clear(); V3.clear(); M.clear(); sementara.clear();
   }
  dat.clear();S.clear();
}
\end{verbatim}



% \chapter{Lampiran 5: Poisson.h dan Poisson.cpp}
\section*{poisson.h}
\scriptsize
\begin{verbatim}
#ifndef POISSON_H
#define POISSON_H

void poisson(std::vector<double> r, std::vector<int> S, std::vector<std::vector<int> > R,std::vector<double> G2);

void pois_dplot(std::vector<std::vector<double> > M,int b, int k, int ii,std::string q);

//c-%pois_fftw3() paket untuk menghitung nilai transformasi forier
fftw_complex *pois_fftw3(fftw_complex *dat,std::vector<int> N, int s);

//c-%cJ() Paket untuk menghitung hasil transformasi dibagi dengan Total matrix
fftw_complex *pois_cJ(std::vector<double> n,std::vector<int> S);

//c-%O() menghitung nilai transformasi/total matrix di kali dengan determinan matriks R.
fftw_complex *pois_O(fftw_complex *n,std::vector<int>S,std::vector<std::vector<int> > R);

//c-%pois_Linv()
fftw_complex *pois_Linv(fftw_complex *n,std::vector<std::vector<int> > R,std::vector<double> g2);

//c-pois_cI()
std::vector<double> pois_cI(fftw_complex *n,std::vector<int> S);

//c-%pois_real_kali() adalah mengalikan nilai dua imaginer dengan hasil real.
double pois_real_kali(fftw_complex *n,fftw_complex *m,int batas);
#endif
\end{verbatim}
\normalsize
\section*{poisson.cpp}
\scriptsize
\begin{verbatim}
 /*
==============================================================================================================
Nama package:
Fungsi Package:
Pembuat :
$$$$$$$$$$$$$$$$$$$$$$$$$$$$$$$$$$$$$$$$$$$$$$$$$$$$$$$$$$$$$$$$$$$$$$$$$$$$$$$$$$$$$$$$$$$$$$$$$$$$$$$$$$$$$$
*/

#include "edft.h"
#include "setup.h"
#include "opmat.h"
#include "package.h"
#include "out.h"
#include "poisson.h"

void poisson(std::vector<double> r, std::vector<int> S, std::vector<std::vector<int> > R, std::vector<double> G2)
{
 std::cout<<"menghitung poisson \n";
 //c-ps=prod(s)
 int ps=pac_prod(S);
// std::cout<<ps; 
 
 //c-one:membuat matriks satu. //o-ones(prod(S),1)
 std::vector<double> one; 
 one=opmat_ones(ps,1);// out_dm(one,one.size(),1);
// std::vector<double> onev;
// onev=opmat_mtv(one,one.size(),1);
// out_dv(onev);

 //o-sum(R,2)'
 std::vector<double> sR;
 sR=opmat_vkali_titik(opmat_sumd(opmat_intodoub(R,3,3),3,3),0.5); //c-sR, sum R
// out_dv(sR); 

 //c-A adalah versi vektor dari perkalian sR*onev. 
 //o- ones(prod(S),1)*sum(R,2)' 
 std::vector<double> A(ps*3);
 cblas_dger(CblasRowMajor,ps,sR.size(),1.0,&one[0],1,&sR[0],1,&A[0],3);
 std::vector<std::vector<double> >mA;
// out_dv(A); 
 mA=opmat_vtm(A,ps,3);
// out_dm(mA,ps,3);
 
 //c-rminA= r dikurang A
 //o-r-(ones....)
 std::vector<std::vector<double> > rminA;
 rminA=opmat_mmin(opmat_vtm(r,ps,3),mA,ps,3);
// out_dm(rminA,ps,3);
 
 //c-rminA2 matriks rminA di kuadratkan
 //o-().^2
 std::vector<std::vector<double> > rminA2;
 rminA2=opmat_vtm(opmat_vkuadrat(opmat_mtv(rminA,ps,3)),ps,3);
// out_dm(rminA2,ps,3);

 //c-eminA2s vektor dari matriks rminA2
 //o-sum(,2)
 std::vector<double> rminA2s;
 rminA2s=opmat_sumd(rminA2,ps,3);
// out_dv(rminA2s);

 //c- nilai jarak dr.
 //o-dr=sqrt(...)...
 std::vector<double> dr(ps);
 dr=opmat_sqrt(rminA2s); //c-menghitung jarak dr ke titik pusat
// out_dv(dr);

 //c-deklarasi nilai sigma.
 double sigma1,sigma2;
 sigma1=0.75; sigma2=0.50;
 
 //c-menghitung -dr.^2
 std::vector<double> mindr2(ps);
 mindr2=opmat_vkali_titik(opmat_vkuadrat(dr),-1.);
//out_dv(mindr2);

 //c-matrix atas1, atas2. bagian atas dari g1
 double pengali1,pengali2;
 std::vector<double> atas1(ps),atas2(ps);
 pengali1=1/(2*sigma1*sigma1); //std::cout<<pengali1<<"\n"; 
 pengali2=1/(2*sigma2*sigma2);// std::cout<<pengali2<<"\n";
 atas1=opmat_vexp(opmat_vkali_titik(mindr2,pengali1)); // out_dv(atas1);
 atas2=opmat_vexp(opmat_vkali_titik(mindr2,pengali2)); // out_dv(atas2);

 //c-elemen bawah pengali.
 double bawah1,bawah2;
 bawah1=pow(sqrt(2*PI*sigma1*sigma1),3); //std::cout<<bawah1<<"\n";
 bawah2=pow(sqrt(2*PI*sigma2*sigma2),3); //std::cout<<bawah2<<"\n";
 
 //c-g1 dan g2: menghitung nilai normalisasi gaussian (0.5 dan 0.75)
 std::vector<double> g1(ps),g2(ps);
 g1=opmat_vkali_titik(atas1,1/bawah1); //out_dv(g1);
 g2=opmat_vkali_titik(atas2,1/bawah2); //out_dv(g2);

 //c-membuat nilai n //o-n=g2-g1; //c-mendefinisikan rapat muatan.
 std::vector<double> n;
 n=opmat_vmin(g2,g1); //out_dv(n);
 
 //c-menghitung norm dan integral harus mendekati 1 dan 0.
 double normg1,normg2,detR,tcharge;
 detR=opmat_det3(opmat_intodoub(R,3,3));
 normg1=opmat_sum(g1)*detR/ps;
 normg2=opmat_sum(g2)*detR/ps;
 tcharge=opmat_sum(n)*detR/ps;
 std::cout<<"Check nilai normalisasi g1 = "<<normg1<<"\n";
 std::cout<<"Check nilai normalisasi g2 = "<<normg2<<"\n";
 std::cout<<"Cek muatan total : "<<tcharge<<"\n";

 //c-plot data yang sudah di normalisasi dengan gaussian
 std::vector<std::vector<double> >sl; //c-%sl adalah slice yang akan di plot
 std::string q="gaussian_";
 for (int i=0;i<3;i++)
   {
    int b,k;
    sl=set_slice(n,S,S[i]/2-1,i+1,&b,&k); 
   // out_dm(sl,b,k);
    pois_dplot(sl,b,k,i,q);
    sl.clear();
   }

 //c-mengeluarkan nilai setelah di transform
 std::vector<double> phi(ps);
// fourier=pois_fftw3(n,S,1); //out_dv(fourier);

 //c-%phi adalah nilai solusi dari poisson.
 phi=pois_cI(pois_Linv(pois_O(pois_cJ(n,S),S,R),R,G2),S); 
// out_dv(phi);

 //c-plot data yang dihasilkan dari solusi poisson
 std::vector<std::vector<double> > plotphi; //c-%plotphi adalah nilai phi yang dipersiapkan untuk diplot
 q="phi_";
 for (int i=0;i<3;i++)
   {
    int b,k;
    plotphi=set_slice(phi,S,S[i]/2-1,i+1,&b,&k); 
   // out_dm(sl,b,k);
    pois_dplot(plotphi,b,k,i,q);
    plotphi.clear();
   }
   
//c- Check energi total Coulomb
double Unum,Uanal; //c-%unum adalah energi total coulomb numerik, Uanal=analitik
Unum=0.5*pois_real_kali(pois_cJ(phi,S),pois_O(pois_cJ(n,S),S,R),ps);
std::cout<<"Nilai energi total Coulomb numerik = "<<Unum<<"\n";
Uanal=((1/sigma1+1/sigma2)/2-sqrt(2)/sqrt(sigma1*sigma1+sigma2*sigma2))/sqrt(PI);
std::cout<<"Nilai energi total Coulomb analitik = "<<Uanal<<"\n";



 //c- menghapus sampah....
 r.clear(); S.clear(); R.clear(); one.clear(); sR.clear(); A.clear(); mA.clear(); rminA.clear(); rminA2.clear(); dr.clear(); mindr2.clear(); atas1.clear(); atas2.clear(); g1.clear(); g2.clear(); n.clear(); sl.clear(); phi.clear(); plotphi.clear();
}

//=====================================================================
//c- %pois_dplot untuk membuat data plot poisson.
void pois_dplot(std::vector<std::vector<double> > M,int b, int k, int ii,std::string q)
{
 std::vector<double> v1(b*k), v2(b*k), v3(b*k);
 std::vector<std::vector<double> > hasil,plot;
 for(int i=0;i<b;i++)
       {
        for(int j=0;j<k;j++)
          {
           v1[i*k+j]=i*1.0;
           v2[i*k+j]=j*1.0;
           v3[i*k+j]=M[i][j];
          }
       } 
  hasil.push_back(v1);
  hasil.push_back(v2);
  hasil.push_back(v3);
  
  plot=opmat_transpose_double(hasil,3,b*k);
//  std::string q="gaussian_";
  fout_dm(plot,q,ii,b*k,3);
 
  M.clear(); v1.clear(); v2.clear(); v3.clear(); hasil.clear(); plot.clear();
}

//=====================================================================
//c-%pois_fftw3() paket untuk menghitung nilai transformasi forier
fftw_complex *pois_fftw3(fftw_complex* dat,std::vector<int> N, int s)
{
  int x,y,z;
  x=N[0]; y=N[1]; z=N[2];
  int batas=x*y*z;
 
  //c-mendefinisikan nilai input output.
  fftw_complex *in, *out;
  fftw_plan my_plan;
  in=(fftw_complex*) fftw_malloc(sizeof(fftw_complex)*batas);
  out=(fftw_complex*) fftw_malloc(sizeof(fftw_complex)*batas);

  if (s==1)
   {
   // std::cout<<"belum didefinisikan";
     my_plan=fftw_plan_dft_3d(z,y,x,dat,out,FFTW_BACKWARD,FFTW_ESTIMATE);
      fftw_execute(my_plan);
   }

  else
   {
     my_plan=fftw_plan_dft_3d(z,y,x,dat,out,FFTW_FORWARD,FFTW_ESTIMATE); 
      fftw_execute(my_plan);
    }
  return out;

 //c- Bersih bersih
 fftw_destroy_plan(my_plan); fftw_free(in); fftw_free(out); fftw_free(dat); N.clear();
}

//=====================================================================
//c-%pois_cJ() Paket untuk menghitung hasil transformasi dibagi dengan Total matrix
//o-%cJ(in)
fftw_complex *pois_cJ(std::vector<double> n,std::vector<int> S)
{
 int batas=S[0]*S[1]*S[2]; 
 fftw_complex *hasil,*sementara,*bagi,*in;

 hasil=(fftw_complex*) fftw_malloc(sizeof(fftw_complex)*batas);
 sementara=(fftw_complex*) fftw_malloc(sizeof(fftw_complex)*batas);
 in=(fftw_complex*) fftw_malloc(sizeof(fftw_complex)*batas);
 bagi=(fftw_complex*) fftw_malloc(sizeof(fftw_complex)*1);

 for(int i=0;i<batas;i++)
 in[i]=(fftw_complex) n[i];
  
 sementara=pois_fftw3(in,S,-1);

 bagi[0]=(fftw_complex) batas;
 
 for(int i=0;i<batas;i++)
 hasil[i]=sementara[i]/bagi[0];

 return hasil;
 //c-bersih bersih....
 fftw_free(hasil); fftw_free(sementara); n.clear();S.clear();
}

//=====================================================================
//c-%O() menghitung nilai transformasi/total matrix di kali dengan determinan matriks R.
fftw_complex *pois_O(fftw_complex *n,std::vector<int>S,std::vector<std::vector<int> > R)
{
 int batas=S[0]*S[1]*S[2];
 double determinan=opmat_det3(opmat_intodoub(R,3,3));

 fftw_complex *hasil,*sementara,*kali;

 hasil=(fftw_complex*) fftw_malloc(sizeof(fftw_complex)*batas);
 sementara=(fftw_complex*) fftw_malloc(sizeof(fftw_complex)*batas);
 kali=(fftw_complex*) fftw_malloc(sizeof(fftw_complex)*1); 
 
 kali[0]=(fftw_complex) determinan;

 for(int i=0;i<batas;i++)
 hasil[i]=n[i]*kali[0];

 return hasil;
 //c-bersih bersih
 fftw_free(hasil); fftw_free(sementara); fftw_free(kali); S.clear(); R.clear();
}

//======================================================================
//c-pois_Linv()
fftw_complex *pois_Linv(fftw_complex *n,std::vector<std::vector<int> > R,std::vector<double> g2)
{
 int batas=g2.size();
 double pengali=-4*PI;
 double det=opmat_det3(opmat_intodoub(R,3,3));
 std::vector<double> bagi;
 
 bagi=opmat_vkali_titik(g2,-1.*det); 

 fftw_complex *hasil,*sementara,*cbagi,*cpengali;

 cpengali=(fftw_complex*) fftw_malloc(sizeof(fftw_complex)*1);
 cbagi=(fftw_complex*) fftw_malloc(sizeof(fftw_complex)*batas);
 hasil=(fftw_complex*) fftw_malloc(sizeof(fftw_complex)*batas);
 sementara=(fftw_complex*) fftw_malloc(sizeof(fftw_complex)*batas);
 
 cpengali[0]=(fftw_complex) pengali;
 
 for (int i=0;i<batas;i++)
 cbagi[i]=(fftw_complex) bagi[i];

 for (int i=0;i<batas;i++)
 hasil[i]=cpengali[0]*n[i]/cbagi[i];
 hasil[0]=(fftw_complex) 0;

 return hasil;
 //c-bersih bersih
 fftw_free(n);fftw_free(cbagi);fftw_free(hasil);fftw_free(sementara);fftw_free(cpengali); bagi.clear(); g2.clear(); R.clear();
}

//======================================================================
//c-%pois_cI()
std::vector<double> pois_cI(fftw_complex *n,std::vector<int> S)
{
 int batas=S[0]*S[1]*S[2]; 
 fftw_complex *sementara,*bagi;
 std::vector<double> hasil(batas);

 sementara=(fftw_complex*) fftw_malloc(sizeof(fftw_complex)*batas);
 bagi=(fftw_complex*) fftw_malloc(sizeof(fftw_complex)*1);
 
 sementara=pois_fftw3(n,S,1);

 bagi[0]=(fftw_complex) batas;
 
 for(int i=0;i<batas;i++)
 hasil[i]=creal(sementara[i]);

 return hasil;
 //c-bersih bersih....
 hasil.clear(); fftw_free(sementara); fftw_free(n);S.clear();
}

//=====================================================================
//c-%pois_real_kali() adalah mengalikan nilai dua imaginer dengan hasil real.
double pois_real_kali(fftw_complex *n,fftw_complex *m,int batas)
{
 fftw_complex sementara=(fftw_complex) 0;
 double hasil;

 for(int i=0;i<batas;i++)
   {
    sementara+=n[i]*m[i];
   }
 hasil=creal(sementara);
 return hasil;
}
\end{verbatim}


% \include{backmatter/lampiranF}
\end{document}

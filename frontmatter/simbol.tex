\chapter{DAFTAR SIMBOL}

\begin{table}[htbp]
\centering
\footnotesize % 10pt
\renewcommand{\arraystretch}{1.25}
\begin{tabular}{|c|p{10cm}|}
\hline
\textbf{Simbol} & \textbf{Definisi} \\
\hline
$\langle \cdot \rangle$ 
& Notasi tuple atau struktur terurut. \\
\hline
$\times$ 
& Produk kartesian antara dua himpunan. \\
\hline
$\to$ 
& Pemetaan fungsi dari domain ke kodomain. \\
\hline
$\Delta(\cdot)$
& Himpunan semua distribusi probabilitas atas suatu himpunan. \\
\hline
$\cdot$
& Placeholder untuk argumen fungsi atau distribusi. \\
\hline
$\sim$
& Relasi bahwa suatu variabel diambil dari distribusi tertentu. \\
\hline
$\mathbb{E}[\cdot]$
& Ekspektasi matematis terhadap suatu variabel acak. \\
\hline
$\mathbb{P}(\cdot)$
& Probabilitas suatu peristiwa. \\
\hline
$\sum$
& Operator penjumlahan. \\
\hline
$\prod$
& Operator perkalian produk atas indeks. \\
\hline
$\arg\max$
& Operator pencari argumen yang memaksimalkan suatu fungsi. \\
\hline
$\approx$
& Relasi aproksimasi antara dua nilai. \\
\hline
$|\cdot|$
& Kardinalitas atau panjang suatu urutan. \\
\hline
$o_{1:t}$, $o_{\le t}$
& Notasi sejarah observasi dari langkah awal hingga langkah ke-$t$. \\
\hline
$a_t^{*}$
& Aksi rujukan atau aksi benar pada langkah ke-$t$. \\
\hline
$p_t$
& Probabilitas bahwa aksi pada langkah ke-$t$ benar. \\
\hline
$f_\theta$
& Fungsi keputusan parametrik. \\
\hline
$:$ 
& Penanda bahwa suatu simbol adalah fungsi dari satu himpunan ke himpunan lain. \\
\hline
\end{tabular}
\end{table}
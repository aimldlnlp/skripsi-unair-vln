\chapter{\fontsize{12}{12}\selectfont KETENTUAN PENGGUNAAN SKRIPSI}

\noindent Ketentuan hak cipta bagi skripsi yang tidak dipublikasikan, terdaftar, tersedia, serta terbuka untuk umum di Perpustakaan Universitas Airlangga, dimiliki penulis dengan mengikuti aturan HKI yang berlaku di Universitas Airlangga. Referensi kepustakaan diperkenankan dicatat, tetapi pengutipan atau peringkasan hanya dapat dilakukan dengan seizin penulis dan harus disitasi sesuai dengan kaidah ilmiah. Memperbanyak atau menerbitkan sebagian atau seluruh skripsi haruslah seizin Penulis\\

\noindent Sitasi Skripsi ini dapat ditulis sebagai berikut:\\
Nama belakang, Nama depan. (Tahun): Judul Skripsi. Skripsi. Surabaya: Universitas Airlangga.\\

\noindent \textit{Contoh:}\\
Sofiah, A. (2015). Desain dan Implementasi Piranti EMG Mutlikanal Berbasis IIR Filter dalam Penyadapan Sinyal Elektrik Otot Ekstrimitas. Skripsi. Universitas Airlangga.\\

\noindent Sofiah, A. (2015). \textit{Design and Implementation of IIR Filter-Based Multichannel EMG for Myoelectric Signal of Extremity Muscles Recording}. Undergraduate Thesis. Universitas Airlangga.
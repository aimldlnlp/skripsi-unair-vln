% Auto-generated: glossary entries sorted alphabetically by KEY.
% Note: keys (inside {...}) are used with \gls{key}. The visible term is in name={...}.

\newglossaryentry{agen-embodied}{
  name={Agen \textit{embodied}},
  description={Agen yang memiliki tubuh/sensor dan berinteraksi langsung dengan lingkungan (mis.\ bergerak dan menerima observasi visual) untuk menyelesaikan tugas.}
}

\newglossaryentry{alternation}{
  name={Alternation},
  description={Tipologi code-switching ketika penutur berganti bahasa dengan memanfaatkan struktur masing-masing bahasa secara relatif terpisah (bergantian).}
}

\newglossaryentry{asr}{
  name={ASR},
  description={\textit{Automatic Speech Recognition}. Modul pengenalan ujaran otomatis yang mengubah sinyal audio menjadi teks.}
}

\newglossaryentry{audit-subtugas}{
  name={Audit sub-tugas},
  description={Kemampuan menelusuri dan mengevaluasi keberhasilan serta kegagalan pada tingkat sub-tugas (bukan hanya tujuan akhir), termasuk keterlacakan rencana--instruksi.}
}

\newglossaryentry{belief-state}{
  name={\textit{Belief state}},
  description={Distribusi probabilitas atas status laten ketika agen tidak dapat mengamati keadaan lingkungan secara penuh.}
}

\newglossaryentry{blackbox}{
  name={\textit{Black box}},
  description={Asumsi bahwa model diperlakukan sebagai komponen yang hanya diakses melalui input--output, tanpa memodifikasi arsitektur atau bobot internalnya.}
}

\newglossaryentry{cgt}{
  name={CGT},
  description={Metrik CSR yang dibobot oleh panjang lintasan rujukan tiap sub-tugas (mengaproksimasi tingkat kesulitan relatif).}
}

\newglossaryentry{cmi}{
  name={CMI},
  description={\textit{Code-Mixing Index}. Indeks pencampuran bahasa yang mengukur derajat code-mixing berdasarkan proporsi token bahasa dominan vs.\ non-dominan.}
}

\newglossaryentry{codemixing}{
  name={\textit{Code-mixing}},
  description={Pencampuran unsur dua bahasa dalam satu segmen ujaran; sering dipakai bergantian dengan istilah code-switching pada literatur komputasional.}
}

\newglossaryentry{codeswitching}{
  name={\textit{Code-switching}},
  description={Pergantian bahasa dalam satu ujaran/teks; pada penelitian ini antara Bahasa Indonesia dan Bahasa Inggris pada berbagai tingkat (antar-kalimat, intra-kalimat, hingga kata).}
}

\newglossaryentry{compounding-error}{
  name={\textit{Compounding error}},
  description={Akumulasi kesalahan kecil per langkah yang dapat menyebabkan deviasi besar pada episode panjang.}
}

\newglossaryentry{congruent-lex}{
  name={\textit{Congruent lexicalization}},
  description={Tipologi code-switching ketika kedua bahasa berbagi kerangka sintaks yang kompatibel sehingga unsur leksikal dari kedua bahasa dapat bercampur lebih bebas.}
}

\newglossaryentry{continuous-control}{
  name={\textit{Continuous control}},
  description={Pengaturan aksi kontinu (mis.\ kecepatan linear dan sudut) yang lebih realistis dibanding aksi diskret berbasis node.}
}

\newglossaryentry{crf}{
  name={CRF},
  description={\textit{Conditional Random Field}. Model probabilistik untuk \textit{sequence labeling} yang memodelkan dependensi antar label berurutan (mis.\ pada LID).}
}

\newglossaryentry{csr}{
  name={CSR},
  description={\textit{Conditional Success Rate}. Metrik sub-tugas yang memberi bobot lebih tinggi pada keberhasilan yang mempertahankan ketergantungan urutan antar-tahap.}
}

\newglossaryentry{cvdn}{
  name={CVDN},
  description={\textit{Cooperative Vision-and-Dialog Navigation}. Dataset VLN berbasis dialog yang menekankan konteks percakapan selama navigasi.}
}

\newglossaryentry{dataset}{
  name={Dataset},
  description={Kumpulan data terstruktur (mis.\ episode, trajektori, instruksi, dan metadata) yang digunakan untuk pelatihan dan evaluasi model.}
}

\newglossaryentry{embodied-ai}{
  name={\textit{Embodied AI}},
  description={Bidang AI yang mempelajari agen yang berinteraksi melalui siklus persepsi--aksi di lingkungan nyata atau simulasi.}
}

\newglossaryentry{eos}{
  name={EOS},
  description={\textit{End-of-sequence}. Token khusus yang menandai akhir urutan keluaran pada proses dekoding model generatif.}
}

\newglossaryentry{episode}{
  name={Episode},
  description={Satu sesi interaksi agen--lingkungan untuk menuntaskan sebuah tugas, terdiri dari rangkaian langkah observasi dan aksi.}
}

\newglossaryentry{filter-heuristik}{
  name={Filter heuristik},
  description={Aturan/pengecek sederhana berbasis fitur atau kamus untuk menyaring keluaran agar memenuhi kendala tertentu (mis.\ kepatuhan rasio bahasa atau kamus landmark).}
}

\newglossaryentry{foundation-model}{
  name={\textit{Foundation model}},
  description={Model berukuran besar yang dilatih pada data masif dan dapat diadaptasi ke berbagai tugas (mis.\ melalui \textit{zero-shot} atau \textit{few-shot}).}
}

\newglossaryentry{fvm}{
  name={FVM},
  description={\textit{Foundation Vision Model}. Istilah umum untuk model visi fondasi yang dilatih pada data masif dan dapat diadaptasi lintas tugas.}
}

\newglossaryentry{ganjaran}{
  name={Ganjaran (\textit{reward})},
  description={Sinyal nilai yang dipakai untuk mengevaluasi tindakan agen pada setiap langkah dalam kerangka pembelajaran penguatan.}
}

\newglossaryentry{gibson}{
  name={Gibson},
  description={Dataset/lingkungan 3D lain yang sering digunakan untuk simulasi embodied AI dan navigasi.}
}

\newglossaryentry{gpu}{
  name={GPU},
  description={\textit{Graphics Processing Unit}. Perangkat keras komputasi paralel yang umum dipakai untuk mempercepat pelatihan/inferensi model deep learning.}
}

\newglossaryentry{grounding}{
  name={\textit{Visual grounding}},
  description={Penyelarasan token/frasa pada instruksi dengan entitas visual (objek, landmark), relasi spasial, dan konteks lingkungan yang diamati agen.}
}

\newglossaryentry{habitat}{
  name={Habitat},
  description={Simulator untuk embodied AI yang menyediakan sensor (RGB, kedalaman, dan lain-lain) dan eksekusi episode yang efisien.}
}

\newglossaryentry{hm3d}{
  name={HM3D},
  description={\textit{Habitat-Matterport 3D}. Dataset rekonstruksi indoor 3D untuk Habitat yang berisi ribuan scene dengan variasi layout dan fidelitas visual.}
}

\newglossaryentry{horizon}{
  name={Horizon},
  description={Panjang sebuah episode (jumlah langkah aksi/observasi) pada tugas navigasi.}
}

\newglossaryentry{iindex}{
  name={I-index},
  description={Indeks yang mengukur probabilitas terjadinya perpindahan label bahasa antar token berurutan dalam suatu deret.}
}

\newglossaryentry{image-tagging}{
  name={\textit{Image tagging}},
  description={Tugas memberikan label/tag semantik pada sebuah citra (mis.\ objek, atribut, atau konteks) yang relevan untuk deskripsi atau navigasi.}
}

\newglossaryentry{imu}{
  name={IMU},
  description={\textit{Inertial Measurement Unit}. Sensor inersia yang mengukur percepatan dan kecepatan sudut untuk membantu estimasi gerak.}
}

\newglossaryentry{insertion}{
  name={Insertion},
  description={Tipologi code-switching ketika unsur (kata/frasa) dari bahasa kedua disisipkan ke kerangka sintaks bahasa utama.}
}

\newglossaryentry{instruksi-navigasi}{
  name={Instruksi navigasi},
  description={Teks perintah yang menjelaskan urutan tindakan agar agen bergerak dari posisi awal menuju tujuan di lingkungan.}
}

\newglossaryentry{inter-sentential}{
  name={\textit{Inter-sentential switching}},
  description={Pergantian bahasa antar-kalimat (satu kalimat Indonesia lalu kalimat berikutnya Inggris, atau sebaliknya).}
}

\newglossaryentry{intra-sentential}{
  name={\textit{Intra-sentential switching}},
  description={Pergantian bahasa di dalam satu kalimat, mis.\ penyisipan frasa/clauses bahasa lain ke struktur kalimat utama.}
}

\newglossaryentry{isr}{
  name={ISR},
  description={\textit{Independent Success Rate}. Rata-rata keberhasilan pada tingkat sub-tugas tanpa mempertimbangkan ketergantungan antar-tahap.}
}

\newglossaryentry{kebijakan}{
  name={Kebijakan (\textit{policy})},
  description={Aturan/fungsi yang memetakan observasi (dan instruksi) menjadi aksi yang dipilih agen.}
}

\newglossaryentry{le}{
  name={LE},
  description={\textit{Language Entropy}. Entropi distribusi bahasa pada deret token; makin tinggi berarti komposisi bahasa makin beragam/seimbang.}
}

\newglossaryentry{lhprvln}{
  name={LHPR-VLN},
  description={\textit{Long-Horizon Planning and Reasoning VLN}. Benchmark long-horizon yang mengevaluasi penyelesaian multi-tahap beserta metrik per-sub-tugas.}
}

\newglossaryentry{lhvln}{
  name={LH-VLN},
  description={\textit{Long-Horizon Vision--Language Navigation}. Varian VLN dengan tujuan global dicapai melalui rangkaian sub-tugas multi-target yang saling bergantung, sehingga menuntut konsistensi rencana lintas langkah.}
}

\newglossaryentry{lid}{
  name={LID},
  description={\textit{Language Identification}. Tugas mengidentifikasi bahasa (mis.\ ID/EN) pada tingkat token atau ujaran.}
}

\newglossaryentry{llm}{
  name={LLM},
  description={\textit{Large Language Model}. Model bahasa berukuran besar yang dapat menghasilkan teks dan mengikuti instruksi, digunakan sebagai komponen generatif pada pipeline.}
}

\newglossaryentry{llm-in-the-loop}{
  name={LLM \textit{in-the-loop}},
  description={Skema yang menempatkan LLM sebagai komponen di dalam loop proses (membangkitkan sekaligus membantu pemeriksaan kualitas) selama pembangkitan data.}
}

\newglossaryentry{marianmt}{
  name={MarianMT},
  description={Model terjemahan mesin berbasis Marian yang digunakan untuk menghitung skor pada metrik T-Index.}
}

\newglossaryentry{memori}{
  name={Memori},
  description={Mekanisme penyimpanan konteks (mis.\ peta internal, progres, atau ringkasan langkah) untuk menjaga konsistensi keputusan pada episode panjang.}
}

\newglossaryentry{metadata}{
  name={Metadata},
  description={Informasi tambahan yang menyertai data utama (mis.\ label bahasa per token, indeks code-switching, atau identitas sub-tugas) untuk analisis dan audit.}
}

\newglossaryentry{metric}{
  name={Metrik evaluasi},
  description={Ukuran kuantitatif untuk menilai kualitas sistem/dataset, mis.\ keberhasilan navigasi atau intensitas pencampuran bahasa.}
}

\newglossaryentry{mindex}{
  name={M-index},
  description={Indeks yang menangkap keseimbangan distribusi bahasa (dua bahasa) berbasis konsentrasi vs.\ keragaman proporsi token.}
}

\newglossaryentry{model-visi-fondasi}{
  name={Model visi fondasi},
  description={\textit{Foundation model} pada ranah visi komputer yang menghasilkan representasi visual general dan dapat dipakai lintas tugas.}
}

\newglossaryentry{navgen}{
  name={NavGen},
  description={Platform/pipeline untuk menghasilkan episode long-horizon terstruktur dengan rencana lintasan dan deskripsi yang dapat diaudit.}
}

\newglossaryentry{navigation-graph}{
  name={\textit{Navigation graph}},
  description={Representasi topologi lingkungan dalam bentuk graf; node biasanya merepresentasikan lokasi/sudut pandang, dan edge merepresentasikan aksi perpindahan yang valid.}
}

\newglossaryentry{nlp}{
  name={NLP},
  description={\textit{Natural Language Processing}. Bidang yang memproses dan menganalisis bahasa manusia dengan metode komputasional.}
}

\newglossaryentry{perencanaan}{
  name={\textit{Planning} (perencanaan)},
  description={Proses menyusun strategi atau rute (sering kali hierarkis) agar agen mencapai target dengan mempertimbangkan ketergantungan langkah dan bukti visual.}
}

\newglossaryentry{pipeline}{
  name={\textit{Pipeline}},
  description={Rangkaian tahap terurut untuk menghasilkan, memproses, dan memvalidasi data (mis.\ perancangan tugas, eksekusi trajektori, tagging, hingga pembangkitan instruksi).}
}

\newglossaryentry{pomdp}{
  name={POMDP},
  description={\textit{Partially Observable Markov Decision Process}. Kerangka keputusan di bawah ketakteramatan parsial yang memodelkan status laten, observasi, dan transisi dinamis.}
}

\newglossaryentry{prompt}{
  name={\textit{Prompt}},
  description={Teks masukan yang diberikan kepada model bahasa untuk mengarahkan gaya, struktur, dan batasan keluaran (mis.\ rasio bahasa dan larangan kosakata).}
}

\newglossaryentry{r2r}{
  name={R2R},
  description={\textit{Room-to-Room}. Dataset VLN klasik berbasis graf sudut pandang pada Matterport3D dengan instruksi bahasa untuk mengikuti rute rujukan.}
}

\newglossaryentry{ram}{
  name={RAM},
  description={\textit{Recognize Anything Model}. Model \textit{image tagging} pralatih yang menghasilkan tag semantik kuat untuk berbagai objek/atribut pada citra.}
}

\newglossaryentry{rampp}{
  name={RAM++},
  description={Varian RAM yang memperluas penyelarasan citra--tag--teks untuk representasi semantik yang lebih kaya.}
}

\newglossaryentry{replanning}{
  name={\textit{Re-planning}},
  description={Penyesuaian rencana saat agen menyimpang, menemukan ambiguitas, atau mengalami kegagalan lokal selama eksekusi.}
}

\newglossaryentry{rgb}{
  name={RGB},
  description={\textit{Red, Green, Blue}. Format representasi warna pada citra yang umum digunakan pada sensor visual.}
}

\newglossaryentry{rlhf}{
  name={RLHF},
  description={\textit{Reinforcement Learning from Human Feedback}. Pemelajaran penguatan berbasis umpan balik manusia untuk menyelaraskan keluaran LLM agar lebih sesuai preferensi/kebijakan.}
}

\newglossaryentry{rxr}{
  name={RxR},
  description={\textit{Room-across-Room}. Dataset VLN multibahasa yang memperluas R2R dengan instruksi yang lebih beragam dan panjang.}
}

\newglossaryentry{sam}{
  name={SAM},
  description={\textit{Segment Anything Model}. Model segmentasi berbasis \textit{prompt} yang menghasilkan mask objek dari input visual.}
}

\newglossaryentry{scene}{
  name={\textit{Scene}},
  description={Satu lingkungan/adegan 3D (mis.\ satu rumah atau lantai bangunan) tempat episode navigasi dijalankan.}
}

\newglossaryentry{se}{
  name={SE},
  description={\textit{Span Entropy}. Entropi panjang span monolingual; mengukur variasi segmentasi monolingual dalam urutan token.}
}

\newglossaryentry{simulasi}{
  name={Simulasi},
  description={Pemodelan/perhitungan untuk meniru perilaku suatu sistem nyata dalam kondisi tertentu.}
}

\newglossaryentry{simulator}{
  name={Simulator},
  description={Perangkat lunak yang meniru lingkungan dan sensor agar eksperimen navigasi dapat dijalankan secara terkontrol tanpa pengambilan data dunia nyata.}
}

\newglossaryentry{skripsi}{
  name={Skripsi},
  description={Karya tulis ilmiah sebagai salah satu syarat kelulusan program sarjana.}
}

\newglossaryentry{spl}{
  name={SPL},
  description={\textit{Success weighted by Path Length}. Metrik SR yang dibobot oleh efisiensi lintasan (perbandingan lintasan aktual dengan lintasan rujukan terpendek).}
}

\newglossaryentry{sr}{
  name={SR},
  description={\textit{Success Rate}. Metrik keberhasilan tugas kompleks secara keseluruhan; bernilai berhasil hanya bila seluruh sub-tugas berhasil.}
}

\newglossaryentry{subtugas}{
  name={Sub-tugas},
  description={Bagian dari tugas long-horizon yang memiliki target antara dan harus diselesaikan berurutan sebelum mencapai tujuan global.}
}

\newglossaryentry{tag-semantik}{
  name={Tag semantik},
  description={Label kata/frasa yang merepresentasikan objek/atribut scene (mis.\ \textit{sofa}, \textit{door}) yang dipakai sebagai ringkasan semantik observasi visual.}
}

\newglossaryentry{tag-switching}{
  name={\textit{Tag switching}},
  description={Pergantian bahasa berupa sisipan tag/ungkapan pendek (mis.\ \textit{you know}, \textit{right}) di dalam ujaran bahasa utama.}
}

\newglossaryentry{tindex}{
  name={T-Index},
  description={Metrik kualitas titik switch yang memakai skor model terjemahan mesin untuk menilai transisi antarbahasa pada korpus.}
}

\newglossaryentry{trajektori}{
  name={Trajektori},
  description={Urutan pasangan observasi--aksi (atau posisi/pose) yang ditempuh agen selama episode.}
}

\newglossaryentry{tts}{
  name={TTS},
  description={\textit{Text-to-Speech}. Modul sintesis ujaran yang mengubah teks menjadi sinyal audio.}
}

\newglossaryentry{unk}{
  name={UNK},
  description={\textit{Unknown Token}. Token/label untuk kata yang tidak dikenali oleh kosakata model atau sistem pelabelan.}
}

\newglossaryentry{viewpoint}{
  name={\textit{Viewpoint}},
  description={Titik sudut pandang/kamera diskret pada dataset berbasis graf (mis.\ R2R/RxR) yang menjadi node pada navigation graph.}
}

\newglossaryentry{vln}{
  name={VLN},
  description={\textit{Vision--Language Navigation}. Tugas memetakan instruksi bahasa alami menjadi rangkaian aksi agar agen menavigasi lingkungan 3D.}
}

\newglossaryentry{vlnce}{
  name={VLN-CE},
  description={\textit{VLN in Continuous Environments}. Benchmark VLN dengan kontrol kontinu yang memasukkan dinamika dan ketidakpastian sensor sehingga lebih realistis.}
}

\newglossaryentry{waypoint}{
  name={\textit{Waypoint}},
  description={Titik antara/penanda target lokal yang membantu memecah navigasi panjang menjadi segmen-segmen yang lebih mudah diikuti dan dievaluasi.}
}
